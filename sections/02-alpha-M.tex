%!TEX root = ../lingnote-verbmorphs.tex

\subsection{M}\label{sec:alphalist-m}
\begin{morphdesc}[resume*=alphalist]
\item[m-]\label{m:m-}
	allomorph of perfective \X{wu-} in the coda of a syllable;
	currently used instead of perfective \X[w-pfv]{w-}
		only in some Inland Northern Tlingit varieties (Teslin and Carcross-Tagish),
	but may also occur elsewhere in older Tlingit (e.g.\ song lyrics, set phrases,
	19th century transcriptions)
	\begin{itemize}
	\item	\vbform{amsiteen}{pfv}[tr, \fm{g̱}, ach]{s/he/it caught sight of (saw) him/her/it}
			\vbmorph{a-&\gm{m-}&s-&i-&\rt[²]{tin}&-μμL}
				{\xx{3>3}&\xx{pfv}&\xx{xtn}&\xx{stv}&\rt[²]{see}&\·\xx{var}}
		\versus \vbform{x̱at wusiteen}{pfv}{she/he/it caught sight of (saw) me}
			\vbmorph{x̱at=&wu-&s-&i-&\rt[²]{tin}&-μμL}
				{\xx{1sg.o}&\xx{pfv}&\xx{xtn}&\xx{stv}&\rt[²]{see}&\·\xx{var}}
	\end{itemize}
\end{morphdesc}
