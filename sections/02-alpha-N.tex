%!TEX root = ../lingnote-verbmorphs.tex

\subsection{N}\label{sec:alphalist-n}
\begin{morphdesc}[resume*=alphalist]
\item[n-]\label{m:n-}
	\begin{enumerate}
	\item	\fm{n} conjugation class prefix, horizontal spatial orientation
		\begin{itemize}
		\item	\fm{nayx̱éixʼw!} (imp; subj intr, \fm{n}, \fm{-μμH} act) ‘you guys (go to) sleep!’
		\item	\fm{naḵahoon} (hort; tr, \fm{n}, \fm{-μμH} act) ‘let me sell it’
		\end{itemize}
	\item	progressive aspect prefix
		\begin{itemize}
		\item	\fm{yaa nx̱ax̱éin} (prog; tr, \fm{∅}, act) ‘I am going along eating it’\newline
			versus \fm{x̱ax̱á} (impfv) ‘I am eating it’
		\end{itemize}
	\end{enumerate}

\item[-n]\label{m:-n}
	stem suffix of uncertain meaning;
	causes ablaut /\ipa{a}, \ipa{u}/ → [\ipa{eː}] of \fm{\rt{Ca}} and \fm{\rt{Cu}} roots
	except for \fm{\rt[¹]{naᴸ}} ‘die’ and \fm{\rt[²]{ya}} ‘pack’
	\begin{enumerate}
	\item	with progressive aspect
		\begin{itemize}
		\item	\fm{yaa anax̱éin} (prog; tr, \fm{∅}, \fm{-μH} act) ‘s/he/it is going along eating him/her/it’ with root \fm{\rt[²]{x̱a}} ‘eat’\newline
			(not \fm[*]{yaa anax̱áan})\newline
			versus \fm{aawax̱áa} (pfv) ‘s/he/it ate him/her/it’
		\item	\fm{yaa anaskwéin} (prog; subj intr, \fm{∅}, ach) ‘s/he/it is coming to know him/her/it’ with root \fm{\rt[²]{kuᴸ}} ‘know’\newline
			(not \fm[*]{yaa anaskóon})\newline
			versus \fm{awsikóo} (pfv) ‘s/he/it came to know him/her/it’
		\end{itemize}
	\item	with conditional mood
	\item	with contingent mood
	\item	irregularly in a few imperfective state verbs
	\end{enumerate}

\item[na-]\label{m:na-}
	allomorph of \fm{n} conjugation prefix \X{n-} with epenthetic (filler) vowel \fm{a}

\item[-nás]\label{m:-nás}
	≡ \fm{-n-ás}
	combination of stem suffix \X{-n}
		and unknown \X{-ás};
	only occurs with the root \fm{\rt[²]{ḵe}} \~\ \fm{\rt[²]{ḵi}} ‘pay’;
	see \X{-ás} for more details
	and see \X{-s} for forms without \fm{-n}
\end{morphdesc}
