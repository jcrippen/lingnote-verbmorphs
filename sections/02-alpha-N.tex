%!TEX root = ../lingnote-verbmorphs.tex

\subsection{N}\label{sec:alphalist-n}
\begin{morphdesc}[resume*=alphalist]
\item[n-]\label{m:n-}
	\begin{enumerate}
	\item	\fm{n} conjugation class prefix, horizontal spatial orientation
		\begin{itemize}
		\item	\fm{nayx̱éixʼw!} (imp; subj intr, \fm{n}, \fm{-μμH} act) ‘you guys (go to) sleep!’
		\item	\fm{naḵahoon} (hort; tr, \fm{n}, \fm{-μμH} act) ‘let me sell it’
		\end{itemize}
	\item	progressive aspect prefix
		\begin{itemize}
		\item	\fm{yaa nx̱ax̱éin} (prog; tr, \fm{∅}, act) ‘I am going along eating it’\newline
			versus \fm{x̱ax̱á} (impfv) ‘I am eating it’
		\end{itemize}
	\end{enumerate}

\item[-n]\label{m:-n}
	stem suffix of uncertain meaning;
	causes ablaut /\ipa{a}, \ipa{u}/ → [\ipa{eː}] of \fm{\rt{Ca}} and \fm{\rt{Cu}} roots
	except for \fm{\rt[¹]{naᴸ}} ‘die’ and \fm{\rt[²]{ya}} ‘pack’
	\begin{enumerate}
	\item	with progressive aspect
		\begin{itemize}
		\item	\fm{yaa anax̱éin} (prog; tr, \fm{∅}, \fm{-μH} act) ‘s/he/it is going along eating him/her/it’ with root \fm{\rt[²]{x̱a}} ‘eat’\newline
			(not \fm[*]{yaa anax̱áan})\newline
			versus \fm{aawax̱áa} (pfv) ‘s/he/it ate him/her/it’
		\item	\fm{yaa anaskwéin} (prog; subj intr, \fm{∅}, ach) ‘s/he/it is coming to know him/her/it’ with root \fm{\rt[²]{kuᴸ}} ‘know’\newline
			(not \fm[*]{yaa anaskóon})\newline
			versus \fm{awsikóo} (pfv) ‘s/he/it came to know him/her/it’
		\end{itemize}
	\item	with conditional mood
	\item	with contingent mood
	\item	irregularly in a few imperfective state verbs
	\end{enumerate}

\item[na-]\label{m:na-}
	allomorph of \fm{n} conjugation prefix \X{n-} with epenthetic (filler) vowel \fm{a}

\item[-nás]\label{m:-nás}
	≡ \fm{-n-ás}
	combination of stem suffix \X{-n}
		and unknown \X{-ás};
	only occurs with the root \fm{\rt[²]{ḵe}} \~\ \fm{\rt[²]{ḵi}} ‘pay’;
	see \X{-ás} for more details
	and see \X{-s} for forms without \fm{-n}

\item[neil=]\label{m:neil=}
	Directional preverb ‘inside; home’ indicating direction to the inside of a building
		or to some location the speaker considers to be home.
	Derived from the noun \fm{neil} ‘home’ and the motion derivation
		\motderiv{NP-t/x̱/dé}{∅, \fm{-μμL} rep}{arriving at NP}
		with the punctual postposition \fm{-t} ‘to a point’.
	The punctual postposition \fm{-t} is normally absent with this preverb in common with a
		few others such as \X{ÿan=} ‘ashore’ and \X{ḵux̱=} ‘back’,
		having disappeared for as yet inadequately explained historical reasons.
	There are however a few rare cases of \fm{neil} with \fm{-t} for which see \X{neilt=}.
	This has two meanings depending on context, one where it means ‘inside’
		and the other where it means ‘home’.
	Generally a gloss should reflect whichever meaning is most appropriate.
	\begin{enumerate}
	\item	Motion verbs with \fm{neil=}.
		\begin{itemize}
		\item	\vbform{atkʼátskʼu g̱óot neil uwagút}{pfv}[subj intr, \fm{∅}, mot]{he went home without the boy}
			\parencite[168.6]{boas:1917}
				\vbmorph{atkʼátksʼu&g̱óot&\gm{neil=}&u-&wa-&\rt[¹]{gut}&-μH}
					{boy&without&home&\xx{zpfv}&\xx{stv}&\rt[¹]{go.\xx{sg}&\·\xx{var}}
		\end{itemize}
	\end{enumerate}


\item[neildé=]\label{m:neildé=}

\item[neilí=]\label{m:neilí=}
	Locational preverb ‘inside’ indicating location inside a building or cave.
	Derived from the noun \fm{neil} ‘home’ 
		with the special locative postposition allomorph
		\X[-i-loc]{-i} \~\ \X[-í-loc]{-í} ‘at’
		(instead of the regular allomorphs \fm{-xʼ} and \fm{-μ}).
	This locative allomorph is unique in that it only occurs with a handful of preverbs,
		for which see the detailed entry of \X[-í-loc]{-í}.
	Although the noun \fm{neil} only refers to ‘home’ in general, it has been extended
		metonymically to mean ‘inside’ in preverbs such as this one
		and the related directional preverb \X{neil=} / \X{neildé=} ‘home; inside’.
	Unlike \fm{neil=} / \fm{neildé=} which is attested with both ‘home’ and ‘inside’ meanings,
		the preverb \fm{neilí=} is only attested with the ‘inside’ meaning.
	This preverb is homophonous with the possessed form of the noun \fm{neil} as in
		\fm{ax̱ neilí} ‘my home’, but the preverb \fm{neilí=} never occurs with a possessor
		pronoun or noun phrase.
	\begin{itemize}
	\item	\vbform{neilí wuḵeiyi shaawát}{rel pfv}[subj intr, \fm{g̱}, ach]{women who sat inside}
		\parencite[266.132]{dauenhauer-dauenhauer:1987}
			\vbmorph{\gm{neil}&\gm{-í=}&wu-&\rt[¹]{ḵe}&-μμL&-yi&shaawát}
				{inside&\·\xx{loc}&\xx{pfv}&\rt[¹]{sit.\xx{pl}}&\·\xx{var}&\·\xx{rel}&woman}
	\item	\vbform{du neilí daak awlitlʼit}{pfv}[tr, \fm{∅}, mot]{he threw away all the rubbish from his house}
		\parencite[228.3228]{story-naish:1973}
		This form has a possessive pronoun preceding it and so would ordinarily be considered
			to be a possessed noun phrase and not a preverb.
		But the translation ‘from his house’ suggests that there should be an
			ablative postposition \fm{-dáx̱} ‘from, out of’ after \fm{neilí} 
			where none occurs.
		This lack of an ablative postposition makes it seem like possessed \fm{du neilí}
			‘his house’ is the object of the verb, with the whole thus meaning something
			like ‘he threw his house outside’.
		Although it is possible to reconcile this analysis with the translation,
			another possibility is that the object is covert and instead
			\fm{du neilí} is a locative postposition phrase ‘at his house’
			in which case the sentence would mean something like
			‘at his house he threw it out’.
		If this analysis is correct then \fm{-í} would be the allomorph of the locative
			postposition.
		We need other examples of the preverb \fm{neilí=} with a possessor to confirm
			that this is possible before deciding on the analysis for this form.
	\end{itemize}

\item[neilx̱=]\label{m:neilx̱=}

\end{morphdesc}
