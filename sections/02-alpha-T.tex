%!TEX root = ../lingnote-verbmorphs.tex

\subsection{T}\label{sec:alphalist-t}
\begin{morphdesc}[resume*=alphalist]
\item[-t]\label{m:-t}
	Ictive repetitive suffix describing repeated contact with a target.
	Unlike the \X{-ch}, \X{-k}, and \X{-x̱} repetitive suffixes,
		this suffix is never specified by conjugation class
		or as part of a motion derivation.
	Called ‘ictive’ from Latin \fm{ictus} ‘a blow, a strike, a beat’
		and so glossed \xx{ict},
		but may be instead be glossed \xx{rep} like other repetitive suffixes.
	Apparently productive given its occurrence with a fairly large number of verbs,
		but the grammatical and semantic details have yet to be analyzed.
		
	This suffix is historically related to the
		punctual postposition \fm{-t} ‘at/to/around a point’
		though the two are now functionally independent.
	It is also plausibly related to the obscure suffix \X{-át}
		but \fm{-t} normally does not occur with an epenthetic (filler) vowel.
	\begin{enumerate}
	\item	Repetitive suffix with any verb that denotes an event involving
			striking (or aiming at) a target.
		This suffix appears in a repetitive imperfective form
			or in forms derived from the repetitive imperfective
			(secondary aspectual derivations, 
			\citeauthor{leer:1991}’s (\citeyear{leer:1991}) “epiaspect”).
		It is attested with at least 35 different verb roots
			including both activity and achievement verbs;
			compare to verbs that occur with \X{-x̱aa}.
		It may be limited to transitive verbs but this is unverified.
		Predominantly found with \fm{∅} conjugation class though there are also examples
			with the three non-\fm{∅} classes (\fm{n}, \fm{g̱}, \fm{g}).
		As with other repetitives there may be multiple entities involved (plural),
			or a single entity multiple times (pluractional),
			or both.
		\begin{itemize}
		\item	\fm{–.únt} ‘repeatedly shoot and hit (with gun)’
			from \fm{\rt{.un}} ‘shoot (gun)’ in
			\newline
			\vbform{a.únt}{rep impfv}[tr, \fm{∅}, ach]{she/he/it repeatedly shoots and hits him/her/it}
			\parencites[63]{leer:1963}[58]{story:1966}[02/312]{leer:1973}[154]{leer:1976}
				\vbmorph{a-&\rt[²]{.un}&-μH&\gm{-t}}
					{\xx{3>3}&\rt[²]{shoot}&\·\xx{var}&\·\xx{ict}}
			\exand \vbform{aawa.únt}{rep pfv}{she/he/it repeatedly shot and hit him/her/it}
			\parencite[02/312]{leer:1973}
				\vbmorph{a-&μʷ-&wa-&\rt[²]{.un}&-μH&\gm{-t}}
					{\xx{3>3}&\xx{pfv}&\xx{stv}&\rt[²]{shoot}&\·\xx{var}&\·\xx{rep}}
			\versus \vbform{a.únx̱}{rep impfv}{she/he/it is repeatedly shooting him/her/it}
			\parencites[02/312]{leer:1973}[153]{leer:1976}
				\vbmorph{a-&\rt[²]{.un}&-μH&-x̱}
					{\xx{3>3}&\rt[²]{shoot}&\·\xx{var}&\·\xx{rep}}
			\exand \vbform{aawa.ún}{pfv}{she/he/it shot him/her/it}
				\vbmorph{a-&μʷ-&wa-&\rt[²]{.un}&-μH}
					{\xx{3>3}&\xx{pfv}&\xx{stv}&\rt[²]{shoot}&\·\xx{var}}
			\newline
			Also the derived noun \fm{at.úndi} ‘shooter’ \parencites[58]{story:1966}[02/312]{leer:1973}
				\vbmorph{at=&\rt{.un}&-μH&\gm{-t}&-i}
					{\xx{ind.n.o}&\rt{shoot}&\·\xx{var}&\·\xx{ict}&\·\xx{nmz}}
		\item	\fm{–sʼélʼt} ‘repeatedly tear, rip’
			from \fm{\rt{sʼelʼ}} ‘tear, rip’ in
			\newline
			\vbform{asʼélʼt}{rep impfv}[tr, \fm{n}, \fm{-μμH} act]{she/he/it repeatedly tears, rips him/her/it}
			\parencites[224.3161]{story-naish:1973}[09/218]{leer:1973}[518]{leer:1976}
				\vbmorph{a-&\rt{sʼelʼ}&-μH&\gm{-t}}
					{\xx{3>3}&\rt{tear}&\·\xx{var}&\·\xx{ict}}
			\versus \vbform{asʼéilʼ}{impfv}{she/he/it is tearing, ripping him/her/it}
			\parencites[09/218]{leer:1973}[518]{leer:1976}
				\vbmorph{a-&\rt{sʼelʼ}&-μμH}
					{\xx{3>3}&\rt{tear}&\·\xx{var}}
			\newline
			Also the nouns \fm{sʼéilʼ} ‘rip, tear, wound’
				and \fm{sʼélʼ} ‘rubber’.
		\item	Roots attested with ictive \fm{-t} in a repetitive imperfective form
			include \parencite[532]{crippen:2019}:
			\begin{inlinelist}
			\item	\fm{\rt{chʼex̱ʼ}} ‘point’
			\item	\fm{\rt{dax̱}} ‘adze out’
			\item	\fm{\rt{dlakw}} ‘scratch’
			\item	\fm{\rt{dzuᴸ}} ‘throw missile’ (\fm{–dzeit})
			\item	\fm{\rt{guᴴk}} ‘peck’
			\item	\fm{\rt{han}} ‘cut into strips’
			\item	\fm{\rt{jux̱ʼ}} ‘sling missile’
			\item	\fm{\rt{kitʼ}} ‘pry’
			\item	\fm{\rt{kʼix̱ʼ}} \~\ \fm{\rt{kʼex̱ʼ}} ‘gaff, snag’
			\item	\fm{\rt{ḵasʼ}} ‘split’
			\item	\fm{\rt{ḵʼish}} ‘swat’
			\item	\fm{\rt{sʼaxw}} ‘stack’
			\item	\fm{\rt{sʼelʼ}} ‘tear’
			\item	\fm{\rt{sʼuᴴw}} ‘chop’
			\item	\fm{\rt{taḵ}} ‘poke, spear’
			\item	\fm{\rt{teḵʼ}} ‘twist’
			\item	\fm{\rt{tex̱ʼ}} ‘wring’
			\item	\fm{\rt{taxʼ}} ‘bite’
			\item	\fm{\rt{tʼach}} ‘slap, swim’
			\item	\fm{\rt{tʼaxʼ}} ‘flick’
			\item	\fm{\rt{tʼax̱ʼ}} ‘pop’
			\item	\fm{\rt{tʼex̱ʼ}} ‘pound’
			\item	\fm{\rt{tʼiᴴÿ}} ‘elbow’
			\item	\fm{\rt{tʼuᴴk}} ‘shoot (bow)’
			\item	\fm{\rt{tsaḵ}} ‘poke’
			\item	\fm{\rt{tsix̱}} \~\ \fm{\rt{tsex̱}} ’kick’
			\item	\fm{\rt{tsix̱ʼ}} \~\ \fm{\rt{tsex̱ʼ}} ‘strangle’
			\item	\fm{\rt{tsuᴴw}} ‘push’
			\item	\fm{\rt{tsux̱}} ‘block’
			\item	\fm{\rt{tsʼikʼw}} \~\ \fm{\rt{tsʼukʼ}} ‘pinch’
			\item	\fm{\rt{.uᴴn}} ‘shoot (gun)’
			\item	\fm{\rt{walʼ}} ‘break’
			\item	\fm{\rt{xitʼ}} ‘sweep’
			\item	\fm{\rt{xʼasʼ}} ‘slice’
			\item	\fm{\rt{x̱ich}} ‘club’
			\item	\fm{\rt{x̱utʼ}} ‘adze’
			\end{inlinelist}
		\end{itemize}
	\item	In combination with \X{-kw} as \X{-kwt} \~\ \X{-kt},
			occurring with multipositional repetitive state imperfectives
			and with tendency state imperfectives.
		See \X{-kwt} \~\ \X{-kt} for details.
	\item	Repetitive suffix in some tendency state imperfectives based on CVC roots
			instead of \X{-kt} \~\ \X{-kwt} with CV roots (see above)
			or \X{-k} \~\ \X{-kw} with CVC roots.
		\begin{itemize}
		\item	\fm{–níkt} ‘tattle’
			from \fm{\rt{nik}} ‘tell, report’ in
			\newline
			\vbform{dliníkt}{rep impfv}[subj intr, conj?, state]{she/he/it tattles}
			\parencites[58]{story:1966}[101.1296]{story-naish:1973}
				\vbmorph{d-&l-&i-&\rt{nik}&-μH&\gm{-t}}
					{\xx{mid}&\xx{xtn}&\xx{stv}&\rt{tell}&\·\xx{var}&\·\xx{ict}}
		\item	\fm{–.úkt} ‘boil easily’
			from \fm{\rt[²]{.uᴴk}} ‘boil’ in
			\newline
			\vbform{dli.úkt}{rep impfv}[obj intr, conj?, state]{she/he/it boils easily}
			\parencites[02/181]{leer:1973}[167]{leer:1976}
				\vbmorph{d-&l-&i-&\rt{.uᴴk}&-μH&\gm{-t}}
					{\xx{mid}&\xx{xtn}&\xx{stv}&\rt{boil}&\·\xx{var}&\·\xx{ict}}
		\item	\fm{–.áḵwt} ‘quick to plan’
			from \fm{\rt[²]{.aḵw}} ‘plan, direct, command, try’ in
			\newline
			\vbform{at kaÿa.áḵwt}{rep impfv}[tr, \fm{n}?, state]{she/he/it is quick to plan/suggest things}
			\parencite[123]{leer:1976}
				\vbmorph{at=&ka-&ÿa-&\rt[²]{.aḵw}&-μH&\gm{-t}}
					{\xx{ind.n.o}&\xx{qual}&\xx{stv}&\rt[²]{plan}&\·\xx{var}&\·\xx{ict}}
		\item	\fm{–níkwt} ‘gets sick easily’
			from \fm{\rt[¹]{nikw}} ‘sick’ in
			\newline
			\vbform{yaníkwt}{impfv}[obj intr, \fm{n}?, state]{she/he/it gets sick easily}
			\parencite[04/177]{leer:1973}
				\vbmorph{ÿa-&\rt[¹]{nikw}&-μH&\gm{-t}}
					{\xx{stv}&\rt[¹]{sick}&\·\xx{var}&\·\xx{ict}}
		\end{itemize}
	\item	Frozen in a few CVC roots originally based on CV roots.
		\begin{itemize}
		\item	\fm{–tʼéet}
			from \fm{\rt[²]{tʼit}} ‘beachcomb, scavenge’
			from \fm{\rt[²]{tʼiᴸ}} \~\ \fm{\rt[²]{tʼeᴸ}} ‘find’ in
			\newline
			\vbform{ḵultʼéet}{impfv}[subj intr, \fm{n}, inv act]{she/he/it is beachcombing, scavenging}
				\parencites[128.1711]{story-naish:1973}[07/131]{leer:1973}
				\vbmorph{ḵu-&d-&l-&\rt[²]{tʼit}&-μμH}
					{\xx{areal}&\xx{apsv}&\xx{xtn}&\rt[²]{scavenge}&\·\xx{var}}
			\exand \vbform{at ḵunaltʼéetch}{hab}[tr, \fm{n}, inv act]{she/he/it is always looking for things}
				\parencite[07/131]{leer:1973}
				\vbmorph{at=&ḵu-&na-&d-&l-&\rt[²]{tʼit}&-μμH&-ch}
					{\xx{ind.n.o}&\xx{areal}&\xx{ncnj}&\xx{mid}&\xx{xtn}&\rt[²]{scavenge}&\·\xx{var}&\·\xx{rep}}
			\exalso \vbform{atʼéetxʼ}{rep impfv}[tr, \fm{n}, inv act]{she/he/it finds them}
				\parencite[07/130–131]{leer:1973}
				\vbmorph{a-&\rt[²]{tʼit}&-μμH&-xʼ}
					{\xx{3>3}&\rt[²]	{scavenge}&\·\xx{var}&\·\xx{pl}}
			\versus \vbform{aawatʼee}{pfv}[tr, \fm{n}/\fm{g}, ach]{she/he/it found him/her/it}
				\vbmorph{a-&μʷ-&wa-&\rt[²]{tʼiᴸ}&-μμL}
					{\xx{3>3}&\xx{pfv}&\xx{stv}&\rt[²]{find}&\·\xx{var}}
			\newline
			The root \fm{\rt[²]{tʼit}} ‘beachcomb, scavenge’ must be distinct
				from the root \fm{\rt[²]{tʼiᴸ}} \~\ \fm{\rt[²]{tʼeᴸ}} ‘find’
				for three reasons:
				(a) the root \fm{\rt[²]{tʼiᴸ}} should have \fm{-μμL}
					not \fm{-μμH} with an obstruent suffix like \fm{-t},
				(b) the stems \fm{–tʼéetxʼ} and \fm{–tʼéetch} have \fm{-μμH}
					instead of expected \fm{-μH} with two obstruent suffixes,
				and (c) the root \fm{\rt[²]{tʼiᴸ}} has become \fm{\rt[²]{tʼeᴸ}}
					in many Northern varieties but \fm{–tʼéet} does
					not show the same \fm{i} > \fm{e} sound change.
		\end{itemize}
	\item	Unclear meaning with two verb roots of unknown meaning.
		\begin{itemize}
		\item	\fm{–tált} ‘dissuade’
			from unknown \fm{\rt{tal}} in
			\newline
			\vbform{atált}{rep impfv}[tr, \fm{n}, inv act]{she/he/it is trying to dissuade, discourage him/her/it}
			\parencites[06/72]{leer:1973}[69.831]{story-naish:1973}
				\vbmorph{a-&\rt{tal}&-μH&\gm{-t}}
					{\xx{3>3}&\rt{dissuade?}&\·\xx{var}&\·\xx{ict}}
			\versus \vbform{ash wootált}{pfv}{she/he/it dissuaded, discouraged him/her/it}
			\parencites[06/72]{leer:1973}
				\vbmorph{ash=&wu-&μ-&\rt{tal}&-μH&\gm{-t}}
					{\xx{3prx.o}&\xx{pfv}&\xx{stv}&\rt{dissuade?}&\·\xx{var}&\·\xx{ict}}
		\item	\fm{–shéesht} ‘squeeze out’
			from unknown \fm{\rt{shish}} in
			\newline
			\vbform{alshéesht}{impfv}[tr, conj?, inv act]{she/he/it is cleaning him/her/it out by squeezing contents}
			\parencites[552]{leer:1976}
				\vbmorph{a-&lˢ-&\rt{shish}&-μμH&\gm{-t}}
					{\xx{3>3}&\xx{csv}&\rt{squeeze.out}&\·\xx{var}&\·\xx{ict}}
			\versus \vbform{awlishéesht}{pfv}{she/he/it is cleaned him/her/it out by squeezing contents}
			\parencites[10/88]{leer:1973}[552]{leer:1976}
				\vbmorph{a-&w-&lˢ-&i-&\rt{shish}&-μμH&\gm{-t}}
					{\xx{3>3}&\xx{pfv}&\xx{csv}&\xx{stv}&\rt{squeeze.out}&\·\xx{var}&\·\xx{ict}}
			\newline
			Compare 
			\begin{inlinelist}
			\item	\fm{\rt{shisʼ}} ‘strip, squeeze out’
				(thus likely \fm{–shéesht} < \fm[*]{\rt{shishʼ}} with */\ipa{ʃʼ}/)
			\item	\fm{\rt{shiᴴsh}} ‘try to outdo in eating competition’
				(\fm{ᴴ} suggests \fm[*]{\rt{shiˀsh}} < \fm[*]{\rt{shishʼ}})
			\item	\fm{shéesht} ‘lucky gambling stick’
			\item	\fm{keishísh} ‘mountain alder’
			\item	\fm{sʼelasheesh} ‘flathead duck’
			\item	\fm{sheesh} ‘inconnu’
			\item	\fm{\rt{shish}} ‘skim off, sip, slurp’
			\end{inlinelist}
		\end{itemize}
	\item	In some complex nouns derived from verbs.
		Some of these are only attested as nouns, but they imply the possibility
			of a verb with \fm{-t} from which they must be derived.
		\begin{itemize}
		\item	\fm{atkʼíx̱ʼdi} ‘gaffer, gaff fisherman’
			from \fm{\rt[²]{kʼix̱ʼ}} \~\ \fm{\rt[²]{kʼex̱ʼ}} ‘gaff, snag’
			\parencites[f04/130]{leer:1973}
			\vbmorph{at=&\rt[²]{kʼix̱ʼ}&-μH&\gm{-t}&-i}
				{\xx{ind.n.o}&\rt[²]{gaff}&\·\xx{var}&\·\xx{ict}&\·\xx{nmz}}
			\newline
			Compare \vbform{at kʼéx̱ʼt}{rep impfv}[tr, \fm{∅}/\fm{g}, ach]{she/he/it is repeatedly gaffing something}
			\parencites[91.1149]{story-naish:1973}[f04/130]{leer:1973}[778]{leer:1976}
				\vbmorph{at=&\rt[²]{kʼex̱ʼ}&-μH&\gm{-t}}
					{\xx{ind.n.o}&\rt[²]{gaff}&\·\xx{var}&\·\xx{ict}}
			\exand
				\begin{inlinelist}
				\item	\fm{kʼéx̱ʼaa} ‘gaff hook’
				\item	\fm{kakʼéx̱ʼaa} ‘crochet hook’
				\item	\fm{wóoshnáx̱ x̱ʼakakʼéix̱'i} ‘chain’
				\end{inlinelist}
		\item	\fm{kalḵásʼdi} ‘splitting knife/stick’
			from \fm{\rt[¹]{ḵasʼ}} ‘split; stick’
			\vbmorph{ka-&lˢ-&\rt[¹]{ḵasʼ}&-μH&\gm{-t}&-i}
				{\xx{qual}&\xx{csv}&\rt[¹]{split}&\·\xx{var}&\·\xx{ict}&\·\xx{nmz}}
			\newline
			Compare \vbform{aklaḵásʼt}{rep impfv}[tr, \fm{∅}/\fm{n}, ach]{she/he/it splits him/her/it}
			\parencites[205.2866]{story-naish:1973}[f01/47]{leer:1973}
				\vbmorph{a-&k-&lˢa-&\rt[¹]{ḵasʼ}&-μH&\gm{-t}}
					{\xx{3>3}&\xx{qual}&\xx{csv}&\rt[¹]{split}&\·\xx{var}&\·\xx{ict}}
		\item	\fm{taay kasʼúkdi} ‘cooked fat’
			from \fm{\rt[¹]{sʼikw}} \~\ \fm{\rt[¹]{sʼuk}} ‘crisp, toast, fry to crisp’
			\vbmorph{taaÿ&ka-&\rt{sʼuk}&-μH&\gm{-t}&-i}
				{fat&\xx{qual}&\rt{crisp}&\·\xx{var}&\·\xx{ict}&\·\xx{poss}}
			\newline
			Compare \vbform{akawlisʼúk}{pfv}[tr, \fm{∅}, ach]{she/he/it crisped, toasted, fried him/her/it}
			\parencites[97.1243, 231.3282]{story-naish:1973}[09/250–251]{leer:1973}[523]{leer:1976}
			\vbmorph{a-&ka-&w-&lˢ-&i-&\rt[¹]{sʼuk}&-μH}
				{\xx{3>3}&\xx{qual}&\xx{pfv}&\xx{csv}&\xx{stv}&\rt[¹]{crisp}&\·\xx{var}}
			\newline
			No verb based on this root is attested with \fm{-t} but this predicts
				repetitive imperfective forms like \fm{akasʼúkt} and \fm{aklasʼúkt}.
		\item	\fm{xákwdi} ‘empty shell of house, empty container’
			probably from \fm{\rt{xak}} \~\ \fm{\rt{xakw}} ‘skeleton, empty shell; dessicated’
			\vbmorph{\rt{xakw}&-μH&\gm{-t}&-i}
				{\rt{dessicated}&\·\xx{var}&\·\xx{ict}&\·\xx{nmz}}
			\newline
			Compare
			\begin{inlinelist}
			\item	\fm{ḵaa shakaxaagú} ‘skull’
				\parencite[f03/37]{leer:1973}
			\item	\fm{xákw} ‘sandbar’
				\parencite[f03/36]{leer:1973};
			\end{inlinelist}
			and further compare
			\begin{inlinelist}
			\item	\fm{xáak} ‘empty shell’
			\item	\fm{du xaagí} ‘her/his skeleton’
			\item	\vbform{wusixaak}{pfv}[obj intr, \fm{g̱}, ach]{she/he/it became a dried out shell, skeleton}
				\parencite[193.2687]{story-naish:1973}
			\item	\vbform{ḵukaawaxaak}{pfv}[impers, conj?, ach?]{weather became dry and crisp}
				\parencite[76.923]{story-naish:1973}
			\item	\fm{ḵuxaak} ‘dry weather’
				\parencite[f03/30]{leer:1973}
			\item	\vbform{ashxáak}{impfv}[tr, \fm{∅}, \fm{-μμH} act]{she/he/it is steaming him/her/it (shell) open}
				\parencite[f03/28]{leer:1973}
			\item	\vbform{x̱ʼawsixaak}{pfv}[obj intr?, conj?, ach?]{his/her/its mouth fell open (got dry?)}
				\parencite[f03/29]{leer:1973}
			\end{inlinelist}
		\item	\fm{atxáshdi} ‘cut leather’
			from \fm{\rt{xash}} ‘cut’
			\vbmorph{at=&\rt{xash}&-μH&\gm{-t}&-i}
				{\xx{ind.n.o}&\rt{cut}&\·\xx{var}&\·\xx{ict}&\·\xx{nmz}}
			\newline
			Compare
			\vbform{axáash}{impfv}[tr, \fm{n}/\fm{∅}, \fm{-μμH} act]{she/he/it cuts him/her/it}
			\vbmorph{a-&\rt{xash}&-μμH}
				{\xx{3>3}&\rt{cut}&\·\xx{var}}
			\parencite[613]{leer:1976}
			\newline
			No verb based on this root is attested with \fm{-t} but this predicts
				repetitive imperfective forms like \fm{axásht} and \fm{akaxásht}.
		\item	\fm{kaxʼásʼdi} ‘lumber’
			from \fm{\rt[²]{xʼasʼ}} ‘slice’
			\vbmorph{ka-&\rt[²]{xʼasʼ}&-μH&\gm{-t}&-i}
				{\xx{qual}&\rt{slice}&\·\xx{var}&\·\xx{ict}&\·\xx{nmz}}
			\exand \fm{sh kadaxʼásʼdi hít} ‘sawmill’ = ‘house that slices itself’
			\parencite[f04/20]{leer:1973}
			\vbmorph{sh=&ka-&da-&\rt[²]{xʼasʼ}&-μH&\gm{-t}&-i&hít}
				{\xx{rflx.o}&\xx{qual}&\xx{mid}&\rt[²]{slice}&\·\xx{var}&\·\xx{ict}&\·\xx{rel}&house}
			\newline
			compare \vbform{akaxʼásʼt}{rep impfv}[tr, \fm{∅}/\fm{g̱}, \fm{-μμH} act]{she/he/it repeatedly slices him/her/it}
			\parencites[173.2384]{story-naish:1973}[739]{leer:1976}
			\vbmorph{a-&ka-&\rt[²]{xʼasʼ}&-μH&\gm{-t}}
				{\xx{3>3}&\xx{qual}&\rt[²]{slice}&\·\xx{var}&\·\xx{rep}}
		\item	\fm{kax̱ʼeiltí} ‘crumbs’
			from \fm{\rt[²]{x̱ʼeᴴl}} ‘crumble, break into pieces’
			(but \X{-μμL})
			\parencite[f01/163]{leer:1973}
			\vbmorph{ka-&\rt[²]{x̱ʼeᴴl}&-μμL&\gm{-t}&-í}
				{\xx{sro}&\rt[²]{crumble}&\·\xx{var}&\·\xx{ict}&\·\xx{nmz}}
			\newline
			Compare
			\vbform{akaawax̱ʼéil}{pfv}[tr, \fm{g̱}, ach]{she/he/it crumbled, broke him/her/it into pieces}
			\parencite[f01/163]{leer:1973}
			\vbmorph{a-&ka-&μʷ-&wa-&\rt[²]{x̱ʼeᴴl}&-μμH}
				{\xx{3>3}&\xx{sro}&\xx{pfv}&\xx{stv}&\rt[²]{crumble}&\·\xx{var}}
			\exand \vbform{yei kx̱ax̱ʼéilt}{rep impfv}{I’m (trying to) break it apart}
			\parencite[f01/163]{leer:1973}
			\vbmorph{yei=&k-&x̱a-&\rt[²]{x̱ʼeᴴl}&-μμH&\gm{-t}}
				{down&\xx{sro}&\xx{1sg.s}&\rt[²]{crumble}&\·\xx{var}&\·\xx{ict}}
			\newline
			although this may instead be from contraction of \fm{kax̱ʼeil(í)} ‘crumble’ +
				\fm{eetí} ‘remains’.
		\end{itemize}
	\item	Possibly analyzable in some CVCC nouns with unexplained coda /\ipa{t}/.
		\begin{itemize}
		\item	\fm{káast} ‘barrel’
			from unknown \fm{\rt{kas}}
			(possibly a loanword but its source is unknown).
		\item	\fm{láḵt} ‘bentwood box’
			from unknown \fm{\rt{laḵ}};
			compare \fm{laax̱} ‘redcedar’.
		\item	\fm{núkt} ‘dusky grouse’
			perhaps from \fm{\rt{nuk}} ‘sg.\ sit’;
			compare \fm{–núkts} ‘sweet’ (with \X{-ts})
			and \fm{–núkch} ‘helpless’ (with \X{-ch}).
		\item	\fm{síxwdi} \~\ \fm{sáxwdi} \~\ \fm{súxdi} ‘handle, shaft’
			from unknown \fm{\rt{sixw}} \~\ \fm{\rt{saxw}};
			compare \fm{saaxw} ‘cockles’
			and \fm{séek} ‘belt’.
		\item	\fm{shéesht} ‘lucky gambling stick’
			from unknown \fm{\rt{shish}};
			see \fm{–shéesht} ‘squeeze out’ above for more detail.
		\end{itemize}
	\end{enumerate}

\item[-tʼ]\label{m:-tʼ}
	repetitive suffix only occurring with specific verbs;
	unlike the \X{-ch}, \X{-k}, and \X{-x̱} repetitive suffixes,
		this suffix is never specified by conjugation class
		or as part of a motion derivation;
	for some verbs \fm{-tʼ} occurs instead of
		the conjugation class-specific repetitive suffix (\fm{-ch}, \fm{-k}, \fm{-x̱})
		but other verbs can use either the conjugation class-specific suffix
		or \fm{-tʼ};
	this suffix seems to be productive
		and apparently involves the destruction of an entity
		but there are some puzzling exceptions (especially \fm{–kweitʼ} ‘know’ below);
	there may be a historical relationship 
		between \X{-lʼ}, \X{-sʼ}, \X{-tʼ}, and \X{-xʼ}
		since all of these are ejective
	\begin{enumerate}
	\item	repetitive suffix attested with eleven roots,
		appearing in a repetitive imperfective form
		or in forms derived from the repetitive imperfective
			(secondary aspectual derivations, 
			\citeauthor{leer:1991}’s (\citeyear{leer:1991}) “epiaspect”);
		the meaning involves destruction or dissolution of an entity
			with the exceptions of \fm{–kweitʼ} ‘know’ and \fm{–géitʼ} ‘predict’
				where \fm{-tʼ} is apparently just iterative
			and \fm{–sháttʼ} ‘capture’
				where \fm{-tʼ} is reportedly plural;
		as with other repetitives there may be multiple entities involved (plural),
			or a single entity multiple times (pluractional),
			or both
		\begin{itemize}
		\item	\fm{–géitʼ} ‘predict’
			from \fm{\rt[²]{ga}} ‘predict’ in
			\newline
			\vbform{ashoosgéitʼ}{rep impfv}[tr, \fm{n}, ach]{she/he/it repeatedly predicts him/her/it}
			\parencite[22097]{eggleston:2017}
				\vbmorph{a-&shu-&u-&s-&\rt[²]{ga}&-μᵉμH&\gm{-tʼ}}
					{\xx{3>3}&end&\xx{irr}&\xx{xtn}&\rt[²]{predict}&\·\xx{var}&\·\xx{rep}}
			\versus \vbform{ashoowsigaa}{pfv}{she/he/it predicted him/her/it}
				\vbmorph{a-&shu-&μw-&s-&i-&\rt[²]{ga}&-μμL}
					{\xx{3>3}&end&\xx{pfv}&\xx{xtn}&\xx{stv}&\rt[²]{predict}&\·\xx{var}}
		\item	\fm{–gántʼ} ‘burn’
			from \fm{\rt[¹]{gan}} ‘burn’ in
			\newline
			\vbform{aksagántʼ}{rep impfv}[tr, \fm{n}, ach]{she/he/it repeatedly burns him/her/it}
			\parencite[38.361]{story-naish:1973}
				\vbmorph{a-&k-&sa-&\rt[¹]{gan}&-μH&\gm{-tʼ}}
					{\xx{3>3}&\xx{qual}&\xx{csv}&\rt[¹]{burn}&\·\xx{var}&\·\xx{rep}}
			\versus \vbform{akawsigaan}{pfv}{she/he/it burned him/her/it}
				\vbmorph{a-&ka-&w-&s-&i-&\rt[¹]{gan}&-μμL}
					{\xx{3>3}&\xx{qual}&\xx{pfv}&\xx{csv}&\xx{stv}&\rt[¹]{burn}&\·\xx{var}}
			\newline
			also in nouns:
			\begin{itemize}
			\item	\fm{káx̱ gántʼi} ‘something roasted’
				\parencite[f05/20]{leer:1973}
				\vbmorph{ká&-x̱&\rt{gan}&-μH&\gm{-tʼ}&-i}
					{\xx{hsfc}&\·\xx{pert}&\rt{burn}&\·\xx{var}&\·\xx{rep}&\·\xx{nmz}}
			\item	\fm{koogántʼi} ‘burned area’
				\parencite[223.3147]{story-naish:1973}
				\vbmorph{ka-&u-&\rt{gan}&-μH&\gm{-tʼ}&-i}
					{\xx{hsfc}&\xx{irr}&\rt{burn}&\·\xx{var}&\·\xx{rep}&\·\xx{nmz}}
				\newline
				\citeauthor{story-naish:1973} mistranslate this as ‘windfall’;
				\textcite[T·39]{leer-hitch-ritter:2001} gives \fm{koogánti} 
					with \X{-t} instead of \X{-tʼ}
			\end{itemize}
		\item	\fm{–héitʼ} ‘erase’
			from \fm{\rt[¹]{ha}} ‘(dis)appear, move imperceptibly’ in
			\newline
			\vbform{aadáx̱ as.héitʼ}{rep impfv}[tr, \fm{n}, ach]{she/he/it is erasing it from it}
			\parencite[01/7]{leer:1973}
				\vbmorph{aa&-dáx̱&a-&s-&\rt[¹]{ha}&-μμᵉH&\gm{-tʼ}}
					{\xx{3n}&\·\xx{abl}&\xx{3>3}&\xx{csv}&\rt[¹]{disappear}&\·\xx{var}&\·\xx{rep}}
			\exand \vbform{aadáx̱ as.héix̱}{rep impfv}{she/he/it is erasing it from it}
			\parencite[7]{leer:1976}
				\vbmorph{aa&-dáx̱&a-&s-&\rt[¹]{ha}&-μμᵉH&-x̱}
					{\xx{3n}&\·\xx{abl}&\xx{3>3}&\xx{csv}&\rt[¹]{disappear}&\·\xx{var}&\·\xx{rep}}
			\versus \vbform{aadáx̱ awsihaa}{pfv}{she/he/it erased it from it}
			\parencite[01/7]{leer:1973}
				\vbmorph{aa&-dáx̱&a-&w-&s-&i-&\rt[¹]{ha}&-μμL}
					{\xx{3n}&\·\xx{abl}&\xx{3>3}&\xx{pfv}&\xx{csv}&\xx{stv}&\rt[¹]{disappear}&\·\xx{var}}
			\newline
			\textcite[7]{leer:1976} also translates these as
				“he is making off with them”
				and “made off with it (took it without owner’s knowledge)”;
			\textcite[80.990]{story-naish:1973} add “rub off”;
			with areal \X[ḵu-areal]{ḵu-} as the object the meaning is ‘remove dirt, polish’
				but this is only attested with \X{-x̱} and not \fm{-tʼ}
		\item	\fm{–húntʼ} ‘sell off’
			from \fm{\rt[²]{hun}} ‘sell’ in
			\newline
			\vbform{ahúntʼ}{rep impfv}[tr, \fm{n}, \fm{-μμH} act]{she/he/it sells off him/her/it}
			\parencite[67]{leer:1976}
			also note “if it gets to him he sells it” \parencite[68]{leer:1976}
				\vbmorph{a-&\rt[²]{hun}&-μH&\gm{-tʼ}}
					{\xx{3>3}&\rt[²]{sell}&\·\xx{var}&\·\xx{rep}}
			\versus \vbform{ahóon}{impfv}{she/he/it sells him/her/it}
				\vbmorph{a-&\rt[²]{hun}&-μμH}
					{\xx{3>3}&\rt[²]{sell}&\·\xx{var}}
		\item	\fm{–kélʼtʼ} ‘ash’
			from \fm{\rt{kelʼ}} ‘ash’ in
			\newline
			\vbform{ashkélʼtʼ}{rep impfv}[tr, \fm{∅}/\fm{g̱}, ach]{she/he/it makes ash of him/her/it}
				\vbmorph{a-&sh-&\rt{kelʼ}&-μH&\gm{-tʼ}}
					{\xx{3>3}&\xx{pej}&\rt{ash}&\·\xx{var}&\·\xx{rep}}
			\versus \vbform{awshikélʼ}{pfv}[\fm{∅}]{she/he/it made ash of him/her/it}
				\vbmorph{a-&w-&sh-&i-&\rt{kelʼ}&-μH}
					{\xx{3>3}&\xx{pfv}&\xx{pej}&\xx{stv}&\rt{ash}&\·\xx{var}}
			\newline
			the root \fm{\rt{kelʼ}} can mean ‘undo, untie, unravel’
				and ‘flee, chase’
				as well as ‘ash’,
				but \fm{-tʼ} occurs only with the ‘ash’ meaning;
			the verb is attested both as \fm{∅} conjugation class (perfective stem \fm{-μH})
				and as \fm{g̱} conjugation class (\fm{yei=}…\fm{-ch}, perfective stem \fm{-μμH})
			\newline
			also in nouns:
			\begin{itemize}
			\item	\fm{kélʼtʼ} ‘ash’
				(also \fm{kéilʼ} ‘ash; dandruff’ (Southern, Tongass) without \fm{-tʼ})
				\vbmorph{\rt{kelʼ}&-μH&\gm{-tʼ}}
					{\rt{ash}&\·\xx{var}&\·\xx{rep}}
			\item	\fm{dleit kakélʼtʼ} ‘light snow’
				\vbmorph{dleit&ka-&\rt{kelʼ}&-μH&\gm{-tʼ}}
					{snow&\xx{hsfc}&\rt{ash}&\·\xx{var}&\·\xx{rep}}
			\item	\fm{sʼaḵkélʼtʼi} \~\ \fm{sʼax̱kélʼtʼi} ‘fine powder, bone ash’
				\vbmorph{sʼaḵ-&\rt{kelʼ}&-μH&\gm{-tʼ}&-i}
					{bone&\rt{ash}&\·\xx{var}&\·\xx{rep}&\·\xx{poss}}
			\item	\fm{sʼeeḵ x̱ʼakélʼtʼi} ‘cigarette ash’
				\vbmorph{sʼeeḵ&x̱ʼe-&\rt{kelʼ}&-μH&\gm{-tʼ}&-i}
					{smoke&mouth&\rt{ash}&\·\xx{var}&\·\xx{rep}&\·\xx{poss}}
			\end{itemize}
		\item	\fm{–kweitʼ} ‘know’
			from \fm{\rt{kuᴸ}} ‘know’ in
			\newline
			\vbform{ash ée askweitʼ}{rep impfv}[tr, \fm{∅⁺}, ach]{she/he/it is acquainting him/her/it with him/her/it}
				\parencite[f06/126]{leer:1973}
				\vbmorph{ash&ee&-H&a-&s-&s-&\rt{kuᴸ}&-μμᵉL&\gm{-tʼ}}
					{\xx{3prx}&\xx{base}&\·\xx{loc}&\xx{3>3}&\xx{appl}&\xx{xtn}&\rt{know}&\·\xx{var}&\·\xx{rep}}
			\versus \vbform{ash ée awsikóo}{pfv}{she/he/it acquainted him/her/it with him/her/it}
				\parencite[716]{leer:1976}
				\vbmorph{ash&ee&-H&a-&w-&s-&s-&i-&\rt{kuᴸ}&-μμH}
					{\xx{3prx}&\xx{base}&\·\xx{loc}&\xx{3>3}&\xx{pfv}&\xx{appl}&\xx{xtn}&\xx{stv}&\rt{know}&\·\xx{var}}
		\item	\fm{–láxwtʼ} ‘starve to death’
			from \fm{\rt[¹]{laxw}} ‘starve’ in
			\newline
			\vbform{a x̱oo aa láxwtʼ}{rep impfv}[obj intr, \fm{∅}, ach]{some among them are starving to death}
			\parencite[265.1]{swanton:1909}
				\vbmorph{a&x̱oo&aa&\rt{laxw}&-μH&\gm{-tʼ}}
					{\xx{3n}&among&\xx{part}&\rt{starve}&\·\xx{var}&\·\xx{rep}}\
			\versus \vbform{uwaláxw}{pfv}{she/he/it starved}
				\vbmorph{u-&wa-&\rt{laxw}&-μH}
					{\xx{zpfv}&\xx{stv}&\rt{starve}&\·\xx{var}}	
		\item	\fm{–náatʼ} ‘plural die’
			from \fm{\rt[¹]{na⁽ʷ⁾}} ‘die’ in
			\newline
			\vbform{has náatʼ}{rep impfv}[obj intr, \fm{n}, ach]{they (always) die}
			\parencite[56]{leer:1963}
				\vbmorph{has=&\rt[¹]{na⁽ʷ⁾}&-μμH&\gm{-tʼ}}
					{\xx{plh}&\rt[¹]{die}&\·\xx{var}&\·\xx{rep}}
			\exand \vbform{has woonáatʼ}{rep pfv}{they died}
			\parencite[58]{leer:1963}
				\vbmorph{has=&wu-&μ-&\rt[¹]{na⁽ʷ⁾}&-μμH&\gm{-tʼ}}
					{\xx{plh}&\xx{pfv}&\xx{stv}&\rt[¹]{die}&\·\xx{var}&\·\xx{rep}}
			\exand \vbform{dax̱ woonáatʼ}{rep pfv}{they died}
			\parencite[58]{leer:1963}
				\vbmorph{dax̱=&wu-&μ-&\rt[¹]{na⁽ʷ⁾}&-μμH&\gm{-tʼ}}
					{\xx{dpl}&\xx{pfv}&\xx{stv}&\rt[¹]{die}&\·\xx{var}&\·\xx{rep}}
			\versus \vbform{has woonaa}{pfv}{they died}
				\vbmorph{has=&wu-&μ-&\rt[¹]{na⁽ʷ⁾}&-μμL}
					{\xx{plh}&\xx{pfv}&\xx{stv}&\rt[¹]{die}&\·\xx{var}}
		\item	\fm{–sháttʼ} ‘capture’
			from \fm{\rt[²]{shaᴴt}} ‘grab’ in
			\newline
			\vbform{sháa wududlisháttʼ}{pfv}[tr, \fm{g̱}?, ach?]{they captured the women}
			\parencite[57]{story:1966}
				\vbmorph{sháa&wu-&du-&d-&l-&i-&\rt[²]{shaᴴt}&-μH&\gm{-tʼ}}
					{woman&\xx{pfv}&\xx{ind.h.s}&\xx{mid}&\xx{xtn}&\xx{stv}&\rt[²]{grab}&\·\xx{var}&\·\xx{rep}}
			\versus \vbform{x̱at wududlisháat}{pfv}[tr, \fm{g̱}, ach]{I was captured}
			\parencite[41.410]{story-naish:1973}
				\vbmorph{x̱at=&wu-&du-&d-&l-&i-&\rt[²]{shaᴴt}&-μμH}
					{\xx{1sg.o}&\xx{pfv}&\xx{ind.h.s}&\xx{mid}&\xx{xtn}&\xx{stv}&\rt[²]{grab}&\·\xx{var}}
		\item	\fm{–.úwtʼ} ‘buy’
			from \fm{\rt[²]{.uw}} ‘buy’ in
			\newline
			\vbform{da.úwtʼ}{rep impfv}[subj intr, \fm{∅}?, ach]{she/he/it buys a bunch at a time}
			\parencite[149]{leer:1976}
				\vbmorph{da-&\rt[²]{.uw}&-μH&\gm{-tʼ}}
					{\xx{apsv}&\rt[²]{buy}&\·\xx{var}&\·\xx{rep}}
			\versus \vbform{aawa.úw}{pfv}[\fm{∅}]{she/he/it bought him/her/it}
				\vbmorph{a-&μʷ-&wa-&\rt[²]{.uw}&-μH}
					{\xx{3>3}&\xx{pfv}&\xx{stv}&\rt[²]{buy}&\·\xx{var}}
			\exand \vbform{a.óow}{impfv}[tr, \fm{n}, inv act]{she/he/it buys him/her/it}
				\vbmorph{a-&\rt[²]{.uw}&-μμH}
					{\xx{3>3}&\rt[²]{buy}&\·\xx{var}}
		\item	\fm{–xóoshtʼ} ‘scorch, singe’
			from \fm{\rt{xuᴴsh}} ‘scorch, singe’ in
			\newline
			\vbform{alxóoshtʼ}{rep impfv}[obj intr, \fm{n}/\fm{g̱}, ach]{she/he/it repeatedly scorches, singes it}
			\parencite[635]{leer:1976}
				\vbmorph{a-&lˢ-&\rt{xuᴴsh}&-μμH&\gm{-tʼ}}
					{\xx{3>3}&\xx{csv}&\rt{scorch}&\·\xx{var}&\·\xx{rep}}
			\exand \vbform{awlixóoshtʼ}{rep pfv}{she/he/it repeatedly scorched, singed him/her/it}
			\parencite[f03/125]{leer:1973}
				\vbmorph{a-&w-&lˢ-&i-&\rt{xuᴴsh}&-μμH&\gm{-tʼ}}
					{\xx{3>3}&\xx{pfv}&\xx{xtn}&\xx{stv}&\rt{scorch}&\·\xx{var}&\·\xx{rep}}
			\versus \vbform{awlixóosh}{pfv}{she/he/it scorched, singed it}
				\vbmorph{a-&w-&lˢ-&i-&\rt{xuᴴsh}&-μμH}
					{\xx{3>3}&\xx{pfv}&\xx{xtn}&\xx{stv}&\rt{scorch}&\·\xx{var}}
			\newline
			also noun noun \fm{xóoshtʼ} ‘scorched or singed matter’;
			the unusual stem variation \fm{-μμH} instead of \fm{-μH} expected with \fm{-tʼ}
				in the verb forms suggests a derivation process,
				probably from the noun \fm{xóoshtʼ} to the verbs,
				but this does not explain the stem variation of the noun;
			compare related \fm{\rt{xuᴴts}} ‘char’ and \fm{\rt{xwaᴴts}} ‘paint black’
		\end{itemize}
	\item	in some complex nouns apparently derived from unknown verbs
		\begin{itemize}
		\item	\fm{daakaleiltʼí} ‘husk, chaff, dried remains of berry’
			from \fm{\rt{lel}} ‘limp, lax, baggy, flabby’
			\vbmorph{daa-&ka-&\rt{lel}&-μμL&\gm{-tʼ}&-í}
				{around&\xx{hsfc}&\rt{limp}&\·\xx{var}&\·\xx{rep}&\·\xx{poss}}
			\newline
			compare
			\begin{inlinelist}
			\item	\vbform{kawlilél}{pfv}[obj intr, \fm{∅}, ach]{she/he/it became limp, baggy, flabby}
			\item	\fm{leilí} ‘flab; scrotum’
			\item	\fm{a daaleilí} ‘its saggy skin’;
			\end{inlinelist}
			related to \fm{létlʼk} \~\ \fm{lélʼk} ‘soft’
				and \fm{\rt{dletl}} ‘hang slack’
		\item	\fm{kayéx̱tʼi} ‘wood shavings’
			from \fm{\rt[²]{yex̱}} ‘shave; build’
			\vbmorph{ka-&\rt{yex̱}&-μH&\gm{-tʼ}&-i}
				{\xx{hsfc}&\rt{shave}&\·\xx{var}&\·\xx{rep}&\·\xx{nmz}}
			\newline
			compare \vbform{akayeix̱}{impfv}[tr, \fm{∅}, \fm{-μμL} act]{she/he/it is shaving, whittling, planing him/her/it}
				\vbmorph{a-&ka-&\rt[²]{yex̱}&-μμL}
					{\xx{3>3}&\xx{hsfc}&\rt[²]{shave}&\·\xx{var}}
			\newline
			the noun \fm{kayéx̱tʼi} implies a verb with \fm{-tʼ}
				but no other forms of \fm{\rt[²]{yex̱}} with \fm{-tʼ} are attested
		\item	\fm{kakúshtʼi} ‘pitch blister on tree’
			from unknown \fm{\rt{kush}}
			\vbmorph{ka&\rt{kush}&-μH&\gm{-tʼ}&-i}
				{\xx{hsfc}&\rt{\xx{unkn}}&\·\xx{var}&\·\xx{rep}&\·\xx{nmz}}
			\newline
			also in \fm{xʼaan kakúshtʼi} ‘cottonwood fluff with seeds’
			\vbmorph{xʼaan&ka&\rt{kush}&-μH&\gm{-tʼ}&-i}
				{branch.tip&\xx{hsfc}&\rt{\xx{unkn}}&\·\xx{var}&\·\xx{rep}&\·\xx{nmz}}
			\newline
			compare \fm{kóoshdaa} ‘land otter’
				(perhaps including \fm{dáa} ‘weasel’
					or alternatively \X{-t} + \X{-aa})
				and \fm{kweiḵ} ‘finger abscess, felon, whitlow’;
			perhaps related to \fm{\rt{ḵush}} ‘purulent, unclean’ in
				\begin{inlinelist}
				\item	\fm{ḵóosh} ‘open sore; unclean thing’
				\item	\fm{ḵóosh kadáan!} (interj.)\ ‘damn! woe!’
				\item	\vbform{wuliḵoosh}{pfv}[obj intr, \fm{n}, ach]{she/he/it became open sore}
				\item	\vbform{liḵooshí}{impfv}[obj intr, conj?, inv state]{she/he/it is unclean};
				\end{inlinelist}
			perhaps also related to \fm{\rt[²]{ḵush}} ‘handle blob’ in
				\vbform{x̱waaḵoosh}{pfv}[tr, \fm{n}?, mot]{I carried it (blob, blubber, mud, snake)}
				\parencite[874]{leer:1976}
				and thus \fm{\rt{ḵutlʼ}} ‘mud’ in
				\fm{ḵútlʼkw} ‘mud, mire’,
				\vbform{kawshiḵútlʼ}{pfv}[obj intr, \fm{∅}, ach]{she/he/it became muddy},
				and \vbform{kashiḵútlʼkw}{rep impfv}[obj intr, conj?, rep state]{she/he/it is muddy}
		\item	\fm{ḵʼalukakúltʼi} ‘slit, eyehole in thong end’
			from unknown \fm{\rt{kul}}
			\vbmorph{ḵʼa-&lu-&ka-&\rt{kul}&-μH&\gm{-tʼ}&-i}
				{mouth&nose&\xx{hsfc}&\rt{\xx{unkn}}&\·\xx{var}&\·\xx{rep}&\·\xx{nmz}}
			\newline
			compare
			\begin{inlinelist}
			\item	\fm{kool} ‘navel, whorl’
			\item	\fm{jikóol} ‘back of hand’
			\end{inlinelist}
		\item	\fm{at yasaḵéitʼi} ‘lawyer, judge’
			from \fm{\rt[¹]{ḵa}} ‘say, speak’
			\vbmorph{at=&ÿa-&sa-&\rt{ḵa}&-μμᵉH&\gm{-tʼ}&-i}
				{\xx{ind.n.o}&\xx{qual}&\xx{tr}&\rt[¹]{say}&\·\xx{var}&\·\xx{rep}&\·\xx{nmz}}
			\newline
			also in
			\fm{ḵuyasaḵéitʼi} ‘judge, commissioner’
			\vbmorph{ḵu-&ÿa-&sa-&\rt{ḵa}&-μμᵉH&\gm{-tʼ}&-i}
				{\xx{ind.h.o}&\xx{qual}&\xx{tr}&\rt[¹]{say}&\·\xx{var}&\·\xx{rep}&\·\xx{nmz}}
		\item	\fm{ḵusanéx̱tʼi} ‘curer’
			from \fm{\rt[¹]{nex̱}} \~\ \fm{\rt[¹]{nix̱}} ‘safe’
			\parencite[04/189]{leer:1973}
			\vbmorph{ḵu-&sa-&\rt[¹]{nex̱}&-μH&\gm{-tʼ}&-i}
				{\xx{ind.h.o}&\xx{csv}&\rt[¹]{safe}&\·\xx{var}&\·\xx{rep}&\·\xx{nmz}}
			\newline
			the noun \fm{ḵusanéx̱tʼi} implies a verb with \fm{-tʼ}
				but no other forms of \fm{\rt[¹]{nex̱}} \~\ \fm{\rt[¹]{nix̱}}
				with \fm{-tʼ} are attested;
			\textcite[04/190]{leer:1973} gives an alternative form \fm{kusanéx̱tʼi}
				with \X[ka-qual]{ka-} and \X[u-irr]{u-} instead
		\end{itemize}
	\item	possibly analyzable in some CVCC nouns with unexplained coda /\ipa{tʼ}/
		\begin{itemize}
		\item	\fm{íx̱tʼ} ‘shaman, medicine man, Indian doctor’
			from unknown \fm{\rt{.ix̱}}
			\vbmorph{\rt{.ix̱}&-μH&\gm{-tʼ}}
				{\rt{\xx{unkn}}&\·\xx{var}&\·\xx{rep}}
			\newline
			also in a few obscure verbs apparently derived from the noun
			\parencite[all from][02/281–282]{leer:1973}
			\begin{itemize}
			\item	\vbform{woosh da.íx̱tʼ}{rep impfv}[tr?, conj?, ach?]{they (shamans) are trying to outdo each other}
				\vbmorph{woosh=&da-&\rt{.ix̱}&-μH&\gm{-tʼ}}
					{\xx{recip.o}&\xx{mid}&\rt{\xx{unkn}}&\·\xx{var}&\·\xx{rep}}
			\item	\vbform{áa ḵuwdi.íx̱tʼ}{pfv}[subj intr?, conj?, ach?]{there got to be a lot of shamans there}
				\vbmorph{á&-μ&ḵu-&w-&d-&i-&\rt{.ix̱}&-μH&\gm{-tʼ}}
					{\xx{3n}&\·\xx{loc}&\xx{areal}&\xx{pfv}&\xx{mid}&\xx{stv}&\rt{\xx{unkn}}&\·\xx{var}&\·\xx{rep}}
			\item	\vbform{wuli.íx̱tʼ}{pfv}[obj intr?, conj?, ach?]{she/he/it became a shaman}
				\vbmorph{wu-&l-&i-&\rt{.ix̱}&-μH&\gm{-tʼ}}
					{\xx{pfv}&\xx{intr}&\xx{stv}&\rt{\xx{unkn}}&\·\xx{var}&\·\xx{rep}}
			\end{itemize}
		\item	\fm{sʼáxtʼ} ‘devils club’
			from unknown \fm{\rt{sʼax}}
			\vbmorph{\rt{sʼax}&-μH&\gm{-tʼ}}
				{\rt{\xx{unkn}}&\·\xx{var}&\·\xx{rep}}
			\newline
			compare
			\begin{inlinelist}
			\item	\fm{sʼáx} ‘starfish’
			\item	\fm{\rt{sʼaxw}} ‘stack’
			\item	\fm{\rt{sʼix}} ‘aged, fermented, rotten’
			\item	\fm{\rt{sʼixw}} \~\ \fm{\rt{sʼux}} ‘sour’
			\item	\fm{\rt{sʼikw}} \~\ \fm{\rt{sʼuk}} ‘crisp’
			\item	\fm{\rt{sʼixʼw}} ‘sticky’
			\item	\fm{\rt{tsʼikʼw}} \~\ \fm{\rt{tsʼikʼ}} ‘pinch’
			\item	\fm{\rt{tsʼikw}} ‘delicate’ in \fm{–tsʼígwaa} (see \X{-aa})
			\end{inlinelist}
		\item	\fm{shakalḵʼíshtʼ} ‘deer with two-point horns’
			from unknown \fm{\rt{ḵʼish}}
			\vbmorph{sha-&ka-&d-&l-&\rt{ḵʼish}&-μH&\gm{-tʼ}}
				{head&\xx{qual}&\xx{mid}&\xx{xtn}&\rt{\xx{unkn}}&\·\xx{var}&\·\xx{rep}}
			\newline
			compare
			\begin{inlinelist}
			\item	\fm{ḵʼeishtʼóo} ‘larva cyst in skin; hard ball’ (see below)
			\item	\fm{\rt{ḵʼichʼ}} ‘scabbed’ and noun \fm{ḵʼéechʼ} ‘scab, dried blood’
			\item	\fm{\rt{ḵʼish}} \~\ \fm{\rt{ḵʼesh}} ‘swat puck, ball’
				and nouns \fm{ḵʼísht} ‘puck’ (with \X{-t})
				and \fm{ḵʼeish} ‘batting, striking a ball’
			\item	\fm{\rt{ḵʼetlʼ}} ‘cut open, split down center’
			\end{inlinelist}
		\end{itemize}
	\item	possibly part of three nouns ending with \fm{…tʼu}
		\begin{itemize}
		\item	\fm{kéitʼu} ‘pick, pickaxe’ 
			also \fm{sheey kéitʼu} ‘prying tool (made from limb)’
			and \fm{ḵaashakéitʼu} ‘freshwater insect’
			from unknown \fm{\rt{ke}} or \fm{\rt{ketʼ}};
			compare
			\begin{inlinelist}
			\item	\fm{káatʼ} ‘digging stick’
			\item	\fm{\rt{kitʼ}} ‘jam; pry’ and noun \fm{kítʼaa} ‘prybar’
			\item	Haida \fm{kitʼuu} ‘harpoon, seafood spear’ \parencite[1062]{enrico:2005}
			\end{inlinelist}
		\item	\fm{ḵʼeishtʼóo} ‘larva cyst in skin; hard ball’
			from unknown \fm{\rt{ḵʼesh}};
			compare
			\begin{inlinelist}
			\item	\fm{\rt{ḵʼichʼ}} ‘scabbed’ and noun \fm{ḵʼéechʼ} ‘scab, dried blood’
			\item	\fm{\rt{ḵʼish}} \~\ \fm{\rt{ḵʼesh}} ‘swat puck, ball’
				and nouns \fm{ḵʼísht} ‘puck’ (with \X{-t})
				and \fm{ḵʼeish} ‘batting, striking a ball’
			\item	\fm{ḵʼeishkaháagu} ‘very small cranberry’
			\item	\fm{\rt{ḵʼetlʼ}} ‘cut open, split down center’
			\end{inlinelist}
		\item	\fm{x̱éitʼu} ‘white film on tongue’
			from unknown \fm{\rt{x̱e}} or \fm{\rt{x̱etʼ}};
			possibly includes \fm{\rt{wu}} ‘pale, fair’ like in
			\begin{inlinelist}
			\item	\fm{chichwú} ‘white porpoise’
			\item	\fm{chʼáatwu} ‘epidermis’
			\item	\fm{chʼeetwú} ‘white auklet or murrelet’
			\item	\fm{jánwu} \~\ \fm{jánu} ‘mountain goat’
			\item	\fm{keetwú} \~\ \fm{kitwú} ‘white killerwhale’
			\item	\fm{lakʼeechʼwú} ‘scoter’
			\item	\fm{taanwú} \~\ \fm{taanú} ‘umbilical cord’
			\item	\fm{wú} ‘white side of it (flounder, halibut)’;
			\end{inlinelist}
			compare
			\begin{inlinelist}
			\item	\fm{x̱aatlʼáḵw} ‘mouth ulcer’ (with \X{-áḵw}?)
			\item	\fm{x̱aatlʼ} ‘freshwater grass’
			\item	\fm{x̱éetʼ} ‘giant clam’
			\end{inlinelist}
		\end{itemize}
	\end{enumerate}

\item[tu-]
	first person plural subject;
	note that \cite{story-naish:1973} write all cases of \fm{tu-} as \fm{too-}
		so they do not distinguish the two allomorphs
	\begin{itemize}
	\item	\fm{wutuwax̱áa} (pfv; tr, \fm{∅}, \fm{-μH} act) ‘we ate it’\newline
		versus \fm{toox̱á} (impfv) ‘we eat it; we are eating it’
	\end{itemize}

\item[tu-]\label{m:tu-inside}
	incorporated noun ‘inside; mind, emotion, bodily spirit’;
	derived from relational noun \fm{tú} ‘inside of (hollow object)’
	used metaphorically as ‘mind, emotion, bodily spirit’ as in \fm{ax̱ toowú yanéekw} ‘my mind hurts’

\item[too-]
	allomorph of first person plural subject \fm{tu-}
	\begin{itemize}
	\item	\fm{toox̱á} (pfv; tr, \fm{∅}, \fm{-μH} act) ‘we eat it; we are eating it’\newline
		versus \fm{wutuwax̱áa} (pfv) ‘we ate it’
	\end{itemize}

\item[-ts]\label{m:-ts}
	unknown suffix which occurs only in the stem \fm{–núkts} ‘sweet, delicious’
		which is analyzed as \fm{\rt{nuk}-μH-ts};
	the underlying root may be \fm{\rt{nikw}} \~\ \fm{\rt{nuk}} ‘feel’
		but the composition of meaning is unclear;
	this suffix disappears when the stem \fm{–núkts} is combined with \X{-chʼán},
		which see
	\begin{itemize}
	\item	\vbform{linúkts}{impfv}[obj intr, \fm{g}, inv state]{she/he/it is sweet (tasting)}
			\vbmorph{lˢ-&i-&\rt[⁰]{nuk}&-μH&\gm{-ts}}
				{\xx{intr}&\xx{stv}&\rt[⁰]{sweet}&\·\xx{var}&\·\xx{unkn}}
	\end{itemize}
\end{morphdesc}
