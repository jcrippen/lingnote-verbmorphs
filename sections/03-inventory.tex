%!TEX root = ../lingnote-verbmorphs.tex

\section{Inventory of verb morphemes by position and function}\label{sec:inventory}

\subsection{Preverbs}\label{sec:inventory-preverb}

The preverbs are adverbs and postposition phrases that indicate manner, location, or direction
	of the eventuality described by the verb.

\textcite[132–134]{leer:1991} identified six classes of preverbs and \textcite[686]{crippen:2019} expanded this to eight by dividing two of \citeauthor{leer:1991}’s classes.
The eight subclasses are labeled by capital letters from inside out (right to left).

\subsubsection{Preverb class H}\label{sec:inventory-preverb-H}

The preverbs in class H are the outermost (leftmost) of the preverbs.
They are all derived from a noun or pronoun with the locative postposition allomorph \X[-μ-loc]{-μ}
	although this suffix is essentially frozen in these forms.

\begin{morphdesc}
\item[\X{g̱unayéi=}]
\item[\X{g̱unéi=}]
\item[\X{g̱unyéi=}]
	Inceptive manner preverb ‘starting, beginning’ indicating initiation of motion
	or the beginning of an eventuality.

\item[\X{áa=}]
	Locational preverb ‘there, at it’.

\item[\X{shóo=}]
	Locational preverb ‘at end of, in front of’.
\end{morphdesc}

\subsubsection{Preverb class G}\label{sec:inventory-preverb-G}

The preverbs in class G are all derived from a noun
	with the locative postposition allomorph \X[-í-loc]{-í} \~\ \X[-i-loc]{-i}.
See that entry for further discussion.

\begin{morphdesc}
\item[\X{dáag̱i=}]
	Locational preverb ‘inland’ indicating location on land away from a body of water.
\item[\X{éeg̱i=}]
\item[\X{éig̱i=}]
	Locational preverb ‘beach’ indicating location on land above the shoreline of a body of water.
\item[\X{gági=}]
	Locational preverb ‘emerging, out in the open’ indicating location in an open or visible area,
		usually as a result of movement from an enclosed or obscured location.
\item[\X{héeni=}]
	Locational preverb ‘in water’ indicating location in a body of water,
		usually as a result of movement from dry land.
\item[\X{neilí=}]
	Locational preverb ‘inside’ indicating location inside a building, cave,
		or other covered enclosure.
\item[\X{x̱áni=}]
	Locational preverb ‘near’ indicating location near some entity.
\end{morphdesc}

\subsubsection{Preverb class F}\label{sec:inventory-preverb-F}

The preverbs in class F are all originally from postposition phrases, but there is no single
	generalization that applies to all of them.
Most of the class F preverbs are fossilized so that speakers do not generally perceive them as having
	clear internal structure; the exception is \fm{héenx̱=} which is transparently from
	\fm{héen} ‘fresh water, river’ and the pertingent postposition
	\fm{-x̱} ‘of, at, contacting’.

\begin{morphdesc}
\item[\X{héenx̱=}]
	Locational preverb ‘in water’ indicating a location in a body of water,
		usually as a result of movement from dry land.

\item[\X{ḵáaḵw=}]
\item[\X{ḵáaḵwt=}]
\item[\X{ḵáaḵwx̱=}]
	Variant form of manner preverb  \X{ḵwáaḵ=} / \X{ḵwáaḵt=} / \X{ḵwáaḵx̱=} ‘wrongly’
		indicating that an event was performed incorrectly
		or happened in an unfortunate or undesirable way.

\item[\X{ḵut=}]
	Manner preverb ‘astray, lost, excess’ indicating that eventuality involves becoming lost
		either literally or metaphorically.
	Originally from areal \fm{ḵu-}
		and punctual postposition \fm{-t} ‘to, at, around’.

\item[\X{ḵwáaḵ=}]
\item[\X{ḵwáaḵt=}]
\item[\X{ḵwáaḵx̱=}]
	Manner preverb ‘wrongly’ indicating that an event was performed incorrectly
		or happened in an unfortunate or undesirable way.

\item[\X{ux̱=}]

\item[\X{yaax̱=}]
	Directional preverb ‘aboard’ indicating motion into a boat or other vehicle.
	Originally from \fm{yaakw} ‘canoe, boat’
		and pertingent postposition \fm{-x̱} ‘of, at, contacting’.

\item[\X{ÿanax̱=}]
	Directional preverb ‘into ground’ indicating motion into the ground, specifically
		below the surface of the ground.
	Originally from \fm{ÿán} ‘shore’ (earlier ‘ground’)
		and perlative postposition \fm{-náx̱} ‘via, through, across’.

\item[\X{yatx̱=}]

\item[\X{yux̱=}]
	Directional preverb ‘out, outside’ indicating motion out of a building, cave,
		or other covered enclosure.
	Probably originally from distal \fm{yú} ‘far off’
		and pertingent postposition \fm{-x̱} ‘of, at, contacting’.
\end{morphdesc}

\subsection{Quantifier proclitics}\label{sec:inventory-qfr}

\subsection{Object prefixes}\label{sec:inventory-object}

\subsection{Qualifiers and incorporated nouns}\label{sec:inventory-qualinc}

\subsection{Aspect and conjugation class prefixes}\label{sec:inventory-aspconj}

\subsection{Subject prefixes}\label{sec:inventory-subject}

\subsection{Classifier prefixes}\label{sec:inventory-classifier}

\subsection{Roots}\label{sec:inventory-root}

\subsection{Stem variation suffixes}\label{sec:inventory-stemvar}

The stem variation suffixes are representations of the different stem variation patterns
as notional suffixes that attach to roots and give rise to the core syllable of stems.
Stem variation suffixes are symbolic representations of four different processes:

\begin{itemize}
\item	vowel length measured in moras, symbolized by μ (Greek letter \textit{mu})
	\begin{itemize}
	\item	one mora μ = short vowel
	\item	two moras μμ = long vowel
	\end{itemize}
\item	tone symbolized by H and L or both
	\begin{itemize}
	\item	H = high tone
	\item	L = low tone
	\item	HL = falling tone (Southern dialects only)
	\end{itemize}
\item	ablaut (change of a lexical /\ipa{a}/ or /\ipa{u}/ vowel into [\ipa{e}])
	\begin{itemize}
	\item	μᵉ = ablauted vowel
	\end{itemize}
\item	deletion of a final consonant represented by ⊗,
	consequently appearing as a short vowel with high tone (μH)
\end{itemize}

Tongass Tlingit has a distinct stem variation system because it does not have tone
and instead has laryngealized vowels.
Vowel length is the same as in other dialects with short vowels (μ) and long vowels (μμ).
Long vowels may be a plain vowel (no special indication so just μμ),
laryngealized with a breathy/fading vowel (μμʰ),
or laryngealized with a glottalized vowel (μμˀ).

\begin{morphdesc}
\item[\X{-μL}]
	short vowel (μ) with low tone (L) so […\ipa{V̀}…]

\item[\X{-μH}]
	short vowel (μμ) with high tone (H) so […\ipa{V́}…]

\item[\X{-μμL}]
	long vowel (μμ) with low tone (L) so […\ipa{V̀ː}…]

\item[\X{-μμH}]
	long vowel (μμ) with high tone (H) so […\ipa{V́ː}…]

\item[\X{-μμHL}]
	long vowel (μμ) with falling tone (HL) so […\ipa{V̂ː}…];
	only in Southern Tlingit dialects (Sanya, Henya) that have phonologically
		contrastive falling tone

\item[\X{-μᵉμL}]
	ablaut (/\ipa{a, u}/ → [\ipa{e}]) long vowel (μμ)
		with low tone (L) so […\ipa{èː}…];
	normally occurs only with \fm{\rt{CVᴸ}} roots

\item[\X{-μᵉμH}]
	ablaut (/\ipa{a, u}/ → [\ipa{e}]) long vowel (μμ)
		with high tone (H) so […\ipa{éː}…];
	normally occurs only with \fm{\rt{CV}} or \fm{\rt{CVᴸ}} roots

\item[\X{-μᵉμHL}]
	ablaut (/\ipa{a, u}/ → [\ipa{e}]) long vowel (μμ)
		with falling tone (HL) so […\ipa{êː}…];
	only in Southern Tlingit dialects (Sanya, Henya) that have phonologically
		contrastive falling tone;
	normally occurs only with \fm{\rt{CV}} or \fm{\rt{CVᴸ}} roots
		that contain /\ipa{a}/ or /\ipa{u}/

\item[{\X[-DEL]{-⊗}}]
	irregular deletion (⊗) of final consonant
		resulting in a short vowel with high tone (μH)
		so […\ipa{V́}]
		(no tone […\ipa{V}] in Tongass Tlingit);
	only occurs in imperatives with
		\begin{inlinelist}
		\item	\fm{\rt[¹]{gut}} ‘sg go’
		\item	\fm{\rt[¹]{.at}} ‘pl go’
		\item	\fm{\rt[¹]{nuk}} ‘sg sit’
		\end{inlinelist}
\end{morphdesc}

\subsection{Repetitive and derivation suffixes}\label{sec:inventory-repderiv}

\subsection{Tense, modality, and clause type suffixes}\label{sec:inventory-tmc}

\subsection{Enclitic (postverbal) auxiliaries}\label{sec:inventory-aux}
