%!TEX root = ../lingnote-verbmorphs.tex

\section{Inventory of verb morphemes by position and function}\label{sec:inventory}

This is an inventory of all the verb morphemes except for roots, organized by their linear positions and functions.
This organization corresponds to the traditional template-based analyses of Tlingit verb morphology.

\subsection{Preverbs}\label{sec:inventory-preverb}

The preverbs are adverbs and postposition phrases that indicate manner, location, or direction
	of the eventuality described by the verb.
\textcite[132–134]{leer:1991} identified six classes of preverbs and \textcite[686]{crippen:2019} expanded this to eight by dividing two of \citeauthor{leer:1991}’s classes.
The eight classes are labeled by capital letters from right to left: H > G > F > E > D > C > B > A.
The order of these classes reflects the relative orders between various preverbs, but there seems to be some variation in order that is still unexplored and unexplained.

% This is a hack to eliminate vertical space between items that have no content.
% It is used in the morphdesc lists of preverbs for grouping related items
% together with a single item description.
% It needs to appear after the item label \item[…] but before any content.
% In this position it affects the space between the current item and anything
% above this item.
\ProvideDocumentCommand{\removeitemvspace} {} {\vspace{\dimexpr-\itemsep-\parsep+1pt}}

\subsubsection{Preverb class H}\label{sec:inventory-preverb-H}

The preverbs in class H are the outermost (leftmost) of the preverbs.
They are all derived from a noun or pronoun with the locative postposition allomorph \X[-μ-loc]{-μ}
	although this suffix is essentially fossilized in these forms.

\begin{morphdesc}
\item[\X{g̱unayéi=}]
\item[\X{g̱unéi=}] \removeitemvspace
\item[\X{g̱unyéi=}] \removeitemvspace
	Inceptive manner preverb ‘starting, beginning’ indicating initiation of motion
	or the beginning of an eventuality.
	Derived from the noun \fm{g̱unayé} ‘elsewhere, different place’
		which itself is from \fm{g̱una} ‘different’ and \fm{yé} ‘place, way’,
		with the locative postposition allomorph \fm{-μ}.

\item[\X{áa=}]
	Locational preverb ‘there, at it’.
	Composition is \fm{áa} ≡ \fm{á} + \fm{-μ}
		from the third person nonhuman independent pronoun \fm{á} ‘it’
		and the locative postposition allomorph \fm{-μ}.

\item[\X{shóo=}]
	Locational preverb ‘at end of, in front of’.
	Composition is \fm{shóo} ≡ \fm{shú} + \fm{-μ}
		from the relational noun \fm{shú} ‘end’
		and the locative postposition allomorph \fm{-μ}.
\end{morphdesc}

\subsubsection{Preverb class G}\label{sec:inventory-preverb-G}

The preverbs in class G are all derived from a noun
	with the locative postposition allomorph \X[-í-loc]{-í} \~\ \X[-i-loc]{-i}.
See that entry for further discussion.

\begin{morphdesc}
\item[\X{dáag̱i=}]
	Locational preverb ‘inland’ indicating location on land away from a body of water.

\item[\X{éeg̱i=}]
\item[\X{éig̱i=}] \removeitemvspace
	Locational preverb ‘beach’ indicating location on land above the shoreline of a body of water.

\item[\X{gági=}]
	Locational preverb ‘emerging, out in the open’ indicating location in an open or visible area,
		usually as a result of movement from an enclosed or obscured location.

\item[\X{héeni=}]
	Locational preverb ‘in water’ indicating location in a body of water,
		usually as a result of movement from dry land.

\item[\X{neilí=}]
	Locational preverb ‘inside’ indicating location inside a building, cave,
		or other covered enclosure.

\item[\X{x̱áni=}]
	Locational preverb ‘near’ indicating location near some entity.
\end{morphdesc}

\subsubsection{Preverb class F}\label{sec:inventory-preverb-F}

The preverbs in class F are all originally from postposition phrases, but there is no single
	generalization that applies to all of them.
Most of the class F preverbs are fossilized so that speakers do not generally perceive them as having
	clear internal structure; the exception is \fm{héenx̱=} which is transparently from
	\fm{héen} ‘fresh water, river’ and the pertingent postposition
	\fm{-x̱} ‘of, at, contacting’.

\begin{morphdesc}
\item[\X{héenx̱=}]
	Locational preverb ‘in water’ indicating a location in a body of water,
		usually as a result of movement from dry land.

\item[\X{ḵáaḵw=}]
\item[\X{ḵáaḵwt=}] \removeitemvspace
\item[\X{ḵáaḵwx̱=}] \removeitemvspace
	Variant form of manner preverb  \X{ḵwáaḵ=} / \X{ḵwáaḵt=} / \X{ḵwáaḵx̱=} ‘wrongly’
		indicating that an event was performed incorrectly
		or happened in an unfortunate or undesirable way.

\item[\X{ḵut=}]
	Manner preverb ‘astray, lost, excess’ indicating that eventuality involves becoming lost
		either literally or metaphorically.
	Originally from areal \fm{ḵu-}
		and punctual postposition \fm{-t} ‘to, at, around’.

\item[\X{ḵwáaḵ=}]
\item[\X{ḵwáaḵt=}] \removeitemvspace
\item[\X{ḵwáaḵx̱=}] \removeitemvspace
	Manner preverb ‘wrongly’ indicating that an event was performed incorrectly
		or happened in an unfortunate or undesirable way.

\item[\X{ux̱=}]
\item[\X{úx̱=}] \removeitemvspace
	Manner preverb ‘out of control, blindly, amiss’
		indicating that the eventuality happens in a manner that is somehow unintentional,
		unconsidered, without clear vision, in error, without care, or unacceptable
	Usually occurs with class D preverb \X{kei=} \~\ \X{kéi=} ‘up’.

\item[\X{yaax̱=}]
	Directional preverb ‘aboard’ indicating motion into a boat or other vehicle.
	Originally from \fm{yaakw} ‘canoe, boat’
		and pertingent postposition \fm{-x̱} ‘of, at, contacting’.

\item[\X{ÿanax̱=}]
	Directional preverb ‘into ground’ indicating motion into the ground, specifically
		below the surface of the ground.
	Originally from \fm{ÿán} ‘shore’ (earlier ‘ground’)
		and perlative postposition \fm{-náx̱} ‘via, through, across’.

\item[\X{yatx̱=}]
\item[\X{yetx̱=}] \removeitemvspace
\item[\X{yedax̱=}] \removeitemvspace
	Directional preverb ‘lifting, picking up’ indicating movement upward
		from an unspecified surface which is usually the ground.
	From either \fm{ÿán} ‘ground’ or \fm{yé} ‘place’
		and ablative postposition \fm{-dáx̱} ‘from, off of, out of’.

\item[\X{yux̱=}]
	Directional preverb ‘out, outside’ indicating motion out of a building, cave,
		or other covered enclosure.
	Probably originally from distal \fm{yú} ‘far off’
		and pertingent postposition \fm{-x̱} ‘of, at, contacting’.
\end{morphdesc}

\subsubsection{Preverb class E}\label{sec:inventory-preverb-E}

The preverbs in class E all reflect a regular pattern of alternation between three postpositions:
	punctual \fm{-t} ‘to, arriving at’,
	pertingent \fm{-x̱} ‘at, contacting’,
	and allative \fm{-dé} \~\ \fm{-de} ‘toward’.
Although each of these postpositions are unrelated in many contexts,
	they are related together in the motion derivation
	\motderiv{NP-t/x̱/dé}{∅, \fm{-μμL} rep}{arriving at NP}.
The noun phrase NP describes the destination of the event of motion
	and the postposition used with this NP depends on the aspect of the verb.
For the prospective (‘future’) and progressive aspects the allative postposition \fm{-dé} \~\ \fm{-de}
	is used on the NP,
	for the repetitive imperfective aspect the pertingent postposition \fm{-x̱} is used on the NP,
	and for all other aspects the punctual postposition \fm{-t} is used on the NP.

\pex\label{ex:inventory-preverb-F-prospprog}%
\a\label{ex:inventory-preverb-F-prospprog-prosp}%
\exrtcmt{prospective aspect}
\begingl
	\gla	Wé \rlap{aan\gm{dé}} @ {} \rlap{kḵwagóot.} @ {} @ {} @ {} @ {} @ {} //
	\glb	wé aan -\gm{dé} g- ʷ- g̱- x̱a- \rt[¹]{gut} -μμH //
	\glc	\xx{mdst} town \·\xx{all} \xx{gcnj}\· \xx{irr}\· \xx{mod}\· \xx{1sg.s}\· \rt[¹]{go.\xx{sg}} \·\xx{var} //
	\gld	the town -to \rlap{\xx{prosp}.I.go} {} {} {} {} {} //
	\glft	‘I will go to the town.’
		//
\endgl
\a\label{ex:inventory-preverb-F-prospprog-prog}%
\exrtcmt{progressive aspect}
\begingl
	\gla	Wé \rlap{aan\gm{dé}} @ {} yaa @ \rlap{nx̱agút.} @ {} @ {} @ {} //
	\glb	wé aan -\gm{dé} ÿaa= n- x̱a- \rt[¹]{gut} -μH //
	\glc	\xx{mdst} town \·\xx{all} along\· \xx{ncnj}\· \xx{1sg.s}\· \rt[¹]{go.\xx{sg}} \·\xx{var} //
	\gld	the town -to along\· \rlap{\xx{prog}.I.go} {} {} {} //
	\glft	‘I am going to the town.’
		//
\endgl
\xe

\ex\label{ex:inventory-preverb-F-repimpfv}%
\exrtcmt{repetitive imperfective aspect}
\begingl
	\gla	Wé \rlap{aan\gm{x̱}} @ {} \rlap{x̱agoot.} @ {} @ {} //
	\glb	wé aan -\gm{x̱} x̱a- \rt[¹]{gut} -μμL //
	\glc	\xx{mdst} town \·\xx{pert} \xx{1sg.s}\· \rt[¹]{go.\xx{sg}} \·\xx{var} //
	\gld	the town -to \rlap{\xx{impfv}.I.go.\xx{rep}} {} {} //
	\glft	‘I repeatedly go to the town.’
		//
\endgl
\xe

\pex\label{ex:inventory-preverb-F-rest}%
\a\label{ex:inventory-preverb-F-rest-pfv}%
\exrtcmt{perfective aspect}%
\begingl
	\gla	Wé \rlap{aan\gm{t}} @ {} \rlap{x̱waagút.} @ {} @ {} @ {} @ {} //
	\glb	wé aan -\gm{t} ʷ- x̱a- μ- \rt[¹]{gut} -μH //
	\glc	\xx{mdst} town \·\xx{pnct} \xx{pfv}\· \xx{1sg.s}\· \xx{stv}\· \rt[¹]{go.\xx{sg}} \·\xx{var} //
	\gld	the town -to \rlap{\xx{pfv}.I.go} {} {} {} {} //
	\glft	‘I went to the town (and arrived).’, ‘I got to town.’
		//
\endgl
\a\label{ex:inventory-preverb-F-rest-hab}%
\exrtcmt{habitual aspect}%
\begingl
	\gla	Wé \rlap{aan\gm{t}} @ {} \rlap{x̱wagútch.} @ {} @ {} @ {} @ {} //
	\glb	wé aan -\gm{t} ʷ- x̱a- \rt[¹]{gut} -μH -ch //
	\glc	\xx{mdst} town \·\xx{pnct} \xx{zpfv}\· \xx{1sg.s}\· \xx{stv}\· \rt[¹]{go.\xx{sg}} \·\xx{var} //
	\gld	the town -to \rlap{\xx{hab}.I.go} {} {} {} {} //
	\glft	‘I always go to town (arriving every time).’, ‘I always get to town.’
		//
\endgl
\a\label{ex:inventory-preverb-F-rest-imp}%
\exrtcmt{imperative mood}%
\begingl
	\gla	Wé \rlap{aan\gm{t}} @ {} \rlap{nagú!} @ {} @ {} @ {} //
	\glb	wé aan -\gm{t} na- {} \rt[¹]{gut} -⊗ //
	\glc	\xx{mdst} town \·\xx{pnct} \xx{ncnj}\· \xx{2sg.s}\· \rt[¹]{go.\xx{sg}} \·\xx{var} //
	\gld	the town -to \rlap{\xx{imp}.you.go} {} {} {} //
	\glft	‘Go to town!, ‘Get to town!’
		//
\endgl
\xe

The preverbs \fm{haa=} ‘here’, \fm{ḵux̱=} ‘back’, \fm{neil=} ‘home, inside’,
	and \fm{ÿan=} ‘ashore, ending’ have some unpredictable forms
	with the three postpositions \fm{-t}, \fm{-x̱}, and \fm{-dé}
	from the motion derivation \motderiv{NP-t/x̱/dé}{∅, \fm{-μμL} rep}{arriving at NP}.
The preverb \fm{haa=} ‘here’ when combined with the allative postposition \fm{-dé} ‘toward’
	usually is pronounced \fm{haandé=} [\ipa{ˈhàːn.té}] with an extra \fm{n}.
The preverbs \fm{ḵux̱=} ‘back’, \fm{neil=} ‘home, inside’, and \fm{ÿan=} ‘ashore, ending’
	all normally lack the punctual postposition \fm{-t} ‘to, arriving at’
	when it would be expected from the \fm{NP-t/x̱/dé} motion derivation
	so that \fm[*]{ḵux̱t=} and \fm[*]{ÿant=} do not occur and \fm{neilt=} is rare.
Similarly, \fm{ḵux̱=} ‘back’ also normally lacks the pertingent postposition \fm{-x̱} ‘at, contacting’
	when it would be expected from the \fm{NP-t/x̱/dé} motion derivation
	so that \fm[*]{ḵux̱x̱=} does not occur.
The preverb \fm{ÿan=} ‘ashore, ending’ has a unique form \fm{ÿax̱=}
	with the pertingent postposition \fm{-x̱} ‘at, contacting’
	instead of the predicted \fm[*]{ÿanx̱=}.
See the entries for these preverbs for more details and discussion.

\begin{morphdesc}
\item[\X{haat=}]
\item[\X{haax̱=}] \removeitemvspace
\item[\X{haadé=}] \removeitemvspace
\item[\X{haandé=}] \removeitemvspace
	Directional preverb meaning ‘here, hither’.

\item[\X{kux=}]
\item[\X{kuxx̱=}] \removeitemvspace
\item[\X{kúxde=}] \removeitemvspace
	Directional preverb meaning ‘aground, shallows’.

\item[\X{ḵux̱=}]
\item[\X{ḵúx̱de=}] \removeitemvspace
	Directional preverb meaning ‘back, returning’.

\item[\X{neil=}]
\item[\X{neilt=}] \removeitemvspace
\item[\X{neilx̱=}] \removeitemvspace
\item[\X{neildé=}] \removeitemvspace
	Directional preverb meaning ‘home’ or ‘inside of building, cave’.
	
\item[\X{ÿan=}]
\item[{\X[ÿax̱=ashore]{ÿax̱=}}] \removeitemvspace
\item[\X{ÿánde=}] \removeitemvspace
	Directional preverb meaning ‘ashore’ with secondary metaphorical meaning ‘ending, terminating’.

\item[\X{yóot=}]
\item[\X{yóox̱=}] \removeitemvspace
\item[\X{yóode=}] \removeitemvspace
	Directional preverb meaning ‘away’.
\end{morphdesc}

\subsubsection{Preverb class D}\label{sec:inventory-preverb-D}

\begin{morphdesc}
\item[\X{daak=}]
	Directional preverb ‘out to sea’.
	Derived from the directional noun \fm{dáak} ‘out at sea’.

\item[\X{daaḵ=}]
	Directional preverb ‘inland’ indicating motion on land away from a body of water.
	Derived from the directional noun \fm{dáaḵ} ‘inland’.

\item[\X{eeḵ=}]
\item[\X{eèḵ=}] \removeitemvspace
\item[\X{eiḵ=}] \removeitemvspace
\item[\X{ÿeeḵ=}] \removeitemvspace
\item[\X{ÿeiḵ=}] \removeitemvspace
	Directional preverb ‘beach’ indicating motion from some upland area down to a beach
		or shoreline along a body of water.
	Derived from the noun \fm{éeḵ} \~\ \fm{éiḵ} ‘beach’.

\item[kei=]
\item[kéi=] \removeitemvspace
	Directional preverb meaning ‘up‘.

\item[yei=]
	Directional preverb meaning ‘down’.
\end{morphdesc}

\subsubsection{Preverb class C}\label{sec:inventory-preverb-C}

\begin{morphdesc}
\item[\X{yéi=}]
	Manner preverb ‘thus, so’.
	Derived from noun \fm{yé} \~\ \fm{yéi} ‘place, way, manner’.

\item[{\X[yóo=quot]{yóo=}}]
	Quotative preverb ‘thus, so’ indicating that the material preceding the verb is quoted speech.
\end{morphdesc}

\subsubsection{Preverb class B}\label{sec:inventory-preverb-B}

\begin{morphdesc}
\item[{\X[ÿaa=mind]{ÿaa=}}]
	Manner preverb ‘mental’ uniquely associated with verbs that describe mental phenomena.
	No obvious etymological connections to other vocabulary
		but may be related to Proto-Dene \fm[*]{yənə-} \~\ \fm[*]{yiːn-} ‘mind’
		and Eyak \fm{ʔiːlih=} ‘mental’.
\end{morphdesc}

\subsubsection{Preverb class A}\label{sec:inventory-preverb-A}

\begin{morphdesc}
\item[{\X[yoo=alt]{yoo=}}]
	Manner preverb ‘alternating, back and forth, to and fro’
		indicating a repeated alternation between two locations, dispositions, or situations.

\item[{\X[ÿaa=along]{ÿaa=}}]
	Directional preverb ‘along’ indicating progression horizontally or laterally.
	Related to the directional noun \fm{diÿáa} ‘across, other side’
		and the noun \fm{niÿaa} ‘direction’.

\item[{\X[ÿax̱=exh]{ÿax̱=}}]
	Exhaustive manner preverb ‘completely, using up, affecting all of’
		that occurs as part of the exhaustive derivation
		\motderiv{ÿax̱= ÿa-s-}{∅, \fm{-x̱} rep}{completely, using up, affecting all of}.

\item[{\X[ÿax̱=facing]{ÿax̱=}}]
	Directional preverb ‘facing’
		that occurs as part of the motion derivation
		\motderiv{NP-xʼ ÿax̱=}{∅, \fm{-x̱} rep}{turning over by NP}
		and other motion derivations based on this one.
\end{morphdesc}

\subsection{Quantifier proclitics}\label{sec:inventory-qfr}

\subsection{Object prefixes}\label{sec:inventory-object}

\subsection{Qualifiers and incorporated nouns}\label{sec:inventory-qualinc}

\subsection{Aspect and conjugation class prefixes}\label{sec:inventory-aspconj}

\subsection{Subject prefixes}\label{sec:inventory-subject}

\subsection{Classifier prefixes}\label{sec:inventory-classifier}

\subsection{Roots}\label{sec:inventory-root}

\subsection{Stem variation suffixes}\label{sec:inventory-stemvar}

The stem variation suffixes are representations of the different stem variation patterns
as notional suffixes that attach to roots and give rise to the core syllable of stems.
Stem variation suffixes are symbolic representations of four different processes:

\begin{itemize}
\item	vowel length measured in moras, symbolized by μ (Greek letter \textit{mu})
	\begin{itemize}
	\item	one mora μ = short vowel
	\item	two moras μμ = long vowel
	\end{itemize}
\item	tone symbolized by H and L or both
	\begin{itemize}
	\item	H = high tone
	\item	L = low tone
	\item	HL = falling tone (Southern dialects only)
	\end{itemize}
\item	ablaut (change of a lexical /\ipa{a}/ or /\ipa{u}/ vowel into [\ipa{e}])
	\begin{itemize}
	\item	μᵉ = ablauted vowel
	\end{itemize}
\item	deletion of a final consonant represented by ⊗,
	consequently appearing as a short vowel with high tone (μH)
\end{itemize}

Tongass Tlingit has a distinct stem variation system because it does not have tone
and instead has laryngealized vowels.
Vowel length is the same as in other dialects with short vowels (μ) and long vowels (μμ).
Long vowels may be a plain vowel (no special indication so just μμ),
laryngealized with a breathy/fading vowel (μμʰ),
or laryngealized with a glottalized vowel (μμˀ).

\begin{morphdesc}
\item[\X{-μL}]
	short vowel (μ) with low tone (L) so […\ipa{V̀}…]

\item[\X{-μH}]
	short vowel (μμ) with high tone (H) so […\ipa{V́}…]

\item[\X{-μμL}]
	long vowel (μμ) with low tone (L) so […\ipa{V̀ː}…]

\item[\X{-μμH}]
	long vowel (μμ) with high tone (H) so […\ipa{V́ː}…]

\item[\X{-μμHL}]
	long vowel (μμ) with falling tone (HL) so […\ipa{V̂ː}…];
	only in Southern Tlingit dialects (Sanya, Henya) that have phonologically
		contrastive falling tone

\item[\X{-μᵉμL}]
	ablaut (/\ipa{a, u}/ → [\ipa{e}]) long vowel (μμ)
		with low tone (L) so […\ipa{èː}…];
	normally occurs only with \fm{\rt{CVᴸ}} roots

\item[\X{-μᵉμH}]
	ablaut (/\ipa{a, u}/ → [\ipa{e}]) long vowel (μμ)
		with high tone (H) so […\ipa{éː}…];
	normally occurs only with \fm{\rt{CV}} or \fm{\rt{CVᴸ}} roots

\item[\X{-μᵉμHL}]
	ablaut (/\ipa{a, u}/ → [\ipa{e}]) long vowel (μμ)
		with falling tone (HL) so […\ipa{êː}…];
	only in Southern Tlingit dialects (Sanya, Henya) that have phonologically
		contrastive falling tone;
	normally occurs only with \fm{\rt{CV}} or \fm{\rt{CVᴸ}} roots
		that contain /\ipa{a}/ or /\ipa{u}/

\item[{\X[-DEL]{-⊗}}]
	irregular deletion (⊗) of final consonant
		resulting in a short vowel with high tone (μH)
		so […\ipa{V́}]
		(no tone […\ipa{V}] in Tongass Tlingit);
	only occurs in imperatives with
		\begin{inlinelist}
		\item	\fm{\rt[¹]{gut}} ‘sg go’
		\item	\fm{\rt[¹]{.at}} ‘pl go’
		\item	\fm{\rt[¹]{nuk}} ‘sg sit’
		\end{inlinelist}
\end{morphdesc}

\subsection{Repetitive and derivation suffixes}\label{sec:inventory-repderiv}

\subsection{Tense, modality, and clause type suffixes}\label{sec:inventory-tmc}

\subsection{Enclitic (postverbal) auxiliaries}\label{sec:inventory-aux}
