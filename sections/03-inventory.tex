%!TEX root = ../lingnote-verbmorphs.tex

\section{Inventory of verb morphemes by position and function}\label{sec:inventory}

\subsection{Preverbs}\label{sec:inventory-preverb}

\subsection{Quantifier proclitics}\label{sec:inventory-qfr}

\subsection{Object prefixes}\label{sec:inventory-object}

\subsection{Qualifiers and incorporated nouns}\label{sec:inventory-qualinc}

\subsection{Aspect and conjugation class prefixes}\label{sec:inventory-aspconj}

\subsection{Subject prefixes}\label{sec:inventory-subject}

\subsection{Classifier prefixes}\label{sec:inventory-classifier}

\subsection{Roots}\label{sec:inventory-root}

\subsection{Stem variation suffixes}\label{sec:inventory-stemvar}

The stem variation suffixes are representations of the different stem variation patterns
as notional suffixes that attach to roots and give rise to the core syllable of stems.
Stem variation suffixes are symbolic representations of four different processes:

\begin{itemize}
\item	vowel length measured in moras, symbolized by μ (Greek letter \textit{mu})
	\begin{itemize}
	\item	one mora μ = short vowel
	\item	two moras μμ = long vowel
	\end{itemize}
\item	tone symbolized by H and L or both
	\begin{itemize}
	\item	H = high tone
	\item	L = low tone
	\item	HL = falling tone (Southern dialects only)
	\end{itemize}
\item	ablaut (change of a lexical /\ipa{a}/ or /\ipa{u}/ vowel into [\ipa{e}])
	\begin{itemize}
	\item	μᵉ = ablauted vowel
	\end{itemize}
\item	deletion of a final consonant represented by ⊗,
	consequently appearing as a short vowel with high tone (μH)
\end{itemize}

Tongass Tlingit has a distinct stem variation system because it does not have tone
and instead has laryngealized vowels.
Vowel length is the same as in other dialects with short vowels (μ) and long vowels (μμ).
Long vowels may be a plain vowel (no special indication so just μμ),
laryngealized with a breathy/fading vowel (μμʰ),
or laryngealized with a glottalized vowel (μμˀ).

\begin{morphdesc}
\item[\X{-μL}]
	short vowel (μ) with low tone (L) so […\ipa{V̀}…]

\item[\X{-μH}]
	short vowel (μμ) with high tone (H) so […\ipa{V́}…]

\item[\X{-μμL}]
	long vowel (μμ) with low tone (L) so […\ipa{V̀ː}…]

\item[\X{-μμH}]
	long vowel (μμ) with high tone (H) so […\ipa{V́ː}…]

\item[\X{-μμHL}]
	long vowel (μμ) with falling tone (HL) so […\ipa{V̂ː}…];
	only in Southern Tlingit dialects (Sanya, Henya) that have phonologically
		contrastive falling tone

\item[\X{-μᵉμL}]
	ablaut (/\ipa{a, u}/ → [\ipa{e}]) long vowel (μμ)
		with low tone (L) so […\ipa{èː}…];
	normally occurs only with \fm{\rt{CVᴸ}} roots

\item[\X{-μᵉμH}]
	ablaut (/\ipa{a, u}/ → [\ipa{e}]) long vowel (μμ)
		with high tone (H) so […\ipa{éː}…];
	normally occurs only with \fm{\rt{CV}} or \fm{\rt{CVᴸ}} roots

\item[\X{-μᵉμHL}]
	ablaut (/\ipa{a, u}/ → [\ipa{e}]) long vowel (μμ)
		with falling tone (HL) so […\ipa{êː}…];
	only in Southern Tlingit dialects (Sanya, Henya) that have phonologically
		contrastive falling tone;
	normally occurs only with \fm{\rt{CV}} or \fm{\rt{CVᴸ}} roots
		that contain /\ipa{a}/ or /\ipa{u}/

\item[{\X[-DEL]{-⊗}}]
	irregular deletion (⊗) of final consonant
		resulting in a short vowel with high tone (μH)
		so […\ipa{V́}]
		(no tone […\ipa{V}] in Tongass Tlingit);
	only occurs in imperatives with
		\begin{inlinelist}
		\item	\fm{\rt[¹]{gut}} ‘sg go’
		\item	\fm{\rt[¹]{.at}} ‘pl go’
		\item	\fm{\rt[¹]{nuk}} ‘sg sit’
		\end{inlinelist}
\end{morphdesc}

\subsection{Repetitive and derivation suffixes}\label{sec:inventory-repderiv}

\subsection{Tense, modality, and clause type suffixes}\label{sec:inventory-tmc}

\subsection{Enclitic (postverbal) auxiliaries}\label{sec:inventory-aux}
