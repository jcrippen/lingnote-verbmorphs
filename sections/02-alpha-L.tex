%!TEX root = ../lingnote-verbmorphs.tex

\subsection{L}\label{sec:alphalist-l}
\begin{morphdesc}[resume*=alphalist]
\item[l-]\label{m:l-}
	valency prefix of classifier
	\begin{enumerate}
	\item	argument addition
		\begin{enumerate}
		\item	lone argument of intransitive
		\item	causative
		\item	applicative
		\end{enumerate}
	\item	spatial extension
		\begin{enumerate}
		\item	extended entity
		\item	extended eventuality
		\end{enumerate}
	\end{enumerate}

\item[lˢ-]\label{m:lˢ-}
	allomorph of valency \fm{s-} \~\ \fm{sa-};
	occurs when any fricative
		\fm{s}, \fm{sʼ}, \fm{l}, \fm{lʼ}, \fm{sh}
	or any affricate
		\fm{dz}, \fm{ts}, \fm{tsʼ},
		\fm{dl}, \fm{tl}, \fm{tlʼ},
		\fm{j}, \fm{ch}, \fm{chʼ}
	occurs in the onset or coda of the stem syllable;
	phonetically indistinguishable from \fm{l-} \~\ \fm{la-} and may be represented as such
	if the distinction is not important
	\begin{itemize}
	\item	\fm{lichán} (impfv; obj intr, \fm{g}, \fm{-μH} invar.\ state) ‘it stinks’
		(not *\fm{sichán})
	\item	\fm{wutuliyíḵsʼ} (rep pfv; tr, \fm{n}, mot) ‘we repeatedly pulled it (long obj.)’\newline
		versus \fm{wutusiyeeḵ} (pfv) ‘we pulled it (long obj.)’
	\end{itemize}

\item[…l]\label{m:…l}
	≡ \fm{d-l-}
	combination of voice \X{d-}
		and valency \X{l-} or \X{lˢ-},
	appears only as a coda consonant and so requires a preceding vowel
	\begin{itemize}
	\item	\fm{sh ilg̱ásʼx̱} (rep impfv; tr, \fm{∅}, ach) ‘s/he/it repeatedly scratches self’
			with \fm{d-l-}\newline
		versus \fm{sh wudlig̱ásʼ} (pfv) ‘s/he/it scratched self’
			with \fm{d-l-i-}
	\item	\fm{tléil sh kawulháachʼ} (neg pfv; tr, \fm{n}, \fm{-μμH} state) ‘s/he/it didn’t shame self’
			with \fm{d-l-}
		versus \fm{sh kawdliháachʼ} (pfv) ‘s/he/it shamed self’
			with \fm{d-l-i-}
	\end{itemize}

\item[-l]\label{m:-l}
	fossilized instrument noun suffix,
		cognate with Eyak \fm{-ł} \parencite[9]{krauss:1981a}
		but no longer productive in Tlingit and limited to a handful of nouns and
		possibly a few verbs;
	this suffix is glossed here as \xx{inst} for ‘instrument’ but in most contexts
		it need not be either segmented or glossed because it has no meaning
		for modern speakers;
	possibly related to the unknown suffix \X{-sh} which is limited to one verb root
		and a handful of nouns;
	the repetitive suffix \X{-lʼ} is occasionally delaryngealized to \fm{-l}
		by some speakers (such as \fm{shantudínl} = \fm{shantudínlʼ} ‘idiot, airhead’)
		reflecting the fact that \fm{-lʼ} is not obviously meaningful in some forms
	\begin{enumerate}
	\item	nouns with a frozen \fm{-l} suffix
		\begin{itemize}
		\item	\fm{áatʼl} ‘pit for aging hooligan grease; cold storage pit’
		\item	\fm{chʼúḵʼl} ‘sand lance, surf smelt’
		\item	\fm{táḵl} ‘hammer’ (also \fm{shakatáḵl} ‘bobby pin’)
		\item	\fm{tsʼáḵl} ‘black paint’
		\item	\fm{x̱ákwl} ‘grappling hook (orig. made of eagle claw)’
		\end{itemize}
	\end{enumerate}

\item[-lʼ]\label{m:-lʼ}
	repetitive suffix limited to only a few verbs,
	most often with \fm{\rt[²]{xakw}} ‘whip up, beat (eggs, soapberries, etc.)’;
	probably historically related to \X{-sʼ};
	\newline
	allomorphs:
	\begin{allolist}
	\item[-lʼ]	single consonant form
	\item[\X{-álʼ}]	form with epenthetic (filler) vowel \fm{á}
	\end{allolist}
	\begin{enumerate}
	\item	repetitive suffix attested with three verb roots
		\begin{itemize}
		\item	\vbform{aklaxákwlʼ}{rep impfv}[tr, \fm{∅}, ach]{she/he/it whips up him/her/it}
				\vbmorph{a-&k-&la-&\rt[²]{xakw}&-μH&\gm{-lʼ}}
					{\xx{3>3}&\xx{qual}&\xx{xtn}&\rt[²]{whip}&\·\xx{var}&\·\xx{rep}}
			\versus \vbform{akawlixákw}{pfv}{she/he/it whipped up him/her/it}
				\vbmorph{a-&ka-&w-&l-&i-&\rt[²]{xakw}&-μH}
					{\xx{3>3}&\xx{qual}&\xx{pfv}&\xx{xtn}&\xx{stv}&\rt[²]{whip}&\·\xx{var}}
				
		\item	\vbform{akaagúklʼ}{rep impfv}[tr, \fm{∅}?, state]{she/he/it tries to be skilled at it}
			\parencites[f05/176]{leer:1973}[671]{leer:1976}
				\vbmorph{a-&ka-&μ-&\rt[²]{guᴴk}&-μH&\gm{-lʼ}}
					{\xx{3>3}&\xx{qual}&\xx{stv}&\rt[²]{know.how}&\·\xx{var}&\·\xx{rep}}
			\versus \vbform{ashigóok}{impfv}[tr, \fm{g}/\fm{g̱}, state]{she/he/it knows how to do it}
				\vbmorph{a-&sh-&i-&\rt[²]{guᴴk}&-μμH}
					{\xx{3>3}&\xx{xtn}&\xx{stv}&\rt[²]{know.how}&\·\xx{var}}
		\item	\vbform{kax̱aagúnlʼx̱}{rep impfv}[tr?, conj?, state]{I am trying hard}
			\parencite[f05/126]{leer:1973}
				\vbmorph{ka-&x̱a-&μ-&\rt{gun}&\gm{-lʼ}&-x̱}
					{\xx{qual}&\xx{1sg.s}&\xx{stv}&\rt{try.hard}&\·\xx{rep}&\·\xx{rep}}
		\end{itemize}
	\item	unclear meaning in a handful of nouns
		\begin{itemize}
		\item	\fm{dínlʼ} ‘obstruction’ from unknown \fm{\rt{din}},
			possibly related to \fm{\rt[²]{diᴴn}} ‘trouble’;
			occurs as a part of:
			\begin{itemize}
			\item	\fm{gukyikdínlʼ} ‘hard of hearing’
				\vbmorph{guk-&yik-&\rt{din}&-μH&\gm{-lʼ}}
					{ear&within&\rt{obstruction?}&\·\xx{var}&\·\xx{rep}}
			\item	\fm{shantudínlʼ} ‘idiot, airhead’
				\vbmorph{shaⁿ-&tu-&\rt{din}&-μH&\gm{-lʼ}}
					{head&inside&\rt{obstruction?}&\·\xx{var}&\·\xx{rep}}
			\end{itemize}
		\item	\fm{gúnlʼ} ‘burl, lump’ from unknown \fm{\rt{gun}}
			\vbmorph{\rt{gun}&-μH&\gm{-lʼ}}
				{\rt{lump?}&\·\xx{var}&\·\xx{rep}}
			\newline
			also occurs in
				\begin{inlinelist}
				\item	\fm{aasdaagúnlʼi} ‘burl’
				\item	\fm{jigúnlʼi} ‘wrist knob’
				\item	\fm{lakʼichʼgúnlʼi} ‘occipital knob’
				\item	\fm{leikachóox̱'u gúnlʼi} ‘adam’s apple’
				\item	\fm{leitux̱gúnlʼi} ‘adam’s apple’
				\item	\fm{tlʼeḵkagúnlʼi} ‘knuckle’
				\item	\fm{x̱ʼagúnlʼi} ‘lumpy mouth’
				\item	\fm{x̱ʼusgúnlʼi} ‘ankle knob’
				\end{inlinelist};
			the noun is also the basis of the derived verb
				\vbform{kashigúnlʼ}{impfv}[obj intr, conj?, inv state]{it has a burl};
			probably related to \vbform{kax̱aagúnlʼx̱}{rep impfv}{I am trying hard}
				discussed above	but the meaning relationship is unclear
		\item	\fm{g̱úḵlʼ} ‘swan’ from unknown \fm{\rt{g̱uḵ}}
			\vbmorph{\rt{g̱uḵ}&-μH&\gm{-lʼ}}
				{\rt{\xx{unkn}}&\·\xx{var}&\·\xx{rep}}
			\newline
			also occurs in \fm{áa tug̱úḵlʼi} ‘kind of whitefish’
		\item	\fm{túḵlʼ} ‘young spruce or hemlock; cartilage’ from unknown \fm{\rt{tuḵ}}
			\vbmorph{\rt{tuḵ}&-μH&-lʼ}
				{\rt{flex?}&\·\xx{var}&\·\xx{rep}}
			\newline
			occurs in 
				\begin{inlinelist}
				\item	\fm{lututúḵlʼi} ‘nasal cartilage’
				\item	\fm{sʼaḵshutúḵlʼi} ‘bone end cartilage’
				\item	\fm{sʼaḵx̱ʼaaktúḵlʼi} ‘cartilage between bones’
				\item	\fm{yuwshutúḵlʼi} ‘xiphoid process’
				\end{inlinelist};
			probably related to \fm{dúḵ} ‘cottonwood’;
			possibly related to \fm{tóox̱ʼ} in \fm{shutóox̱ʼ} ‘ankle’
				and \fm{x̱ʼusʼgukshtóox̱ʼ} ‘ankle bone’;
			less likely related to \fm{tooḵ} ‘butt’ or \fm{tóoḵ} ‘needlefish’
		\item	\fm{tʼáḵlʼ} ‘projecting bone’ from unknown \fm{\rt{tʼaḵ}}
			possibly related to \fm{tʼaaḵ} ‘beside’;
			occurs as a part of:
			\begin{itemize}
			\item	\fm{jitʼáḵlʼi} ‘wrist knob’
				\vbmorph{ji-&\rt{tʼaḵ}&-μH&\gm{-lʼ}&-i}
					{hand&\rt{beside?}&\·\xx{var}&\·\xx{rep}&\·\xx{poss}}
			\item	\fm{x̱ʼustʼáḵlʼi} ‘ankle knob’
				\vbmorph{x̱ʼus-&\rt{tʼaḵ}&-μH&\gm{-lʼ}&-i}
					{foot&\rt{beside?}&\·\xx{var}&\·\xx{rep}&\·\xx{poss}}
			\end{itemize}
			also compare \fm{tʼáaḵw} ‘joist, timber; carpentry joint’
		\item	\fm{xákwlʼi} ‘soapberries’ from \fm{\rt{xakw}} ‘whip’
			\vbmorph{\rt{xakw}&-μH&\gm{-lʼ}&-i}
				{\rt{whip}&\·\xx{var}&\·\xx{rep}&\·\xx{nmz}}
			\newline
			nominalization of the verb ‘whip up’ above
		\item	\fm{yadzánlʼ} ‘lumpy face’ from unknown \fm{\rt{dzan}}
			\vbmorph{ÿa-&\rt{dzan}&-μH&\gm{-lʼ}}
				{face&\rt{lump?}&\·\xx{var}&\·\xx{rep}}
			\newline
			possibly related to \fm{dzánti} ‘flounder’;
			may derive from irregular palatalization and affrication of
				\fm{gúnlʼ} ‘burl, lump’ above
		\end{itemize}
	\item	potentially identifiable as a frozen suffix in some CVC roots
		\begin{itemize}
		\item	\fm{\rt{sʼelʼ}} ‘tear, rip’ in
			\newline
			\vbform{asʼéilʼ}{impfv}[tr, \fm{n}, \fm{-μμH} act]{she/he/it is tearing, ripping him/her/it}
			\parencites[09/218]{leer:1973}[518]{leer:1976}
				\vbmorph{a-&\rt{sʼelʼ}&-μμH}
					{\xx{3>3}&\rt{tear}&\·\xx{var}}
			\newline
			also nouns \fm{sʼéilʼ} ‘rip, tear, wound’
				and \fm{sʼélʼ} ‘rubber’;
			perhaps from an unknown \fm[*]{\rt{sʼa}} with ablaut \X{-μᵉμH} and \fm{-lʼ};
			compare \fm{\rt[²]{sʼa}} ‘claim item about to be destroyed;
				claim as payment, plunder’
		\end{itemize}
	\end{enumerate}

\item[la-]\label{m:la-val}
	allomorph of valency \X{l-}

\item[la-]\label{m:la-throat}
	allomorph of incorporated noun \X{le-} ‘throat’

\item[lˢa-]\label{m:lˢa-}
	allomorph of valency \X{lˢ-}

\item[le-]\label{m:le-}
	incorporated noun indicating throat or inside of mouth

\item[li]
	≡ \fm{l-i-}
	combination of valency \X{l-} or \X{lˢ-}
		and stative \X[i-stv]{i-}

\item[-lʼútʼ]\label{m:-lʼútʼ}
	allomorph of noun \fm{lʼóotʼ} ‘tongue’ used as a suffix in the stem
		\fm{–ḵéilʼútʼ} ‘lick seam’;
	this stem is formed from the root \fm{\rt[²]{ḵa}} ‘stitch, sew’ 
		with stem variation \X{-μᵉμH}
		and \fm{-lʼútʼ};
	compare the noun \fm{ḵéichʼálʼ} ‘seam’ with \fm{-μᵉμH} and \X{-chʼálʼ}
	\begin{itemize}
	\item	\vbform{aḵéilʼútʼ}{impfv}[tr, conj?, inv act]{she/he/it is licking it (seam) to make it hard}
		\parencite[f01/26]{leer:1973}
			\vbmorph{a-&\rt[²]{ḵa}&-μᵉμH&\gm{-lʼútʼ}}
				{\xx{3>3}&\rt[²]{stitch}&\·\xx{var}&\·tongue}
	\end{itemize}
		
		
\end{morphdesc}
