%!TEX root = ../lingnote-verbmorphs.tex

\subsection{H}\label{sec:alphalist-h}
\begin{morphdesc}[resume*=alphalist]
\item[haa=]\label{m:haa=}
	first person plural object
	(compare \fm{haa keidlí áwé} ‘it is our dog’);
	although this is homophonous with the possessive pronoun,
		\fm{haa=} as an object is not necessarily possessive
	\begin{itemize}
	\item	\fm{haa yisiteen} (pfv; tr, \fm{g̱}, ach) ‘you sg.\ saw us’\newline
		versus \fm{x̱at yisiteen} (pfv) ‘you sg.\ saw me’
	\end{itemize}

\item[has=]\label{m:has=}
	human pluralizer for third person;
	\newline
	allomorphs:
	\begin{allolist}
	\item[as=]	onset glottal stop instead of \fm{h} used in Southern and Tongass varieties
	\item[s=]	lone consonant, usually coda of a preceding syllable
	\end{allolist}
	note that since Tlingit is number neutral (nouns are not singular by default),
		a form without \fm{has=} may still refer to plural humans,
		i.e.\ \fm{has=} is not required for third person human plural arguments
	\begin{enumerate}
	\item	human pluralizer for third person subject
		\begin{itemize}
		\item	\fm{tʼá aawax̱áa} (pfv; tr, \fm{∅}, \fm{-μH} act) ‘s/he/it ate king salmon’\newline
			versus \fm{tʼá has aawax̱áa} (pfv) ‘they (humans) ate king salmon’
		\end{itemize}
	\item	human pluralizer for third person object
		\begin{itemize}
		\item	\fm{has tushikʼáan} (impfv; tr, \fm{g}, \fm{-μμH} state) ‘we hate them’\newline
			versus \fm{yee tushikʼáan} (impfv) ‘we hate you guys’
		\end{itemize}
	\item	human pluralizer for both third person subject and third person object;
		some speakers do not accept this use of \fm{has=}
			for both subject and object at the same time
		\begin{itemize}
		\item	\fm{has awsiteen} (pfv; tr, \fm{g}, ach) ‘they saw them’
			or ‘s/he/it saw them’ or ‘they saw him/her/it’
		\end{itemize}
	\end{enumerate}

\item[héeni=]\label{m:héeni=}

\end{morphdesc}

