%!TEX root = ../lingnote-verbmorphs.tex

\subsection{H}\label{sec:alphalist-h}
\begin{morphdesc}[resume*=alphalist]
\item[haa=]\label{m:haa=}
	first person plural object
	(compare \fm{haa keidlí áwé} ‘it is our dog’);
	although this is homophonous with the possessive pronoun,
		\fm{haa=} as an object is not necessarily possessive
	\begin{itemize}
	\item	\fm{haa yisiteen} (pfv; tr, \fm{g̱}, ach) ‘you sg.\ saw us’\newline
		versus \fm{x̱at yisiteen} (pfv) ‘you sg.\ saw me’
	\end{itemize}

\item[has=]\label{m:has=}
	human pluralizer for third person;
	\newline
	allomorphs:
	\begin{allolist}
	\item[as=]	onset glottal stop instead of \fm{h} used in Southern and Tongass varieties
	\item[s=]	lone consonant, usually coda of a preceding syllable
	\end{allolist}
	note that since Tlingit is number neutral (nouns are not singular by default),
		a form without \fm{has=} may still refer to plural humans,
		i.e.\ \fm{has=} is not required for third person human plural arguments
	\begin{enumerate}
	\item	human pluralizer for third person subject
		\begin{itemize}
		\item	\fm{tʼá aawax̱áa} (pfv; tr, \fm{∅}, \fm{-μH} act) ‘s/he/it ate king salmon’\newline
			versus \fm{tʼá has aawax̱áa} (pfv) ‘they (humans) ate king salmon’
		\end{itemize}
	\item	human pluralizer for third person object
		\begin{itemize}
		\item	\fm{has tushikʼáan} (impfv; tr, \fm{g}, \fm{-μμH} state) ‘we hate them’\newline
			versus \fm{yee tushikʼáan} (impfv) ‘we hate you guys’
		\end{itemize}
	\item	human pluralizer for both third person subject and third person object;
		some speakers do not accept this use of \fm{has=}
			for both subject and object at the same time
		\begin{itemize}
		\item	\fm{has awsiteen} (pfv; tr, \fm{g}, ach) ‘they saw them’
			or ‘s/he/it saw them’ or ‘they saw him/her/it’
		\end{itemize}
	\end{enumerate}

\item[héeni=]\label{m:héeni=}
	Locational preverb ‘in water’ indicating location in a body of water,
		usually as a result of movement from dry land.
	Derived from the noun \fm{héen} ‘fresh water; river’
		with the special locative postposition allomorph
		\X[-i-loc]{-i} \~\ \X[-í-loc]{-í} ‘at’
		(instead of the regular allomorphs \fm{-xʼ} and \fm{-μ}).
	This locative allomorph is unique in that it only occurs with a handful of preverbs,
		for which see the detailed entry of \X[-í-loc]{-í}.
	The noun \fm{héen} normally only refers to either (a) fresh water
		as opposed to \fm{éilʼ} ‘salt; salt water, ocean’
		or (b) a river or other flowing body of water
		as opposed to \fm{áa} ‘lake’,
		but the preverb \fm{héeni=} is used more broadly
		for both salt water and fresh water
		as well as for flowing water bodies and static water bodies.
	As with other preverbs including \X[-i-loc]{-i}, \fm{héeni=} can be represented either
		fully segmented as \fm{héen-i=}
		or unsegmented as \fm{héeni=}.
	The preverb \fm{héeni=} is easily confused with the homophonous \fm{héen-i} ‘river-\xx{poss}’
		which is a possessed noun phrase; this possessed noun phrase will always have a
		possessor – e.g. \fm{x̱áat héen-i} ‘salmon river-\xx{poss}’ – whereas the preverb
		does not have a possessor.
	The form \fm{héèni=} [\ipa{hîː.nì}] is predicted in Southern Tlingit varieties
		and \fm{heéni=} [\ipa{hiːˀ.niʰ}] in Tongass Tlingit.
	\begin{enumerate}
	\item	In forms with \fm{∅} conjugation class that reflect the motion derivation
			\motderiv{héeni}{∅, \fm{-x̱} rep}{in water}.
		\begin{itemize}
		\item	\vbform{héeni yoo kax̱wlitséx̱}{pfv}[tr, \fm{∅}, mot]{I trampled on them in the water (to get them clean)}
			\parencite[233.3303]{story-naish:1973}
				\vbmorph{\gm{héen}&\gm{-i=}&yoo=&ka-&ʷ-&x̱-&l-&i-&\rt[²]{tsex̱}&-μH}
					{water&\·\xx{loc}&\xx{alt}&\xx{qual}&\xx{pfv}&\xx{1sg.s}&\xx{xtn}&\xx{stv}&\rt[²]{kick}&\·\xx{var}}
		\item	\vbform{héeni wugoodí}{sub pfv}[subj intr, \fm{∅}, mot]{as he stepped into the water}
			\parencite[86.83]{dauenhauer-dauenhauer:1987}
				\vbmorph{\gm{héen}&\gm{-i=}&wu-&\rt[¹]{gut}&-μμL&-í}
					{water&\·\xx{loc}&\xx{pfv}&\rt[¹]{go.\xx{sg}}&\·\xx{var}&\·\xx{sub}}
		\end{itemize}
	\item	In forms with \fm{∅} conjugation class
			and some form of \X{ÿeiḵ=} \~\ \X{eiḵ=} \~\ \X{eeḵ=} ‘beach’
			that reflect the motion derivation
			\motderiv{héeni ÿeiḵ}{∅, \fm{-ch} rep}{from shore into water}.
		\begin{itemize}
		\item	\vbform{héeni eeḵ has atáan}{csec}[tr, \fm{∅}, mot]{them having brought it down into the water}
			\parencite[14.115]{nyman-leer:1993}
				\vbmorph{\gm{héen}&\gm{-i=}&eeḵ=&has=&a-&\rt[²]{tan}&-μμH&}
					{water&\·\xx{loc}&beach&\xx{plh}&\xx{3>3}&\rt[²]{handle.w/ec}&\·\xx{var}&\xx{sub}}
		\end{itemize}
	\item	In forms with something other than \fm{∅} conjugation class.
		These may reflect undocumented motion derivations that use \fm{héeni=} with
			other conjugation classes or they may reflect one of the motion
			derivations above with an additional motion derivation that changes
			the conjugation class from \fm{∅} to something else.
		Alternatively, these may be cases where \fm{héeni=} is used without a motion derivation,
			in which case \fm{héeni=} is probably a kind of adjunct postposition phrase.
		\begin{itemize}
		\item	\vbform{héeni kawdax̱dudliyaa}{pfv}[tr, \fm{g̱}, ach]{people lowered each into the sea}
			\parencite[96.300]{dauenhauer-dauenhauer:1987}
				\vbmorph{\gm{héen}&\gm{-i=}&ka-&w-&dax̱-&du-&d-&l-&i-&\rt[²]{ÿa}&-μμL}
					{water&\·\xx{loc}&\xx{qual}&\xx{pfv}&\xx{distb}&\xx{ind.h.s}&\xx{mid}&\xx{xtn}&\xx{stv}&\rt[²]{lower}&\·\xx{var}}
		\item	\vbform{héeni wtuwa.aat}{pfv}[subj intr, \fm{g}?, mot]{we waded ashore}
			\parencite[290.562]{dauenhauer-dauenhauer:1987}
				\vbmorph{\gm{héen}&\gm{-i=}&w-&tu-&wa-&\rt[¹]{.at}&-μμL}
					{water&\·\xx{loc}&\xx{pfv}&\xx{1pl.s}&\xx{stv}&\rt[¹]{go.\xx{pl}}&\·\xx{var}}
			\newline
			Also \vbform{áxʼ héeni aawa.aat}{pfv}{people waded ashore there} (line 565)
				and \vbform{áxʼ héeni has woo.aat}{pfv}{they waded ashore there} (line 569)
				in the same text.
			All three are translated as ‘ashore’ although it looks like they would mean
				‘into water’.
		\end{itemize}
	\end{enumerate}
\end{morphdesc}

