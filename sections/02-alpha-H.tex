%!TEX root = ../lingnote-verbmorphs.tex

\subsection{H}\label{sec:alphalist-h}
\begin{morphdesc}[resume*=alphalist]
\item[haa=]\label{m:haa=1plo}
	first person plural object
	(compare \fm{haa keidlí áwé} ‘it is our dog’);
	although this is homophonous with the possessive pronoun,
		\fm{haa=} as an object is not necessarily possessive
	\begin{itemize}
	\item	\fm{haa yisiteen} (pfv; tr, \fm{g̱}, ach) ‘you sg.\ saw us’\newline
		versus \fm{x̱at yisiteen} (pfv) ‘you sg.\ saw me’
	\end{itemize}

\item[haa=]\label{m:haa=here}
	Directional preverb meaning ‘here, hither’ indicating a direction toward the speaker.
	Can be glossed simply as ‘here’, but can also be \xx{cis} for ‘cislocative’.
	This preverb never occurs bare, instead always appearing as one of the allomorphs listed below
		which arise from the motion derivation
		\motderiv{NP-t/x̱/dé}{∅, \fm{-μμL} rep}{arriving at NP}.
	See the entries of each allomorph for more detail.
	
	Probably connected historically to first person plural object pronoun \X[haa=1plo]{haa=} ‘us’
		and possessive pronoun \fm{haa} ‘our’,
		but the relationship is not clear
		and there is no meaningful connection between them in the modern language.
	The \fm{…háan} element in the first person plural independent pronoun \fm{uháan} ‘us’
		and the second person plural independent pronoun \fm{yeewháan} ‘you pl.’\
		is likely also related, but this is even less clear.
	Another possible historical relationship is with the root \fm{\rt{han}} ‘sg.\ stand’
		that is probably related to \fm{ḵwáan} ‘people (of a place)’
		as seen by its archaic form \fm{ḵuháan} \parencite[01/67]{leer:1973}.
	\newline
	Allomorphs:
	\begin{allolist}
	\item[\X{haat=}]		allomorph with punctual postposition \fm{-t} ‘to, arriving at’
	\item[\X{haax̱=}]		allomorph with pertingent postposition \fm{-x̱} ‘at, contacting’
	\item[\X{haadé=}]	allomorph with allative postposition \fm{-dé} ‘toward’
	\item[\X{haandé=}]	allomorph with allative postposition \fm{-dé} ‘toward’
				and unexpected \fm{n}
	\end{allolist}
	\begin{enumerate}
	\item	In motion verbs with the motion derivation
			\motderiv{NP-t/x̱/dé}{∅, \fm{-μμL} rep}{arriving at NP}
		where the NP is this preverb \fm{haa=} ‘here’.
		\begin{itemize}
		\item	\vbform{haat uwagút}{pfv}[subj intr, \fm{∅}, mot]{she/he/it came here}
				\vbmorph{\gm{haa}&-t=&u-&wa-&\rt[¹]{gut}&-μH}
					{here&\·\xx{pnct}&\xx{zpfv}&\xx{stv}&\rt[¹]{go.\xx{sg}}&\·\xx{var}}
			\versus \vbform{haax̱ goot}{rep impfv}{she/he/it repeatedly comes here}
				\vbmorph{\gm{haa}&-x̱=&\rt[¹]{gut}&-μμL}
					{here&\·\xx{pert}&\rt[¹]{go.\xx{sg}}&\·\xx{var}}
			\versus \vbform{haandé kg̱wagóot}{prosp}{she/he/it will come here}
				\vbmorph{\gm{haan}&-dé=&k-&ʷ-&g̱a-&\rt[¹]{gut}&-μμH}
					{here&-\xx{all}=&\xx{gcnj}&\xx{irr}&\xx{mod}&\rt[¹]{go.\xx{sg}}&\·\xx{var}}
			\exor \vbform{haadé kg̱wagóot}{prosp}{she/he/it will come here}
				\vbmorph{\gm{haa}&-dé=&k-&ʷ-&g̱a-&\rt[¹]{gut}&-μμH}
					{here&-\xx{all}=&\xx{gcnj}&\xx{irr}&\xx{mod}&\rt[¹]{go.\xx{sg}}&\·\xx{var}}
		\end{itemize}
	\item	In a few other contexts where it occurs in interjections and
		postposition phrases that are not bound to a motion verb.
		\begin{itemize}
		\item	\fm{haahée} \~\ \fm{haahéi} \~\ \fm{háay} ‘give it here’
			\parencite[01/49]{leer:1973}
			with unknown \fm{…hée} \~\ \fm{…héi} \~\ \fm{…y}
			perhaps related to the mesio-proximal deictic element
				\fm{hé} ‘over here (near speaker)’
		\item	\fm{shkʼáay} ‘give it here’
			\parencite[01/49]{leer:1973}
			from \fm{shkʼé} and \fm{háay} above
		\item	\fm{haandé i jín!} ‘shake hands!’ lit.\ ‘hither, your hand!’
			\parencites[01/48]{leer:1973}[146.997]{nyman-leer:1993}[10]{dauenhauer-dauenhauer:2002}
		\item	\fm{shkʼaandé} ‘give it here’
			\parencite[01/48]{leer:1973}
			from \fm{shʼké} and \fm{haandé}
		\end{itemize}
	\end{enumerate}

\item[haadé=]\label{m:haadé=}
	Allomorph of the directional preverb \X{haa=} ‘here, hither’
		with the allative postposition \fm{-dé} ‘to, toward’.
	Derived from \fm{haa=} and the motion derivation
		\motderiv{NP-t/x̱/dé}{∅, \fm{-μμL} rep}{arriving at NP}
		with the allative postposition used for prospective and progressive aspect forms.
	This is a variant of the more conservative form \X{haandé=}.
	\begin{enumerate}
	\item	In motion verbs with the motion derivation
			\motderiv{haat= / haax̱= / haa(n)dé=}{∅, \fm{-μμL} rep}{arriving at NP}
			when the aspect is either prospective or progressive.
		\begin{itemize}
		\item	\vbform{haadé yaa naḵúx̱}{prog}[subj intr, \fm{n}, mot]{he/she is coming here (by boat); it (boat) is coming here}
			\parencite[23.117]{story-naish:1973}[102.403, 102.411, 118.188]{dauenhauer-dauenhauer:1987}
				\vbmorph{\gm{haa}&\gm{-dé=}&ÿaa=&na-&\rt[¹]{ḵux̱}&-μH}
					{here&\·\xx{all}&along&\xx{ncnj}&\rt[¹]{go.boat}&\·\xx{var}}
			\versus \vbform{haadé kg̱waḵóox̱}{prosp}{he/she will come here (by boat); it (boat) will come here}
			\parencite[118.181]{dauenhauer-dauenhauer:1987}
				\vbmorph{\gm{haa}&\gm{-dé=}&k-&ʷ-&g̱a-&\rt[¹]{ḵux̱}&-μμH}
					{here&\·\xx{all}&\xx{gcnj}&\xx{irr}&\xx{mod}&\rt[¹]{go.boat}&\·\xx{var}}
		\end{itemize}
	\item	In other verbs with the motion derivation
			\motderiv{haat= / haax̱= / haa(n)dé=}{∅, \fm{-μμL} rep}{arriving at NP}
			when the aspect is either prospective or progressive.
		\begin{itemize}
		\item	\vbform{Tlákw.aan ḵwáani haadé s akwg̱alʼeix̱}{prosp}[sub intr, \fm{n}, inv act]{The Klukwan people are coming here to dance}
			\parencite[87.1084]{story-naish:1973}
				\vbmorph{\gm{haa}&\gm{-dé=}&s=&a-&k-&ʷ-&g̱a-&\rt[²]{lʼex̱}&-μμH}
					{here&\·\xx{all}&\xx{plh}&\xx{xpl}&\xx{gcnj}&\xx{irr}&\xx{mod}&\rt[²]{dance}&\·\xx{var}}
		\end{itemize}
	\item	As a postposition phrase in other contexts.
		\begin{itemize}
		\item	\fm{Haadé wé atshéeÿi.} ‘Come here, those singers.’
			\parencite[270.10]{swanton:1909}
		\end{itemize}
	\end{enumerate}

\item[haandé=]\label{m:haandé=}
	Allomorph of the directional preverb \X{haa=} ‘here, hither’
		with the allative postposition \fm{-dé} ‘to, toward’.
	Same as \X{haadé=} but with an additional \fm{n};
		speakers generally prefer one or the other of these two forms,
		but will always accept both.
	As with \X{haadé=}, \fm{haandé=} is derived from \fm{haa=} and the motion derivation
		\motderiv{haat= / haax̱= / haa(n)dé=}{∅, \fm{-μμL} rep}{arriving at NP}
		with the allative postposition used for prospective and progressive aspect forms.

	The origin of the \fm{n} is unclear.
	It may be excrescent, arising as a prenasalization of \fm{d} that is then perceived as
		a separate consonant in the coda of the preceding syllable.
	It may however be etymological, with the \fm{haadé=} form arising from
		leveling and regularization of the \fm{haat=} / \fm{haax̱=} / \fm{haadé=} paradigm.
	Compare the \fm{…haan} in \fm{uháan} ‘us, first person plural’
		and \fm{yeewháan} ’you guys, second person plural’, 
		though the origin and meaning of this element are both unknown.
	\begin{enumerate}
	\item	In motion verbs with the motion derivation
			\motderiv{haat= / haax̱= / haa(n)dé=}{∅, \fm{-μμL} rep}{arriving at NP}
			when the aspect is either prospective or progressive.
		\begin{itemize}
		\item	\vbform{Haandé kḵwagóot wé i éekʼ}{prog}[subj intr, \fm{∅}, mot]{your brother is coming here}
			\parencite[178.246]{dauenhauer-dauenhauer:1987}
				\vbmorph{\gm{haan}&\gm{-dé}=&k-&ʷ-&g̱a-&\rt[¹]{gut}&-μμH}
					{here&\·\xx{all}&\xx{gcnj}&\·\xx{irr}&\xx{mod}&\rt[¹]{go.\xx{sg}}
		\end{itemize}
	\item	In other verbs with the motion derivation
			\motderiv{haat= / haax̱= / haa(n)dé=}{∅, \fm{-μμL} rep}{arriving at NP}
			when the aspect is either prospective or progressive.
		\begin{itemize}
		\item	
		\end{itemize}
	\item	As a postposition phrase in other contexts.
		\begin{itemize}
		\item	\fm{haandé i jín!} ‘shake hands!’ lit.\ ‘hither, your hand!’
			\parencites[01/48]{leer:1973}[146.997]{nyman-leer:1993}[10]{dauenhauer-dauenhauer:2002}
		\item	\fm{shkʼaandé} ‘give it here’
			\parencite[01/48]{leer:1973}
			from \fm{shʼké} and \fm{haandé}
		\item	\fm{Ha, haandé déi!} ‘Well, come over now!’
			\parencite[118.180]{dauenhauer-dauenhauer:1987}
		\end{itemize}
	\end{enumerate}

\item[haat=]\label{m:haat=}

\item[haax̱=]\label{m:haat=}

\item[has=]\label{m:has=}
	human pluralizer for third person;
	\newline
	allomorphs:
	\begin{allolist}
	\item[as=]	onset glottal stop instead of \fm{h} used in Southern and Tongass varieties
	\item[s=]	lone consonant, usually coda of a preceding syllable
	\end{allolist}
	note that since Tlingit is number neutral (nouns are not singular by default),
		a form without \fm{has=} may still refer to plural humans,
		i.e.\ \fm{has=} is not required for third person human plural arguments
	\begin{enumerate}
	\item	human pluralizer for third person subject
		\begin{itemize}
		\item	\fm{tʼá aawax̱áa} (pfv; tr, \fm{∅}, \fm{-μH} act) ‘s/he/it ate king salmon’\newline
			versus \fm{tʼá has aawax̱áa} (pfv) ‘they (humans) ate king salmon’
		\end{itemize}
	\item	human pluralizer for third person object
		\begin{itemize}
		\item	\fm{has tushikʼáan} (impfv; tr, \fm{g}, \fm{-μμH} state) ‘we hate them’\newline
			versus \fm{yee tushikʼáan} (impfv) ‘we hate you guys’
		\end{itemize}
	\item	human pluralizer for both third person subject and third person object;
		some speakers do not accept this use of \fm{has=}
			for both subject and object at the same time
		\begin{itemize}
		\item	\fm{has awsiteen} (pfv; tr, \fm{g}, ach) ‘they saw them’
			or ‘s/he/it saw them’ or ‘they saw him/her/it’
		\end{itemize}
	\end{enumerate}

\item[héeni=]\label{m:héeni=}
	Locational preverb ‘in water’ indicating location in a body of water,
		usually as a result of movement from dry land.
	Derived from the noun \fm{héen} ‘fresh water; river’
		with the special locative postposition allomorph
		\X[-i-loc]{-i} \~\ \X[-í-loc]{-í} ‘at’
		(instead of the regular allomorphs \fm{-xʼ} and \X[-μ-loc]{-μ}).
	This locative allomorph is unique in that it only occurs with a handful of preverbs,
		for which see the detailed entry of \X[-í-loc]{-í}.
	The noun \fm{héen} normally only refers to either (a) fresh water
		as opposed to \fm{éilʼ} ‘salt; salt water, ocean’
		or (b) a river or other flowing body of water
		as opposed to \fm{áa} ‘lake’,
		but the preverb \fm{héeni=} is used more broadly
		for both salt water and fresh water
		as well as for flowing water bodies and static water bodies.
	As with other preverbs including \X[-i-loc]{-i}, \fm{héeni=} can be represented either
		fully segmented as \fm{héen-i=}
		or unsegmented as \fm{héeni=}.
	The preverb \fm{héeni=} is easily confused with the homophonous \fm{héen-i} ‘river-\xx{poss}’
		which is a possessed noun phrase; this possessed noun phrase will always have a
		possessor – e.g. \fm{x̱áat héen-i} ‘salmon river-\xx{poss}’ – whereas the preverb
		does not have a possessor.
	The form \fm{héèni=} [\ipa{hîː.nì}] is predicted in Southern Tlingit varieties
		and \fm{heéni=} [\ipa{hiːˀ.niʰ}] in Tongass Tlingit.
	\begin{enumerate}
	\item	In forms with \fm{∅} conjugation class that reflect the motion derivation
			\motderiv{héeni}{∅, \fm{-x̱} rep}{in water}.
		\begin{itemize}
		\item	\vbform{héeni yoo kax̱wlitséx̱}{pfv}[tr, \fm{∅}, mot]{I trampled on them in the water (to get them clean)}
			\parencite[233.3303]{story-naish:1973}
				\vbmorph{\gm{héen}&\gm{-i=}&yoo=&ka-&ʷ-&x̱-&l-&i-&\rt[²]{tsex̱}&-μH}
					{water&\·\xx{loc}&\xx{alt}&\xx{qual}&\xx{pfv}&\xx{1sg.s}&\xx{xtn}&\xx{stv}&\rt[²]{kick}&\·\xx{var}}
		\item	\vbform{héeni wugoodí}{sub pfv}[subj intr, \fm{∅}, mot]{as he stepped into the water}
			\parencite[86.83]{dauenhauer-dauenhauer:1987}
				\vbmorph{\gm{héen}&\gm{-i=}&wu-&\rt[¹]{gut}&-μμL&-í}
					{water&\·\xx{loc}&\xx{pfv}&\rt[¹]{go.\xx{sg}}&\·\xx{var}&\·\xx{sub}}
		\end{itemize}
	\item	In forms with \fm{∅} conjugation class
			and some form of \X{ÿeiḵ=} \~\ \X{eiḵ=} \~\ \X{eeḵ=} ‘beach’
			that reflect the motion derivation
			\motderiv{héeni ÿeiḵ}{∅, \fm{-ch} rep}{from shore into water}.
		\begin{itemize}
		\item	\vbform{héeni eeḵ has atáan}{csec}[tr, \fm{∅}, mot]{them having brought it down into the water}
			\parencite[14.115]{nyman-leer:1993}
				\vbmorph{\gm{héen}&\gm{-i=}&eeḵ=&has=&a-&\rt[²]{tan}&-μμH&}
					{water&\·\xx{loc}&beach&\xx{plh}&\xx{3>3}&\rt[²]{handle.w/ec}&\·\xx{var}&\xx{sub}}
		\end{itemize}
	\item	In forms with something other than \fm{∅} conjugation class.
		These may reflect undocumented motion derivations that use \fm{héeni=} with
			other conjugation classes or they may reflect one of the motion
			derivations above with an additional motion derivation that changes
			the conjugation class from \fm{∅} to something else.
		Alternatively, these may be cases where \fm{héeni=} is used without a motion derivation,
			in which case \fm{héeni=} is probably a kind of adjunct postposition phrase.
		\begin{itemize}
		\item	\vbform{héeni kawdax̱dudliyaa}{pfv}[tr, \fm{g̱}, ach]{people lowered each into the sea}
			\parencite[96.300]{dauenhauer-dauenhauer:1987}
				\vbmorph{\gm{héen}&\gm{-i=}&ka-&w-&dax̱-&du-&d-&l-&i-&\rt[²]{ÿa}&-μμL}
					{water&\·\xx{loc}&\xx{qual}&\xx{pfv}&\xx{distb}&\xx{ind.h.s}&\xx{mid}&\xx{xtn}&\xx{stv}&\rt[²]{lower}&\·\xx{var}}
		\item	\vbform{héeni wtuwa.aat}{pfv}[subj intr, \fm{g}?, mot]{we waded ashore}
			\parencite[290.562]{dauenhauer-dauenhauer:1987}
				\vbmorph{\gm{héen}&\gm{-i=}&w-&tu-&wa-&\rt[¹]{.at}&-μμL}
					{water&\·\xx{loc}&\xx{pfv}&\xx{1pl.s}&\xx{stv}&\rt[¹]{go.\xx{pl}}&\·\xx{var}}
			\newline
			Also \vbform{áxʼ héeni aawa.aat}{pfv}{people waded ashore there} (line 565)
				and \vbform{áxʼ héeni has woo.aat}{pfv}{they waded ashore there} (line 569)
				in the same text.
			All three are translated as ‘ashore’ although it looks like they would mean
				‘into water’.
		\end{itemize}
	\end{enumerate}

\item[héenx̱=]\label{m:héenx̱=}
	Locational preverb ‘in water’ indicating a location in a body of water, usually as a result
		of movement from dry land.
	Derived from the noun \fm{héen} ‘fresh water; river’ with the pertingent postposition
		\fm{-x̱} ‘at, contacting, part/made of’.
	The noun \fm{héen} normally only refers to either (a) fresh water
		as opposed to \fm{éilʼ} ‘salt; salt water, ocean’
		or (b) a river or other flowing body of water
		as opposed to \fm{áa} ‘lake’,
		but the preverb \fm{héenx̱=} can be used more broadly
		for both salt water and fresh water
		as well as for flowing water bodies and static water bodies.
	As with other preverbs derived from postposition phrases, \fm{héenx̱=} can be represented either
		fully segmented as \fm{héen-x̱=}
		or unsegmented as \fm{héenx̱=}.
	If segmented, the pertingent postposition \fm{-x̱} can be glossed as \xx{pert}
		or more transparently as ‘in’ or ‘on’ depending on context.

	The distinction between preverb and ordinary postposition phrase is hazy in this case.
	Some instances of \fm{héenx̱} are clearly just an ordinary postposition phrase where
		\fm{héen} occurs in the place of some other noun phrase.
	In these cases the postposition phrase \fm{héenx̱} may be an argument or an adjunct
		depending on various factors, and \fm{héenx̱} may be moved around to other
		places in the sentence besides immediately preceding the verb word.
	But when \fm{héenx̱} is required as part of a motion derivation it behaves more like a preverb
		in that its position is restricted to immediately before the verb word
		and it may not have a clearly referential meaning,
		for example ‘in water’ instead of ‘in the water’ or ‘in the river’.
	For brevity only cases of \fm{héenx̱} with motion derivations are listed here.
	Even in motion derivation cases the distinction is still unclear because its appearance with
		some verbs is more lexicalized like other preverbs but with some verbs it seems to
		be less predictable and so has a more transparent interpretation.
	\begin{enumerate}
	\item	In forms with \fm{g̱} conjugation class that reflect the motion derivation
			\motderiv{NP-x̱}{g̱, \fm{yei=…-ch} rep}{down into/onto NP}.
		\begin{itemize}
		\item	\vbform{chʼu tle héenx̱ has akawu̬sikei}{pfv}[tr, \fm{g̱}, \fm{yei=…-ch} rep]{they followed them into the water}
			\parencite[312.29]{swanton:1909}
				\vbmorph{\gm{héen}&\gm{-x̱=}&has=&a-&ka-&wu-&s-&i-&\rt[¹]{ke}&-μμL}
					{water&\·in&\xx{plh}&3>3&\xx{qual}&\xx{pfv}&\xx{csv}&\xx{stv}&\rt[¹]{unravel}&\·\xx{var}}
		\end{itemize}
	\end{enumerate}
\end{morphdesc}

