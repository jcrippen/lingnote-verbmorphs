%!TEX root = ../lingnote-verbmorphs.tex

\subsection{O}\label{sec:alphalist-o}
\begin{morphdesc}[resume*=alphalist]
\item[oo-]\label{m:oo-}
	allomorph of irrealis \fm{u-}
	
\item[oo]\label{m:oo}
	≡ \fm{a-u-}
	combination of argument marking \X{a-}
		and either irrealis \X[u-irr]{u-}
			or \fm{∅} conjugation perfective \X[u-pfv]{u-}

\item[oowa]\label{m:oowa}
	≡ \fm{a-u-wa-}
	combination of argument marking \X{a-}
		and irrealis \X[u-irr]{u-}
		and stative \X{wa-};
	unlike \X[eeÿa-a-i-ÿa]{eeÿa} ≡ \fm{a-i-ÿa} versus \X[eeÿa-a-ʷ-i-ÿa]{eeÿa} ≡ \fm{a-ʷ-i-ÿa},
		the combination \fm{oowa} cannot be perfective
		because \fm{∅} conjugation perfective \X[u-pfv]{u-}
			cannot occur with both argument marking \X{a-}
			and stative \X{wa-} at the same time
		and with perfective \X{wu-} (instead of \X[u-pfv]{u-})
			the form is always \X{aawa} never \fm{oowa}
	\begin{itemize}
	\item	\vbform{yéi oowajée}{impfv}[tr, \fm{n}, \fm{-μμH} state]{she/he/it thinks so about him/her/it}
			\vbmorph{yéi=&\gm{a-}&\gm{u-}&\gm{wa-}&\rt[²]{jiᴸ}&-μμH}
				{thus&\xx{3>3}&\xx{irr}&\xx{stv}&\rt[²]{think}&\·\xx{var}}
		\versus \vbform{yéi aawajee}{pfv}{she/he/it thought so about him/her/it}
			\vbmorph{yéi=&a-&ʷ-&μʷ-&wa-&\rt[²]{jiᴸ}&-μμL}
				{thus&\xx{3>3}&\xx{irr}&\xx{pfv}&\xx{stv}&\rt[²]{think}&\·\xx{var}}
	\end{itemize}

\item[oox̱]\label{m:oox̱}
	≡ \fm{a-u-x̱-}
	combination of argument marking \X{a-}
		and either irrealis \X[u-irr]{u-}
			or \fm{∅} conjugation perfective \X[u-pfv]{u-}
		and either \fm{g̱} conjugation \X[x̱-g̱cnj]{x̱-}
			or modal \X[x̱-mod]{x̱-}
			or first person singular subject \X[x̱-1sg]{x̱-}
\end{morphdesc}
