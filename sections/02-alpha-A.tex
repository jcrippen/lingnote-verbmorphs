%!TEX root = ../lingnote-verbmorphs.tex

\subsection{A}\label{sec:alphalist-a}
\raggedright
\begin{morphdesc}[series=alphalist]
\item[a-]\label{m:a-}
	argument marking prefix in the same position as object prefixes/proclitics;
	\newline
	allomorphs:
	\begin{allolist}
	\item[\X{ⱥ-}]	symbol used in glosses indicating absence of expected \fm{a-}
			due to preceding ergative \fm{-ch}
	\end{allolist}
	combinations:
	\begin{allolist}
	\item[\X{aawa}]	≡ \fm{a-μʷ-wa-} with perfective \X{μʷ-} and stative \X{wa-}
	\item[\X{am}]	≡ \fm{a-m-} with perfective \X{m-}
	\item[\X{aw}]	≡ \fm{a-w-} with perfective \X[w-pfv]{w-}
	\item[\X{awu}]	≡ \fm{a-wu-} with perfective \X{wu-}
	\item[\X{ax̱}]	≡ \fm{a-x̱-} with first person singular subject \X[x̱-1sg]{x̱-}
				or \fm{g̱} conjugation \X[x̱-g̱cnj]{x̱-}
				or modal \X[x̱-mod]{x̱-}
	\item[{\X[aÿ-a-ÿ]{aÿ}}]
			≡ \fm{a-ÿ-} with second person plural subject \X[ÿ-2pl]{ÿ-}
	\item[{\X[aÿ-a-ʷ-ÿ]{aÿ}}]
			≡ \fm{a-ʷ-ÿ-} with perfective \X[ʷ-pfv]{ʷ-}
				and second person singular subject \X[ÿ-2sg]{ÿ-}
	\item[\X{ee}]	≡ \fm{a-i-} with second person singular subject \X[i-2sg]{i-}
	\item[{\X[eeÿa-a-i-ÿa]{eeÿa}}]
			≡ \fm{a-i-ÿa-} with second person singular subject \X[i-2sg]{i-}
				and stative \X[ÿa-stv]{ÿa-}
	\item[{\X[eeÿa-a-ʷ-i-ÿa]{eeÿa}}]
			≡ \fm{a-ʷ-i-ÿa-} with perfective \X[ʷ-pfv]{ʷ-}
				and second person singular subject \X[i-2sg]{i-}
				and stative \X[ÿa-stv]{ÿa-}
	\item[\X{oo}]	≡ \fm{a-u-} with irrealis \X[u-irr]{u-}
				or \fm{∅} conj perfective \X[u-pfv]{u-}
	\item[\X{oowa}]	≡ \fm{a-u-wa-} with irrealis \X[u-irr]{u-}
				and stative \X{wa-}
	\item[\X{oox̱}]	≡ \fm{a-u-x̱-} with irrealis \X[u-irr]{u-}
				and first person singular subject \X[x̱-1sg]{x̱-}
				or \fm{g̱} conjugation \X[x̱-g̱cnj]{x̱-}
				or modal \X[x̱-mod]{x̱-}
	\end{allolist}
	\begin{enumerate}
	\item\label{m:a-3>3}
		3>3 agreement of transitive verb: indicates existence of third person subject
		and third person object (regardless of whether these are actually spoken)
		\begin{itemize}
		\item	\vbform{aawax̱áa}{pfv}[tr, \fm{∅}, \fm{-μH} act]{she/he/it ate him/her/it}
			\vbmorph{\gm{a-}&μʷ-&wa-&\rt[²]{x̱a}&-μμH}
				{\xx{3>3}&\xx{pfv}&\xx{stv}&\rt[²]{eat}&\·\xx{var}}
			\versus \vbform{wutuwax̱áa}{pfv}{we ate it}
			\vbmorph{wu-&tu-&wa-&\rt[²]{x̱a}&-μμH}
				{\xx{pfv}&\xx{1pl.s}&\xx{stv}&\rt[²]{eat}&\·\xx{var}}
		\end{itemize}
	\item\label{m:a-ind.h.o}
		indefinite nonhuman object of some transitive verbs
			instead of \X{at=};
		verbs that use \fm{a-} in this way usually also use \fm{a-} 3>3 (\ref{m:a-3>3})
			and the reason for using \fm{a-} instead of \fm{at=} is still unclear;
		this may be the same as the expletive \fm{a-} (\ref{m:a-xpl}) differing only
			in conventional translation;
		this may instead be the equivalent of a cognate object structure as in English
			“he danced a dance” and “she smiled a smile”
			(compare \fm{alʼeix̱} ‘s/he is dancing’)
		\begin{itemize}
		\item	\vbform{alʼóon}{impfv}[tr, \fm{n}, \fm{-μμH} act]{s/he/it is hunting something}
			\vbmorph{\gm{a-}&\rt[²]{lʼuᴴn}&-μμH}
				{\xx{ind.n.o}&\rt[²]{hunt}&\·\xx{var}}
			\andnot{	\vbform[*]{at lʼóon}{impfv}{s/he/it is hunting something}
				using \fm{at=}}
			\versus \vbform{alʼóon}{impfv}{s/he/it is hunting it}
			\vbmorph{a-&\rt[²]{lʼuᴴn}&-μμH}
				{\xx{3>3}&\rt[²]{hunt}&\·\xx{var}}
			\notealso{compare \vbform{g̱áx̱ alʼóon}{impfv}{s/he/it is hunting rabbits}}
			\notealso{(\fm{g̱áx̱ alʼóon} cannot mean
				‘s/he/it is hunting something rabbits’)}
		\end{itemize}
	\item\label{m:a-ind.h.s}
		indefinite human subject (not object!)\ of subject intransitive verbs
		instead of \X{du-};
		possibly all subject intransitive verbs use \fm{a-} rather than \fm{du-}?
		\begin{itemize}
		\item	\vbform{ax̱éxʼw}{impfv}[subj intr, \fm{n}, \fm{-μH} act]{people are sleeping}
				\vbmorph{\gm{a-}&\rt[¹]{x̱exʼw}&-μH}
					{\xx{ind.h.s}&\rt[¹]{sleep·\xx{pl}}&\·\xx{var}}
				\andnot{\fm[*]{dux̱éxʼw} ‘people are sleeping’
					using \fm{du-} \xx{ind.n.s} ‘people’}
			\versus \vbform{toox̱éxʼw}{impfv}{we are sleeping}
				\vbmorph{too- \rt[¹]{x̱exʼw} -μH}
					{\xx{1pl·s} \rt[¹]{sleep·\xx{pl}} \·\xx{var}}
		\item	\vbform{aawa.aat}{pfv}[subj intr, \fm{n}, mot]{people went}
			\vbmorph{\gm{a-}&μʷ-&wa-&\rt[¹]{.at}&-μμL}
				{\xx{ind.h.s}&\xx{pfv}&\xx{stv}&\rt[¹]{go·\xx{pl}}&\·\xx{var}}
			\andnot{\fm[*]{wuduwa.aat} ‘people went’ 
				using \fm{du-} \xx{ind.h.s} ‘someone, people’}
			\versus \vbform{wutuwa.aat}{pfv}{we went}
				\vbmorph{wu- &tu-&wa- &\rt[¹]{.at}&-μμL}
					{\xx{pfv}&\xx{1pl.s}&\xx{stv}&\rt[¹]{go·\xx{pl}}&\·\xx{var}}
		\end{itemize}
	\item\label{m:a-xpl}
		nonreferential expletive (filler) object, does not refer to anything;
		it is unclear why this is used instead of suppressing the object with
			antipassive voice \X{d-}, although in some cases \fm{d-} might
			already be in use to suppress the subject as passive voice
		\begin{itemize}
		\item	\vbform{awdigaan}{pfv}[impers, \fm{g̱}, ach]{it sunshined}
			\vbmorph{\gm{a-}&w-&d-&i-&\rt[¹]{gan}&-μμL}
				{\xx{xpl}&\xx{pfv}&\xx{mid}&\xx{stv}&\rt[²]{burn}&\·\xx{var}}
				\andnot{\fm[*]{g̱agaan awdigaan} ‘sun sunshined’: no argument allowed}
			\versus \vbform{sʼeenáa kawdigán}{pfv}[obj intr, \fm{∅}, ach]{the lamp shone, lit up}
				\vbmorph{sʼeenáa&ka-&w-&d-&i-&\rt[¹]{gan}&-μH}
					{lamp&\xx{hsfc}&\xx{pfv}&\xx{mid}&\xx{stv}&\rt[¹]{burn}&\·\xx{var}}
				\notealso{(object \fm{sʼeenáa} ‘lamp’ allowed)}
		\end{itemize}
	\end{enumerate}

\item[ⱥ-]\label{m:ⱥ-}
	allomorph of 3>3 agreement \X{a-} specifically when preceded by ergative \fm{-ch};
	this is not actually a real morpheme but is used in glosses to indicate that
		an otherwise expected \fm{a-} is absent
		due to the presence of ergative \fm{-ch} immediately preceding the verb;
	certain elements (some preverbs, focus particles) are transparent to \fm{-ch … ⱥ-}
		but other things (NPs, PPs, adverbs) block the effect
	\begin{itemize}
	\item	\vbform{ax̱ x̱úx̱ch uwa.ún}{pfv}[tr, \fm{∅}, ach]{my husband shot him/her/it}
		\parencite[183.303]{nyman-leer:1993}
			\vbmorph{ax̱&x̱úx̱&-ch&\gm{ⱥ-}&u-&wa-&\rt[²]{.uᴴn}&-μH}
				{\xx{1sg.psr}&husband&\·\xx{erg}&\xx{3>3}&\xx{zpfv}&\xx{stv}&\rt[²]{shoot}&\·\xx{var}}
		\versus \vbform{ax̱ x̱úx̱ aawa.ún}{pfv}{she/he/it shot my husband}
			\vbmorph{ax̱&x̱úx̱&a-&μʷ-&wa-&\rt[²]{.uᴴn}&-μH}
				{\xx{1sg.psr}&husband&\xx{3>3}&\xx{pfv}&\xx{stv}&\rt[²]{shoot}&\·\xx{var}}
	\item	\vbform{Lingítch áwé yéi uwasáa}{pfv}[tr, \fm{∅}, ach]{it was Tlingits that named it so}
		\parencite[144.128]{dauenhauer-dauenhauer:1987}
			\vbmorph{Lingít&-ch&áwé&yéi=&\gm{ⱥ-}&u-&wa-&\rt[²]{sa}&-μμH}
				{Tlingit&\·\xx{erg}&\xx{foc}&thus&\xx{3>3}&\xx{zpfv}&\xx{stv}&\rt[²]{name}&\·\xx{var}}
		\versus \vbform{aawasáa}{pfv}{she/he/it named him/her/it so}
			\vbmorph{yéi=&ⱥ-&u-&wa-&\rt[²]{sa}&-μμH}
				{thus&\xx{3>3}&\xx{zpfv}&\xx{stv}&\rt[²]{name}&\·\xx{var}}		\item	\vbform{x̱áat ḵwáanich ásgíyú wusineex̱}{pfv}[tr, \fm{g̱}, ach]{salmon people apparently rescued him}
		\parencite[312.38]{swanton:1909}
			\vbmorph{x̱áat&ḵwáan&-i&-ch&ásgíyú&\gm{ⱥ-}&wu-&s-&i-&\rt[¹]{nix̱}&-μμL}
				{salmon&people&\xx{poss}&\xx{erg}&\xx{foc}&\xx{3>3}&\xx{pfv}&\xx{csv}&\xx{stv}&\rt[¹]{safe}&\·\xx{var}}
		\versus \vbform{awsineex̱}{pfv}{she/he/it rescued him/her/it}
			\vbmorph{a-&w-&s-&i-&\rt[¹]{nix̱}&-μμL}
				{\xx{3>3}&\xx{pfv}&\xx{csv}&\xx{stv}&\rt[¹]{safe}&\·\xx{var}}
	\end{itemize}

\item[aa=]\label{m:aa=}
	partitive proclitic ‘one of, some of’;
	can apply to either singular or plural quantity of the referent depending on context;
	derived from the independent partitive pronoun \fm{aa} ‘one, some’
		but distinguished from it by position because unlike independent pronouns
		the proclitic can occur between preverbs and verb
	\begin{enumerate}
	\item	partitive third person object of transitive verb
		\begin{itemize}
		\item	\vbform{aa wutusi.ée}{pfv}[tr, \fm{∅}, \fm{-μμH} act]{we cooked one/some of them}
				\vbmorph{\gm{aa=}&wu-&tu-&s-&i-&\rt[¹]{.i}&-μμH}
					{\xx{part.o}&\xx{pfv}&\xx{1pl.s}&\xx{csv}&\xx{stv}&\rt[¹]{cooked}&\·\xx{var}}
			\versus \vbform{wutusi.ée}{pfv}{we cooked him/her/it/them}
				\vbmorph{wu-&tu-&s-&i-&\rt[¹]{.i}&-μμH}
					{\xx{pfv}&\xx{1pl.s}&\xx{csv}&\xx{stv}&\rt[¹]{cooked}&\·\xx{var}}
		\end{itemize}
	\item	partitive third person object of object intransitive verb
		\begin{itemize}
		\item	\vbform{yéi aa yatee}{impfv}[obj intr, \fm{n}, \fm{-μμL} state]{one is/some are that way}
				\vbmorph{yéi=&\gm{aa=}&ÿa-&\rt[¹]{tiᴸ}&-μμL}
					{thus&\xx{part.o}&\xx{stv}&\rt[¹]{be}&\·\xx{var}}
			\versus \vbform{yéi yatee}{impfv}{she/he/it is that way}
				\vbmorph{yéi=&ÿa-&\rt[¹]{tiᴸ}&-μμL}
					{thus&\xx{stv}&\rt[¹]{be}&\·\xx{var}}
		\end{itemize}
	\item	partitive third person subject of subject intransitive verb
		\begin{itemize}
		\item	\vbform{aa woo.aat}{pfv}[subj intr, \fm{n}, mot]{some of them went}
				\vbmorph{\gm{aa=}&wu-&μ-&\rt[ˢ]{.at}&-μμL}
					{\xx{part.s}&\xx{pfv}&\xx{stv}&\rt[ˢ]{go·\xx{pl}}&\·\xx{var}}
			\versus \vbform{woo.aat}{pfv}[subj intr, \fm{n}, mot]{they went}
				\vbmorph{wu-&μ-&\rt[ˢ]{.at}&-μμL}
					{\xx{pfv}&\xx{stv}&\rt[ˢ]{go·\xx{pl}}&\·\xx{var}}
		\item	\vbform{aa woonook}{pfv}[subj intr, \fm{g̱}, mot]{one of them sat down}
				\vbmorph{\gm{aa=}&wu-&μ-&\rt[ˢ]{nuk}&-μμL}
					{\xx{part.s}&\xx{pfv}&\xx{stv}&\rt[ˢ]{sit·\xx{sg}}&\·\xx{var}}
			\versus \vbform{woonook}{pfv}{she/he/it sat down}
				\vbmorph{wu-&μ-&\rt[ˢ]{nuk}&-μμL}
					{\xx{pfv}&\xx{stv}&\rt[ˢ]{sit·\xx{sg}}&\·\xx{var}}
		\end{itemize}
	\end{enumerate}

\item[-a]\label{m:-a}
	allomorph of instrument suffix \X{-aa} \~\ \X{-áa};
	occurs between an H tone syllable and \fm{-ÿi}
		so that \fm{-aa} becomes a short vowel
	\begin{itemize}
	\item	\vbform{koolitsʼígwayi át}{rel impfv}[obj intr, \fm{g}?, inv state]{thing which is a delicate, touchy issue}
			\vbmorph{ka-&u-&lˢ-&i-&\rt{tsʼikw}&-μH&\gm{-a}&-ÿi&át}
				{\xx{qual}&\xx{irr}&\xx{xtn}&\xx{stv}&\rt{delicate}&\·\xx{var}&\·\xx{inst}&\·\xx{rel}&thing}
		\versus \vbform{koolitsʼígwaa}{impfv}{she/he/it is a delicate, touchy issue}	
			\vbmorph{ka-&u-&lˢ-&i-&\rt{tsʼikw}&-μH&-aa}
				{\xx{qual}&\xx{irr}&\xx{xtn}&\xx{stv}&\rt{delicate}&\·\xx{var}&\·\xx{inst}}
	\item	predicted to occur with subordinate \X[-ÿi-sub]{-ÿi}
		but no examples
	\item	\fm{ax̱ gúxʼayi} ‘my cup, can, dipper’
			\vbmorph{ax̱&\rt{guxʼ}&-μH&\gm{-a}&-ÿi}
				{\xx{1sg.psr}&\rt{dip}&\·\xx{var}&\·\xx{inst}&\·\xx{poss}}
		\versus \fm{gúxʼaa} ‘cup, can, dipper’
			\vbmorph{\rt{guxʼ}&-μL&-aa}
				{\rt{dip}&\·\xx{var}&\·\xx{inst}}
	\end{itemize}

\item[-á]\label{m:-á}
	allomorph of instrument suffix \X{-aa} \~\ \X{-áa};
	occurs between an L tone syllable
		so that \fm{-aa} becomes a short vowel
	\begin{itemize}
	\item	predicted to occur with relative \X[-ÿi-rel]{-ÿi} and subordinate \X[-ÿi-sub]{-ÿi}
		but no examples
	\item	\fm{ax̱ sʼeenáyi} ‘my lamp’
			\vbmorph{ax̱&\rt{sʼin}&-μμL&\gm{-á}&-ÿi}
				{\xx{1sg.psr}&\rt{lamp}&\·\xx{var}&\·\xx{inst}&\·\xx{poss}}
		\versus \fm{sʼeenáa} ‘lamp’
			\vbmorph{\rt{sʼin}&-μμL&-áa}
				{\rt{lamp}&\·\xx{var}&\·\xx{inst}}
	\end{itemize}

\item[-aa]\label{m:-aa}
	instrument suffix, indicates an instrument used for an event;
	has polar tone opposite the preceding syllable
		so \fm{-aa} after an H tone syllable
		and \X{-áa} after an L tone syllable;
	becomes a short vowel when followed by \fm{-ÿi}
		so \X{-a} + \fm{-ÿi}
			(not \fm[*]{-a-ÿí})
		and \X{-á} + \fm{-ÿi};
	this suffix usually occurs in nouns derived from verbs (nominalizations)
		but it also occurs in some verbs where the the verb is probably
		derived from the noun that includes \fm{-aa} \~\ \fm{-áa},
		supported by the fact that when this suffix occurs in verbs
		the verb stem is always invariable;
	the meaning of \fm{-aa} \~\ \fm{-áa} is similar to English \fm{-er} as in \fm{dipper}
		but it only applies to an instrument and never an agent,
		so \fm{gúxʼaa} is only a tool and never a person;
	glossed \xx{inst} versus the instrumental postposition \fm{-n} \~\ \fm{een} \~\ \fm{teen}
		which is glossed \xx{instr};
	probably also part of \X{-jaa} \~\ \X{-jáa} with repetitive \X{-ch}
		and part of \X{-x̱aa} \~\ \X{-x̱áa} with repetitive \X{-x̱}
		but the composition of meaning in these is unclear
	\newline
	allomorphs:
	\begin{allolist}
	\item[-aa]	L tone form after used after H tone syllable
	\item[\X{-áa}]	H tone form after used after L tone syllable
	\item[\X{-a}]	short vowel form of \fm{-aa} when followed by \fm{-ÿi}
	\item[\X{-á}]	short vowel form of \fm{-áa} when followed by \fm{-ÿi}
	\end{allolist}
	part of:
	\begin{allolist}
	\item[\X{-jaa} \~\ \X{-jáa}]
			with repetitive \X{-ch} as a suffix in some nouns
	\item[\X{-x̱aa} \~\ \X{-x̱áa}]
			with repetitive \X{-x̱} in ‘miss target’ derivation
	\end{allolist}
	\begin{enumerate}
	\item	instrument suffix forming nouns from verb roots,
		denoting an instrument used for the event described by the root
		\begin{itemize}
		\item	\fm{óonaa} ‘rifle, gun’ (literally ‘instrument for shooting’)
				\vbmorph{\rt{.uᴴn}&-μμH&\gm{-aa}}
					{\rt{shoot}&\·\xx{var}&\·\xx{inst}}
			\versus \vbform{aawa.ún}{pfv}[tr, \fm{∅}, ach]{she/he/it shot him/her/it}
				\vbmorph{a-&μʷ-&wa-&\rt[²]{.uᴴn}&-μH}
					{\xx{3>3}&\xx{pfv}&\xx{stv}&\rt[²]{shoot}&\·\xx{var}}
		\item	\fm{gúxʼaa} ‘cup, can, dipper’ (literally ‘instrument for dipping’)
				\vbmorph{\rt{guxʼ}&-μH&\gm{-aa}}
					{\rt{dip}&\·\xx{var}&\·\xx{inst}}
			\versus \vbform{agóoxʼ}{impfv}[tr, \fm{∅}, \fm{-μμH} act]{she/he/it dips up/out him/her/it}
				\vbmorph{a-&\rt[²]{guxʼ}&-μμH}
					{\xx{3>3}&\rt[²]{dip}&\·\xx{var}}
		\end{itemize}
	\item	instrument suffix retained in verbs derived from nouns with this suffix;
		in some cases only the verb based on the noun with \fm{-aa} is attested,
			but in other cases both the verb without \fm{-aa}
			and the verb from the noun with \fm{-aa} are attested,
			though the difference in meaning between these, if any, is unclear
		\begin{itemize}
		\item	\vbform{dusdeegáa}{impfv}[tr, \fm{∅}?, inv act]{people dipnet for it}
			\parencite[91.1143]{story-naish:1973}
				\vbmorph{du-&d-&s-&\rt{dik}&-μμL&\gm{-áa}}
					{\xx{ind.h.s}&\xx{mid}&\xx{tr}&\rt{dipnet}&\·\xx{var}&\xx{inst}}
			\versus \fm{deegáa} ‘dipnet’ (literally ‘instrument for dipnetting’)
				\vbmorph{\rt{dik}&-μμL&\gm{-áa}}
					{\rt{dipnet}&\·\xx{var}&\·\xx{inst}}
			\versus \vbform{awdzidéek}{pfv}[tr, \fm{∅}?, ach?]{she/he/it dipnetted for things}
			\parencite[91.1142]{story-naish:1973}
				\vbmorph{a-&w-&d-&s-&i-&\rt{dik}&-μμH}
					{\xx{ind.n.o}&\xx{pfv}&\xx{mid}&\xx{tr}&\xx{stv}&\rt{dipnet}&\·\xx{var}}
		\end{itemize}
	\item	unknown suffix in verbs with the ‘pretend activity’ derivation
			made up of:
			reflexive \X{ash=} \~\ \X{ach=}
			+ qualifier \X[ka-qual]{ka-}
			+ irrealis \X[u-irr]{u-}
			± extensional \X{s-}/\X{lˢ-} or \X{l-} or pejorative \X{sh-}
			± unknown \fm{-aa}
			\parencites[55]{story:1966}[654]{crippen:2019};
			may be glossed as \xx{inst} or \xx{unkn};
			attested with the roots
			\begin{inlinelist}
			\item	\fm{\rt{chʼitʼ}} \~\ \fm{\rt{chʼetʼ}} ‘ball’
			\item	\fm{\rt{dlen}} ‘tempt’
			\item	\fm{\rt{gulʼ}} ‘one-eye’
			\item	\fm{\rt{g̱iḵ}} \~\ \fm{\rt{g̱eḵ}} ‘swing’
			\item	\fm{\rt{g̱ixʼ}} ‘creak, squeak’
			\item	\fm{\rt{kitsʼ}} ‘rock’
			\item	\fm{\rt{kʼeᴴn}} ‘jump’
			\item	\fm{\rt{ḵux̱}} ‘go by boat’
			\item	\fm{\rt{ḵʼish}} ‘swat, hit with stick’
			\item	\fm{\rt{taḵ}} ‘poke’
			\item	\fm{\rt{tʼach}} ‘slap; swim’
			\item	\fm{\rt{tʼaxʼ}} ‘flick’
			\item	\fm{\rt{x̱ʼilʼ}} ‘slide’
			\end{inlinelist}
			see also \fm{\rt{tsin}} ‘alive, strong’ under \X{-áa} allomorph
		\begin{itemize}
		\item	\vbform{has ash koosḵúx̱aa}{impfv}[subj intr, \fm{∅}?, inv act]{they are playing toy boat}
			\parencite[152.2071]{story-naish:1973}
				\vbmorph{has=&ash=&ka-&u-&d-&s-&\rt[¹]{ḵux̱}&-μμH&\gm{-aa}}
					{\xx{plh}&\xx{rflx.o}&\xx{qual}&\xx{irr}&\xx{mid}&\xx{csv}&\rt[¹]{go.boat}&\·\xx{var}&\·\xx{inst}}
			\versus \vbform{wooḵoox̱}{pfv}[subj intr, \fm{n}, mot]{she/he/it went by boat}
				\vbmorph{wu-&μ-&\rt[¹]{ḵux̱}&-μμL}
					{\xx{pfv}&\xx{stv}&\rt[¹]{go.boat}&\·\xx{var}}
		\item	\vbform{has ash koolḵʼíshaa}{impfv}[subj intr, \fm{∅}?, inv act]{they are playing baseball, softball, etc.}
			\parencite[152.2070]{story-naish:1973}
				\vbmorph{has=&ash=&ka-&u-&d-&lˢ-&\rt[²]{ḵʼish}&-μμH&\gm{-aa}}
					{\xx{plh}&\xx{rflx.o}&\xx{qual}&\xx{irr}&\xx{mid}&\xx{csv}&\rt[²]{swat}&\·\xx{var}&\·\xx{inst}}
			\versus \vbform{aawaḵʼísh}{pfv}[tr, \fm{∅}, ach]{she/he/it swatted, batted him/her/it}
				\vbmorph{a-&μʷ-&wa-&\rt[²]{ḵʼish}&-μH}
					{\xx{3>3}&\xx{pfv}&\xx{stv}&\rt[²]{swat}&\·\xx{var}}
		\item	\vbform{has ash koolchʼéitʼaa}{impfv}[subj intr, \fm{∅}?, inv act]{they are playing ball}
			\parencite[152.2069]{story-naish:1973}
				\vbmorph{has=&ash=&ka-&u-&d-&lˢ-&\rt{chʼetʼ}&-μμH&\gm{-aa}}
					{\xx{plh}&\xx{rflx.o}&\xx{qual}&\xx{irr}&\xx{mid}&\xx{csv}&\rt{ball}&\·\xx{var}&\·\xx{inst}}
			\versus \fm{koochʼéitʼaa} ‘ball’
				\vbmorph{ka-&u-&\rt{chʼetʼ}&-μμH&\gm{-aa}}
					{\xx{qual}&\xx{irr}&\rt{ball}&\·\xx{var}&\·\xx{inst}}
		\item	\vbform{has ash koolkʼéinaa}{impfv}[subj intr, \fm{∅}?, inv act]{they are playing at jumping}
			\parencite[152.2073]{story-naish:1973}
				\vbmorph{has=&ash=&ka-&u-&d-&l-&\rt[¹]{kʼeᴴn}&-μμH&\gm{-aa}}
					{\xx{plh}&\xx{rflx.o}&\xx{qual}&\xx{irr}&\xx{mid}&\xx{csv}&\rt{ball}&\·\xx{var}&\·\xx{inst}}
			\versus \vbform{héent wujikʼén}{pfv}[subj intr, \fm{∅}, mot]{she/he/it jumped into the water}
			\parencite[71.850]{story-naish:1973}
				\vbmorph{héen&-t&wu-&d-&sh-&i-&\rt[¹]{kʼeᴴn}&-μH}
					{water&\·\xx{pnct}&\xx{pfv}&\xx{mid}&\xx{pej}&\xx{stv}&\rt[¹]{jump}&\·\xx{var}}
		\item	\vbform{a kát ash kux̱ashx̱ʼílʼaa}{impfv}[subj intr, \fm{∅}?, inv act]{I am sledding on it}
			\parencite[98]{leer:1963}
				\vbmorph{a&ká&-t&ash=&ka-&u-&x̱a-&d-&sh-&\rt[¹]{x̱ʼilʼ}&-μμH&\gm{-aa}}
					{\xx{3n.psr}&\xx{hsfc}&\xx{pnct}&\xx{rflx.o}&\xx{qual}&\xx{irr}&\xx{1sg.s}&\xx{mid}&\xx{pej}&\rt[¹]{slide}&\·\xx{var}&\·\xx{inst}}
			\versus \fm{koox̱ʼílʼaa yeit} ‘recreational sled’
				\vbmorph{ka-&u-&\rt[¹]{x̱ʼilʼ}&-μH&\gm{-aa}&yee-&át}
					{\xx{qual}&\xx{irr}&\rt[¹]{slide}&\·\xx{var}&\·\xx{instr}&below&thing}
			\versus \vbform{wushix̱ʼéelʼ}{pfv}[subj intr, \fm{n}, mot]{she/he/it slid}
			\parencite[98]{leer:1963}
				\vbmorph{wu-&sh-&i-&\rt[¹]{x̱ʼilʼ}&-μμH}
					{\xx{pfv}&\xx{pej}&\xx{stv}&\rt[¹]{slide}&\·\xx{var}}
		\end{itemize}
	\item	unknown suffix in a few verbs, possibly onomatopoeia
		\begin{itemize}
		\item	\vbform{aatlein ax̱altsʼíxaa}{impfv}[subj intr, \fm{∅}?, inv act]{I sneeze a lot}
			\parencite[200.2788]{story-naish:1973}
				\vbmorph{aatlein&a-&x̱a-&d-&lˢ-&\rt{tsʼix}&-μH&\gm{-aa}}
					{lots&\xx{xpl}&\xx{1sg.s}&\xx{mid}&\xx{xtn}&\rt{sneeze}&\·\xx{var}&\·\xx{unkn}}
			\versus	\vbform{akḵwaltsʼíxaa}{prosp}{I’m going to sneeze}
			\parencite[200.2789]{story-naish:1973}
				\vbmorph{a-&k-&w-&g̱-&x̱a-&d-&lˢ-&\rt{tsʼix}&-μH&\gm{-aa}}
					{\xx{xpl}&\xx{gcnj}&\xx{irr}&\xx{mod}&\xx{1sg.s}&\xx{mid}&\xx{xtn}&\rt{sneeze}&\·\xx{var}&\·\xx{unkn}}
		\item	\vbform{kalitsʼígwaa}{impfv}[obj intr, \fm{g}?, inv state]{she/he/it is a delicate matter}
				\vbmorph{ka-&lˢ-&i-&\rt{tsʼikw}&-μH&\gm{-aa}}
					{\xx{qual}&\xx{xtn}&\xx{stv}&\rt{delicate}&\·\xx{var}&\·\xx{unkn}}
		\end{itemize}
	\item	meaningless suffix in a few borrowed nouns;
		although this is not originally a suffix it acts like the instrument suffix
			in the same phonological way despite not contributing any meaning;
		for convenience this can be glossed as \xx{inst} like the real instrument suffix
		\begin{itemize}
		\item	\fm{dáanaa} ‘dollar, money, silver’
			from Chinook Jargon \fm{dála} (same meaning)
			ultimately from English \fm{dollar}
				\vbmorph{\rt{dan}&-μμH&\gm{-aa}}
					{\rt{money}&\·\xx{var}&\·\xx{instr}}
			\versus \fm{ax̱ dáanayi} ‘my dollar, money, silver’
				\vbmorph{ax̱&\rt{dan}&-μμH&\gm{-a}&-ÿi}
					{\xx{1sg.psr}&\rt{money}&\·\xx{var}&\·\xx{inst}&\·\xx{poss}}
		\item	\fm{shgóonaa} ‘sailboat, schooner’
			from English \fm{schooner}
				\vbmorph{sh=&\rt{gun}&-μμH&\gm{-aa}}
					{\xx{unkn}&\rt{sailboat}&\·\xx{var}&\·\xx{instr}}
			\versus \fm{ax̱ shgóonayi} ‘my sailboat, my schooner’
				\vbmorph{ax̱&sh=&\rt{gun}&-μμH&\gm{-a}&-ÿi}
					{\xx{1sg.psr}&\xx{unkn}&\rt{sailboat}&\·\xx{var}&\·\xx{instr}&\·\xx{poss}}
		\end{itemize}
	\item	meaningless suffix retained in verbs derived from borrowed nouns with this suffix
		\begin{itemize}
		\item	\vbform{jididáanaa}{impfv}[obj intr, \fm{g}?, inv state]{she/he/it has (lots of) money, is rich}
			\parencite[05/41]{leer:1973}
				\vbmorph{ji-&d-&i-&\rt[¹]{dan}&-μμH&\gm{-aa}}
					{hand&\xx{mid}&\xx{stv}&\rt[¹]{money}&\·\xx{var}&\·\xx{inst}}
			\versus \vbform{x̱at jidadáanayi kát}{sub impfv}{if I were rich}
			\parencite[15]{leer:1963}
				\vbmorph{x̱at=&ji-&da-&\rt[¹]{dan}&-μμH&\gm{-a}&-ÿi&ká&-t}
					{\xx{1sg.o}&hand&\xx{mid}&\rt[¹]{money}&\·\xx{var}&\·\xx{inst}&\·\xx{sub}&\xx{hsfc}&\·\xx{pnct}}
		\end{itemize}
	\item	part of unknown suffix \X{-jaa} \~\ \X{-jáa} in nouns derived from verb roots,
			see that entry for more detail
	\item	part of amissive \X{-x̱aa} \~\ \X{-x̱áa} in verbs with the ‘miss target’ derivation made up of:
		qualifier \X[ÿa-qual]{ÿa-}
		+ extensional \X{s-}/\X{lˢ-}
		+ repetitive \X{-x̱}
		+ unknown \fm{-aa} \~\ \X{-áa}
		with \fm{∅} conjugation class;
		see amissive \X{-x̱aa} \~\ \X{-x̱áa} for more detail;
		if treated as a single suffix \fm{-x̱aa} \~\ \fm{-x̱áa}
			this is glossed as \xx{miss},
			otherwise \xx{rep} + \xx{unkn}
		\begin{itemize}
		\item	\vbform{ayawsi.únx̱aa}{pfv}[tr, \fm{∅}, ach]{she/he/it shot at him/her/it and missed}
				\vbmorph{a-&ÿa-&w-&s-&i-&\rt[²]{.uᴴn}&-μH&\gm{-x̱}&\gm{-aa}}
					{\xx{3>3}&\xx{qual}&\xx{pfv}&\xx{xtn}&\xx{stv}&\rt[²]{shoot}&\·\xx{var}&\·\xx{rep}&\·\xx{unkn}} or
				\vbmorph{a-&ÿa-&w-&s-&i-&\rt[²]{.uᴴn}&-μH&\gm{-x̱aa}}
					{\xx{3>3}&\xx{qual}&\xx{pfv}&\xx{xtn}&\xx{stv}&\rt[²]{shoot}&\·\xx{var}&\·\xx{miss}}				
			\versus \vbform{aawa.ún}{pfv}{she/he/it shot him/her/it}
				\vbmorph{a-&μʷ-&wa-&\rt[²]{.uᴴn}&-μH}
					{\xx{3>3}&\xx{pfv}&\xx{stv}&\rt[²]{shoot}&\·\xx{var}}
		\end{itemize}
	\end{enumerate}

\item[-áa]\label{m:-áa}
	allomorph of \X{-aa} with H tone, used after L tone syllable (polar tone);
	for phonological reasons this allomorph is less common than \fm{-aa};
	attested only with nouns except for \vbform{sh yáx̱ awooltseenáa}{impfv}{she/he/it is exercising}
		and \vbform{dusdeegáa}{impfv}{people dipnet for it} shown below
	\begin{itemize}
	\item	\vbform{sh yáx̱ awooltseenáa}{impfv}[subj intr, conj?, inv act]{she/he/it is exercising}
		\parencite[09/162]{leer:1973} from \fm{\rt{tsin}} ‘alive, strong’
		\vbmorph{sh&yá&-x̱&a-&ÿa-&u-&d-&lˢ-&\rt{tsin}&-μμL&-áa}
			{\xx{rflx}&face&\·\xx{pert}&\xx{xpl}&face&\xx{irr}&\xx{mid}&\xx{csv}&\rt[¹]{alive}&\·\xx{var}&\·\xx{inst}}
		\newline
		includes motion derivation \motderiv{ÿaa= \~\ ÿa-u-}{∅, \fm{-ch} rep}{obliquely, circuitously, aside};
		similar in meaning to ‘pretend activity’ derivation (see \X{-aa})
			but with different structure
	\item	\fm{deegáa} ‘dipnet’
			\vbmorph{\rt{dik}&-μμL&\gm{-áa}}
				{\rt{dipnet}&\·\xx{var}&\·\xx{inst}}
		\versus \vbform{asdéek}{impfv}[tr, \fm{∅}, \fm{-μμH} act]{she/he/it dipnets for him/her/it}
			\vbmorph{a-&d-&s-&\rt{dik}&-μμH}
				{\xx{3>3}&\xx{mid}&\xx{csv}&\rt{dipnet}&\·\xx{var}}
		\versus \vbform{dusdeegáa}{impfv}[tr, \fm{∅}?, inv act]{people dipnet for it}
			\parencite[91.1143]{story-naish:1973}
				\vbmorph{du-&d-&s-&\rt{dik}&-μμL&\gm{-áa}}
					{\xx{ind.h.s}&\xx{mid}&\xx{tr}&\rt{dipnet}&\·\xx{var}&\xx{inst}}
	\end{itemize}
	because there are not many nouns with \fm{-áa} they are listed here for reference:
	\begin{itemize}
	\item	\fm{deegáa} ‘dipnet’
		from \fm{\rt{dik}} ‘fish by dipnet’
			\vbmorph{\rt{dik}&-μμL&\gm{-áa}}
				{\rt{dipnet}&\·\xx{var}&\·\xx{inst}}
	\item	\fm{dzeenáa} \~\ \fm{dzeináa} ‘small animal leg snare’
		from unknown \fm{\rt{dzin}} \~\ \fm{\rt{dzen}}
			\vbmorph{\rt{dzin}&-μμL&\gm{-áa}}
				{\rt{\xx{unkn}}&\·\xx{var}&\·\xx{inst}}
	\item	\fm{dzoonáa} \~\ \fm{dzeenáa} \~\ \fm{dzanáa} ‘dart, missile’
		from \fm{\rt{dzuᴸ}} ‘throw at’
			\vbmorph{\rt{dzuᴸ}&-μμL&-n&\gm{-áa}}
				{\rt{throw}&\·\xx{var}&\·\xx{nsfx}&\·\xx{inst}}
		\newline
		the form \fm{dzeináa} with ablaut \fm{-μᵉμL} from \fm{-n} is expected
			but apparently does not occur, perhaps because
			the stem has been reanalyzed
	\item	\fm{sʼeenáa} ‘lamp, light’
		from unknown \fm{\rt{sʼin}}
			\vbmorph{\rt{sʼin}&-μμL&\gm{-áa}}
				{\rt{\xx{unkn}}&\·\xx{var}&\·\xx{inst}}
	\item	\fm{g̱aatáa} ‘trap’
		from \fm{\rt{g̱at}} ‘split, fall apart’
			\vbmorph{\rt{g̱at}&-μμL&\gm{-áa}}
				{\rt{fall.apart}&\·\xx{var}&\·\xx{inst}}
	\item	\fm{woosheenáa} ‘staff, cane, walking stick’
		probably from \fm{\rt{shi}} ‘reach out hand, touch’ (but \X{-μμL})
			\vbmorph{ÿa-&u-&\rt{shi}&-μμL&-n&\gm{-áa}}
				{\xx{qual}&\xx{irr}&\rt{\xx{unkn}}&\·\xx{var}&\·\xx{nsfx}&\·\xx{inst}}
		\newline
		includes motion derivation \motderiv{ÿaa= \~\ ÿa-u-}{∅, \fm{-ch} rep}{obliquely, circuitously, aside};
		compare \fm{sheeÿ} ‘right (hand)’ and \fm{sheeÿ} ‘limb, knot’
	\item	\fm{wootsaag̱áa} ‘staff, cane, walking stick’
		from \fm{\rt{tsaḵ}} ‘poke, prod’
			\vbmorph{ÿa-&u-&\rt{tsaḵ}&-μμL&\gm{-áa}}
				{\xx{qual}&\xx{irr}&\rt{\xx{unkn}}&\·\xx{var}&\·\xx{inst}}
		\newline
		includes motion derivation \motderiv{ÿaa= \~\ ÿa-u-}{∅, \fm{-ch} rep}{obliquely, circuitously, aside}
	\item	\fm{tináa} \~\ \fm{teenáa} ‘copper shield’
		perhaps from \fm{\rt{tin}} ‘see’
			\vbmorph{\rt{tin}&-μμL&\gm{-áa}}
				{\rt{see}&\·\xx{var}&\·\xx{inst}}
	\item	\fm{xeisáa} ‘small animal or bird trap’
		from \fm{\rt{xes}} ‘catch with container’
			\vbmorph{\rt{xes}&-μμL&\gm{-áa}}
				{\rt{capture}&\·\xx{var}&\·\xx{inst}}
	\item	\fm{xaadáa} ‘veil, fine netting’
		from \fm{\rt{xat}} ‘pull, tighten; fasten’
			\vbmorph{\rt{xat}&-μμL&\gm{-áa}}
				{\rt{pull}&\·\xx{var}&\·\xx{inst}}
	\item	\fm{xeejáa} ‘springpole (of snare)’
		from \fm{\rt{xich}} ‘bend over’
			\vbmorph{\rt{x̱ich}&-μμL&\gm{-áa}}
				{\rt{bend.over}&\·\xx{var}&\·\xx{inst}}
	\item	\fm{xʼeesháa} ‘bucket’
		from unknown \fm{\rt{xʼish}}
			\vbmorph{\rt{xʼish}&-μμL&\gm{-áa}}
				{\rt{\xx{unkn}}&\·\xx{var}&\·\xx{inst}}
		\newline
		compare \fm{\rt{xʼish}} ‘skin’ and \fm{xʼíshaa} ‘skinning knife’
	\item	\fm{x̱oonáa} ‘tree debarker’
		from \fm{\rt{x̱uᴴw}} ‘peel bark’ (noun \fm{x̱óow} ‘slab of bark’)
			\vbmorph{\rt{x̱uᴴw}&-μμL&-n&\gm{-áa}}
				{\rt{debark}&\·\xx{var}&\·\xx{nsfx}&\·\xx{inst}}
	\end{itemize}

\item[aawa]\label{m:aawa}
	≡ \fm{a-μʷ-wa-}
	combination of argument marking \X{a-},
		perfective \X{μʷ-},
		and stative \X{wa-};
	compare \X{awu} ≡ \fm{a-wu-} and \X{aw} ≡ \fm{a-w-}
		as well as \X[eeÿa-a-ʷ-i-ÿa]{eeÿa} ≡ \fm{a-ʷ-i-ÿa-}
		and \X{oowa} ≡ \fm{a-u-wa-}
	\begin{itemize}
	\item	\vbform{aawajáḵ}{pfv}[tr, \fm{∅}, ach]{she/he/it killed him/her/it}
			\vbmorph{\gm{a-}&\gm{μʷ-}&\gm{wa-}&\rt[²]{jaḵ}&-μH}
				{\xx{3>3}&\xx{pfv}&\xx{stv}&\rt[²]{kill}&\·\xx{var}}
		\versus \vbform{awsi.átʼ}{pfv}[tr, \fm{∅}, ach]{she/he/it cooled him/her/it}
			\vbmorph{a-&w-&s-&i-&\rt[¹]{.átʼ}&-μH}
				{\xx{3>3}&\xx{pfv}&\xx{csv}&\xx{stv}&\rt[⁰]{cold}&\·\xx{var}}
	\end{itemize}

\item[ach=]\label{m:ach=}
	allomorph of third person proximate human object \X{ash=};
	may reflect older forms of reflexive with affricate \fm{ch} (compare \X{chush=})
		that otherwise became \fm{sh} (compare \X{sh=}),
	or perhaps derived from third person pronoun \fm{á} + postposition \fm{-ch}
		(compare \X{ách});
	apparently no reliable difference in meaning or context between \fm{ash=} and \fm{ach=},
		nor any phonological contexts that distinguish them;
	dialect distribution remains to be studied, but mostly attested in Inland Northern
	\begin{itemize}
	\item	\vbform{has ach x̱ʼawóosʼ}{impfv}[tr, \fm{n}, \fm{-μμH} act]{they question him/her}
		\parencite[60.396]{nyman-leer:1993}
			\vbmorph{has=&\gm{ach=}&x̱ʼe-&\rt[²]{wusʼ}&-μμH}
				{\xx{plh}&\xx{3prx.o}&mouth&\rt[²]{question}&\·\xx{var}}
		\versus \vbform{has ax̱ʼawóosʼ}{impfv}{they question him/her/it}
			\vbmorph{has=&a-&x̱ʼe-&\rt[²]{wusʼ}&-μμH}
				{\xx{plh}&\xx{3>3}&mouth&\rt[²]{question}&\·\xx{var}}
	\item	\vbform{du een ach katoolyádi}{sub impfv}[subj intr, \fm{n}, inv \fm{-μH} act]{that we play with him/her}
		\parencite[188.434]{dauenhauer-dauenhauer:1987}
			\vbmorph{du&ee&-n&\gm{ach=}&ka-&too-&d-&l-&\rt[¹]{ÿat}&-μH&-i}
				{\xx{3h}&\xx{base}&\·\xx{instr}&\xx{rflx.o}&\xx{qual}&\xx{1pl.s}&\xx{mid}&\xx{csv}&\rt[¹]{child}&\·\xx{var}&\·\xx{sub}}
		\versus \vbform{du een ash kanax̱toolyát}{hort}{let us play with him/her}
			\parencite[189.452]{dauenhauer-dauenhauer:1987}
			\vbmorph{du&ee&-n&ash=&ka-&na-&x̱-&too-&d-&l-&\rt[¹]{ÿat}&-μH}
				{\xx{3h}&\xx{base}&\·\xx{instr}&\xx{rflx.o}&\xx{qual}&\xx{ncnj}&\xx{mod}&\xx{1pl.s}&\xx{mid}&\xx{csv}&\rt[¹]{child}&\·\xx{var}&}
		\newline
		(note that these two examples are from the same speaker)
	\end{itemize}

\item[ách]\label{m:ách}
	≡ \fm{á-ch}
	combination of third person nonhuman pronoun \fm{á}
		and applicative instrumental postposition \fm{-ch};
	this is a postposition phrase required immediately before a few verbs and is
	not actually a verb morpheme although it resembles (and may be becoming) a preverb
	\begin{itemize}
	\item	\vbform{du jeedé ách ax̱wsiwóo}{pfv}[subj intr, \fm{∅}, ach]{I sent him/her/it to him/her/it}
		\parencite[03/296]{leer:1973}
			\vbmorph{du&jee&-dé&\gm{á}&\gm{-ch}&a-&ʷ-&x̱-&s-&i-&\rt[²]{wu}&-μμH}
				{\xx{3h.psr}&pos’n&\·\xx{all}&\xx{3n}&\·\xx{instr}&\xx{xpl}&\xx{pfv}&\xx{1sg.s}&\xx{appl}&\xx{stv}&\rt[²]{send}&\·\xx{var}}
	\end{itemize}

\item[-áchʼ]\label{m:-áchʼ}
	allomorph of unknown suffix \X{-chʼ} with epenthetic (filler) vowel \fm{á};
	this \fm{-áchʼ} is attested only in the noun
		\fm{g̱eeg̱áchʼ} \~\ \fm{g̱eig̱áchʼ} ‘hammock, swing for baby’
		as well as in verbs and nouns derived from this noun;
	since the meaning of \fm{-chʼ} \~\ \fm{-áchʼ} is unknown it is glossed \xx{unkn}
	\begin{itemize}
	\item	\fm{g̱eeg̱áchʼ} \~\ \fm{g̱eig̱áchʼ} ‘hammock, swing for baby’
		\vbmorph{\rt{g̱eḵ}&-μμL&\gm{-áchʼ}}
			{\rt{swing}&\·\xx{var}&\·\xx{unkn}}
		\begin{itemize}
		\item	\fm{g̱eeg̱áchʼaa} \~\ \fm{g̱eig̱áchʼaa} ‘swing’
			\parencite[f02/193]{leer:1973} with suffix \X{-aa} ‘instrument for’;
			implies verb → noun → verb → noun
		\end{itemize}
	\item	\vbform{ash koolg̱eig̱áchʼ}{impfv}[subj intr, conj?, act]{she/he/it is playing on a swing/hammock}
			\vbmorph{ash=&ka-&u-&d-&l-&\rt[¹]{g̱eḵ}&-μμL&\gm{-áchʼ}}
				{\xx{rflx.o}&\xx{qual}&\xx{irr}&\xx{mid}&\xx{csv}&\rt[¹]{swing}&\·\xx{var}&\·\xx{unkn}}
		\versus \vbform{awlig̱eiḵ}{pfv}[tr, \fm{n}, mot]{she/he/it swung him/her/it}
			\vbmorph{a-&w-&l-&i-&\rt[¹]{g̱eḵ}&-μμL}
				{\xx{3>3}&\xx{pfv}&\xx{csv}&\xx{stv}&\rt[¹]{swing}&\·\xx{var}}
		\notealso{(\fm{ash koolg̱eig̱áchʼ} is presumably verb → noun → verb)}
	\item	possibly in \fm{tlʼaaḵʼwáchʼ} ‘sourdock, wild rhubarb (\textit{Rumex} spp.)’
		\parencite[f01/251]{leer:1973} implying a stem \fm{tlʼaaḵʼw},
		although \textcite[79]{leer:1978b} analyzes this with a stem \fm{ḵʼwáchʼ}
		perhaps related to obscure \fm{\rt{ḵʼwash}} ‘peel’
		and \fm{x̱ʼwaash} ‘large sea urchin’ \parencite[f01/215]{leer:1973}
	\end{itemize}

\item[-álʼ]\label{m:-álʼ}
	allomorph of repetitive \X{-lʼ} with epenthetic (filler) vowel \fm{á};
	attested only with two verbs but may be possible with others given attested nouns
	\begin{enumerate}
	\item	repetitive suffix attested with two verb roots;
		with \fm{\rt[²]{chuxʼ}} ‘touch lightly, graze’ the \fm{-álʼ} occurs only in the
			repetitive imperfective form
		but with \fm{…néegwálʼ} ‘paint’ the \fm{-álʼ} occurs in all forms
			and so forms a frozen disyllabic stem;
		the order of \fm{-álʼ}\X{-k} in \fm{yoo ayanéegwálʼk} is interesting
		\begin{itemize}
		\item	\vbform{achóoxʼálʼ}{rep impfv}[tr, \fm{∅}, \fm{-μμH} act]{she/he/it repeatedly touches him/her/it}
			\parencites[10/227]{leer:1973}[598]{leer:1976}
				\vbmorph{a-&\rt[²]{chuxʼ}&-μμH&\gm{-álʼ}}
					{\xx{3>3}&\rt[²]{graze}&\·\xx{var}&\·\xx{rep}}
			\versus \vbform{achóoxʼ}{impfv}{she/he/it touches him/her/it}
				\vbmorph{a-&\rt[²]{chuxʼ}&-μμH}
					{\xx{3>3}&\rt[²]{graze}&\·\xx{var}}
		\item	\vbform{anéegwálʼ}{impfv}[tr, \fm{n}, inv act]{she/he/it paints him/her/it}
				\vbmorph{a-&\rt[²]{nikw}&-μμH&\gm{-álʼ}}
					{\xx{3>3}&\rt[²]{paint}&\·\xx{var}&\·\xx{rep}}
			\versus \vbform{yoo ayanéegwálʼk}{rep impfv}{she/he/it repeatedly paints him/her/it}
				\vbmorph{yoo=&a-&ÿa-&\rt[²]{nikw}&-μμH&\gm{-álʼ}&-k}
					{\xx{alt}&\xx{3>3}&\xx{stv}&\rt[²]{paint}&\·\xx{var}&\·\xx{rep}&\·\xx{rep}}
		\end{itemize}
	\item	unclear meaning in a handful of nouns
		\begin{itemize}
		\item	\fm{ḵéichʼálʼ} \~\ \fm{ḵéechʼálʼ} ‘seam’ from \fm{\rt{ḵa}} ‘stitch, sew’
			\vbmorph{\rt{ḵa}&-μᵉμH&-chʼ&\gm{-álʼ}}
				{\rt{stitch}&\·\xx{var}&\·\xx{unkn}&\·\xx{rep}}
			\newline
			see \X{-chʼálʼ} for more detail on this noun
		\item	\fm{néegwálʼ} ‘paint’ from unknown \fm{\rt{nikw}}
			possibly originally from \fm{\rt{ni}} \~\ \fm{\rt{ne}} ‘occur, do’ 
			+ repetitive \X{-kw}
				\vbmorph{\rt{nikw}&-μμH&\gm{-álʼ}}
					{\rt{paint?}&\·\xx{var}&\·\xx{rep}}
			\newline
			also occurs in:
			\begin{itemize}
			\item	\fm{kanéegwálʼ} ‘berries and salmon eggs’
				\vbmorph{ka-&\rt{nikw}&-μμH&\gm{-álʼ}}
					{\xx{sro}&\rt{paint?}&\·\xx{var}&\·\xx{rep}}
			\item	\fm{x̱ʼanéegwálʼ} ‘lipstick’
				\vbmorph{x̱ʼe-&\rt{nikw}&-μμH&\gm{-álʼ}}
					{mouth&\rt{paint?}&\·\xx{var}&\·\xx{rep}}
			\end{itemize}
		\item	\fm{táaxʼálʼ} ‘needle’ from \fm{\rt{taxʼ}} ‘bite’
			\vbmorph{\rt{taxʼ}&-μμH&\gm{-álʼ}}
				{\rt{bite}&\·\xx{var}&\·\xx{rep}}
		\item	\fm{tʼaag̱álʼ} ‘fastening peg’ perhaps from
			\fm{\rt{tʼaḵ}} ‘shift position, aside’ (noun \fm{tʼaaḵ} ‘beside’)
			\vbmorph{\rt{tʼaḵ}&-μμL&\gm{-álʼ}}
				{\rt{aside}&\·\xx{var}&\·\xx{rep}}
			\newline
			compare 
			\begin{inlinelist}
			\item	\fm{tʼáḵlʼ} ‘projecting bone’ (see \X{-lʼ}\!)
			\item	\fm{tʼáaḵw} ‘joist’
			\item	\fm{tʼáax̱ʼw} ‘wart’
			\item	\fm{\rt{tʼak}} ‘dent’
			\item	\fm{\rt{tʼakw}} ‘slap tail, smack’
			\item	\fm{\rt{tʼax̱}} ‘gape’
			\end{inlinelist}
		\item	\fm{tsaag̱álʼ} ‘spear’ from \fm{\rt{tsaḵ}} ‘poke, prod’
			\vbmorph{\rt{tsaḵ}&-μμL&\gm{-álʼ}}
				{\rt{poke}&\·\xx{var}&\·\xx{rep}}
		\item	\fm{x̱eeygwálʼ} ‘pack strap, lashing’ from \fm{\rt{x̱iÿ}} ‘pack’
			\vbmorph{\rt{x̱iÿ}&-μμL&-kw&\gm{-álʼ}}
				{\rt{pack}&\·\xx{var}&\·\xx{rep}&\·\xx{rep}}
		\item	\fm{x̱ʼéexʼwálʼ} \~\ \fm{x̱ʼéixʼwálʼ} ‘safety pin’ from \fm{\rt{x̱ʼixʼ}} ‘wedge’
			\vbmorph{\rt{x̱ʼixʼ}&-μμH&\gm{-álʼ}}
				{\rt{wedge}&\·\xx{var}&\·\xx{rep}}
			\newline
			also occurs in:
			\begin{itemize}
			\item	\fm{shax̱ʼéexʼwálʼ} ‘hair clip, barette’
				\vbmorph{sha-&\rt{x̱ʼixʼ}&-μμH&\gm{-álʼ}}
					{head&\rt{wedge}&\·\xx{var}&\·\xx{rep}}
			\end{itemize}
		\item	\fm{x̱ʼéigwálʼ} ‘safety pin’ from \fm{\rt{x̱ʼe}} ‘mouth’
			\vbmorph{\rt{x̱ʼe}&-μμH&-kw&\gm{-álʼ}}
				{\rt{mouth}&\·\xx{var}&\·\xx{rep}&\·\xx{rep}}
			\newline
			probably a reanalysis of \fm{x̱ʼéexʼwálʼ} \~\ \fm{x̱ʼéixʼwálʼ} above;
			also occurs in:
			\begin{itemize}
			\item	\fm{shax̱ʼéegwálʼ} ‘hair clip, barette’
				\vbmorph{sha-&\rt{x̱ʼe}&-μμH&-kw&\gm{-álʼ}}
					{head&\rt{mouth}&\·\xx{var}&\·\xx{rep}&\·\xx{rep}}
			\end{itemize}
		\end{itemize}
	\end{enumerate}

\item[-áḵw]\label{m:-áḵw}
	allomorph of deprivative \X[-ḵ-dprv]{-ḵ} \~\ \X[-ḵw-dprv]{-ḵw}
		with epenthetic (filler) vowel \fm{á};
	generally describes a situation where something is deprived, lacking, or removed;
	this \fm{-áḵw} form is attested specifically in the following verb stems
		where it has a clear compositional meaning of ‘remove, deprive’:
	\begin{enumerate}
	\item	derivational suffix meaning ‘remove, deprive’ in six verb stems
		\begin{itemize}
		\item	\fm{–geiÿáḵw} ‘claim as payment’
			from \fm{\rt{geᴴÿ}} ‘repay’ (but \X{-μμL}) in
			\newline
			\vbform{algeiÿáḵw}{impfv}[tr, \fm{∅}?, inv act]{she/he/it claims him/her/it as/for payment}
			\parencite[f05/74]{leer:1973}
				\vbmorph{a-&l-&\rt[²]{geᴴÿ}&-μμL&\gm{-áḵw}}
					{\xx{3>3}&\xx{xtn}&\rt[²]{repay}&\·\xx{var}&\·\xx{dprv}}
			\versus \vbform{awsigéÿ}{pfv}[tr, \fm{∅}, ach]{she/he/it has repaid for him/her/it}
				\vbmorph{a-&w-&s-&i-&\rt[²]{geᴴÿ}&-μH}
					{\xx{3>3}&\xx{pfv}&\xx{xtn}&\xx{stv}&\rt[²]{repay}&\·\xx{var}}
		\item	\fm{–g̱eiÿáḵw} ‘scoop out (of shell)’
			from \fm{\rt{g̱e}} ‘between’ (noun \fm{g̱ei}) in
			\newline
			\vbform{alg̱eiÿáḵw}{impfv}[tr, \fm{n}, inv act]{she/he/it scoops it out (of shell)}
			\parencite[f02/147]{leer:1973}
				\vbmorph{a-&l-&\rt{g̱e}&-μμL&\gm{-ÿáḵw}}
					{\xx{3>3}&\xx{tr}&\rt{between}&\·\xx{var}&\·\xx{dprv}}
		\item	\fm{–nóoxʼáḵw} ‘remove shell of’
			from \fm{\rt{nuxʼ}} ‘shell’ (noun \fm{nóoxʼ}) in
			\newline
			\vbform{kadulnóoxʼáḵw}{impfv}[tr, \fm{∅}?, inv act]{they remove the shell from it}
			\parencite[171.2350]{story-naish:1973}
				\vbmorph{ka-&du-&d-&l-&\rt{nuxʼ}&-μμH&\gm{-áḵw}}
					{\xx{qual}&\xx{ind.h.s}&\xx{mid}&\xx{csv}&\rt{shell}&\·\xx{var}&\·\xx{dprv}}
		\item	\fm{–tlʼéiláḵw} ‘remove milt of’
			from \fm{\rt{tlʼeᴴl}} ‘milt’ (noun \fm{tlʼéil}) in
			\newline
			\vbform{altlʼéiláḵw}{impfv}[tr, \fm{n}, inv act]{she/he/it removes the milt from him/her/it}
				\vbmorph{a-&l-&\rt{tlʼeᴴl}&-μμH&\gm{-áḵw}}
					{\xx{3>3}&\xx{tr}&\rt{milt}&\·\xx{var}&\·\xx{dprv}}
		\item	\fm{–xʼwánjáḵw} ‘remove boots’
			from \fm{\rt{xʼwan}} ‘boot’ (noun \fm{xʼwán}) in
			\newline
			\vbform{kawdlixʼwánjáḵw}{pfv}[subj intr?, \fm{∅}?, ach?]{she/he/it removed boots}
			\parencite[f04/77]{leer:1973}
				\vbmorph{ka-&w-&d-&lˢ-&i-&\rt{xʼwan}&-μH&-ch&\gm{-áḵw}}
					{\xx{qual}&\xx{pfv}&\xx{mid}&\xx{tr}&\xx{stv}&\rt{boot}&\·\xx{var}&\·\xx{rep}&\·\xx{dprv}}
		\item	\fm{–x̱aaÿáḵw} ‘shed, lose hair, go bald’
			from \fm{\rt{x̱aw}} ‘fur, hair’ (noun \fm{x̱aaw}) in
			\newline
			\vbform{wudlix̱aaÿáḵw}{pfv}[obj intr, \fm{n}, ach]{she/he/it lost hair, shed, went bald}
			\parencite[786]{leer:1976}
				\vbmorph{wu-&d-&l-&i-&\rt{x̱aw}&-μμL&\gm{-áḵw}}
					{\xx{pfv}&\xx{pasv}&\xx{xtn}&\xx{stv}&\rt{fur}&\·\xx{var}&\·\xx{dprv}}
		\end{itemize}
	\item	derivational suffix ‘remove, deprive’ in verb stems
		where the phonological composition is irregular
		or the underlying root has an unknown meaning:
		\begin{itemize}
		\item	\fm{–.éiÿáḵw} ‘have injured limb’
			from unknown \fm{\rt{.e}} or \fm{\rt{.eÿ}} or \fm{\rt{.a}} in
			\newline
			\vbform{kawdi.éiÿáḵw}{pfv}[obj intr, \fm{g̱}?, ach?]{she/he/it is incapacitated; it (body part) is dislocated, useless}
			\parencite[02/21]{leer:1973}
				\vbmorph{ka-&w-&d-&i-&\rt{.eÿ}&-μμH&\gm{-áḵw}}
					{\xx{qual}&\xx{pfv}&\xx{mid}&\xx{stv}&\rt{\xx{unkn}}&\·\xx{var}&\·\xx{dprv}}
			\newline
			\citeauthor{leer:1976} points to \fm{\rt{.a}} ‘end move, extend’ \parencite[78]{leer:1976}
				but the compositional meaning is unclear;
			possibly related to \fm{\rt{.ek}} ‘weak, paralyzed’
		\item	\fm{–ÿaax̱áḵw} ‘plan, intend’
			from unknown \fm{\rt{ÿax̱}} in
			\newline
			\vbform{yéi awliÿaax̱áḵw}{pfv}[tr, conj?, ach?]{she/he/it wanted, planned it thus}
				\vbmorph{yéi=&a-&w-&l-&i-&\rt{ÿax̱}&-μμL&\gm{-áḵw}}
					{thus&\xx{3>3}&\xx{pfv}&\xx{xtn}&\xx{stv}&\rt{\xx{unkn}}&\·\xx{var}&\·\xx{dprv}}
			\newline
			probably related to \fm{\rt{.aḵw}} ‘direct, command, plan, try’
				and \fm{\rt{ÿaḵw}} ‘compare, liken’;
			\citeauthor{leer:1976} points to \fm{\rt{ÿaḵw}} ‘bequeath, pass on’ \parencite[12]{leer:1978b}
		\item	\fm{–x̱oonáḵw} ‘drowned’
			likely from \fm{\rt{x̱un}} ‘relative, friend’ (noun \fm{x̱oon-í}) in
			\newline
			\vbform{wudix̱oonáḵw}{pfv}[obj intr?, \fm{g̱}?, ach?]{she/he/it drowned}
			\parencite[f02/77]{leer:1973}
				\vbmorph{wu-&d-&i-&\rt{x̱un}&-μμL&\gm{-áḵw}}
					{\xx{pfv}&\xx{pasv}&\xx{stv}&\rt{relative}&\·\xx{var}&\·\xx{dprv}}
		\item	\fm{–chʼéeÿáḵw} ‘slow, late’
			from unknown \fm{\rt{chʼi}} or \fm{\rt{chʼiÿ}} in
			\newline
			\vbform{lichʼéeÿáḵw}{impfv}{obj intr, \fm{g}, inv state}{she/he/it is slow, late}
				\vbmorph{lˢ-&i-&\rt{chʼiÿ}&-μμH&\gm{-áḵw}}
					{\xx{intr}&\xx{stv}&\rt{early?}&\·\xx{var}&\·\xx{dprv}}
		\end{itemize}
	\item	derivational suffix in the stem
		\fm{–séewchʼáḵw} ‘watery, tasteless’
		together with the noun \fm{séew} ‘rain’
		and the suffix \X{-chʼáḵw}
		which is probably a combination of
		the unknown suffix \X{-chʼ}
		and \fm{-áḵw},
		see \X{-chʼáḵw} for details;
	\item	unclear meaning but probably deprivative in several nouns and adjectives
		\begin{itemize}
		\item	\fm{gaawáḵ} ‘saskatoonberry’
			perhaps from \fm{\rt{gaw}} ‘noise’
				\vbmorph{\rt{gaw}&-μμL&\gm{-áḵ}}
					{\rt{noise}&\·\xx{var}&\·\xx{dprv}}
		\item	\fm{gantáḵw} \~\ \fm{kantáḵw} ‘lupine’
			perhaps from \fm{\rt{gan}} ‘burn’ (but \X{-μL})
				\vbmorph{\rt{gan}&-μL&-t&\gm{-áḵw}}
					{\rt{burn}&\·\xx{var}&\·\xx{rep}&\·\xx{dprv}}
			\newline
			but \fm{kantáḵw} instead suggests unknown \fm{\rt{kan}}
			for which compare
			\begin{inlinelist}
			\item	\fm{\rt{kan}} ‘dance in return’
			\item	\fm{kánaa} ‘flag, dance staff’ (with \X{-aa})
			\item	\fm{chookán} ‘grass’
			\item	\fm{keekán} ‘going to see’
			\item	\fm{g̱uwakaan} ‘deer’
			\item	\fm{káani} ‘sibling-in-law’
			\end{inlinelist}
		\item	\fm{goosháḵw} \~\ \fm{gooshúḵ} ‘nine’
			from \fm{\rt{gush}} ‘dorsal fin, thumb’
				\vbmorph{\rt{gush}&-μμL&\gm{-áḵw}}
					{\rt{thumb}&\·\xx{var}&\·\xx{dprv}}
		\item	\fm{katʼéitʼáḵw} ‘berry on stem after freeze’
			from \fm{\rt{tʼaᴸ}} ‘hot; ripe’ (but \X{-μᵉμH})
				\vbmorph{ka-&\rt{tʼaᴸ}&-μᵉμH&-tʼ&\gm{-áḵw}}
					{\xx{sro}&\rt{ripe}&\·\xx{var}&\·\xx{rep}&\·\xx{dprv}}
		\item	\fm{kax̱duḵéisʼáḵw} \~\ \fm{katḵéisʼáḵw} ‘quilt’
			from \fm{\rt{ḵa}} ‘stitch, sew’
				\vbmorph{ká&-x̱&du-&\rt[²]{ḵa}&-μμᵉH&-sʼ&\gm{-áḵw}}
					{\xx{hsfc}&\·\xx{pert}&\xx{ind.h.s}&\rt[²]{stitch}&\·\xx{var}&\·\xx{rep}&\·\xx{dprv}}
				 or \vbmorph{ká&-t&\rt[²]{ḵa}&-μμᵉH&-sʼ&\gm{-áḵw}}
				 	{\xx{hsfc}&\·\xx{pnct}&\rt[²]{stitch}&\·\xx{var}&\·\xx{rep}&\·\xx{dprv}}
		\item	\fm{tlʼeitáḵw} (prenominal adjective) ‘pure, honest’
			from unknown \fm{\rt{tlʼet}}
			probably from \fm{\rt{tlʼeᴴn}} ‘impure, unclean’ (but \X{-μμH})
				\vbmorph{\rt{tlʼet}&-μμL&\gm{-áḵw}}
					{\rt{impure}&\·\xx{var}&\·\xx{dprv}}
		\item	\fm{ḵutx̱ naaxʼáḵw} ‘genocide or extinction of a nation’
			from \fm{\rt{na}} ‘clan, nation’
				\vbmorph{ḵutx̱&\rt{na}&-μμL&-xʼ&-áḵw}
					{too.much&\rt{clan}&\·\xx{var}&\·\xx{pl}&\·\xx{dprv}}
		\item	\fm{laanáḵw} ‘barren female ruminant’
			from unknown \fm{\rt{lan}}
			possibly related to \fm{laaw} ‘penis’ or \fm{lʼaa} ‘breast’
				\vbmorph{\rt{lan}&-μμL&\gm{-áḵw}}
					{\rt{\xx{unkn}}&\·\xx{var}&\·\xx{dprv}}
		\item	\fm{shax̱dáḵw} \~\ \fm{shuḵdáḵ} ‘legendary man-eating shark’
			from unknown \fm{\rt{shax̱}} or \fm{\rt{shuḵ}}
			possibly related to \fm{sháḵʼw} \~\ \fm{shúḵʼ} ‘legendary wolf-like animal’
				\vbmorph{\rt{shax̱}&-μL&-t&\gm{-áḵw}}
					{\rt{\xx{unkn}}&\·\xx{var}&\·\xx{rep}&\·\xx{dprv}}
			\newline
		\item	\fm{tʼaawáḵ} ‘Canada goose’
			from \fm{tʼaaw} ‘feather’
				\vbmorph{\rt{tʼaw}&-μμL&\gm{-áḵ}}
					{\rt{feather}&\·\xx{var}&\·\xx{dprv}}
		\item	\fm{Tlʼanaxéedáḵw} ‘Property/Wealth Woman’
			from earlier \fm{Tlʼeinax̱xéedáḵw} \parencite[“ʟ!ê′nᴀxx̣ī′dᴀq” in][]{swanton:1909}
				\vbmorph{tlʼeiḵ&-náx̱&\rt{xit}&-μμH&\gm{-áḵw}}
					{finger&\·\xx{perl}&\rt{scratch}&\·\xx{var}&\·\xx{dprv}}
		\item	\fm{x̱aatlʼáḵw} ‘mouth ulcer’ from unknown \fm{\rt{x̱atlʼ}}
			possibly related to \fm{\rt[²]{x̱a}} ‘eat’ and/or \fm{x̱aatlʼ} ‘freshwater grass’
				\vbmorph{\rt{x̱atlʼ}&-μμH&\gm{-áḵw}}
					{\rt{\xx{unkn}}&\·\xx{var}&\·\xx{dprv}}
			\newline
			and \fm{washtux̱aatlʼáḵw} ‘cheek ulcer, cold sore’
				with \fm{wásh} ’cheek’ + \fm{tú} ‘inside’
		\end{itemize}
	\end{enumerate}

\item[am]\label{m:am}
	≡ \fm{a-m-}
	combination of argument marking \X{a-}
		and perfective \X{m-};
	compare \X{aw} ≡ \fm{a-w-}, \X{awu} ≡ \fm{a-wu-}, and \X{aawa} ≡ \fm{a-μʷ-wa-};
	because \fm{m-} only occurs in a syllable coda,
		forms like \fm[*]{amu} cannot occur,
		see \X{m-} for more details
	\begin{itemize}
	\item	\vbform{amsi.ée}{pfv}[tr, \fm{∅}, \fm{-μμH} act]{s/he/it cooked him/her/it}
			\vbmorph{\gm{a-}&\gm{m-}&s-&i-&\rt[¹]{.i}&-μμH}
				{\xx{3>3}&\xx{pfv}&\xx{csv}&\xx{stv}&\rt[¹]{cooked}&\·\xx{var}}
		\versus \vbform{awsi.ée}{pfv}{s/he/it cooked him/her/it}
			\vbmorph{a-&w-&s-&i-&\rt[¹]{.i}&-μμH}
				{\xx{3>3}&\xx{pfv}&\xx{csv}&\xx{stv}&\rt[¹]{cooked}&\·\xx{var}}
	\end{itemize}

\item[-án]\label{m:-án}
	restorative suffix, indicates restoration of a previous situation;
	possibly also occurs as part of intensifier \X{-chʼán} \~\ \X{-shán}
		but its contribution to the meaning in this is unclear;
	could be related to \X{-n} on the basis of similar sounds and word final
		positions, with \fm{-án} being an epenthesized version of \fm{-n},
		but there is no other evidence support this
		and the meaning of \fm{-n} is unclear;
	also either or both could be related to the instrumental postposition
		\fm{-n} \~\ \fm{teen} \~\ \fm{een}
		due to similar sound and position, but again there is no other evidence
	\begin{enumerate}
	\item	derivational suffix meaning ‘restore to previous situation’
			\parencites[56]{story:1966}[877]{crippen:2019};
		only attested with \fm{\rt{CVC}} roots, where it occurs with 
			\fm{-μμL} stem variation except for one \fm{-μμH}
			expected with \fm{\rt{CVCʼ}};
		high tone only, insensitive to tone of preceding syllable
		\begin{itemize}
		\item	goodán ‘return to walking’
		\item	\vbform{a tóox̱ yawdudzihaanán}{pfv}[tr, \fm{∅}?, inv ach?]{he was reelected}
			\parencites[56]{story:1966}[172.2372]{story-naish:1973}
				\vbmorph{a&tú&-x̱&ÿa-&w-&du-&d-&s-&i-&\rt[¹]{han}&-μμL&\gm{-án}}
					{\xx{3n}&inside&\·\xx{pert}&face&\xx{pfv}&\xx{ind.h.s}&\xx{mid}&\xx{csv}&\xx{stv}&\rt[¹]{stand·\xx{sg}}&\·\xx{var}&\·\xx{rest}}
			\versus	\vbform{daak wududzihán}{pfv}[tr, \fm{∅}, mot]{he was dropped from office}
			\parencite[56]{story:1966}
				\vbmorph{daak=&wu-&du-&d-&s-&i-&\rt[¹]{han}&-μH}
					{out=&\xx{pfv}&\xx{ind.h.s}&\xx{mid}&\xx{csv}&\xx{stv}&\rt[¹]{stand·\xx{sg}}&\·\xx{var}}
			\versus \vbform{wudihaan}{pfv}[subj intr, \fm{g}, mot]{she/he/it stood up}
			\parencite[40]{leer:1976}
				\vbmorph{wu-&d-&i-&\rt[¹]{han}&-μμL}
					{\xx{pfv}&\xx{mid}&\xx{stv}&\rt[¹]{stand·\xx{sg}}&\·\xx{var}}
		\item	\vbform{a tóox̱ yawdudzinaag̱án}{pfv}[tr, \fm{∅}?, inv ach?]{they reelected them}
			\parencite[172.2373]{story-naish:1973}
				\vbmorph{a&tú&-x̱&ÿa-&w-&du-&d-&s-&i-&\rt[¹]{naḵ}&-μμL&\gm{-án}}
					{\xx{3n}&inside&\·\xx{pert}&face&\xx{pfv}&\xx{ind.h.s}&\xx{mid}&\xx{csv}&\xx{stv}&\rt[¹]{stand·\xx{pl}}&\·\xx{var}}
			\versus \vbform{has wudinaaḵ}{pfv}[subj intr, \fm{g}, mot]{they stood up}
			\parencite[209.2928]{story-naish:1973}
				\vbmorph{has=&wu-&d-&i-&\rt[¹]{naḵ}&-μμL}
					{\xx{plh}&\xx{pfv}&\xx{mid}&\xx{stv}&\rt[¹]{stand·\xx{pl}}&\·\xx{var}}
		\item	taanán ‘reset bone’
		\item	xeexán ‘normalize’
		\item	x̱áatʼán ‘snark, sarcasm’
		\item	\vbform{a yíx̱ guyawjix̱eenán}{pfv}[obj intr, \fm{∅}?, inv ach?]{it got back into joint}
			\parencites[56]{story:1966}[99.1262]{story-naish:1973}
				\vbmorph{a&yíᵏ&-x̱&gu-&ÿa-&w-&d-&sh-&i-&\rt[¹]{x̱in}&-μμL&\gm{-án}}
					{\xx{3n}&within&\·\xx{pert}&base&\xx{qual}&\xx{pfv}&\xx{mid}&\xx{pej}&\xx{stv}&\rt[¹]{fall·w/e}&\·\xx{var}&\·\xx{rest}}
			\versus \vbform{aax̱ kei guwjix̱ín}{pfv}[obj intr, \fm{∅}, ach]{he dislocated it}
			\parencites[56]{story:1966}[70.841]{story-naish:1973}
				\vbmorph{a&-μx̱&kei=&gu-&w-&d-&sh-&i-&\rt[¹]{x̱in}&-μH}
					{\xx{3n}&\·\xx{abl}&up&base&\xx{pfv}&\xx{mid}&\xx{pej}&\xx{stv}&\rt[¹]{fall·w/e}&\·\xx{var}}
			\versus \vbform{wujix̱een}{pfv}[obj intr, \fm{n}, ach]{it (container, board) fell}
			\parencite[796]{leer:1976}
				\vbmorph{wu-&d-&sh-&i-&\rt[¹]{x̱in}&-μμL}
					{\xx{pfv}&\xx{mid}&\xx{pej}&\xx{stv}&\rt[¹]{fall·w/e}&\·\xx{var}}
		\end{itemize}
	\end{enumerate}	

\item[-ani]\label{m:-ani}
	suffix with unknown meaning

\item[as=]\label{m:as=}
	allomorph of human pluralizer \fm{has=} for third person subject or object;
	mostly occurs in Southern \&\ Tongass varieties
	\begin{itemize}
	\item	\vbform{as dustaaÿch}{hab}[tr, \fm{∅}, ach]{they would boil it}
		(Tongass dialect) \parencite[24.80]{leer:1978}
		\vbmorph{\gm{as=}&du-&d-&s-&\rt[¹]{taᴸ}&-μμ&-ÿ&-ch}
			{\xx{plh}&\xx{ind.h.s}&\xx{mid}&\xx{csv}&\rt[¹]{boil}&\·\xx{var}&\·\xx{ÿsfx}&\·\xx{rep}}
		\versus \vbform{has dustáaych}{hab}{they would boil it} (Northern dialect)
		\vbmorph{has=&du-&d-&s-&\rt[¹]{taᴸ}&-μμH&-ÿ&-ch}
			{\xx{plh}&\xx{ind.h.s}&\xx{mid}&\xx{csv}&\rt[¹]{boil}&\·\xx{var}&\·\xx{ÿsfx}&\·\xx{rep}}
	\end{itemize}

\item[-ás]\label{m:-ás}
	allomorph of unknown suffix \X{-s} with epenthetic (filler) vowel \fm{á};
	only attested with the root \fm{\rt[²]{ḵe}} \~\ \fm{\rt[²]{ḵi}} ‘pay’
		so its meaning is unknown;
	this allomorph is found in most varieties of Tlingit together with \X{-n} as \X{-nás},
		but in Tongass Tlingit the \X{-s} allomorph is used without \fm{-n}
	\begin{itemize}
	\item	\vbform{a káxʼ aawaḵéinás}{pfv}[tr, \fm{n}, ach]{she/he/it asked him/her/it for it in exchange}
		\parencites[82.1003]{story-naish:1973}[868]{leer:1973}
			\vbmorph{a-&μʷ-&wa-&\rt[²]{ḵe}&-μμH&-n&\gm{-ás}}
				{\xx{3>3}&\xx{pfv}&\xx{stv}&\rt[²]{pay}&\·\xx{var}&\·\xx{nsfx}&\·\xx{unkn}}
		\versus \vbform{aawaḵéi}{pfv}[tr, \fm{n}, ach]{she/he/it paid him/her/it}
			\vbmorph{a-&μʷ-&wa-&\rt[²]{ḵe}&-μμH}
				{\xx{3>3}&\xx{pfv}&\xx{stv}&\rt[²]{pay}&\·\xx{var}}
	\end{itemize}

\item[-ásʼ]\label{m:-ásʼ}
	allomorph of repetitive \X{-sʼ} with epenthetic (filler) vowel \fm{á};
	attested only in one verb derived from the noun \fm{x̱aanásʼ} ‘raft’
		but may be identified in some other nouns;
	in all cases its meaning is unknown but presumably originally repetitive
	\begin{enumerate}
	\item	in the noun \fm{x̱aanásʼ} ‘raft’ and in verbs derived from this noun
		\begin{itemize}
		\item	\fm{x̱aanásʼ} ‘raft’ from \fm{\rt{x̱a}} ‘paddle (canoe)’
			\vbmorph{\rt{x̱a}&-μμL&-n&\gm{-ásʼ}}
				{\rt{paddle}&\·\xx{var}&\·\xx{nsfx}&\·\xx{unkn}}
		\item	\vbform{awdlix̱aanásʼ}{pfv}[tr, conj?, ach?]{she/he/it made him/her/it into a raft}
			\parencite[787]{leer:1976}
			\vbmorph{a-&w-&d-&lˢ-&i-&\rt[²]{x̱a}&-μμL&-n&\gm{-ásʼ}}
				{\xx{3>3}&\xx{pfv}&\xx{mid}&\xx{xtn}&\xx{stv}&\rt[²]{paddle}&\·\xx{var}&\·\xx{nsfx}&\·\xx{unkn}}
		\item	\vbform{yan awtudlix̱aanásʼ}{pfv}{we went by raft}
			\parencite[165.2271]{story-naish:1973}
			\vbmorph{ÿan=&a-&w-&tu-&d-&lˢ-&i-&\rt[²]{x̱a}&-μμL&-n&\gm{-ásʼ}}
				{\xx{term}&\xx{xpl}&\xx{pfv}&\xx{1pl.s}&\xx{mid}&\xx{xtn}&\xx{stv}&\rt[²]{paddle}&\·\xx{var}&\·\xx{nsfx}&\·\xx{unkn}}
		\end{itemize}
	\item	possibly identifiable in two nouns with unclear etymology
		\begin{itemize}
		\item	\fm{kʼoodásʼ} ‘shirt, tunic’ from unknown \fm{\rt{kʼut}}
			\vbmorph{\rt{kʼut}&-μμL&\gm{-ásʼ}}
				{\rt{\xx{unkn}}&\·\xx{var}&\·\xx{unkn}}
			\newline
			perhaps related to \fm{\rt{kʼut}} ‘bounce back, rebound’
				but the composition of meaning is unclear
		\item	\fm{taagwásʼ} ‘sunray skate’ perhaps from \fm{\rt{takw}} ‘winter’
			\vbmorph{\rt{takw}&-μμL&\gm{-ásʼ}}
				{\rt{winter}&\·\xx{var}&\·\xx{unkn}}
		\end{itemize}
	\end{enumerate}

\item[ash=]\label{m:ash=}
	object proclitic;
	probably derived from third person pronoun \fm{á} + reflexive pronoun \fm{sh}
		or ergative \fm{-ch};
		see also \X{ach=} used instead of \fm{ash=} by some speakers;
	\newline
	allomorphs:
	\begin{allolist}
	\item[ash=]	basic form
	\item[\X{ach=}]	affricate \fm{ch} instead of fricative \fm{sh}
	\end{allolist}
	\begin{enumerate}
	\item	third person proximate human object
	\item	special reflexive object
	\end{enumerate}

\item[at=]\label{m:at=}
	indefinite nonhuman object ‘something, stuff’;
	derived from the noun \fm{át} ‘thing’ (as in \fm{wé át} ‘that thing’);
	see also \fm{a-} used instead of \fm{at=} in some verbs
	\begin{itemize}
	\item	\vbform{at wutusiteen}{pfv}[tr, \fm{g̱}, ach]{we saw something}
			\vbmorph{\gm{at=}&wu-&tu-&s-&i-&\rt[²]{tin}&-μμL}
				{\xx{ind.n.o}&\xx{pfv}&\xx{1pl.s}&\xx{xtn}&\xx{stv}&\rt[²]{see}&\·\xx{var}}
		\versus \vbform{wutusiteen}{pfv}{we saw him/her/it}
			\vbmorph{&wu-&tu-&s-&i-&\rt[²]{tin}&-μμL}
				{&\xx{pfv}&\xx{1pl.s}&\xx{xtn}&\xx{stv}&\rt[²]{see}&\·\xx{var}}
	\item	\vbform{at wusiteen}{pfv}[tr, \fm{g̱}, ach]{s/he/it saw something}
			\vbmorph{\gm{at=}&wu-&s-&i-&\rt[²]{tin}&-μμL}
				{\xx{ind.n.o}&\xx{pfv}&\xx{xtn}&\xx{stv}&\rt[²]{see}&\·\xx{var}}
		\versus \vbform{awsiteen}{pfv}{s/he/it saw him/her/it}
			\vbmorph{a-&w-&s-&i-&\rt[²]{tin}&-μμL}
				{\xx{3>3}&\xx{pfv}&\xx{xtn}&\xx{stv}&\rt[²]{see}&\·\xx{var}}
	\end{itemize}

\item[-át]\label{m:-át}
	suffix with unknown meaning,
		attested only in one verb derived from a noun,
		ultimately from an unknown root;
	possibly related to \X{-t} but this is only speculation;
	could instead be a suffix \fm{-kʼát} but that is also unattested elsewhere,
		though there is a poorly documented particle \fm{kʼát}
		in the phrase \fm{chʼa kʼát} ‘at least’
		which unfortunately does not shed much light on this case;
	another alternative parse is diminutive \fm{-kʼ} + \fm{-át}
		but this also has an unclear meaning;
	sometimes instead attested as \X{-átʼ}
		which may be remodeled by analogy with \X{-tʼ} and 
		and the alternation between \X{-sʼ} \~\ \X{-ásʼ}
	\begin{itemize}
	\item	verb stem \fm{–tlʼéekʼát} ‘thread stick through to stiffen’
		from noun \fm{tlʼéekʼát} ‘barbecue crosspiece’
		from unknown \fm{\rt{tlʼikʼ}}
		possibly related to \fm{\rt{tlʼiᴴn}} ‘tie hair, cloth’
		\newline
		\vbform{aawatlʼéekʼát}{pfv}[tr, conj?, ach?]{she/he/it threaded sticks through him/her/it to stiffen}
		\parencite[08/253]{leer:1973}
			\vbmorph{a-&μʷ-&wa-&\rt{tlʼikʼ}&-μμH&\gm{-át}}
				{\xx{3>3}&\xx{pfv}&\xx{stv}&\rt{thread.thru?}&\·\xx{var}&\·\xx{unkn}}
	\end{itemize}

\item[-átʼ]\label{m:-átʼ}
	possibly an allomorph of repetitive \X{-tʼ} with epenthetic (filler) vowel \fm{á};
	attested only in one verb derived from a noun,
		ultimately from an unknown root;
	sometimes instead attested as \X{-át};
	given the rarity of this form and its variation, ejective \fm{-átʼ}
		may be remodeled from \fm{-át} by analogy with \fm{-tʼ} 
		and the alternation between \X{-sʼ} \~\ \X{-ásʼ},
		suggesting that the \fm{-át} form is more conservative;
	see \X{-át} for more discussion
	\begin{itemize}
	\item	verb stem \fm{–tlʼéekátʼ} ‘thread stick through to stiffen’
		from noun \fm{tlʼéekátʼ} ‘barbecue crosspiece’
		from unknown \fm{\rt{tlʼik}}
		possibly related to \fm{\rt{tlʼiᴴn}} ‘tie hair, cloth’
		\newline
		\vbform{wutuwatlʼéekátʼ}{pfv}[tr, conj?, ach?]{we threaded sticks through it}
		\parencite[227.3206]{story-naish:1973}
			\vbmorph{wu-&tu-&wa-&\rt{tlʼik}&-μμH&\gm{-átʼ}}
				{\xx{pfv}&\xx{1pl.s}&\xx{stv}&\rt{thread.thru?}&\·\xx{var}&\·\xx{unkn}}
	\end{itemize}

\item[aw]\label{m:aw}
	≡ \fm{a-w-}
	combination of argument marking \X{a-}
		and perfective \X[w-pfv]{w-};
	compare \X{awu} ≡ \fm{a-wu-} and \X{aawa} ≡ \fm{a-μʷ-wa-}
	\begin{itemize}
	\item	\vbform{awsi.ée}{pfv}[tr, \fm{∅}, \fm{-μμH} act]{s/he/it cooked him/her/it}
			\vbmorph{\gm{a-}&\gm{w-}&s-&i-&\rt[¹]{.i}&-μμH}
				{\xx{3>3}&\xx{pfv}&\xx{csv}&\xx{stv}&\rt[¹]{cooked}&\·\xx{var}}
	\end{itemize}

\item[awu]\label{m:awu}
	≡ \fm{a-wu-}
	combination of argument marking \X{a-}
		and perfective \X{wu-};
	occurs where stative \fm{ÿa-} \~\ \fm{i-} is suppressed
		such as in negative, past tense, and subordinate clause forms;
	compare \X{aawa} ≡ \fm{a-μʷ-wa-} and \X{aw} ≡ \fm{a-w-}
	\begin{itemize}
	\item	\vbform{tléil awux̱á}{neg pfv}[tr, \fm{∅}, \fm{-μH} act]{she/he/it didn’t eat him/her/it}
			\vbmorph{tléil&\gm{a-}&\gm{wu-}&\rt[²]{x̱a}&-μH}
				{\xx{neg}&\xx{3>3}&\xx{pfv}&\rt[²]{eat}&\·\xx{var}}
		\versus \vbform{aawax̱áa}{pfv}{she/he/it ate him/her/it}
			\vbmorph{a-&μʷ-&wa-&\rt[²]{x̱a}&-μμH}
				{\xx{3>3}&\xx{pfv}&\xx{stv}&\rt[²]{eat}&\·\xx{var}}
	\item	\vbform{awux̱áayin}{past pfv}{she/he/it had eaten him/her/it}
			\vbmorph{\gm{a-}&\gm{wu-}&\rt[²]{x̱a}&-μμH&-yin}
				{\xx{3>3}&\xx{pfv}&\rt[²]{eat}&\·\xx{var}&\·\xx{past}}
	\item	\vbform{awux̱áayi}{sub pfv}{while/when she/he/it had eaten him/her/it}
			\vbmorph{\gm{a-}&\gm{wu-}&\rt[²]{x̱a}&-μμH&-yi}
				{\xx{3>3}&\xx{pfv}&\rt[²]{eat}&\·\xx{var}&\·\xx{sub}}
	\end{itemize}

\item[ax̱]\label{m:ax̱}
	≡ \fm{a-x̱-}
	combination of argument marking \X{a-}
		and either first person singular subject \X[x̱-1sg]{x̱-}
			or \fm{g̱} conjugation \X[x̱-g̱cnj]{x̱-}
			or modal \X[x̱-mod]{x̱-}

\item[ax̱=]\label{m:ax̱=}
	allomorph ‘my’ of \fm{x̱at=} ‘me’ first person singular object,
		only used as possessor of incorporated nouns;
	derived from possessive pronoun \fm{ax̱} ‘my’ (compare \fm{ax̱ keidlí áwé} ‘it is my dog’);
	some speakers disprefer \fm{ax̱=} in verbs and only use \fm{x̱at=}
	\begin{itemize}
	\item	\vbform{ax̱ shalxáash}{impfv}[tr, \fm{n}, \fm{-μμH} act]{she/he/it is cutting my hair}
			\vbmorph{\gm{ax̱=}&sha-&l-&\rt[²]{xash}&-μμH}
				{\xx{1sg.o}&head&\xx{xtn}&\rt[²]{cut}&\·\xx{var}}
		\versus 	\vbform{x̱at shalxáash}{impfv}{she/he/it is cutting my hair}
			\vbmorph{x̱at=&sha-&l-&\rt[²]{xash}&-μμH}
				{\xx{1sg.o}&head&\xx{xtn}&\rt[²]{cut}&\·\xx{var}}
	\end{itemize}

\item[aÿ]\label{m:aÿ-a-ÿ}
	≡ \fm{a-ÿ-}
	combination of argument marking \X{a-}
		and second person plural subject \X[ÿ-2pl]{ÿ-};
	distinct from \X[aÿ-a-ʷ-ÿ]{aÿ} ≡ \fm{a-ʷ-ÿ-}
		which has perfective \X[ʷ-pfv]{ʷ-}
		and second person \emph{singular} subject \X[ÿ-2sg]{ÿ-}

\item[aÿ]\label{m:aÿ-a-ʷ-ÿ}
	≡ \fm{a-ʷ-ÿ-}
	combination of argument marking \X{a-}
		and perfective \X[ʷ-pfv]{ʷ-}
		and second person singular subject \X[ÿ-2sg]{ÿ-};
	distinct from \X[aÿ-a-ÿ]{aÿ} ≡ \fm{a-ÿ-}
		which has second person \emph{plural} subject \X[ÿ-2pl]{ÿ-};
\end{morphdesc}
