%!TEX root = ../lingnote-verbmorphs.tex

\subsection{Y}\label{sec:alphalist-y}
\begin{morphdesc}[resume*=alphalist]
\item[ÿ-]\label{m:ÿ-2sg}
	allomorph of second person singular subject \X[i-2sg]{i-}

\item[ÿ-]\label{m:ÿ-2pl}
	allomorph of second person plural subject \X{ÿi-}
	
\item[ÿ-]\label{m:ÿ-pfv}
	allomorph of perfective \X{wu-}

\item[ÿ-]\label{m:ÿ-face}
	allomorph of incorporated noun \X[ÿa-face]{ÿa-} ‘face’

\item[ÿ-]\label{m:ÿ-qual}
	allomorph of qualifier \X[ÿa-qual]{ÿa-} of unknown meaning

\item[-ÿ]\label{m:-ÿ}
	sonorant suffix of unknown meaning

\item[ÿa-]\label{m:ÿa-stv}
	allomorph of stative \X[i-stv]{i-}
	\begin{itemize}
	\item	\vbform{yadál}{impfv}[obj intr, \fm{g}, \fm{-μH} state]{she/he/it is heavy}
			\vbmorph{\gm{ÿa-}&\rt[¹]{dal}&-μH}
				{\xx{stv}&\rt[¹]{heavy}&\·\xx{var}}
		\versus \vbform{si.áatʼ}{impfv}[obj intr, \fm{g}, \fm{-μμH} state]{she/he/it is cold}
			\vbmorph{s-&i-&\rt[⁰]{.atʼ}&-μμH}
				{\xx{intr}&\xx{stv}&\rt[⁰]{cold}&\·\xx{var}}
	\end{itemize}

\item[ÿa-]\label{m:ÿa-face}
	incorporated noun indicating vertical surface or face,
	derived from the relational noun \fm{ÿá} ‘face’;
	can occur together with qualifier \X[ÿa-qual]{ÿa-} of unknown meaning

\item[ÿa-]\label{m:ÿa-qual}
	qualifier of unknown meaning;
	can occur together with incorporated noun \X[ÿa-face]{ÿa-} ‘face’

\item[ÿaa=]\label{m:ÿaa=along}
	directional preverb indicating progression or movement along a space
	(compare \fm{\rt[²]{ÿa}} ‘move’,
		directional noun \fm{diÿáa} ‘across, other side’,
		\fm{niÿaa} ‘direction’);
	\begin{enumerate}
	\item	progression, used in progressive aspect for \fm{∅} and \fm{n} conjugation class verbs
		\begin{itemize}
		\item	\fm{yaa x̱at nalnítl} (prog; obj intr, \fm{∅}, ach) ‘I am getting fat’\newline
			versus \fm{x̱at wudlinítl} (pfv) ‘I got fat’
		\end{itemize}
	\item	movement along a space
		\begin{enumerate}
		\item	motion derivation
				\fm{ÿaa} (\fm{g̱}, \fm{yei=…-ch} rep) ‘down along’
				(\fm{yei} in repetitive blocks \fm{ÿaa})
		\item	motion derivation
				\fm{ÿaa} \~\ \fm{ÿa-u-} (\fm{∅}, \fm{-ch} rep) ‘obliquely, circuitously’
		\end{enumerate}
	\end{enumerate}

\item[ÿaa=]\label{m:ÿaa=mind}
	preverb indicating mental phenomenon, limited to a couple of verbs;
	uncertain if it can occur together with \fm{ÿaa} ‘along’;
	possibly related to Proto-Dene \fm[*]{yən-} \~\ \fm[*]{yiːn-} ‘mind’ and Eyak \fm{ʔiːlih} ‘mind’
	\begin{itemize}
	\item	\fm{yaa ḵux̱dzigéi} (impfv; subj intr, \fm{g}, \fm{-μμH} state) ‘I am smart, wise’
	\item	\fm{yaa aḵoowlig̱át} (pfv; tr, \fm{∅}, ach) ‘s/he/it forgot him/her/it’
	\end{itemize}

\item[ÿaan=]
	incorporated noun indicating hunger,
	saturates object argument;
	derived from noun \fm{ÿaan} ‘hunger’ (now rare)
	\begin{itemize}
	\item	\fm{ax̱ éet yaan uwaháa} (pfv; obj intr, \fm{∅}, mot) ‘hunger appeared to me’ (i.e.\ ‘I got hungry’)\newline
		(not \fm[*]{yaan ax̱ éet uwaháa})
	\end{itemize}

\item[ÿan=]
	directional preverb indicating motion to shore, motion to ground, or termination;
	allomorphs are \fm{ÿax̱} and \fm{ÿánde}:
		\fm{ÿax̱} is used with repetitive,
		\fm{ÿánde} with progressive and prospective,
		and \fm{ÿan} elsewhere (e.g.\ pfv, imp);
	morphologically a specialization of the
		\fm{NP-\{t,x̱,dé\}} (\fm{∅}, \fm{-μμL} rep) ‘arriving at NP’
		motion derivation,
	so the \fm{ÿan} probably used to end with \fm{-t} ‘to a point’ punctual postposition
		as \fm[*]{ÿant};
	derived from noun \fm{ÿán} ‘shore’
		(< Pre-Tlingit \fm[*]{ŋanʰ} < Proto-Na-Dene \fm[*]{ŋənˀ} ‘ground, earth’)
	\begin{enumerate}
	\item	motion on water to shore,
		can be translated ‘ashore’;
		motion derivation
			\fm{ÿan} / \fm{yax̱} / \fm{ÿánde} (\fm{∅}, \fm{-μμL} rep) ‘ashore’
	\item	motion to ground or other horizontal surface,
		can be translated ‘down’ or ‘on ground’;
		motion derivation
			\fm{ÿan} / \fm{yax̱} / \fm{ÿánde} (\fm{∅}, \fm{-μμL} rep) ‘on ground’
			optionally with incorporates (\fm{kʼi-} ‘base’ for ‘setting up, erecting’,
			\fm{sha-} ‘head’ for ‘leaning against’)
	\item	termination of eventuality,
		can be translated ‘ending, terminating, finishing’;
		eventuality/motion derivation
			\fm{ÿan} \~\ \fm{yax̱} \~\ \fm{ÿánde} (\fm{∅}, \fm{-μμL} rep) ‘ending, finishing’
			optionally with \fm{NP-xʼ} ‘coming to rest at NP’;
		derives from metaphor of ‘shore’ as ‘end of journey’ and thus ‘end of event’
	\end{enumerate}

\item[ÿánde=]
	allomorph of directional preverb \fm{ÿan} ‘ashore’ or ‘ending’
	with allative postposition \fm{-dé} \~\ \fm{-de} ‘toward’
	\begin{itemize}
	\item	\fm{yánde gax̱tooḵóox̱} (prosp; subj intr, \fm{∅}, mot) ‘we are going to boat ashore’\newline
		versus \fm{yan wutuwaḵúx̱} (pfv) ‘we boated ashore’
	\end{itemize}

\item[ÿata=]
	incorporated noun ‘sleep’,
	saturates object argument;
	apparently derived from \fm{ÿá} ‘face’ and \fm{\rt[¹]{taᴸ}} ‘sg.\ sleep’
	\begin{itemize}
	\item	\fm{ax̱ éet yataawaháa} (pfv; obj intr, \fm{∅}, mot) ‘sleep appeared to me’, i.e. ‘I got sleepy’
	\item	\fm{ax̱ yaadáx̱ yataawahaa} (pfv; obj intr, \fm{g̱}, mot) ‘sleep disappeared from my face’,
		i.e.\ ‘I became wakeful’
	\end{itemize}

\item[ÿax̱=]\label{m:ÿax̱=ashore}
	allomorph of directional preverb \fm{ÿan} ‘ashore’ or ‘ending’ (group E)
	with perlative postposition \fm{-x̱} ‘contacting’;
	used only with repetitive versus \fm{ÿánde} (prog, prosp) or \fm{ÿan} (pfv, imp, etc.)
	\begin{itemize}
	\item	\fm{yax̱ tooḵoox̱} (rep impfv; subj intr, \fm{∅}, mot) ‘we repeatedly boat ashore’\newline
		versus \fm{yan wutuwaḵúx̱} (pfv) ‘we boated ashore’
	\end{itemize}

\item[ÿax̱=]\label{m:ÿax̱=exh}
	manner preverb ‘using up, completely’ occuring as part of exhaustive derivation
		\motderiv{ÿax̱= + ÿa- + s-}{\fm{∅}, \fm{-x̱} rep}{affecting all of, using up};
	possibly could be equated with \X[ÿax̱=facing]{ÿax̱=} ‘facing’
		but the two meanings seem to be distinct;
	not directly related to \X[ÿax̱=ashore]{ÿax̱=} which is instead from the noun \fm{ÿán} ‘shore’

\item[ÿee-]
	allomorph of second person singular subject \fm{ÿi-}

\item[ÿee=]
	allomorph of second person singular object \fm{ÿi-}

\item[ÿee=]\label{m:ÿee=time}
	incorporated noun indicating time;
	may or may not saturate the object argument of the verb;
	derived from a no longer independent noun \fm{ÿee} ‘time’
	that can also be identified in some nouns, adjectives, and adverbs
	including
		\begin{inlinelist}
		\item	\fm{hóochʼeenís} ‘for the last time’
		\item	\fm{ḵinxʼiyís} ‘just in case’
		\item	\fm{niyís} ‘in preparation for (time)’
		\item	\fm{yagiÿee} \~\ \fm{yakÿee} \~\ \fm{yagee} ‘day’
		\item	\fm{ÿeedát} ‘moment; now’
		\item	\fm{ÿeen} ‘during, in the middle of (duration)’
		\item	\fm{ÿées} ‘new, young’
		\item	\fm{ÿéeÿi} ‘former; subordinate clause past tense’
		\item	\fm{yeis} ‘autumn, fall’
		\item	\fm{ÿeisú} ‘still, yet, recently’
		\end{inlinelist}
	\begin{itemize}
	\item	\vbform{yeeyayátʼ}{impfv}[obj intr, \fm{n}, \fm{-μH} state]{it is a long time}
			\vbmorph{ÿee=&ÿa-&\rt[¹]{ÿatʼ}&-μH}
				{time&\xx{stv}&\rt[¹]{long}&\·\xx{var}}
		\versus \vbform{yayátʼ}{impfv}[obj intr, \fm{n}, \fm{-μH} state]{she/he/it is long}
			\vbmorph{ÿa-&\rt[¹]{ÿatʼ}&-μH}
				{\xx{stv}&\rt[¹]{long}&\·\xx{var}}
	\end{itemize}

\item[ÿee]
	≡ \fm{wu-i-μ}
	combination of perfective \fm{wu-}
		and second person singular subject \fm{i-}
		and stative \fm{μ-}

\item[ÿeeÿ]
	second person plural subject \fm{ÿi-} combined with either one or both of
		perfective \fm{wu-}
		and stative \fm{ÿa-} \~\ \fm{i-}
	\begin{enumerate}
	\item	\fm{ÿeeÿ} ≡ \fm{ÿi-ÿa-}
		with stative \fm{ÿa-}
	\item	\fm{ÿeeÿ} ≡ \fm{wu-ÿi-}
		with perfective \fm{wu-}
	\item	\fm{ÿeeÿ} ≡ \fm{wu-ÿi-ÿa-}
		with perfective \fm{wu-}
		and stative \fm{ÿa-}
	\item	\fm{ÿeeÿsi} ≡ \fm{ÿi-s-i-}
		with valency \fm{s-}
			(or \fm{ÿeeÿli} \fm{l-} or \fm{ÿeeÿshi} \fm{sh-})
		and stative \fm{i-}
	\item	\fm{ÿeeÿsi} ≡ \fm{wu-ÿi-s-i-}
		with perfective \fm{wu-}
		and valency \fm{s-}
			(or \fm{ÿeeÿli} \fm{l-} or \fm{ÿeeÿshi} \fm{sh-})
		and stative \fm{i-}
	\end{enumerate}

\item[ÿeeḵ=]
	directional preverb ‘beach’, variant forms \fm{ÿeiḵ=} and \fm{eèḵ=};
	derived from noun \fm{éeḵ} \~\ \fm{éiḵ} ‘beach’;
	compare \fm{éeg̱i=} \~\ \fm{éig̱i=}

\item[yei=]\label{m:yei=}
	direction preverb ‘down’;
	may reflect \fm{g̱} conjugation class or a \fm{∅} conjugation class motion derivation;
	part of directional element paradigm of \fm{\rt{ÿiⁿ}} ‘down’:
		\fm{(di)ÿée} ‘below’, \fm{(di)ÿín-de} ‘to below’, \fm{(di)ÿee-naa} ‘downward’;
	related to \fm{ÿee} ‘beneath, below’
	\begin{enumerate}
	\item	reflects \fm{g̱} conjugation class in prospective, progressive, and repetitive imperfective
	\item	\fm{g̱} conjugation class motion derivation
	\item	\fm{∅} conjugation class motion derivation
	\end{enumerate}

\item[yéi=]
	manner preverb ‘thus, so’;
	derived from noun \fm{yéi} \~\ \fm{yé} ‘place, way, manner’

\item[ÿeiḵ=]\label{m:ÿeiḵ=}
	variant form of directional preverb \fm{ÿeeḵ=} ‘beach’ used in some Northern varieties
	arises from uvular lowering of \fm{ée} to \fm{éi};
	derived from noun \fm{éeḵ} \~\ \fm{éiḵ} ‘beach’

\item[ÿi-]\label{m:ÿi-}
	second person plural subject or object; long vowel allomorphs are \fm{ÿee-} and \fm{ÿee=}
	\begin{enumerate}
	\item	second person plural subject
	\item	second person plural object
	\end{enumerate}

\item[ÿi]
	≡ \fm{wu-i-}
	combination of perfective \fm{wu-} and
		second person singular subject \fm{i-}

\item[-ÿi]\label{m:-ÿi-rel}
	allomorph of relative clause \fm{-i}

\item[-ÿi]\label{m:-ÿi-sub}
	allomorph of subordinate \fm{-i}

\item[ÿu-]\label{m:ÿu-}
	abstract representation of perfective \fm{wu-};
	this form does not actually occur in speech, instead see
		\fm{wu-}, \fm{w-}, \fm{m-}, \fm{μʷ-} \fm{ÿi}, \fm{ÿee}, \fm{ÿeeÿ}

\item[yoo=]
	alternating eventuality preverb

\item[yóo=]
	quotative preverb

\end{morphdesc}
