%!TEX root = ../lingnote-verbmorphs.tex

\subsection{Y}\label{sec:alphalist-y}
\begin{morphdesc}[resume*=alphalist]
\item[ÿ-]\label{m:ÿ-2sg}
	allomorph of second person singular subject \X[i-2sg]{i-}

\item[ÿ-]\label{m:ÿ-2pl}
	allomorph of second person plural subject \X{ÿi-}
	
\item[ÿ-]\label{m:ÿ-pfv}
	allomorph of perfective \X{wu-}

\item[ÿ-]\label{m:ÿ-face}
	allomorph of incorporated noun \X[ÿa-face]{ÿa-} ‘face’

\item[ÿ-]\label{m:ÿ-qual}
	allomorph of qualifier \X[ÿa-qual]{ÿa-} of unknown meaning

\item[-ÿ]\label{m:-ÿ}
	sonorant suffix of unknown meaning

\item[ÿa-]\label{m:ÿa-stv}
	allomorph of stative \X[i-stv]{i-}
	\begin{itemize}
	\item	\vbform{yadál}{impfv}[obj intr, \fm{g}, \fm{-μH} state]{she/he/it is heavy}
			\vbmorph{\gm{ÿa-}&\rt[¹]{dal}&-μH}
				{\xx{stv}&\rt[¹]{heavy}&\·\xx{var}}
		\versus \vbform{si.áatʼ}{impfv}[obj intr, \fm{g}, \fm{-μμH} state]{she/he/it is cold}
			\vbmorph{s-&i-&\rt[⁰]{.atʼ}&-μμH}
				{\xx{intr}&\xx{stv}&\rt[⁰]{cold}&\·\xx{var}}
	\end{itemize}

\item[ÿa-]\label{m:ÿa-face}
	incorporated noun indicating vertical surface or face,
	derived from the relational noun \fm{ÿá} ‘face’;
	can occur together with qualifier \X[ÿa-qual]{ÿa-} of unknown meaning

\item[ÿa-]\label{m:ÿa-qual}
	qualifier of unknown meaning;
	can occur together with incorporated noun \X[ÿa-face]{ÿa-} ‘face’

\item[ÿaa=]\label{m:ÿaa=along}
	directional preverb indicating progression or movement along a space
	(compare \fm{\rt[²]{ÿa}} ‘move’,
		directional noun \fm{diÿáa} ‘across, other side’,
		\fm{niÿaa} ‘direction’);
	\begin{enumerate}
	\item	progression, used in progressive aspect for \fm{∅} and \fm{n} conjugation class verbs
		\begin{itemize}
		\item	\fm{yaa x̱at nalnítl} (prog; obj intr, \fm{∅}, ach) ‘I am getting fat’\newline
			versus \fm{x̱at wudlinítl} (pfv) ‘I got fat’
		\end{itemize}
	\item	movement along a space
		\begin{enumerate}
		\item	motion derivation
				\fm{ÿaa} (\fm{g̱}, \fm{yei=…-ch} rep) ‘down along’
				(\fm{yei} in repetitive blocks \fm{ÿaa})
		\item	motion derivation
				\fm{ÿaa} \~\ \fm{ÿa-u-} (\fm{∅}, \fm{-ch} rep) ‘obliquely, circuitously’
		\end{enumerate}
	\end{enumerate}

\item[ÿaa=]\label{m:ÿaa=mind}
	Preverb uniquely associated with verbs that describe mental phenomena.
	This is glossed as ‘mental’ \xx{ment} although it has no known uses outside of the
		small set of verbs listed below where it is required.
	It does not vary in form and its presence is independent of grammatical aspect, negation,
		or any other verb inflection or clause modification.
	According to its apparent meaning and its distribution among verbs it seems more like
		an incorporated noun than a preverb, but its linear position is always before
		the object prefix so it is not in the same position as the incorporated nouns;
		\textcite[135]{leer:1991} supposes that it was moved from the incorporated noun
		position to the preverb position.
	The \fm{ÿaa=} ‘mental’ preverb can occur together with the homophonous directional preverb
		\X[ÿaa=along]{ÿaa=} ‘along’
		as shown by the form 
		\vbform{ÿaa ÿaa ḵunx̱alg̱át}{prog}{I am wandering along lost, in a daze}
		which is detailed in \ref{item:ÿaa=ment-confusion} below.
	See \X[tu-inside]{tu-} ‘inside; mind’ for the more common representation of the mind
		as an incorporated noun in verbs.
	\begin{enumerate}
	\item	\label{item:ÿaa=ment-smart}
		Verbs of intelligence based on \fm{\rt{ge}} ‘smart, wise, intelligent’.
		All documented verbs based on this root include both \fm{ÿaa=} ‘mental’
			and areal \X[ḵu-areal]{ḵu-} in all forms,
			together suggesting some kind of mental space.
		\begin{itemize}
		\item	\vbform{yaa ḵeedzigéi}{impfv}[subj intr, \fm{g}, \fm{-μμH} state]{you (sg) are smart, wise}
			\parencites[116.1513, 1514]{story-naish:1973}[27]{leer:1963}[f05/63]{leer:1973}[654]{leer:1976}
				\vbmorph{\gm{ÿaa=}&ḵu-&i-&d-&s-&i-&\rt{ge}&-μμH}
					{\xx{ment}&\xx{areal}&\xx{2sg.s}&\xx{mid}&\xx{xtn}&\xx{stv}&\rt{smart}&\·\xx{var}}
			\versus	\vbform{yaa ḵugeesgéi}{imp}{be smart!}
				\parencites[27]{leer:1963}[f05/63]{leer:1973}
				\vbmorph{\gm{ÿaa=}&ḵu-&g-&ee-&d-&s-&\rt{ge}&-μμH}
					{\xx{ment}&\xx{areal}&\xx{gcnj}&\xx{2sg.s}&\xx{mid}&\xx{xtn}&\rt{smart}&\·\xx{var}}
			\versus \vbform{tléil yaa ḵooshgé}{neg impfv}{she/he/it is not smart, wise}
				\parencite[f05/63]{leer:1973}
				\vbmorph{tléil&\gm{ÿaa=}&ḵu-&u-&d-&sh-&\rt{ge}&-μH}
					{\xx{neg}&\xx{ment}&\xx{areal}&\xx{irr}&\xx{mid}&\xx{pej}&\rt{smart}&\·\xx{var}}
		\item	\fm{yaa ḵusgé} \~\ \fm{yaa ḵoosgé} (noun) ‘intelligence, wisdom’
			\parencites[27]{leer:1963}[314.35]{dauenhauer-dauenhauer:1990}
				\vbmorph{\gm{ÿaa=}&ḵu-&d-&s-&\rt{ge}&-μH}
					{\xx{ment}&\xx{areal}&\xx{mid}&\xx{xtn}&\rt{smart}&\·\xx{var}}
			\versus \fm{l yaa ḵooshgé} (noun) ‘craziness, foolishness’
			\parencite[f05/62]{leer:1973}
				\vbmorph{l&\gm{ÿaa=}&ḵu-&u-&d-&sh-&\rt{ge}&-μH}
					{\xx{neg}&\xx{ment}&\xx{areal}&\xx{irr}&\xx{mid}&\xx{xtn}&\rt{smart}&\·\xx{var}}
			\newline
			These are nominalizations of the affirmative and negative state imperfective
				verb forms shown above.
		\item	\vbform{du daa yaa x̱at ḵushusigéi}{impfv}[obj intr, \fm{g}, \fm{-μμH} state]{she/he understands me}
			\parencites[237.3373, 3374]{story-naish:1973}[f05/63]{leer:1973}[654]{leer:1976}[26719]{eggleston:2017}
				\vbmorph{du&daa&\gm{ÿaa=}&x̱at=&ḵu-&shu-&s-&i-&\rt{ge}&-μμH}
					{\xx{3h.poss}&around&\xx{ment}&\xx{1sg.o}&\xx{areal}&end&\xx{xtn}&\xx{stv}&\rt{smart}&\·\xx{var}}
			\versus \vbform{du daa kei yaa ḵushunasgéin}{prog}{she/he/it is beginning to understand him/her/it}
				\parencite[26748]{eggleston:2017}
				\vbmorph{du&daa&kei=&\gm{ÿaa=}&ḵu-&shu-&na-&s-&\rt{ge}&-μμH&-n}
					{\xx{3h.poss}&around&up=&\xx{ment}&\xx{areal}&end&\xx{ncnj}&\xx{xtn}&\rt{smart}&\·\xx{var}&\·\xx{nsfx}}
			\versus \vbform{ax̱ daa kei yaa ḵushunasgéin}{prog}{I am coming to understand}
				\parencite[136]{leer:1991}
				\vbmorph{ax̱&daa&kei=&\gm{ÿaa=}&ḵu-&shu-&na-&s-&\rt{ge}&-μμH&-n}
					{\xx{1sg.poss}&around&up=&\xx{ment}&\xx{areal}&end&\xx{ncnj}&\xx{xtn}&\rt{smart}&\·\xx{var}&\·\xx{nsfx}}
			\newline
			This verb is given by \textcite{story-naish:1973} with an overt object \fm{x̱at=} ‘me’
				but forms from \textcite{leer:1991} and \textcite{eggleston:2017} are
				given as impersonals without an object argument, with \citeauthor{leer:1991}
				adding “the situation” in parentheses.
			It is plausible that when there is no overt object this verb has an unmarked
				third person object which corresponds to \citeauthor{leer:1991}’s
				“the situation”.
			Alternatively, this verb may vary among speakers so that some have it as
				impersonal where others have it as an object intransitive.
			The experiencer is always the possessor of \fm{daa} ‘around, about’.
		\end{itemize}
	\item	\label{item:ÿaa=ment-confusion}
		Verbs of confusion based on \fm{\rt{g̱aᴴt}} ‘fall scattered’.
		This verb root occurs in a wide variety of other verbs meaning things like ‘pile up’,
			‘fall apart in pieces’, ‘split’, ‘sprinkle’, and ‘drift in air’;
			it is probably also the same root in \fm{g̱aatáa} (noun) ‘jawed trap’
			and its associated verbs.
		All verbs based on \fm{\rt{g̱aᴴt}} ‘fall scattered’ with \fm{ÿaa=} ‘mental’
			also include areal \X[ḵu-areal]{ḵu-} suggesting some kind of mental space.
		\begin{itemize}
		\item	\vbform{du daa yaa ḵoowag̱át}{pfv}[impers, \fm{∅}, ach]{she/he/it became dizzy, dazed, disoriented, confused}
			\parencites[34]{leer:1963}[f02/110]{leer:1973}[826]{leer:1976}
				\vbmorph{du&daa&\gm{ÿaa=}&ḵu-&μʷ-&wa-&\rt{g̱aᴴt}&-μH}
					{\xx{3h.poss}&around&\xx{ment}&\xx{areal}&\xx{pfv}&\xx{stv}&\rt{scattered}&\·\xx{var}}
			\versus \fm{daa yaa ḵug̱átch} (noun) ‘dizziness’
				\parencites[760]{kelly-willard:1905}[f02/110]{leer:1973}
				\vbmorph{daa&\gm{ÿaa=}&ḵu-&\rt{g̱aᴴt}&-μH&-ch}
					{around&\xx{ment}&\xx{areal}&\rt{scatter}&\·\xx{var}&\·\xx{rep}}
		\item	\vbform{du taayí daa yaa ḵoowag̱át}{pfv}[impers, \fm{∅}, ach]{she/he/it had a nightmare}
			\parencites[826]{leer:1976}
				\vbmorph{du&taa&-yí&daa&\gm{ÿaa=}&ḵu-&μʷ-&wa-&\rt{g̱aᴴt}&-μH}
					{\xx{3h.poss}&sleep&\xx{poss}&around&\xx{ment}&\xx{areal}&\xx{pfv}&\xx{stv}&\rt{scattered}&\·\xx{var}}
			\versus \fm{taayí daa yaa ḵug̱áat} (noun) ‘nightmare’
				\parencites[34]{leer:1963}[826]{leer:1976}
				\vbmorph{taa&-yí&daa&\gm{ÿaa=}&ḵu-&\rt{g̱aᴴt}&-μμH}
					{sleep&\·\xx{poss}&around&\xx{ment}&\xx{areal}&\rt{scatter}&\·\xx{var}}
		\item	\vbform{ax̱ jidaa yaa ḵug̱átch}{rep impfv}{I make mistakes in working}
			\parencite[f02/110]{leer:1973}
				\vbmorph{ax̱&ji-&daa&\gm{ÿaa=}&ḵu-&\rt{g̱at}&-μH&-ch}
					{\xx{1sg.poss}&hand-&around&\xx{ment}&\xx{areal}&\rt{scatter}&\·\xx{var}&\·\xx{rep}}
			\versus \vbform{ax̱ yadaa yaa ḵug̱átx̱}{rep impfv}{I make a mistake, blunder}
			\parencite[f02/110]{leer:1973}
				\vbmorph{ax̱&ÿa-&daa&\gm{ÿaa=}&ḵu-&\rt{g̱at}&-μH&-x̱}
					{\xx{1sg.poss}&face-&around&\xx{ment}&\xx{areal}&\rt{scatter}&\·\xx{var}&\·\xx{rep}}
		\item	\vbform{ÿaa ÿaa ḵunx̱alg̱át}{prog}[subj intr, \fm{∅}, ach]{I am wandering along lost, in a daze}
			\parencite[136 \#71b]{leer:1991}
				\vbmorph{\gm{ÿaa=}&ÿaa=&ḵu-&n-&x̱a-&d-&l-&\rt[¹]{g̱aᴴt}&-μH}
					{\xx{ment}&along&\xx{areal}&\xx{ncnj}&\xx{1sg.s}&\xx{mid}&\xx{xtn}&\rt[¹]{scattered}&\·\xx{var}}
			\newline
			The order of \fm{ÿaa=} ‘mental’ and \X[ÿaa=along]{ÿaa=} ‘along’ is extrapolated
				from the relative order of \X{kei=} ‘up’ > \fm{ÿaa=} ‘mental’ 
				shown in \ref{item:ÿaa=ment-smart}, with the assumption
				that \fm{kei=} ‘up’ and \fm{ÿaa=} ’along’ are in the same position
				\parencite[136]{leer:1991}.
		\item	\vbform{du daa yaa ḵux̱wlig̱át}{pfv}[subj intr, \fm{∅}, ach]{I made him/her dizzy}
			\parencite[34008]{eggleston:2017}
				\vbmorph{du&daa&\gm{ÿaa=}&ḵu-&ʷ-&x̱-&l-&i-&\rt{g̱aᴴt}&-μH}
					{\xx{3h.poss}&around&\xx{ment}&\xx{areal}&\xx{pfv}&\xx{1sg.s}&\xx{csv}&\xx{stv}&\rt{scattered}&\·\xx{var}}
			\versus \vbform{ḵaa daa yaa ḵulag̱átch}{rep impfv}{it makes people pass out}
			\parencite[149.1972]{story-naish:1973}
				\vbmorph{ḵaa&daa&\gm{ÿaa=}&ḵu-&la-&\rt{g̱aᴴt}&-μH&-ch}
					{\xx{ind.h.poss}&around&\xx{ment}&\xx{areal}&\xx{csv}&\rt{scattered}&\·\xx{var}&\·\xx{rep}}
		\item	\vbform{ax̱ daa yaa ḵusag̱átx̱}{rep impfv}[subj intr, \fm{∅}, ach]{it repeatedly makes me dizzy}
			\parencite[71.851]{story-naish:1973}
				\vbmorph{ax̱&daa&\gm{ÿaa=}&ḵu-&sa-&\rt{g̱aᴴt}&-μH&-x̱}
					{\xx{1sg.poss}&around&\xx{ment}&\xx{areal}&\xx{csv}&\rt{scattered}&\·\xx{var}&\·\xx{rep}}
			\versus \vbform{ax̱ daa yaa ḵoowsig̱át}{pfv}{it confused me}
			\parencite[54.601]{story-naish:1973}
				\vbmorph{ax̱&daa&\gm{ÿaa=}&ḵu-&μʷ-&s-&i-&\rt{g̱aᴴt}&-μH&-x̱}
					{\xx{1sg.poss}&around&\xx{ment}&\xx{areal}&\xx{pfv}&\xx{csv}&\xx{stv}&\rt{scattered}&\·\xx{var}&\·\xx{rep}}
			\versus \vbform{ḵaa daa yaa ḵusagátch}{rep impfv}{it confuses people}
			\parencite[54.602]{story-naish:1973}
				\vbmorph{ḵaa&daa&\gm{ÿaa=}&ḵu-&sa-&\rt{g̱aᴴt}&-μH&-ch}
					{\xx{ind.h.poss}&around&\xx{ment}&\xx{areal}&\xx{csv}&\rt{scattered}&\·\xx{var}&\·\xx{rep}}
			\newline
			This verb has causative/applicative \X{s-} instead of causative/applicative
				\X{l-} like the verb above but there seems to be no difference in
				meaning between the two patterns according to translations.
			If there is in fact no meaning difference this may reflect speaker variation.
		\item	\vbform{du daa yaa ḵusig̱áadi}{impfv}[subj intr, \fm{∅}, ach]{it makes him/her dizzy}
			\parencite[827]{leer:1976}
				\vbmorph{du&daa&\gm{ÿaa=}&ḵu-&s-&i-&\rt{g̱aᴴt}&-μμH&-i}
					{\xx{3h.poss}&around&\xx{ment}&\xx{areal}&\xx{csv}&\xx{stv}&\rt{scattered}&\·\xx{var}&\·\xx{sub}}
			\newline
			This is apparently a state imperfective derived from the preceding verb.
		\item	\vbform{yaa ḵuwdlig̱áat}{pfv}[subj intr, \fm{n}?, mot]{she/he/it wandered in a daze}
			\parencite[827]{leer:1976}
				\vbmorph{\gm{ÿaa=}&ḵu-&w-&d-&l-&i-&\rt{g̱aᴴt}&-μμH}
					{\xx{ment}&\xx{areal}&\xx{pfv}&\xx{mid}&\xx{xtn}&\xx{stv}&\rt{scatter}&\·\xx{var}}
			\versus \vbform{kaawayíkt yaa ḵuwtudlig̱áat}{pfv}{we wandered around dizzily}
				\parencite[f02/110]{leer:1973}
				\vbmorph{\gm{ÿaa=}&ḵu-&w-&tu-&d-&l-&i-&\rt{g̱aᴴt}&-μμH}
					{\xx{ment}&\xx{areal}&\xx{pfv}&\xx{1pl.s}&\xx{mid}&\xx{xtn}&\xx{stv}&\rt{scatter}&\·\xx{var}}
			\versus \vbform{ḵut yaa ḵuwdlig̱áat}{pfv}{she/he went astray (a blind person etc.)}
				\parencite[f02/110]{leer:1973}
				\vbmorph{ḵut=&\gm{ÿaa=}&ḵu-&w-&tu-&d-&l-&i-&\rt{g̱aᴴt}&-μμH}
					{\xx{err}&\xx{ment}&\xx{areal}&\xx{pfv}&\xx{1pl.s}&\xx{mid}&\xx{xtn}&\xx{stv}&\rt{scatter}&\·\xx{var}}
		\end{itemize}
	\item	\label{item:ÿaa=ment-offense}
		Verbs that denote causing offense or making mistakes based on \fm{\rt{g̱aᴴt}}
			‘fall scattered’.
		These are similar to the verbs of confusion above in \ref{item:ÿaa=ment-confusion}
			but with different phrasal material alongside the verb.
		In particular, these verbs occur with two phrases: a postposition phrase
			\fm{NP-x̱} indicating the recipient of offense and an adverbial phrase
			\fm{N-niÿaa} ‘N direction’ including a prefixed body part noun
			(at least \fm{x̱ʼa-} ‘mouth’, \fm{ji-} ’hand’).
		Alternatively, the recipient may be given as a possessive pronoun \fm{a} 
			for the body part like \fm{a x̱ʼanyaa} without the postposition \fm{-x̱}.
		The \fm{niÿaa} ‘direction’ element is most often contracted as \fm{…nyaa} or
			\fm{…naa}; compare \fm{xóon niyaa} ‘north wind direction’ versus
			\fm{Xunyaa} \~\ \fm{Xunaa} ‘Hoonah’ and \fm{naa niyaa} ‘upriver direction’
			versus \fm{naanyaa} \~\ \fm{naanaa}.
		The \fm{ÿaa=} ‘mental’ may fail to occur for some speakers or it may appear
			in place of \fm{…naa}, both reflecting the contraction of two similar
			syllables into one (haplology); thus \fm{x̱ʼanaa yaa=} may instead be
			\fm{x̱ʼanaa=} or \fm{x̱ʼayaa=}.
		\begin{itemize}
		\item	\vbform{áx̱ x̱ʼanyaa yaa ḵuwdlig̱át}{pfv}[subj intr, \fm{∅}, ach]{she/he/it offended it by speech}
			\parencite[136]{leer:1991}
				\vbmorph{á&-x̱&x̱ʼa-&niÿaa&\gm{ÿaa=}&ḵu-&w-&d-&l-&i-&\rt{g̱aᴴt}&-μH}
					{\xx{3n}&\·\xx{pert}&mouth-&dir’n&\xx{ment}&\xx{areal}&\xx{pfv}&\xx{mid}&\xx{xtn}&\xx{stv}&\rt{scatter}&\·\xx{var}}
		\item	\vbform{a jinyaa yaa ḵuwdlig̱át}{pfv}[subj intr, \fm{∅}, ach]{she/he/it offended it by deeds}
			\parencite[827]{leer:1976}
				\vbmorph{a&ji-&niÿaa&\gm{ÿaa=}&ḵu-&w-&d-&l-&i-&\rt{g̱aᴴt}&-μH}
					{\xx{3n.poss}&hand-&dir’n&\xx{ment}&\xx{areal}&\xx{pfv}&\xx{mid}&\xx{xtn}&\xx{stv}&\rt{scatter}&\·\xx{var}}
		\end{itemize}
	\item	\label{item:ÿaa=ment-stingy}
		Perhaps in verbs for being stingy or unsharing based on \fm{\rt{ge}} ‘stingy’.
		This root is not well documented, but all instances of it include a preverb
			\fm{ÿaa=} which does not seem to be the \X[ÿaa=along]{ÿaa=} ‘along’
			preverb and so could be identified as \fm{ÿaa=} ‘mental’.
		\begin{itemize}
		\item	\vbform{yaa ashigéi}{impfv}[tr, \fm{g̱}?, \fm{-μμH} state]{she/he/it is stingy about, unwilling to share him/her/it}
			\parencites[168.2316, 2317]{story-naish:1973}[f05/60]{leer:1973}[654]{leer:1976}
				\vbmorph{\gm{ÿaa=}&a-&sh-&i-&\rt[²]{ge}&-μμH}
					{\xx{ment}&\xx{3>3}&\xx{pej}&\xx{stv}&\rt[²]{stingy}&\·\xx{var}}
		\end{itemize}
	\item	There is no evidence for something like \fm{ÿaa=} ‘mental’ anywhere else,
			such as in a noun compound or as a verb root, unlike nearly all other
			preverbs which have clear correlates among other lexical items and paradigms.
		Internal reconstruction suggests a form like \fm[*]{ŋaːʰ} which is identical to the
			the \X[ÿaa=along]{ÿaa=} ‘along’ preverb and its related lexical items.
		It may be related to one or more of the following roots which
			have similar forms and potentially similar meanings:
		\begin{inlinelist}
		\item	\fm{\rt{ya}} \~\ \fm{\rt{ye}} \~\ \fm{\rt{ÿe}} ‘strange’
		\item	\fm{\rt{ÿaᴸ}} ‘move uncertainly; occur’
		\item	\fm{\rt{ÿaᴸ}} ‘resemble’
		\item	\fm{\rt{ÿakw}} ‘deny’
		\item	\fm{\rt{ÿaḵw}} ‘liken, make simile; bequeath’
		\item	\fm{\rt{ÿax̱}} ‘make plan’
		\item	\fm{\rt{yek}} ‘spirit; animated, too fast’
		\item	\fm{\rt{yel}} ‘deceive; Raven’.
		\end{inlinelist}
	\item	In comparison with other Na-Dene langauges, \fm{ÿaa=} ‘mental’ seems to parallel
			similar patterns with preverbal elements that refer to mental phenomena
			and which are plausibly cognate \parencite[135 fn.\ 50]{leer:1991}.
		Compare Proto-Dene \fm[*]{yənə-} \~\ \fm[*]{yiːn-} ‘mind’
			\parencites[22]{krauss-leer:1981}[11]{leer:2008}
			as for example Slave \fm{ni-} \~\ \fm{ʔeni-} ‘mental’ \parencite[608]{rice:1989}.
		Compare also Eyak \fm{ʔiːlih=} ‘mental’
			\parencites[2087, 2133, 2162]{krauss:1970}[135 fn.\ 50]{leer:1991}[378]{krauss:2015}
			probably from the verb \fm{ʔi-leh} ‘want to, have a mind to’
			\parencites[2148]{krauss:1970}[211, 626]{krauss:2015}
			with related \fm{dəɢ-ə-leh} ‘mind, spirit, feelings’
			\parencites[5]{krauss:1981a}[87, 513, 527]{krauss:2015}.
	\end{enumerate}

\item[ÿaan=]\label{m:ÿaan=}
	Incorporated noun ‘hunger’ from a now archaic noun \fm{ÿaan} ‘hunger’
		(Tongass \fm{ÿaàn}, internal reconstruction \fm[*]{ŋaːʰn}).
	Saturates the object argument.
	Only occurs in one verb (see below) based on \fm{\rt{ha}} ‘appear, disappear, move invisibly’
		as part of the set of verbs describing bodily urges.
	As an independent noun \fm{ÿaan} is now extinct but it is documented in \cite{swanton:1909}
		and other earlier materials.
	Phonologically behaves like a proclitic rather than a prefix: when it is followed by
		\X[u-pfv]{u-} the result is pronounced [\ipa{jàːn.ʔu}] with a glottal stop
		and not as *[\ipa{jàː.nù}] without a glottal stop.
	\begin{itemize}
	\item	\vbform{du eet yaan uwaháa}{pfv}[obj intr, \fm{∅}, mot]{hunger appeared to him/her}
		thus ‘she/he became hungry’
		using \motderiv{NP-t/x̱/dé}{∅, \fm{-μμL} rep}{arriving at NP}
			\vbmorph{du&ee&-t&\gm{ÿaan=}&u-&wa-&\rt[¹]{ha}&-μμL}
				{\xx{3h.poss}&\xx{base}&\·\xx{pnct}&hunger&\xx{zpfv}&\xx{stv}&\rt[¹]{appear}&\·\xx{var}}
		\andnot{	\fm[*]{yaan du eet uwaháa}}
	\item	\vbform{óot yaan g̱ahéinín}{ctng}[obj intr, \fm{∅}, mot]{whenever hunger appears to him/her}
	 	\parencite[255.5]{swanton:1909}
			\vbmorph{óo&-t&\gm{ÿaan=}&g̱a-&\rt[¹]{ha}&-μᵉμH&-n&-ín}
				{\xx{3h}&\·\xx{pnct}&hunger&\xx{mod}&\rt[¹]{appear}&\·\xx{var}&\·\xx{nsfx}&\·\xx{ctng}}
	\end{itemize}
	Compare these examples of the noun \fm{ÿaan} ‘hunger’ used outside of verbs:
	\begin{itemize}
	\item	\fm{Át ÿaanch wudzig̱aax̱.} “There hunger made him cry.”
		\parencite[311.4]{swanton:1909} as a subject noun phrase
			\vbmorph{ÿaan&-ch&ⱥ-&wu-&d-&s-&i-&\rt[¹]{g̱ax̱}&-μμL}
				{hunger&\·\xx{erg}&\xx{3>3}&\xx{pfv}&\xx{mid}&\xx{csv}&\xx{stv}&\rt[¹]{cry}&\·\xx{var}}
	\item	\fm{Á áwé yaandéin ḵoowanei.} “It is there that people had gone hungry.”
		\parencite[262.2]{swanton:1909} as an adverb
			\vbmorph{ÿaan&-déin&ḵu-&μʷ-&wa-&\rt[¹]{neᴸ}&-μμL}
				{hunger&\·\xx{adv}&\xx{areal}&\xx{pfv}&\xx{stv}&\rt[¹]{happen}&\·\xx{var}}
	\end{itemize}

\item[yaax̱=]\label{m:yaax̱=}
	Directional preverb ‘aboard’ indicating motion into a boat or other vehicle.
	The most straightforward gloss is English ‘aboard’ although it also could be given
		as \xx{inveh} for ‘into vehicle’.
	Likely originates from \fm{yaakw} (Tongass \fm{yaàkw} [\ipa{jaːʰkʷ}]) ‘canoe, boat’
		and the pertingent postposition \fm{-x̱} ‘of, contacting’,
		but is now an unmodifiable and undecomposable unit.
	Not to be confused with the homophonous \fm{ÿaax̱} ≡ \fm{ÿaadáx̱} ‘from face’
		which is a contraction of \fm{ÿá} ‘face’ and the ablative postposition
		\fm{-dáx̱} ‘from, off of’, and which generally requires a possessor
		where \fm{yaax̱=} ‘aboard’ does not.
	Also not to be confused with the homophonous noun \fm{yaax̱} ‘bank, side, edge’
		which denotes the edge of a body of water or a river as in
		\fm{héen yaax̱í daaḵ uwagút} “he went inland on a riverbank”
		\parencite[268.5]{swanton:1909}.
	\begin{enumerate}
	\item	Motion verbs with the motion derivation
			\motderiv{yaax̱=}{g̱, \fm{yei=…-ch} rep}{aboard boat, into vehicle, embarking}.
		\begin{itemize}
		\item	\vbform{yaax̱ g̱aag̱agoot}{hort}[subj intr, \fm{g̱}, mot]{let her come aboard}
			\parencite[272.227]{dauenhauer-dauenhauer:1987}
				\vbmorph{\gm{yaax̱=}&g̱aa-&g̱a-&\rt[¹]{gut}&-μμL}
					{aboard&\xx{g̱cnj}&\xx{mod}&\rt[¹]{go.\xx{sg}}&\·\xx{var}}
		\item	\vbform{yaax̱ has awsigoot}{pfv}[tr, \fm{g̱}, mot]{they made him go aboard}
			\parencite[369.4]{swanton:1909}
				\vbmorph{\gm{yaax̱=}&a-&w-&s-&i-&\rt[¹]{gut}&-μμL}
					{aboard&\xx{3>3}&\xx{pfv}&\xx{csv}&\xx{stv}&\rt[¹]{go.\xx{sg}}&\·\xx{var}}
		\item	\vbform{yaax̱ has woo.aat}{pfv}[subj intr, \fm{g̱}, mot]{they went aboard}
			\parencite[88.116]{dauenhauer-dauenhauer:1987}
				\vbmorph{\gm{yaax̱=}&has=&wu-&μ-&\rt[¹]{.at}&-μμL}
					{aboard&\xx{plh}&\xx{pfv}&\xx{stv}&\rt[¹]{go.\xx{pl}}&\·\xx{var}}
		\item	\vbform{yaax̱ wuduwaÿeiḵ}{pfv}[tr, \fm{g̱}, mot]{people pulled her aboard}
			\parencite[254.8]{swanton:1909}
				\vbmorph{\gm{yaax̱=}&wu-&du-&wa-&\rt[²]{ÿeḵ}&-μμL}
					{aboard&\xx{pfv}&\xx{ind.h.s}&\xx{stv}&\rt[²]{pull}&\·\xx{var}}
		\end{itemize}
	\item	The motion derivation above is presumably derived from
			\motderiv{NP-x̱}{g̱, \fm{yei=…-ch} rep}{down along NP}
			with the noun phrase \fm{yaakw}, but there are no attested instances
			of motion verbs with the postposition phrase \fm{yaakw-x̱} so it seems
			that \fm{yaax̱=} has completely replaced this compositional structure.
		This contrasts with the related motion derivation
			\motderiv{héen-x̱}{g̱, \fm{yei=…-ch} rep}{down into water}
			where the noun \fm{héen} ‘water, river’ is still recognizable as
			a distinct noun.
	\end{enumerate}

\item[ÿaḵa-]\label{m:ÿaḵa-}
	Incorporated noun ‘scold, swear’ from the deverbal noun \fm{ÿaḵá} ‘speech, saying’.
	Only known from one verb (see below) based on a two argument root \fm{\rt{ti}} of unclear
		meaning.
	This \fm{\rt{ti}} root is discussed further in the entry for \X{g̱ax̱=} ‘crying’
		and seems to be connected to a network of verbs based on the roots
		\fm{\rt{ti}} ‘handle’, 
		\fm{\rt{tiᴸ}} ‘be’, exist’,
		and \fm{\rt{ti}} ‘imitate’ \parencite[383–396]{leer:1976}.
	In the analyses below it is glossed as ‘handle’ based on the form where \fm{ÿaḵá} is
		an independent noun phrase rather than an incorporated noun.
	The noun \fm{ÿaḵá} is itself derived from the verb
		\vbform{yéi yaawaḵaa}{pfv}[subj intr, \fm{n}, \fm{x̱ʼe-…-μH} act]{she/he said so}
		which has an irregular imperfective form \vbform{yéi x̱ʼayaḵá}{impfv}{she/he says so}
		with incorporated \X{x̱ʼe-} ‘mouth, speech’ not present in other aspects of the verb.
	As an independent noun \fm{ÿaḵá} simply means ‘speech, saying’
		\parencites[766]{kelly-willard:1905}[856]{leer:1976}
		but as an incorporated noun \fm{ÿaḵa-} it only occurs in the verb for cursing or
		scolding and so seems to have drifted in meaning toward ‘scold, swear’.
	The verb with \fm{ÿaḵa-} probably started out as a euphemism meaning something like
		‘handle speech’ used to avoid some unknown verb that explicitly described swearing.
	\begin{itemize}
	\item	\vbform{ash yaadé yaḵatí}{impfv}[subj intr, \fm{∅}, \fm{-μH} act]{she/he is scolding him/her}
		\parencites[06/181]{leer:1973}[387]{leer:1976}
			\vbmorph{ash&ÿaa&-dé&\gm{ÿaḵa}-&\rt[²]{ti}&-μH}
				{\xx{3prx.poss}&face&\·\xx{all}&scold&\rt[²]{handle}&\·\xx{var}}
		\newline
		\citeauthor{leer:1973} gives an alternative translation “he’s taking up words against him”
			which as an English euphemism may reflect the euphemistic origin of the verb.
		The English verb “taking up” might also reflect the \fm{\rt{ti}} ‘handle’ root.
		The recipient is \fm{du yaadé} ‘to his/her face’ which, when compared with the
			perfective form below that instead has \fm{du yát}, looks like the
			motion derivation \motderiv{NP-t/x̱/dé}{∅, \fm{-μμL} rep}{arriving at NP}.
		But it does not seem to be a motion verb since motion verbs do not normally allow
			an activity imperfective form.
	\item	\vbform{ash yát yaḵaawatée}{pfv}{she/he scolded him/her}
		\parencites[06/181]{leer:1973}[387]{leer:1976}
			\vbmorph{ash&ÿá&-t&\gm{ÿaḵa}-&μʷ-&wa-&\rt[²]{ti}&-μμH}
				{\xx{3prx.poss}&face&\·\xx{pnct}&scold&\xx{pfv}&\xx{stv}&\rt[²]{handle}&\·\xx{var}}
	\item	\fm{ÿaḵatí} (noun) ‘vile language, blasphemy; scolding, scoffing’
		\parencites[766]{kelly-willard:1905}[06/181]{leer:1973}[387]{leer:1976}
			\vbmorph{\gm{ÿaḵa}-&\rt{ti}&-μH}
				{scold&\rt{imitate}&\·\xx{var}}
	\item	\vbform{a éet yaḵá aawatée}{pfv}[tr, \fm{∅}, mot]{she insulted it}
		\parencite[168.28]{dauenhauer-dauenhauer:1987}
			\vbmorph{a&ee&-t&yaḵá&a-&μʷ-&wa-&\rt[²]{ti}&-μμH}
				{\xx{3n}&\xx{base}&\·\xx{pnct}&scold&\xx{3>3}&\xx{pfv}&\xx{stv}&\rt[²]{handle}&\·\xx{var}}
		\newline
		This form (from \fm{Yeilnaawú} T.\ Peters) probably illustrates the origin of the
			verb with incorporated \fm{ÿaḵa-}.
		Here the noun \fm{ÿaḵá} is not incorporated as shown by the presence of the \X{a-}
			prefix between \fm{ÿaḵá} and the rest of the verb, so \fm{ÿaḵá} is an
			ordinary noun phrase object.
		The verb is the ordinary transitive handling verb based on \fm{\rt{ti}} ‘handle’ with
			the \motderiv{NP-t/x̱/dé}{∅, \fm{-μμL} rep}{arriving at NP} motion derivation.
	\item	\vbform{yaḵá x̱aatee}{impfv}[tr, cnj?, \fm{-μμL} state]{I am constantly scoffing}
		\parencites[06/181]{leer:1973}[387]{leer:1976}
			\vbmorph{\gm{yaḵá}&x̱a-&μ-&\rt{ti}&-μμL}
				{scold&\xx{1sg.s}&\xx{stv}&\rt{imitate}&\·\xx{var}}
		\newline
		This hapax legomenon is peculiar because it is not the handling verb above,
			instead being a state (and not an activity) with distinct stem variation.
		\citeauthor{leer:1973} does not provide any information on its origin nor any other
			details about it.
	\end{itemize}

\item[ÿan=]\label{m:ÿan=}
	directional preverb indicating motion to shore, motion to ground, or termination;
	allomorphs are \fm{ÿax̱} and \fm{ÿánde}:
		\fm{ÿax̱} is used with repetitive,
		\fm{ÿánde} with progressive and prospective,
		and \fm{ÿan} elsewhere (e.g.\ pfv, imp);
	morphologically a specialization of the
		\fm{NP-\{t,x̱,dé\}} (\fm{∅}, \fm{-μμL} rep) ‘arriving at NP’
		motion derivation,
	so the \fm{ÿan} probably used to end with \fm{-t} ‘to a point’ punctual postposition
		as \fm[*]{ÿant};
	derived from noun \fm{ÿán} ‘shore’
		(< Pre-Tlingit \fm[*]{ŋanʰ} < Proto-Na-Dene \fm[*]{ŋənˀ} ‘ground, earth’)
	\begin{enumerate}
	\item	motion on water to shore,
		can be translated ‘ashore’;
		motion derivation
			\fm{ÿan} / \fm{yax̱} / \fm{ÿánde} (\fm{∅}, \fm{-μμL} rep) ‘ashore’
	\item	motion to ground or other horizontal surface,
		can be translated ‘down’ or ‘on ground’;
		motion derivation
			\fm{ÿan} / \fm{yax̱} / \fm{ÿánde} (\fm{∅}, \fm{-μμL} rep) ‘on ground’
			optionally with incorporates (\fm{kʼi-} ‘base’ for ‘setting up, erecting’,
			\fm{sha-} ‘head’ for ‘leaning against’)
	\item	termination of eventuality,
		can be translated ‘ending, terminating, finishing’;
		eventuality/motion derivation
			\fm{ÿan} \~\ \fm{yax̱} \~\ \fm{ÿánde} (\fm{∅}, \fm{-μμL} rep) ‘ending, finishing’
			optionally with \fm{NP-xʼ} ‘coming to rest at NP’;
		derives from metaphor of ‘shore’ as ‘end of journey’ and thus ‘end of event’
	\end{enumerate}

\item[ÿanax̱=]\label{m:ÿanax̱=}
	Directional preverb ‘into ground’ indicating motion into the ground, specifically below
		the surface of the ground.
	Used both with a literal meaning of physical insertion or travel into the ground
		as well as a metaphorical meaning of burial and thus death by metonymy.
	Derived from a combination of \X{ÿan=} and the perlative postposition \fm{-náx̱}
		‘via, through, across’.
	As detailed in its entry, the preverb \fm{ÿan=} ultimately comes from the same origin
		as the noun \fm{ÿán} ‘shore’. 
	Here in \fm{ÿanax̱=} the earlier meaning of ‘ground’ is retained; this older meaning
		can also be seen in some uses of \X{ÿan=} where it indicates motion to the
		ground or floor rather than to a shoreline.
	\begin{enumerate}
	\item	Motion verbs with the motion derivation
			\motderiv{ÿanax̱=}{g̱, \fm{yei=…-ch} rep}{down into ground}.
		This is a specialization of the more general motion derivation
			\motderiv{NP-náx̱}{g̱, \fm{yei=…-ch} rep}{down via, along, through NP}.
		\begin{itemize}
		\item	\vbform{yanax̱ aawatsaaḵ}{pfv}[tr, \fm{g̱}, mot]{she stuck it into the ground}
			\parencite[285.4]{swanton:1909}
				\vbmorph{\gm{ÿanax̱=}&a-&μʷ-&wa-&\rt[²]{tsaḵ}&-μμL}
					{ground&\xx{3>3}&\xx{pfv}&\xx{stv}&\rt[²]{poke}&\·\xx{var}}
		\end{itemize}
	\item	Motion verbs with the above motion derivation as well as some other motion derivation.
		\begin{itemize}
		\item	\vbform{yanax̱ daaḵ uwagút}{pfv}[subj intr, \fm{∅}, mot]{he went inland underground}
			\parencite[120.217]{dauenhauer-dauenhauer:1987}
				\vbmorph{\gm{ÿanax̱=}&daaḵ=&u-&wa-&\rt[¹]{gut}&-μH}
					{ground&inland&\xx{zpfv}&\xx{stv}&\rt[¹]{go.\xx{sg}}&\·\xx{var}}
			\newline
			This has the motion derivation
				\motderiv{daaḵ=}{∅, \fm{-ch} rep}{inland, back from open, off of fire}
				for which see \X{daaḵ=}.
			The perfective \fm{-μH} stem variation here reflects
				the \fm{∅} conjugation class
				in contrast with the \fm{-μμL} stem variation
				expected for \fm{g̱} conjugation class,
				showing that the motion derivation with \fm{daaḵ=} takes precedence
				over the motion derivation with \fm{ÿanax̱=}.
			This is a rare example of how two motion derivations belonging to different
				conjugation classes compete for assigning conjugation class to a verb.
		\item	\vbform{yanax̱ eiḵ ashux̱sagoodín}{hort past}[tr, \fm{∅}, mot]{she/he/it would have lead him/her/it underground to shore}
			\parencite[104.102]{nyman-leer:1993}
				\vbmorph{\gm{ÿanax̱=}&eiḵ=&a-&shu-&x̱-&sa-&\rt[¹]{gut}&-μμL&-ín}
					{ground&beach&\xx{3>3}&end&\xx{mod}&\xx{csv}&\rt[¹]{go.\xx{sg}}&\·\xx{var}&\·\xx{past}}
			\newline
			As a hortative mood form this lacks an overt conjugation prefix and so signals
				the \fm{∅} conjugation class.
			(This is not a contingent aspect form because the stem is not \fm{-μμH}.)
			This has the motion derivation
				\motderiv{ÿeiḵ=}{∅, \fm{-ch} rep}{down to beach from land}
				with the \X{eiḵ=} variant form of the \X{ÿeeḵ=} \~\ \X{ÿeiḵ=}
				preverb derived from \fm{éiḵ} \~\ \fm{éeḵ} ‘beach’.
			As with the preceding example, the \fm{∅} conjugation class of this
				motion derivation takes precedence over the \fm{g̱} conjugation class
				of the \fm{ÿanax̱=} motion derivation.
		\end{itemize}
	\end{enumerate}

\item[ÿánde=]\label{m:ÿánde=}
	allomorph of directional preverb \fm{ÿan} ‘ashore’ or ‘ending’
	with allative postposition \fm{-dé} \~\ \fm{-de} ‘toward’
	\begin{itemize}
	\item	\fm{yánde gax̱tooḵóox̱} (prosp; subj intr, \fm{∅}, mot) ‘we are going to boat ashore’\newline
		versus \fm{yan wutuwaḵúx̱} (pfv) ‘we boated ashore’
	\end{itemize}

\item[ÿata=]
	incorporated noun ‘sleep’,
	saturates object argument;
	apparently derived from \fm{ÿá} ‘face’ and \fm{\rt[¹]{taᴸ}} ‘sg.\ sleep’
	\begin{itemize}
	\item	\fm{ax̱ éet yataawaháa} (pfv; obj intr, \fm{∅}, mot) ‘sleep appeared to me’, i.e. ‘I got sleepy’
	\item	\fm{ax̱ yaadáx̱ yataawahaa} (pfv; obj intr, \fm{g̱}, mot) ‘sleep disappeared from my face’,
		i.e.\ ‘I became wakeful’
	\end{itemize}

\item[yatx̱=]\label{m:yatx̱=}
	Directional preverb ‘lifting, picking up’ indicating movement upward
		from an unspecified surface which is usually the ground.
	Can be glossed as ‘lifting’ with transitive verbs or as ‘upward’ with intransitive verbs.
	Occurs as part of the motion derivation
			\motderiv{yatx̱= \~\ \X{yetx̱=} \~\ \X{yedax̱=}}{∅, \fm{-x̱} rep}{lifting, picking up}.

	Derived from either \fm{ÿán} ‘shore’ in its original sense of ‘ground’ (see \X{ÿan=})
		or from \fm{yé} \~\ \fm{yéi} ‘place’ (see \X{yéi=})
		together with the ablative postposition \fm{-dáx̱} ‘from, off of, out of’
		in its contracted form \fm{-tx̱}.
	The variant forms \X{yetx̱=} and \X{yedax̱=} suggest an origin in \fm{yé}
		whereas this form \fm{yatx̱} suggests an origin in \fm{ÿán};
		it seems likely that all have the same origin
		and the variation between \fm{a} and \fm{e}
		reflects reanalysis from \fm{yé} to \fm{ÿán} or vice versa.
	Could instead plausibly be from \fm{yáadáx̱} ≡ \fm{yá} ‘proximal, here’ + \fm{-dáx̱}
		but the phonological changes required seem less likely.
	
	\textcite[134, 302]{leer:1991} only lists the \fm{yetx̱=} and \fm{yedax̱=} forms
		(actually \fm{ÿetx̱}, \fm{ÿedax̱}),
		neglecting \fm{yatx̱=} even though it occurs both in \cite{story-naish:1973}
		and in \cite{dauenhauer-dauenhauer:1987}.
	\citeauthor{leer:1991} also glosses this preverb as
		“starting off” and “taking off” \parencite[134]{leer:1991}
		as well as “picking up” \parencite[302]{leer:1991}
		although only the last gloss fits with extant examples;
		‘starting off’ and ‘taking off’ seem more suited to \X{g̱unayéi=}.
	\newline
	Variant forms:
	\begin{allolist}
	\item[\X{yedax̱=}]	form with \fm{e} and uncontracted \fm{-dax̱}
	\item[\X{yetx̱=}]		form with \fm{e} and contracted \fm{-dax̱} > \fm{-tx̱}
	\end{allolist}
	\begin{enumerate}
	\item	Handling (controlled transitive motion) verbs with the motion derivation
			\motderiv{yatx̱= \~\ \X{yetx̱=} \~\ \X{yedax̱=}}{∅, \fm{-x̱} rep}{lifting, picking up}.
		\begin{itemize}
		\item	\vbform{Dikée Aanḵáawuch kée yatx̱ uwatée}{pfv}[tr, \fm{∅}, mot]{God exalted him}
			\parencite[81.998]{story-naish:1973}
				\vbmorph{dikée&aanḵáawu&-ch&kée&\gm{yatx̱=}&ⱥ-&u-&wa-&\rt[²]{ti}&-μμH}
					{above&aristocrat&\·\xx{erg}&above&lifting&\xx{3>3}&\xx{zpfv}&\xx{stv}&\rt[²]{handle}&\·\xx{var}}
			\exalso \vbform{at yáxʼ awooné tóonáx̱ kée yatx̱ akg̱watée}{prosp}{she/he/it will lift him up through honor}
			(orig.\ “he will be lifted up and honored”)
			\parencite[81.999]{story-naish:1973}
				\vbmorph{\gm{yatx̱=}&a-&k-&w-&g̱a-&\rt[²]{ti}&-μμH}
					{lifting&\xx{3>3}&\xx{gcnj}&\xx{irr}&\xx{mod}&\rt[²]{handle}&\·\xx{var}}
		\item	\vbform{Tle ya héen áwé yatx̱ ashoowa.áx̱}{pfv}[tr, \fm{∅}, mot]{this water it lifted up as cloth}
			(orig.\ “It lifted the edge of the sea like a cloth.”)
			\parencite[112.67]{dauenhauer-dauenhauer:1987}
				\vbmorph{\gm{yatx̱=}&a-&shu-&μʷ-&wa-&\rt[²]{.ax̱}&-μH}
					{lifting&\xx{3>3}&end&\xx{pfv}&\xx{stv}&\rt[²]{hdl.fabric}&\·\xx{var}}
		\item	\vbform{Yá teey yee yatx̱ ashoowatán anax̱ áwé}{pfv}[tr, \fm{∅}, mot]{below this cedarbark she lifted it through there}
			(orig.\ “She lifted the cedar bark from there”)
			\parencite[248.64]{dauenhauer-dauenhauer:1987}
				\vbmorph{\gm{yatx̱=}&a-&shu-&μʷ-&wa-&\rt[²]{tan}&-μH}
					{lifting&\xx{3>3}&end&\xx{pfv}&\xx{stv}&\rt[²]{hdl.w/l/e}&\·\xx{var}}
		\end{itemize}
	\item	Other transitive verbs with the same motion derivation.
		\begin{itemize}
		\item	\vbform{tle yetx̱ aÿaawashát}{pfv}[tr, \fm{∅}, mot]{she grabbed up her face}
			\parencite[281.5]{swanton:1909}
				\vbmorph{\gm{yetx̱=}&a-&ÿa-&μʷ-&wa-&\rt[²]{shaᴴt}&-μH}
					{lifting&\xx{3>3}&face&\xx{pfv}&\xx{stv}&\rt[²]{grab}&\·\xx{var}}
		\end{itemize}
	\item	In an intransitive verb that may have the same motion derivation.
		\begin{itemize}
		\item	\vbform{De yatx̱ kawdligéi}{pfv}[obj intr, \fm{∅}, mot]{they had already become grown up}
			(orig.\ “They were already fully grown”)
			\parencite[158.131]{dauenhauer-dauenhauer:1987}
		\end{itemize}
	\end{enumerate}

\item[ÿax̱=]\label{m:ÿax̱=ashore}
	allomorph of directional preverb \fm{ÿan} ‘ashore’ or ‘ending’ (group E)
	with perlative postposition \fm{-x̱} ‘contacting’;
	used only with repetitive versus \fm{ÿánde} (prog, prosp) or \fm{ÿan} (pfv, imp, etc.)
	\begin{itemize}
	\item	\fm{yax̱ tooḵoox̱} (rep impfv; subj intr, \fm{∅}, mot) ‘we repeatedly boat ashore’\newline
		versus \fm{yan wutuwaḵúx̱} (pfv) ‘we boated ashore’
	\end{itemize}

\item[ÿax̱=]\label{m:ÿax̱=exh}
	manner preverb ‘using up, completely’ occuring as part of exhaustive derivation
		\motderiv{ÿax̱= + ÿa- + s-}{\fm{∅}, \fm{-x̱} rep}{affecting all of, using up};
	possibly could be equated with \X[ÿax̱=facing]{ÿax̱=} ‘facing’
		but the two meanings seem to be distinct;
	not directly related to \X[ÿax̱=ashore]{ÿax̱=} which is instead from the noun \fm{ÿán} ‘shore’

\item[ÿax̱=]\label{m:ÿax̱=facing}
	directional preverb ‘facing’ occurring as part of the motion derivation
		\motderiv{NP-xʼ ÿax̱=}{\fm{∅}, \fm{-x̱} rep}{turning over by NP}
		and its further derivations \fm{áa ÿax̱=} ‘turning over’ with \fm{á} ‘there’
		and \fm{shóo ÿax̱=} ‘turning end over end’ with \fm{shú} ‘end’;
	the meaning ‘facing’ is tentatively supposed from its likely etymology of
		the inalienable noun \fm{ÿá} ‘face’ and pertingent postposition \fm{-x̱}
	possibly could be equated with \X[ÿax̱=exh]{ÿax̱=} ‘exhaustively, using up’
		but the two meanings seem to be distinct;
	not directly related to \X[ÿax̱=ashore]{ÿax̱=} which is instead from the noun \fm{ÿán} ‘shore’

\item[ÿee-]
	allomorph of second person singular subject \fm{ÿi-}

\item[ÿee=]
	allomorph of second person singular object \fm{ÿi-}

\item[ÿee=]\label{m:ÿee=time}
	Incorporated noun ‘time’ from an archaic noun \fm{ÿee} ‘time’
		that is no longer in independent use.
	Only attested in verbs based on roots describing a length dimension.
	Resembles but is not related to the spatial noun \fm{ÿee} ‘beneath, below, under’.
	It is unclear if as an incorporated noun \fm{ÿee=} ‘time’ saturates the object argument
		of the verb in which it occurs, but no attestations include a separate object
		noun phrase so it seems likely that \fm{ÿee=} is the actual object in every
		verb where it is used.
	\begin{enumerate}
	\item	In verbs describing a length of time based on \fm{\rt{ÿatʼ}} ‘long’.
		\begin{itemize}
		\item	\vbform{yeeyayátʼ}{impfv}[obj intr, \fm{n}, \fm{-μH} state]{it is a long time}
			\parencite[127.1700]{story-naish:1973}
				\vbmorph{\gm{ÿee=}&ÿa-&\rt[¹]{ÿatʼ}&-μH}
					{time&\xx{stv}&\rt[¹]{long}&\·\xx{var}}
			\versus \vbform{yayátʼ}{impfv}[obj intr, \fm{n}, \fm{-μH} state]{she/he/it is long}
				\vbmorph{ÿa-&\rt[¹]{ÿatʼ}&-μH}
					{\xx{stv}&\rt[¹]{long}&\·\xx{var}}
		\item	\vbform{tléil yeenayátʼch}{hab}{not long ago}
			\parencite[127.1701]{story-naish:1973}
				\vbmorph{tléil&\gm{ÿee=}&na-&\rt[¹]{ÿatʼ}&-μH&-ch}
					{\xx{neg}&time&\xx{ncnj}&\rt[¹]{long}&\·\xx{var}&\·\xx{rep}}
		\item	\vbform{wáa sá áa yeekg̱wayáatʼ?}{prosp}{how long is he going to be there?}
			\parencite[127.1702]{story-naish:1973}
				\vbmorph{wáa&sá&á&-μ&\gm{ÿee=}&k-&ʷ-&g̱a-&\rt[¹]{ÿatʼ}&-μμH}
					{how&\xx{q}&\xx{3n}&\·\xx{loc}&time&\xx{gcnj}&\xx{irr}&\xx{mod}&\rt[¹]{long}&\·\xx{var}}
		\item	\vbform{yeewooyáatʼ}{pfv}{it was a long time}
			\parencite[72.629]{nyman-leer:1993}
				\vbmorph{\gm{ÿee=}&wu-&μ-&\rt[¹]{ÿatʼ}&-μμH}
					{time&\xx{pfv}&\xx{stv}&\rt[¹]{long}&\·\xx{var}}
			\versus \vbform{tlél yeewuyáatʼi}{neg pfv}{it wasn’t long (before)}
			\parencite[86.921]{nyman-leer:1993}
				\vbmorph{tlél&\gm{ÿee=}&wu-&\rt[¹]{ÿatʼ}&-μμH&-i}
					{\xx{neg}&time&\xx{pfv}&\rt[¹]{long}&\·\xx{var}&\·\xx{sub}}
		\item	\vbform{ḵúdáx̱ yeekoowáatʼ}{impfv}{it is too long}
			\parencite[126.560]{nyman-leer:1993}
				\vbmorph{ḵúdáx̱&\gm{ÿee=}&k-&u-&μ-&\rt[¹]{ÿatʼ}&-μμH}
					{too.much&time&\xx{cmpv}&\xx{irr}&\xx{stv}&\rt[¹]{long}&\·\xx{var}}
		\item	\vbform{tle yéi áwé yeekuwátʼx̱}{rep impfv}{it is only so long (a time)}
			\parencite[76.706]{nyman-leer:1993}
				\vbmorph{tle&yéi&á&-wé&\gm{ÿee=}&k-&u-&\rt[¹]{ÿatʼ}&-μH&-x̱}
					{so&thus&\xx{foc}&\·\xx{mdst}&time&\xx{cmpv}&\xx{irr}&\rt[¹]{long}&\·\xx{var}&\·\xx{rep}}
		\end{itemize}
	\item	In verbs describing a short length of time based on \fm{\rt{ÿatlʼ}} ‘short’.
		Apparently does not occur with the related root \fm{\rt{ÿachʼ}} ‘short, too short.’
		\begin{itemize}
		\item	\vbform{chʼa yéi yeegoowáatlʼ}{impfv}[obj intr, \fm{n}, \fm{-μμH} state]{it is a short time}
			\parencite[189.2614]{story-naish:1973}
				\vbmorph{chʼa&yéi=&\gm{ÿee=}&g-&u-&μ-&\rt[¹]{ÿatlʼ}&-μμH}
					{just&thus&time&\xx{cmpv}&\xx{irr}&\xx{stv}&\rt[¹]{short}&\·\xx{var}}
			\versus \vbform{yéi goowáatlʼ}{impfv}[obj intr, \fm{n}, \fm{-μμH} state]{it is short}
			\parencite[100]{leer:1963}
				\vbmorph{yéi=&g-&u-&μ-&\rt[¹]{ÿatlʼ}&-μμH}
					{thus&\xx{cmpv}&\xx{irr}&\xx{stv}&\rt[¹]{short}&\·\xx{var}}
		\end{itemize}
	\item	Identifiable in a number of nouns, adjectives, and adverbs
			where it reflects a formerly independent noun \fm[*]{ÿee} ‘time’
			that is no longer available in the modern language.
		Tongass Tlingit has \fm{ÿeè} [\ipa{ɰiːʰ}] suggesting Pre-Tlingit \fm[*]{ŋiːʰ}.
		\begin{itemize}
		\item	\fm{hóochʼeenís} ‘for the last time’ perhaps
				\vbmorph{hóochʼ&-i&\gm{ÿee}&-n&ÿís}
					{last&\·\xx{poss}&time&\·\xx{instr}&\xx{ben}}
		\item	\fm{ḵinxʼiyís} ‘just in case’ plausibly contains \fm{ÿee}
				but it has unclear morphology.
			If it is based on \fm{niyís} ‘in preparation for (time)’ (see below)
				then perhaps
				\vbmorph{ḵin&-xʼ&ni-&\gm{ÿee}&ÿís}
					{lacking&\·\xx{loc}&\xx{dir}\·&time&\·\xx{ben}}
				
		\item	\fm{saÿeeḵ} \~\ \fm{saÿeiḵ} ‘next day’ possibly has \fm{ÿee}
				with an unidentified \fm{sa-} that may reflect \X{s-}
				and \fm{-ḵ} perhaps reflecting deprivative \X[-ḵ-dprv]{-ḵ}.
			This \fm{saÿeeḵ} is also part of \fm{seig̱án} \~\ \fm{seig̱ánín} ‘tomorrow’
				from earlier \fm[*]{saÿeiḵ-án-ín}.
		\item	\fm{niyís} ‘in preparation for (time)’ perhaps
				\vbmorph{ni-&\gm{ÿee}&ÿís}
					{\xx{dir}&time&\xx{ben}}
			\newline
			Compare the initial \fm{ni-} with \fm{niÿaa} ‘direction’,
				\fm{nisdáat} ‘last night’,
				and \fm{niyaháat} ‘breastplate, body armour’.
			Alternatively the initial \fm{ni} of \fm{niyís} could reflect
				irregular \fm[*]{ŋ} > \fm{n} and thus \fm[*]{ŋiːʰ} ‘time’.
			Possibly \fm{niyís} is part of \fm{nisdáat} ‘last night’
				but the composition of meaning is unclear.
		\item	\fm{yagiÿee} \~\ \fm{yakÿee} \~\ \fm{yagee} ‘day’ probably
				ending with \fm{ÿee} ‘time’ but the preceding material is unclear.
		\item	\fm{ÿeedát} ‘moment; now’ perhaps
				\vbmorph{\gm{ÿee}&-t&át}
					{time&\·\xx{pnct}&thing}
		\item	\fm{ÿeen} ‘during, in the middle of (duration)’ probably
				\vbmorph{\gm{ÿee}&-n}
					{time&\·\xx{instr}}
		\item	\fm{ÿées} ‘new, young’ plausibly contains \fm{ÿee} ‘time’
				but the H tone (Tongass \fm{ÿees} [\ipa{ɰiːs}])
				and final \fm{s} are puzzling.
		\item	\fm{ÿéeÿi} ‘former; subordinate clause past tense’ perhaps
				\vbmorph{\gm{ÿee}&-ÿi}
					{time&\·\xx{poss}}
			\newline
			As with \fm{ÿées} ‘new’, the H tone is unexpected.
		\item	\fm{yeis} ‘autumn, fall’ plausibly contains \fm{ÿee} ‘time’
				but the vowel change /\ipa{i}/ > /\ipa{e}/
				and final \fm{s} are puzzling.
		\item	\fm{ÿeisú} ‘still, yet, recently’ plausibly
				\vbmorph{ÿee&ÿís&-ú}
					{time&\xx{ben}&\·\xx{locp}}
			\newline
			The locative predicate \fm{-ú} seems very likely, but the identification
				of \fm{s} is unclear.
		\end{itemize}
	\end{enumerate}

\item[ÿee]
	≡ \fm{wu-i-μ}
	combination of perfective \fm{wu-}
		and second person singular subject \fm{i-}
		and stative \fm{μ-}

\item[ÿeeÿ]
	second person plural subject \fm{ÿi-} combined with either one or both of
		perfective \fm{wu-}
		and stative \fm{ÿa-} \~\ \fm{i-}
	\begin{enumerate}
	\item	\fm{ÿeeÿ} ≡ \fm{ÿi-ÿa-}
		with stative \fm{ÿa-}
	\item	\fm{ÿeeÿ} ≡ \fm{wu-ÿi-}
		with perfective \fm{wu-}
	\item	\fm{ÿeeÿ} ≡ \fm{wu-ÿi-ÿa-}
		with perfective \fm{wu-}
		and stative \fm{ÿa-}
	\item	\fm{ÿeeÿsi} ≡ \fm{ÿi-s-i-}
		with valency \fm{s-}
			(or \fm{ÿeeÿli} \fm{l-} or \fm{ÿeeÿshi} \fm{sh-})
		and stative \fm{i-}
	\item	\fm{ÿeeÿsi} ≡ \fm{wu-ÿi-s-i-}
		with perfective \fm{wu-}
		and valency \fm{s-}
			(or \fm{ÿeeÿli} \fm{l-} or \fm{ÿeeÿshi} \fm{sh-})
		and stative \fm{i-}
	\end{enumerate}

\item[ÿeeḵ=]\label{m:ÿeeḵ=}
	Variant form of directional preverb \X{ÿeiḵ=} ‘beach’ reported but not attested,
		presumably derived from the same noun \fm{éeḵ} \~\ \fm{éiḵ} ‘beach’.
	\citeauthor{leer:1991} lists this form alongside \fm{ÿeiḵ=} and Tongass \X{eèḵ=}
		\parencite[133, 297]{leer:1991}, implying it as a Southern form.
	But there are no attested examples of it
		from \fm{Yeex̱aas} L.\ Roberts
			\parencites{velten:1939}{velten:1944},
		nor from \fm{Taakw Kʼwátʼi} F.G.\ Johnson
			\parencite[138–151]{dauenhauer-dauenhauer:1987},
		nor from \citeauthor{leer:1991}’s extant notes from Southern speakers
			\parencites{leer:1969}{leer:1975g}{leer:1975h},
		although \citeauthor{harrington:1939c}’s notes with T.\ Skeek still need review
			\parencites{harrington:1939c}
		as do the \citeauthor{waterman:1922} placename records
			\parencites{waterman:1922}{thornton:2012}.
	One instance of \X{eeḵ=} is attested from an unknown Southern speaker
		\parencite[17]{leer:1975g},
		paralleling the \X{eèḵ=} form attested in Tongass
		and the \X{eiḵ=} in Taku Inland Northern.
	Compare \X{éeg̱i=} \~\ \X{éig̱i=} from the same noun with the locative postposition
		allomorph \fm{-i}.

\item[yei=]\label{m:yei=}
	direction preverb ‘down’;
	may reflect \fm{g̱} conjugation class or a \fm{∅} conjugation class motion derivation;
	part of directional element paradigm of \fm{\rt{ÿiⁿ}} ‘down’:
		\fm{(di)ÿée} ‘below’, \fm{(di)ÿín-de} ‘to below’, \fm{(di)ÿee-naa} ‘downward’;
	related to \fm{ÿee} ‘beneath, below’
	\begin{enumerate}
	\item	reflects \fm{g̱} conjugation class in prospective, progressive, and repetitive imperfective
	\item	\fm{g̱} conjugation class motion derivation
	\item	\fm{∅} conjugation class motion derivation
	\end{enumerate}

\item[yéi=]\label{m:yéi=}
	manner preverb ‘thus, so’;
	derived from noun \fm{yéi} \~\ \fm{yé} ‘place, way, manner’

\item[-yéi]\label{m:-yéi}
	≡ \fm{-yi yéi}
	combination of relative clause suffix \X[-ÿi-rel]{-ÿi}
		and noun \fm{yéi} \~\ \fm{yé} ‘place, manner, way’.
	Variant of \X{-éi} that occurs after a stem ending in a vowel
		/\ipa{i}/, /\ipa{a}/, or /\ipa{e}/
		following the same pattern as the distribution of the
		allomorphs of the relative clause suffix \X[-i-rel]{-i}.
	See \X{-éi} for further discussion.
	\begin{itemize}
	\item	\vbform{aadé yeeyaḵaayéi yáx̱}{pfv}[subj intr, \fm{n}, \fm{x̱ʼa-…-μH} act]{like the way that she/he said}
		\parencite[92.1161]{story-naish:1973}
			\vbmorph{aa&-dé&y-&ee-&ya-&\rt[¹]{ḵa}&-μμL&-\gm{yi}&\gm{yéi}&yáx̱}
				{\xx{3n}&\·\xx{all}&\xx{pfv}&\xx{2sg.s}&\xx{stv}&\rt[¹]{say}&\·\xx{var}&\·\xx{rel}&way&like}
	\end{itemize}

\item[ÿeiḵ=]\label{m:ÿeiḵ=}
	Directional preverb ‘beach’ indicating motion from some upland area down to a beach
		or shoreline along a body of water.
	Although canonically translated as ‘down to the beach‘ or ‘to the beach’, this is often
		instead translated as just ‘down’.
	Variation between forms with \fm{ei} [\ipa{è}ː] and forms with \fm{ee} [\ipa{ìː}]
		reflects the more general pattern of uvular lowering where \fm{i} /\ipa{i}/
		is lowered to \fm{e} /\ipa{e}/ next to a uvular sound, seen for example in
		\fm{g̱eey} [\ipa{qìːj}] \~\ \fm{g̱eiy} [\ipa{qèːj}] ‘bay’ as well as in the
		pair of \fm{éeḵ} \~\ \fm{éiḵ} ‘beach’.
		
	In progressive, prospective, and repetitive imperfective aspect forms the
		\fm{ÿeiḵ=} \~\ \fm{ÿeeḵ=} preverb blocks the appearance of
		the preverbs \X[ÿaa=along]{ÿaa=} ‘along’, \X{yei=} ‘down’, and \X{kei=} ‘up’
		that would normally arise from conjugation class specification
		in the same way as \X{daak=} ‘out to sea’ and \X{daaḵ=} ‘inland’
		\parencite[136]{leer:1991}.

	Derived from the noun \fm{éeḵ} \~\ \fm{éiḵ} ‘beach’
		with an initial \fm{ÿ} \~\ \fm{y} that is etymologically unexplained,
		but serves to differentiate the preverb from the noun
		\fm{eeḵ} [\ipa{ʔìːq}] \~\ \fm{eiḵ} [\ipa{ʔèːq}] ‘copper (substance)’
		which is the L tone minimal pair counterpart
		to H tone \fm{éeḵ} [\ipa{ʔíːq}] \~\ \fm{éiḵ} [\ipa{ʔéːq}] ‘beach’.
	Compare \X{éeg̱i=} \~\ \X{éig̱i=} from the same noun with the locative postposition
		allomorph \fm{-i}.
	\newline
	Variant forms:
	\begin{allolist}
	\item[{\X{eeḵ=}}]
		Form in Southern varieties
	\item[{\X{eèḵ=}}]
		Form in Tongass Tlingit
	\item[\X{eiḵ=}]
		Form in Taku Inland Northern
	\item[\X{ÿeeḵ=}]
		Form reported but not attested
	\end{allolist}
	\begin{enumerate}
	\item	\label{item:ÿeiḵ=beach-gosg}
		In verbs based on the root \fm{\rt{gut}} ‘singular go’.
		\begin{itemize}
		\item	\vbform{ÿeiḵ uwagút}{pfv}[subj intr, \fm{∅}, mot]{he came to the beach}
			\parencites[268.9]{swanton:1909}[03/185]{leer:1973}
				\vbmorph{\gm{ÿeiḵ=}&u-&wa-&\rt[¹]{gut}&-μH}
					{beach&\xx{zpfv}&\xx{stv}&\rt[¹]{go.\xx{sg}}&\·\xx{var}}
			\versus \vbform{éig̱i aan ÿeiḵ uwagút}{pfv}[subj intr, \fm{∅}, mot]{he went down to the beach with it}
			\parencite[263.4]{swanton:1909}
				\vbmorph{éig̱i=&aa&-n&\gm{ÿeiḵ=}&u-&wa-&\rt[¹]{gut}&-μH}
					{beach&\xx{3n}&\xx{instr}&beach&\xx{zpfv}&\xx{stv}&\rt[¹]{go.\xx{sg}}&\·\xx{var}}
			\newline
			Notably this form has both \X{éig̱i=} and \fm{ÿeiḵ=} at the same time.
		\item	\vbform{tléil yeiḵ woogoot}{neg pfv}[subj intr, \fm{∅}, mot]{he didn’t come back down}
			\parencite[65]{dauenhauer-dauenhauer:1987}
				\vbmorph{tléil&\gm{ÿeiḵ=}&wu-&\rt[¹]{gut}&-μμL}
					{\xx{neg}&beach&\xx{pfv}&\rt[¹]{go.\xx{sg}}&\·\xx{var}}
			\versus \vbform{tléil yeiḵ wugoodí}{sub neg pfv}{when he didn’t come back down}
			\parencite[62]{dauenhauer-dauenhauer:1987}
				\vbmorph{tléil&\gm{ÿeiḵ=}&wu-&\rt[¹]{gut}&-μμL&-í}
					{\xx{neg}&beach&\xx{pfv}&\rt[¹]{go.\xx{sg}}&\·\xx{var}&-\xx{sub}}
			\versus \vbform{l yeiḵ ugóot}{neg csec}{having not come back down}
			\parencite[66]{dauenhauer-dauenhauer:1987}
				\vbmorph{l&\gm{ÿeiḵ=}&u-&\rt[¹]{gut}&-μμH}
					{\xx{neg}&beach&\xx{irr}&\rt[¹]{go.\xx{sg}}&\·\xx{var}}
			\versus	\vbform{tléil ÿeiḵ ugootch}{neg hab}{he would not come down to the beach}
			\parencite[272.12, 273.9]{swanton:1909}
				\vbmorph{tléil&\gm{yeiḵ=}&u-&\rt[¹]{gut}&-μμL&-ch}
					{\xx{neg}&beach&\xx{zpfv}&\rt[¹]{go.\xx{sg}}&·\xx{var}&\·\xx{rep}}
		\item	\vbform{anax̱ yeiḵ gug̱agut yé}{rel prosp}[subj intr, \fm{∅}, mot]{place where he will come down through}
			\parencite[232.296]{dauenhauer-dauenhauer:1987}
				\vbmorph{a&-nax̱&\gm{ÿeiḵ=}&g-&u-&g̱a-&\rt[¹]{gut}&-μL&&yé}
					{\xx{3n}&\·\xx{perl}&beach&\xx{gcnj}&\xx{irr}&\xx{mod}&\rt[¹]{go.\xx{sg}}&\·\xx{var}&\·\xx{rel}&place}
		\item	\vbform{g̱áach sʼéilʼkʼi gíwé yeiḵ oonasgút}{dub prog}[tr, \fm{∅}, mot]{maybe it is a ragged rug he is carrying}
			\parencite[144.134]{dauenhauer-dauenhauer:1987}
				\vbmorph{g̱áach&sʼéilʼ&-kʼ&-i&gí&-wé&\gm{ÿeiḵ=}&a-&u-&na&s-&\rt[¹]{gut}&-μH}
					{rug&torn&\·\xx{dim}&\·\xx{poss}&\xx{yn}&\xx{mdst}&beach&\xx{3>3}&\xx{irr}&\xx{ncnj}&\xx{csv}&\rt[¹]{go.\xx{sg}}&\·\xx{var}}
			\newline
			The irrealis in this form is from the dubitative use of the polar yes/no
				question particle \fm{gí}.
		\item	\vbform{yeiḵ gagóot}{csec}[obj intr, \fm{g}, mot]{when she was coming down}
			\parencite[214.397]{dauenhauer-dauenhauer:1987}
				\vbmorph{\gm{ÿeiḵ=}&ga-&\rt[¹]{gut}&-μμH}
					{beach&\xx{gcnj}&\rt[¹]{go.sg}&\·\xx{var}}
			\newline	
			The presence of \X[ga-conj]{ga-} indicating \fm{g} conjugation class
				in this form is unexpected in two ways.
			One is that \citeauthor{leer:1991} only documents one motion derivation
				for \fm{ÿeiḵ=} which is \fm{∅} conjugation \parencite[297]{leer:1991}
				so this represents a second undescribed motion derivation in
				a different conjugation class.
			The other is that the \fm{g} conjugation class is associated with upward
				motion where movement to the beach would make more sense with
				the downward motion associated with the \fm{g̱} conjugation class.
			It is possible that the published transcription is in error and instead
				\X[g̱a-conj]{g̱a-} is actually spoken here.
			See (\ref{item:ÿeiḵ=beach-gopl}) below for another form from the same speaker
				that reflects the same issues.
		\end{itemize}
	\item	\label{item:ÿeiḵ=beach-gopl}
		In verbs based on the root \fm{\rt{.at}} ‘plural go’.
		\begin{itemize}
		\item	\vbform{yeiḵ too.áat}{csec}[subj intr, \fm{∅}, mot]{when we got to the beach}
			\parencite[151.59]{naish:1966}
				\vbmorph{\gm{ÿeiḵ=}&too-&\rt[¹]{.at}&-μμH}
					{beach&\xx{1pl.s}&\rt[¹]{go.\xx{pl}}&\·\xx{var}}
		\item	\vbform{anax̱ yeiḵ wutuwa.át}{pfv}[subj intr, \fm{∅}, mot]{we came out through there}
			\parencite[68.129]{dauenhauer-dauenhauer:1987}
				\vbmorph{a&-nax̱&\gm{ÿeiḵ=}&wu-&tu-&wa-&\rt[¹]{.at}&-μH}
					{\xx{3n}&\·\xx{perl}&beach&\xx{pfv}&\xx{1pl.s}&\xx{stv}&\rt[¹]{go.\xx{pl}}&\·\xx{var}}
			\versus \vbform{Jilḵáatnáx̱ yeiḵ uwa.át}{pfv}{they came down via the Chilkat}
			\parencite[68.115]{dauenhauer-dauenhauer:1987}
				\vbmorph{Jilḵáat&-náx̱&\gm{ÿeiḵ=}&u-&wa-&\rt[¹]{.at}&-μH}
					{Chilkat&\·\xx{perl}&beach&\xx{zpfv}&\xx{stv}&\rt[¹]{go.\xx{pl}}&\·\xx{var}}
			\versus \vbform{yá Jilḵáatnáx̱ yeiḵ uwa.adi aa}{rel pfv}{these ones who came down via the Chilkat}
			\parencite[68.121]{dauenhauer-dauenhauer:1987}
				\vbmorph{yá&Jilḵáat&-náx̱&\gm{ÿeiḵ=}&u-&wa-&\rt[¹]{.at}&-μL&-i&aa}
					{\xx{prox}&Chilkat&\·\xx{perl}&beach&\xx{zpfv}&\xx{stv}&\rt[¹]{go.\xx{pl}}&\·\xx{var}&\·\xx{rel}&\xx{part}}
		\item	\vbform{aan yeiḵ ga.áat}{csec}[subj intr, \fm{g}, mot]{while she was walking down with them}
			\parencite[214.403]{dauenhauer-dauenhauer:1987}
				\vbmorph{aa&-n&\gm{ÿeiḵ=}&ga-&\rt[¹]{.at}&-μμH}
					{\xx{3n}&\·\xx{instr}&beach&\xx{gcnj}&\rt[¹]{go.\xx{pl}}&\·\xx{var}}
			\newline
			This form is similar to the \fm{yeiḵ gagóot} form discussed above
				in (\ref{item:ÿeiḵ=beach-gosg}) where the \fm{g} conjugation class
				is represented by \X[g-conj]{g-} but the \fm{g̱} conjugation
				class seems more semantically appropriate so that this may be a
				transcription error.
			Both forms are from the same speaker (\fm{Naakil.aan} F.\ Dick Sr.).
		\end{itemize}
	\item	\label{item:ÿeiḵ=beach-otherrt}
		In verbs based on other roots.
		\begin{itemize}
		\item	\vbform{anax̱ yeiḵ has lugúḵch}{rep impfv}[obj intr, \fm{∅}, mot]{they would come running down to the beach}
			\parencite[240.441]{dauenhauer-dauenhauer:1987}
				\vbmorph{a&-nax̱&\gm{ÿeiḵ=}&has=&lu-&\rt[¹]{guk}&-μH&-ch}
					{\xx{3n}&\·\xx{perl}&beach&\xx{plh}&nose&\rt[¹]{push}&\·\xx{var}&\·\xx{rep}}
		\item	\vbform{áx̱ yeiḵ kaawagwadli yé}{rel pfv}[obj intr, \fm{∅}, mot]{place where it had rolled there down to the beach}
			\parencite[182.299]{dauenhauer-dauenhauer:1987}
				\vbmorph{á&-x̱&\gm{ÿeiḵ=}&ka-&μʷ-&wa-&\rt[¹]{gwatl}&-μL&-i&yé}
					{\xx{3n}&\·\xx{pert}&beach&\xx{hsfc}&\xx{pfv}&\xx{stv}&\rt[¹]{roll}&\·\xx{var}&\·\xx{rel}&place}
		\item	\vbform{yeiḵ ḵukandakʼítʼ}{prog}[obj intr, \fm{∅}, mot]{people are coming down to the beach}
			\parencites[123.1628]{story-naish:1973}[202.178]{dauenhauer-dauenhauer:1987}
				\vbmorph{\gm{ÿeiḵ=}&ḵu-&ka-&n-&da-&\rt[²]{kʼitʼ}&-μH}
					{beach&\xx{ind.h.o}&\xx{qual}&\xx{ncnj}&\xx{pasv}&\rt[²]{use.up}&\·\xx{var}}
		\item	\vbform{ÿeiḵ at koojeilch}{hab}[tr, \fm{∅}, mot]{he would carry things to the beach}
			\parencite[267.4]{swanton:1909}
				\vbmorph{\gm{ÿeiḵ=}&at=&ka-&u-&\rt[²]{jel}&-μμL&-ch}
					{beach&\xx{ind.n.o}&\xx{qual}&\xx{zpfv}&\rt[²]{lug}&\·\xx{var}}
		\item	\vbform{yaakw ÿeiḵ aawashát}{pfv}[tr, \fm{∅}, mot]{he hauled a canoe down to the beach}
			\parencite[278.1]{swanton:1909}
				\vbmorph{yaakw&\gm{ÿeiḵ=}&a-&μʷ-&wa-&\rt[²]{shaᴴt}&-μH}
					{canoe&beach&\xx{3>3}&\xx{pfv}&\xx{stv}&\rt[²]{grab}&\·\xx{var}}
		\item	\vbform{áa ÿeiḵ wujixeex}{pfv}[subj intr, \fm{n}, mot]{he ran there to the beach}
			\parencite[263.12]{swanton:1909}
				\vbmorph{á&-μ&\gm{ÿeiḵ=}&wu-&d-&sh-&i-&\rt[¹]{xix}&-μμL}
					{\xx{3n}&\·\xx{loc}&beach&\xx{pfv}&\xx{mid}&\xx{pej}&\xx{stv}&\rt[¹]{fall}&\·\xx{var}}
			\newline
			This form is interesting because the postposition phrase \fm{áa} ‘there’
				apparently assigns \fm{n} conjugation class instead of the expected
				\fm{∅} conjugation class from \fm{ÿeiḵ=} since the stem variation
				is \X{-μμL} rather than \X{-μH}.
			This implies that the motion derivation
				\motderiv{áa}{n, \fm{yoo=…i-…-k} rep}{around and about there}
				somehow supersedes the motion derivation
				\motderiv{ÿeiḵ=}{∅, \fm{-ch} rep}{down to beach}.
		\end{itemize}
	\end{enumerate}

\item[yedax̱=]\label{m:yedax̱=}
	direction preverb ‘starting off’;
	derived from noun \fm{yé} ‘place’ (see \X{yéi=}) and ablative postposition \fm{-dáx̱};
	occurs as part of the motion derivation
		\motderiv{yetx̱= \~\ yedax̱=}{∅, \fm{-x̱} rep}{starting off};
	varies with \X{yetx̱=} which reflects a more general loss of the vowel in the ablative
		postposition, but some speakers reportedly only have \fm{yetx̱=} and not \fm{yedax̱=}

\item[yetx̱=]\label{m:yetx̱=}
	allomorph of the direction preverb \X{yedax̱=} ‘starting off’;
	derived from noun \fm{yé} ‘place’ (see \X{yéi=}) and ablative postposition \fm{-dáx̱};
	occurs as part of the motion derivation
		\motderiv{yetx̱= \~\ yedax̱=}{∅, \fm{-x̱} rep}{starting off};
	varies with \X{yetx̱=} which reflects a more general loss of the vowel in the ablative
		postposition, but some speakers reportedly only have \fm{yetx̱=} and not \fm{yedax̱=}
	\begin{itemize}
	\item	\vbform{tle yetx̱ aÿaawashát}{pfv}[tr, \fm{∅}, mot]{she grabbed up her face}
		\parencite[281.5]{swanton:1909}
			\vbmorph{\gm{yetx̱=}&a-&ÿa-&μʷ-&wa-&\rt[²]{shaᴴt}&-μH}
				{lifting&\xx{3>3}&face&\xx{pfv}&\xx{stv}&\rt[²]{grab}&\·\xx{var}}
	\end{itemize}

\item[ÿi-]\label{m:ÿi-}
	second person plural subject or object; long vowel allomorphs are \fm{ÿee-} and \fm{ÿee=}
	\begin{enumerate}
	\item	second person plural subject
	\item	second person plural object
	\end{enumerate}

\item[ÿi]
	≡ \fm{wu-i-}
	combination of perfective \fm{wu-} and
		second person singular subject \fm{i-}

\item[-ÿi]\label{m:-ÿi-rel}
	allomorph of relative clause \fm{-i}

\item[-ÿi]\label{m:-ÿi-sub}
	allomorph of subordinate \fm{-i}

\item[yoo=]\label{m:yoo=alt}
	alternating eventuality preverb

\item[yóo=]\label{m:yóo=quot}
	quotative preverb

\item[yóo=]\label{m:yóo=thus}
	Allomorph of the manner preverb \X{yéi=} ‘thus, so’.
		\begin{itemize}
		\item	\vbform{aan du yaadé yóo wdudzigeet}{pfv}{they honored him with it}
			\parencite[111.1445]{story-naish:1973}
			\exalso \vbform{du yaadé yóo kax̱wdzigít}{pfv}{I honored him}
				orig.\ “I gave a memorial party for my maternal uncle (I did all expected of me to honor his name)“
			\parencite[111.1446]{story-naish:1973}
		\end{itemize}


\item[yóo=]\label{m:yóo=away}
	Directional preverb ‘off, away’.
	Usually glossed as ‘off’ or ‘away’ depending on context, with no particular
		abbreviations in common use.
	Phonologically more or less identical to the distal deictic element
		\fm{yú} \~\ \fm{yóo} ‘there (far)’
		but referring to an indefinite, nonspecific location
		(‘off somewhere’)
		rather than picking out a particular location
		(‘over there’).
	Some cases can be difficult or impossible to distinguish from deictic \fm{yóo}.
	
	Described by \textcite[133, 138, 147, 299]{leer:1991} as occurring
		with the motion derivation
		\motderiv{NP-t/x̱/dé}{∅, \fm{-μμL} rep}{arriving at NP}
		with \fm{yóo=} in place of the NP.
	As such, \fm{yóo=} would be expected to never occur alone, instead always
		as one of \fm{yóot=}, \fm{yóox̱=}, or \fm{yóode=}.
	All three of these forms do occur, but \fm{yóo=} also occurs alone
		in contexts where it is not better analyzed as quotative \X[yóo=quot]{yóo=}
		or the \X[yóo=thus]{yóo=} allomorph of \X{yéi=} ‘thus, so’.

	Probably derived from the distal deictic element
		\fm{yú} \~\ \fm{yóo} ‘there (far)’
		like the related \X{yux̱=} ‘out, outside’.
	This \fm{yóo=} does not seem to be related to
		the alternating eventuality \X[yoo=alt]{yoo=} ‘back and forth’,
		the quotative \X[yóo=quot]{yóo=} ‘so said’,
		or the manner \X[yóo=thus]{yóo=} ‘thus’.
	\newline
	Allomorphs:
	\begin{allolist}
	\item[yóo=]		basic form without a postposition
	\item[\X{yóode=}]	Form with allative postposition \fm{-dé} \~\ \fm{-de} ‘to, toward’.
	\item[\X{yóot=}]		Form with punctual postposition \fm{-t} ‘at, to, around’.
	\item[\X{yóox̱=}]		Form with pertingent postposition \fm{-x̱} ‘at, of, contacting’.
	\end{allolist}
	\begin{enumerate}
	\item	\label{item:yóo=quot-mot}
		Forms reflecting the motion derivation
			\motderiv{yóot= / yóox̱= / yóode=}{∅, \fm{-μμL} rep}{off, away}.
		These always occur with one of the three postpositions
			punctual \fm{-t} ‘at, to, around’,
			pertingent \fm{-x̱} ‘at, of, contacting’,
			or allative \fm{-de} ‘to, toward’
			depending on aspect.
		\begin{enumerate}
		\item	\label{item:yóo=quot-mot-de}
			Forms with allative \fm{-de} ‘to, toward’.
			See \X{yóode=} for more.
			\begin{itemize}
			\item	\vbform{haa ítde yóode kg̱wa.áat}{prosp}[subj intr, \fm{∅}, mot]{they will go off following us}
				\parencite[160.1291]{nyman-leer:1993}
					\vbmorph{haa&ít&-de&\gm{yóo}&\gm{-de=}&k-&ʷ-&g̱a-&\rt[¹]{.at}&-μμH}
						{\xx{1pl.psr}&after&\·\xx{all}&off&\·\xx{all}&\xx{gcnj}&\xx{irr}&\xx{mod}&\rt[¹]{go.\xx{pl}}&\·\xx{var}}
			\end{itemize}
		\item	\label{item:yóo=quot-mot-t}
			Forms with punctual \fm{-t} ‘at, to, around’.
			See \X{yóot=} for more.
			\begin{itemize}
			\item	\vbform{yóot uwagút}{pfv}[subj intr, \fm{∅}, mot]{he left}
				\parencite[238.407]{dauenhauer-dauenhauer:1987}
					\vbmorph{\gm{yóo}&\gm{-t=}&u-&wa-&\rt[¹]{gut}&-μH}
						{off&\·\xx{pnct}&\xx{pfv}&\xx{stv}&\rt[¹]{go.\xx{sg}}&\·\xx{var}}
			\item	\vbform{yóot loowagúḵ yú keitlxʼ}{pfv}[obj intr, \fm{∅}, mot]{they ran off, those dogs}
				\parencite[230.266]{dauenhauer-dauenhauer:1987}
					\vbmorph{\gm{yóo}&\gm{-t=}&lu-&μʷ-&wa-&\rt[¹]{guḵ}&-μH}
						{off&\·\xx{pnct}&nose&\xx{pfv}&\xx{stv}&\rt[¹]{push}&\·\xx{var}}
			\end{itemize}
		\item	\label{item:yóo=quot-mot-x̱}
			Forms with pertingent \fm{-x̱} ‘at, of, contacting’.
			See \X{yóox̱=} for more.
			\begin{itemize}
			\item	\vbform{du hídináx̱ yóox̱ yaa yanas.éin}{prog}[subj intr, \fm{∅}, mot]{he is peeping out from his house}
				\parencite[147.1998]{story-naish:1973}
					\vbmorph{\gm{yóo}&\gm{-x̱=}&ÿaa=&ÿa-&na-&d-&s-&\rt[¹]{.a}&-μμᵉH&-n}
						{out&\·\xx{pert}&along&face&\xx{ncnj}&\xx{mid}&\xx{xtn}&\rt[¹]{end.move}&\·\xx{var}&\·\xx{nsfx}}
			\item	\vbform{tléil yóox̱ eeltoox̱úḵ}{phib impfv}[tr, \fm{∅}, mot]{don’t spit it out}
				\parencite[204.2850]{story-naish:1973}
					\vbmorph{tléil&\gm{yóo}&\gm{-x̱=}&u-&i-&l-&\rt[¹]{tux̱}&-μμL&-úḵ}
						{\xx{neg}&out&\·\xx{pert}&\xx{irr}&\xx{2sg.s}&\xx{csv}&\rt[¹]{spit}&\·\xx{var}&\·\xx{phib}}
			\item	\vbform{Tlél chʼu koogéiyi yóox̱ dux̱eech}{neg impfv}[tr, \fm{∅}, mot]{people don’t throw it away carelessly}
				\parencite[184.359]{dauenhauer-dauenhauer:1987}
					\vbmorph{\gm{yóo}&\gm{-x̱=}&du-&\rt[²]{x̱ich}&-μμL}
						{away&\·\xx{pert}&\xx{ind.h.s}&\rt[²]{throw.container}&\·\xx{var}}
			\end{itemize}
		\end{enumerate}
	\item	Forms where \fm{yóo=} appears with a verb that does not
			seem to reflect the motion derivation in \#\ref{item:yóo=quot-mot}.
		\begin{enumerate}
		\item	Forms with one of the \fm{-t}, \fm{-x̱}, or \fm{-de} postpositions
				that are not from the motion derivation above.
			Some of these reflect the distal deictic element
				\fm{yú} \~\ \fm{yóo} ‘there (far)’
				as they seem to refer to a specific location in context.
			\begin{itemize}
			\item	\vbform{yóode yáx̱ koowáatʼ}{impfv}[obj intr, \fm{g}, \fm{-μμH} state]{as long as over to there}
				\parencite[168.7]{nyman-leer:1993}
					\vbmorph{\gm{yóo}&-de&yáx̱&k-&u-&μ-&\rt[¹]{ÿatʼ}&-μμH}
						{\xx{dist}&\·\xx{all}&like&\xx{cmpv}&\xx{irr}&\xx{stv}&\rt[¹]{long}&\·\xx{var}}
			\item	\vbform{yóode nax̱too.aat}{hort}[subj intr, \fm{n}, mot]{let’s go somewhere else}
				\parencite[202.744]{nyman-leer:1993}
					\vbmorph{\gm{yóo}&\gm{-de=}&na-&x̱-&too-&\rt[¹]{.at}&-μμH}
						{&off&\·\xx{all}&\xx{ncnj}&\xx{mod}&\xx{1pl.s}&\rt[¹]{go.\xx{pl}}&\·\xx{var}}
			\item	\vbform{yóode át ḵutootéesʼ}{impfv}[subj intr, \fm{n}, inv act]{we stare off into the distance there}
				\parencite[202.747]{nyman-leer:1993}
					\vbmorph{\gm{yóo}&\gm{-de}&á&-t&ḵu-&too-&\rt[¹]{tisʼ}&-μμH}
						{off&\·\xx{all}&\xx{3n}&\·\xx{pnct}&\xx{areal}&\xx{1pl.s}&\rt[¹]{stare}&\·\xx{var}}
			\end{itemize}
		\item	Forms with other postpositions.
			These probably all reflect the distal deictic element
				\fm{yú} \~\ \fm{yóo} ‘there (far)’
				as they seem to refer to a specific location in context.
			\begin{itemize}
			\item	\vbform{Yóonáx̱ shatán sʼeiḵ}{rel impfv}[obj intr, \fm{g}, \fm{kei=…-ch} rep]{The smoke that is rising over there}
				\parencite[214.409]{dauenhauer-dauenhauer:1987}
					\vbmorph{\gm{yóo}&-náx̱=&sha-&\rt[¹]{tan}&-μH&&sʼeiḵ}
						{\xx{dist}&\·\xx{perl}&head-&\rt[¹]{handle.w/l/e}&\·\xx{var}&\xx{rel}&smoke}
			\end{itemize}
		\end{enumerate}
	\item	Forms where \fm{yóo=} occurs without any postpositions.
		These must be distinguished semantically from
			the quotative \X[yóo=quot]{yóo=}
			and manner \X[yóo=thus]{yóo=}
			which do not have an ‘off’ or ‘away’ interpretation.
		\begin{itemize}
		\item	\vbform{tuháayi yóo akaawataan}{pfv}{he bent the nail}
			\parencite[28.184]{story-naish:1973}
				\vbmorph{\gm{yóo=}&a-&ka-&μʷ-&wa-&\rt[²]{tan}&-μμL}
					{off&\xx{3>3}&\xx{qual}&\xx{pfv}&\xx{stv}&\rt[²]{handle.w/l/e}&\·\xx{var}}
			\exand \vbform{yóo katán}{impfv}{it is curved, bent}
			\parencite[06/38]{leer:1973}
				\vbmorph{\gm{yóo=}&ka-&\rt[¹]{tan}&-μH}
					{off&\xx{qual}&\rt[¹]{position.w/l/e}&\·\xx{var}}
			\exand \vbform{tl.\ yóo yoo sh koodatánk}{neg impfv}{s/he/it can’t bend over}
			\parencite[06/38]{leer:1973}
				\vbmorph{tl.&\gm{yóo=}&yoo=&sh=&ka-&u-&da-&\rt[²]{tan}&-μH&-k}
					{\xx{neg}&off&\xx{alt}&\xx{rflx.o}&\xx{qual}&\xx{irr}&\xx{mid}&\rt[²]{handle.w/l/e}&\·\xx{var}}
			\newline
			Related forms listed by \cite{leer:1973} in the same place are
				\fm{yoo katan lítaa} “curved knife”,
				\fm{yóo yaa (ka)kunatán} (no transl.),
				\fm{yóo kakootán} (no transl.), 
				\fm{yóo kakatán} “is curved, leaning to one side”,
				\fm{yóo kakaksatán} (no transl.),
				\fm{yóot kután} “it’s out of shape”,
				\fm{yóo (ka)kla.át} “pl.”
					\parencite[all][06/38]{leer:1973},
				\fm{yóo sh kawditaan} “bent (self) over, leaned down”
					\parencite[360]{leer:1976}.
			There are also forms without \fm{yóo=}, of which
				some have alternating manner \X[yoo=alt]{yoo=}:
				\fm{yínde sh kawditaan} “he bent over”
					\parencite[28/186]{story-naish:1973},
				\fm{tléil du keey yoo akoostánk} “he can’t bend his knee”
					\parencite[28/190]{story-naish:1973},
				\fm{át sh kandatánch} “keeps on bending”,
				\fm{héende has kawduwatán} “they went under”,
				\fm{héende haa kanatáan} “we’re going to go under”,
				\fm{héende haa kaawataan} “pf.”,
				\fm{kootánaa} “piece of cedar bark used for collecting the layer of oil”
					\parencite[all][06/38]{leer:1973}.
		\item	\vbform{yóo shatán}{impfv}{s/he/it is slanted to one side}
			\parencite[06/39]{leer:1973}
			\newline
			Also \fm{yóo sha(ka)tán} “is slanting, sloping (upright obj., eg tree)”
				\fm{yóo shasatán}, \fm{yóo shaksatán} “is steep, sloped (esp.\ land, mt.)”,
				\fm{yóo ashakaawataan} “etc., made it slant, slope”
				\parencite[all][360]{leer:1976}.
		\item	\vbform{yóo naalée}{impfv}[obj intr, \fm{n}, \fm{-μμH} state]{it is far away}
			\parencite[190.8]{nyman-leer:1993}
				\vbmorph{\gm{yóo=}&na-&μ-&\rt[¹]{li}&-μμH}
					{off&\xx{ncnj}&\xx{stv}&\rt[¹]{far}&\·\xx{var}}
			\exand \vbform{ḵaa ít yóo naaliyéi}{impfv rel}{a place distant from people}
			\parencite[20.210]{nyman-leer:1993}
				\vbmorph{ḵaa&ít&\gm{yóo=}&na-&μ-&\rt[¹]{li}&-μL&-ÿi&yéi}
					{\xx{ind.h.psr}&after&off&\xx{ncnj}&\xx{stv}&\rt[¹]{far}&\·\xx{var}&\·\xx{rel}&place}
		\item	\vbform{yóo awlig̱eech}{pfv}[tr, \fm{n}, ach, \fm{-t} rep]{he felled it (tree)}
			\parencite[176.208]{nyman-leer:1993}
				\vbmorph{\gm{yóo=}&a-&w-&lˢ-&i-&\rt[²]{g̱ich}&-μμL}
					{off&\xx{3>3}&\xx{pfv}&\xx{xtn}&\xx{stv}&\rt[²]{throw}&\·\xx{var}}
			\newline
			Compare the noun \fm{g̱éechadi} \~\ \fm{g̱éejadi} (T.\ \fm{g̱eech.adi})
				‘dead tree blown over by wind’
				\parencite[f02/185]{leer:1973}.
		\end{itemize}
	\item	Forms without a verb, ususally suggesting some kind of ellipsis
			or nonverbal predication.
		\begin{itemize}
		\item	
		\end{itemize}
	\item	Possibly identifiable as part of a few adverbs.
		\begin{itemize}
		\item	\fm{chʼa yóokʼ} ‘immediately, suddenly’
			\begin{itemize}
			\item	\fm{chʼa yóokʼ neil uwagút} ‘he came home right away’
				\parencite[58]{naish:1966}
			\item	\fm{chʼa yóokʼde áwé yéi yaawaḵaa} ‘immediately he said’
				\parencite[133]{naish:1966}
			\end{itemize}
		\end{itemize}
	\end{enumerate}

\item[yóode=]\label{m:yóode=}
	≡ \fm{yóo-de} combination of directional preverb \X[yóo=away]{yóo=} ‘off, away’
		and allative postposition \fm{-dé} \~\ \fm{-de} ‘to, toward’.

\item[yóot=]\label{m:yóot=}
	≡ \fm{yóo-t} combination of directional preverb \X[yóo=away]{yóo=} ‘off, away’
		and punctual postposition \fm{-t} ‘at a point’.

\item[yóox̱=]\label{m:yóox̱=}
	≡ \fm{yóo-t} combination of directional preverb \X[yóo=away]{yóo=} ‘off, away’
		and pertingent postposition \fm{-x̱} ‘of, contacting’.

\item[ÿu-]\label{m:ÿu-} 
	abstract representation of perfective \fm{wu-};
	this form does not actually occur in speech, instead see
		\fm{wu-}, \fm{w-}, \fm{m-}, \fm{μʷ-} \fm{ÿi}, \fm{ÿee}, \fm{ÿeeÿ}

\item[yux̱=]\label{m:yux̱=}
	Directional preverb ‘out, outside’ indicating motion out of a building, cave, or other
		covered enclosure.
	Usually glossed as just ‘out’ or ‘outside’ with no particular abbreviations in common use.
	Probably derived from the distal deictic \fm{yú} \~\ \fm{yóo} ‘there (far)’
		with the pertingent postposition \fm{-x̱} ‘of, contacting’,
		but if so this must have been independent of
		the \fm{yóo-t} \~\ \fm{yóo-x̱} \~\ \fm{yóo-de} pattern
		for which see \X[yóo=away]{yóo=} and its related entries.
	The difference between \fm{yux̱=} and \fm{yóo-x̱} might be best explained by different
		motion derivations.
	Thus \fm{yux̱=} likely developed from the motion derivation
		\motderiv{NP-x̱}{n, \fm{yoo=i-…-k} rep}{along NP}
		with \fm{yú} as the noun phrase
		whereas \fm{yóo-x̱} developed from the motion derivation
		\motderiv{NP-t/x̱/dé}{∅, \fm{-μμL} rep}{arriving at NP}
		with \fm{yú} as the noun phrase.
	The difference in length and tone between \fm{yux̱=} and \fm{yóo-x̱} may be because
		\fm{yux̱=} is more fossilized (and thus phonologically reduced)
		since it does not alternate in postpositions whereas the \fm{yóo-x̱} form
		regularly alternates with \fm{yóo-t} and \fm{yóo-de}
		depending on the grammatical aspect of the verb.
	\begin{enumerate}
	\item	\label{item:yux̱=alone}
		Motion verbs with the motion derivation
			\motderiv{yux̱=}{n, \fm{yoo=i-…-k} rep}{out, outside}.
		These seem to be relatively rare in contrast to the forms in \ref{item:ÿux̱=there}
			with the postposition phrase \fm{áa} ‘there’, suggesting that \fm{áa yux̱=}
			may be tending toward a sort of set phrase that replaces \fm{yux̱=} alone.
		Alternatively, it may be that \fm{yux̱=} has come to need
			an additional postposition phrase for unclear reasons
			and that \fm{áa} ‘there’ is a kind of default in the absence of any others.
		\begin{itemize}
		\item	\vbform{yux̱ naltlʼeet}{imp}[tr, \fm{n}, act]{throw it out}
			\parencite[210.345]{dauenhauer-dauenhauer:1987}
				\vbmorph{\gm{yux̱=}&na-&&lˢ-&\rt[²]{tlʼit}&-μμL}
					{out&\xx{ncnj}&\xx{2sg.s}&\xx{xtn}&\rt[²]{discard}&\·\xx{var}}
		\item	\vbform{yux̱ wujixeex}{pfv}[subj intr, \fm{n}, mot]{he ran outside}
			\parencite[257.9]{swanton:1909}
				\vbmorph{\gm{yux̱=}&wu-&d-&sh-&i-&\rt[¹]{xix}&-μμL}
					{out&\xx{pfv}&\xx{mid}&\xx{pej}&\xx{stv}&\rt[¹]{fall}&\·\xx{var}}
		\end{itemize}
	\item	\label{item:ÿux̱=there}
		Motion verbs with the motion derivation
			\motderiv{yux̱=}{n, \fm{yoo=i-…-k} rep}{out, outside}
			accompanied by a postposition phrase \fm{áa} ‘there’
			which is formed from the third person nonhuman pronoun \fm{á} ‘it’
			and the locative postposition allomorph \X[-μ-loc]{-μ}
			that only occurs with nouns ending with a [\ipa{CV́}] syllable.
		This is by far the most common pattern for verbs with \fm{yux̱=}.
		\begin{itemize}
		\item	\vbform{áa yux̱ woogoot}{pfv}[subj intr, \fm{n}, mot]{she went outside}
			\parencite[255.10]{swanton:1909}
				\vbmorph{á&-μ&\gm{yux̱=}&wu-&μ-&\rt[¹]{gut}&-μμL}
					{\xx{3n}&\·\xx{loc}&out&\xx{pfv}&\xx{stv}&\rt[¹]{go.\xx{sg}}&\·\xx{var}}
			\versus \vbform{áa yux̱ aawagoot}{pfv}{someone went outside there}
			\parencites[287.12]{swanton:1909}[102.409]{dauenhauer-dauenhauer:1987}
				\vbmorph{á&-μ&\gm{yux̱=}&a-&μʷ-&wa-&\rt[¹]{gut}&-μμL}
					{\xx{3n}&\·\xx{loc}&out&\xx{ind.h.s}&\xx{pfv}&\xx{stv}&\rt[¹]{go.\xx{sg}}&\·\xx{var}}
			\versus \vbform{kʼé áa yux̱ nagú}{imp}{good that you go outside there}
			\parencite[287.5]{swanton:1909}
				\vbmorph{kʼe&á&-μ&\gm{yux̱=}&na-&&\rt[¹]{gut}&-⊗}
					{good&\xx{3n}&\·\xx{loc}&out&\xx{ncnj}&\xx{2sg.s}&\rt[¹]{go.\xx{sg}}&\·\xx{var}}
		\item	\vbform{áa yux̱ aawa.aat}{pfv}[subj intr, \fm{n}, mot]{people went outside there}
			\parencite[287.11]{swanton:1909}
				\vbmorph{á&-μ&\gm{yux̱=}&a-&μʷ-&wa-&\rt[¹]{.at}&-μμL}
					{\xx{3n}&\·\xx{loc}&out&\xx{ind.h.s}&\xx{pfv}&\xx{stv}&\rt[¹]{go.\xx{pl}}&\·\xx{var}}
		\item	\vbform{áa yux̱ nalnúkch}{hab}[subj intr, \fm{n}, mot]{she would feel her way outside there}
			\parencite[176.206]{dauenhauer-dauenhauer:1987}
				\vbmorph{á&-μ&\gm{yux̱=}&na-&d-&l-&\rt[¹]{nuk}&-μH&-ch}
					{\xx{3n}&\·\xx{loc}&out&\xx{ncnj}&\xx{mid}&\xx{xtn}&\rt[¹]{feel}&\·\xx{var}&\·\xx{rep}}
			\versus \vbform{áx̱ wudlinook}{pfv}[subj intr, \fm{n}, mot]{s/he felt around for it}
			\parencite[299]{leer:1976}
				\vbmorph{á&-x̱&wu-&d-&l-&i-&\rt[¹]{nuk}&-μμL}
					{\xx{3n}&\·\xx{pert}&\xx{pfv}&\xx{mid}&\xx{xtn}&\xx{stv}&\rt[¹]{feel}&\·\xx{var}}
		\item	\vbform{áa yux̱ wujixeex}{pfv}[subj intr, \fm{n}, mot]{s/he ran outside there}
			\parencites[283.1]{swanton:1909}[162.196]{dauenhauer-dauenhauer:1987}[122.478]{nyman-leer:1993}
				\vbmorph{á&-μ&\gm{yux̱=}&wu-&d-&sh-&i-&\rt[¹]{xix}&-μμL}
					{\xx{3n}&\·\xx{loc}&out&\xx{pfv}&\xx{mid}&\xx{pej}&\xx{stv}&\rt[¹]{fall}&\·\xx{var}}
		\item	\vbform{áa yux̱ awdlig̱ein}{pfv}[subj intr, \fm{n}, mot]{she looked outside there}
			\parencite[287.8]{swanton:1909}
				\vbmorph{á&-μ&\gm{yux̱=}&a-&w-&d-&l-&i-&\rt[²]{g̱eͥn}&-μμL}
					{\xx{3n}&\·\xx{loc}&out&\xx{xpl}&\xx{pfv}&\xx{xtn}&\xx{stv}&\rt[²]{look}&\·\xx{var}}
			\versus \vbform{awdlig̱een}{pfv}[subj intr, \fm{n}, mot]{she/he looked}
			\parencite[835]{leer:1976}
				\vbmorph{a-&w-&d-&l-&i-&\rt[²]{g̱in}&-μμL}
					{\xx{xpl}&\xx{pfv}&\xx{mid}&\xx{xtn}&\xx{stv}&\rt[²]{look}&\·\xx{var}}
		\end{itemize}
	\item	\label{item:ÿux̱=at}
		Motion verbs with the motion derivation
			\motderiv{yux̱=}{n, \fm{yoo=i-…-k} rep}{out, outside}
			accompanied by a postposition phrase \fm{NP-i} ‘at NP’
			which includes the locative postposition allomorph
			\fm{-í} \~\ \fm{-i} that is restricted to preverbal PPs.
		\begin{enumerate}
		\item	\label{item:ÿux̱=at-outside}
			With the locational preverb \X{gáani=} ‘outside’.
			This is like a pleonasm because \fm{yux̱=} can mean ‘outside’ on its own,
				although strictly speaking \fm{gáan} only means ‘outside’
				whereas \fm{yux̱=} can also be interpreted as ‘out’,
				so the two together could be roughly equivalent to ‘out outside’.
			\begin{itemize}
			\item	\vbform{gáani yux̱ woogoot}{pfv}[subj intr, \fm{n}, mot]{it went outside}
				\parencite[220.54]{dauenhauer-dauenhauer:1987}
					\vbmorph{gáan&-i&\gm{yux̱=}&wu-&μ-&\rt[¹]{gut}&-μμL}
						{outside&\·\xx{loc}&out&\xx{pfv}&\xx{stv}&\rt[¹]{go.\xx{sg}}&\·\xx{var}}
			\item	\vbform{gáani yux̱ yaa yanas.éini}{prog sub}[subj intr, \fm{n}, mot]{as it is sticking out its face}
				\parencite[015.336]{dauenhauer-dauenhauer:1987}
					\vbmorph{gáan&-i&\gm{yux̱=}&ÿaa=&ÿa-&na-&d-&s-&\rt[¹]{.a}&-μᵉμH&-n&-i}
						{outside&\·\xx{loc}&out&along&face&\xx{ncnj}&\xx{mid}&\xx{csv}&\rt[¹]{end.move}&\·\xx{var}&\·\xx{nsfx}&\·\xx{sub}}
			\item	\vbform{gáani yux̱ kawdudli.oo}{pfv}[tr, \fm{n}, mot]{people made him dwell outside}
				\parencite[257.4]{swanton:1909}
					\vbmorph{gáan&-i&\gm{yux̱=}&ka-&w-&du-&d-&l-&i-&\rt[¹]{.uᴸ}&-μμL}
						{outside&\·\xx{loc}&out&\xx{qual}&\xx{pfv}&\xx{ind.h.s}&\xx{mid}&\xx{csv}&\rt[¹]{dwell}&\·\xx{var}}
			\item	\vbform{gáani yux̱ has aawax̱óotʼ}{pfv}[tr, \fm{n}, mot]{they dragged her outside}
				\parencite[285.8]{swanton:1909}
					\vbmorph{gáan&-i&\gm{yux̱=}&has=&a-&μʷ-&wa-&\rt[²]{x̱utʼ}&-μμH}
						{outside&\·\xx{loc}&out&\xx{plh}&\xx{3>3}&\xx{pfv}&\xx{stv}&\rt[²]{drag}&\·\xx{var}}
			\end{itemize}
		\item	\label{item:ÿux̱=at-other}
			With other locational preverbs;
				see \X{x̱áni=} ‘near’
				and \X{éig̱i=} ‘beach’
				for details.
			These are only attested in older materials and so are probably archaic.
			\begin{itemize}
			\item	\vbform{hasdu x̱áni yux̱ nag̱asheex}{hort}[subj intr, \fm{n}, mot]{he should run out near them}
				\parencite[286.13]{swanton:1909}
					\vbmorph{has-&du&x̱án&-i&\gm{yux̱=}&na-&g̱a-&d-&sh-&\rt[¹]{xix}&-μμL}
						{\xx{plh}&\xx{3h.psr}&near&\·\xx{loc}&out&\xx{ncnj}&\xx{mod}&\xx{mid}&\xx{pej}&\rt[¹]{fall}&\·\xx{var}}
			\item	\vbform{éig̱i yux̱ wutuwax̱óotʼ}{pfv}[tr, \fm{n}, mot]{we dragged her out on the beach}
				\parencite[285.12]{swanton:1909}
					\vbmorph{éiḵ&-i&\gm{yux̱=}&wu-&tu-&wa-&\rt[²]{x̱utʼ}&-μμH}
						{beach&\·\xx{loc}&out&\xx{pfv}&\xx{1pl.s}&\xx{stv}&\rt[²]{drag}&\·\xx{var}}
			\end{itemize}
		\end{enumerate}
	\item	\label{item:ÿux̱=thru}
		Motion verbs with the motion derivation 
			\motderiv{yux̱=}{n, \fm{yoo=i-…-k} rep}{out, outside}
			accompanied by a postposition phrase \fm{NP-náx̱} ‘via, through, across NP’
			which includes the perlative postposition \fm{-náx̱} ‘via, through, across’.
		\begin{itemize}
		\item	\vbform{anax̱ yux̱ woogoot}{pfv}[subj intr, \fm{n}, mot]{she came out through there}
			\parencite[186.376]{dauenhauer-dauenhauer:1987}
				\vbmorph{á&-náx̱&\gm{yux̱=}&wu-&μ-&\rt[¹]{gut}&-μμL}
					{\xx{3n}&\·\xx{perl}&out&\xx{pfv}&\xx{stv}&\rt[¹]{go.\xx{sg}}&\·\xx{var}}
		\item	\vbform{anax̱ yux̱ jiwdigoot}{pfv}[subj intr, \fm{n}, mot]{she stormed out}
			\parencite[180.291]{nyman-leer:1993}
				\vbmorph{á&-náx̱&\gm{yux̱=}&ji-&w-&d-&i-&\rt[¹]{gut}&-μμL}
					{\xx{3n}&\·\xx{perl}&out&hand&\xx{pfv}&\xx{mid}&\xx{stv}&\rt[¹]{go.\xx{sg}}&\·\xx{var}}
		\item	\vbform{a ítnáx̱ áwé yux̱ jiwdigoot}{pfv}[subj intr, \fm{n}, mot]{it was after her that it stormed out}
			\parencite[182.301]{nyman-leer:1993}
				\vbmorph{a&ít&-náx̱&á&-wé&\gm{yux̱=}&ji-&w-&d-&i-&\rt[¹]{gut}&-μμL}
					{\xx{3n.psr}&after&\·\xx{perl}&\xx{foc}&\·\xx{mdst}&out&hand&\xx{pfv}&\xx{mid}&\xx{stv}&\rt[¹]{go.\xx{sg}}&\·\xx{var}}
		\end{itemize}
	\item	\label{item:ÿux̱=from}
		Motion verbs with the motion derivation 
			\motderiv{yux̱=}{n, \fm{yoo=i-…-k} rep}{out, outside}
			accompanied by a postposition phrase \fm{NP-dáx̱} ‘away from NP’
			which includes the ablative postposition \fm{-dáx̱} ‘from, off of’.
		\begin{itemize}
		\item	\vbform{yú hít yeedáx̱ yux̱ wu.aadí}{pfv sub}[subj intr, \fm{n}, mot]{when they went out from that house}
			\parencite[206.258]{dauenhauer-dauenhauer:1987}
				\vbmorph{yú&hít&yee&-dáx̱&\gm{yux̱=}&wu-&\rt[¹]{.at}&-μμL&-í}
					{\xx{dist}&house&under&\·\xx{abl}&out&\xx{pfv}&\rt[¹]{go.\xx{pl}}&\·\xx{var}&\·\xx{sub}}
		\end{itemize}
	\end{enumerate}
\end{morphdesc}
