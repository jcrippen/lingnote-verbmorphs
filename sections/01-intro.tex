%!TEX root = ../lingnote-verbmorphs.tex

\section{Introduction}\label{sec:intro}

This is a catalogue of morphemes in Tlingit verbs.
It is inspired by \citeauthor{krauss:1981a}’s \textit{Eyak morpheme list} \parencite{krauss:1981a} for the Eyak language but goes beyond his approach to include allomorphs and common combinations of morphemes as well as examples and cross references.
It is meant to be a reference aid for language study and linguistic analysis and is not intended to be a comprehensive description of any particular grammar phenomenon.

This catalogue is organized into two parts:
section \ref{sec:alphalist} on page \pageref{sec:alphalist} is an alphabetic list of verb morphemes
and section \ref{sec:inventory} on page \pageref{sec:inventory} is an inventory of verb morphemes organized by their positions and related functions in verbs.
The alphabetic list allows the reader to look up morphemes by phonological form (spelling) without needing to know a morpheme’s meaning or function.
The position and function list allows the reader to look up morphemes by meaning and function without needing to know a morpheme’s form or allomorphy.

\subsection{Alphabet ordering}\label{sec:intro-alpha}

The alphabetic listing follows a conventional Latin alphabet ordering, so for example \fm{aa} precedes \fm{ach} and \fm{ee} precedes \fm{ei}.
Letters with underline diacritics (\fm{g̱}, \fm{ḵ}, \fm{x̱}) are listed immediately after the corresponding letters without an underline diacritic (\fm{g}, \fm{k}, \fm{x}) so for example \fm{ga} precedes \fm{g̱a}.
The two letters \fm{y} and \fm{ÿ} are not distinguished in order, so for example \fm{ÿa} precedes \fm{ye} precedes \fm{ÿi}.
Letters followed by an apostrophe – i.e.\ the ejective consonants – always follow letters without so for example \fm{ḵa} precedes \fm{ḵʼa}.

\begin{itemize}
\item	\fm{a},
	\fm{ch},
	\fm{d},
	\fm{e},
	\fm{g},
	\fm{g̱},
	\fm{h},
	\fm{i},
	\fm{j},
	\fm{k},
	\fm{ḵ},
	\fm{l},
	\fm{m},
	\fm{n},
	\fm{o},
	\fm{s},
	\fm{t},
	\fm{u},
	\fm{w},
	\fm{x},
	\fm{x̱},
	\fm{y}/\kern-0.125ex\fm{ÿ},
	\fm{ʼ}
\end{itemize}

Symbols other than letters are given after the alphabetic entries.
This includes symbols like \fm{μ} (vowel length), \fm{H} (high tone), and ⊗ (consonant deletion). 

\subsection{Morpheme representations}\label{sec:intro-morphrep}

Different kinds of morphemes are represented in different ways.
Tlingit has both affixes and clitics; among affixes there are both prefixes and suffixes and among clitics there are both proclitics and enclitics.
Prefixes are represented like \fm{x-} with a hyphen “-” following the form \fm{x} and likewise suffixes are represented like \fm{-y} with a hyphen preceding the form \fm{y}.
Proclitics are represented like \fm{x=} with an equals sign “=” following the form \fm{x} and likewise enclitics are represented like \fm{=y} with an equals sign preceding the form \fm{y}.
This representation of affixes and clitics is consistent with the widely followed \textit{Leipzig glossing rules} \parencite{comrie:2008}.

Some morphemes in Tlingit verbs are ‘smaller’ than even a single vowel or consonant.
Most such morphemes involve tone or lengthening of a vowel or sometimes both.
Tone is represented independently of a vowel as \fm{H} for high tone and \fm{L} for low tone.
Lengthening of a vowel is represented independently of the base vowel as \fm{μ} (Greek letter mu, the standard symbol for a mora).
Morphemes consisting of tone or vowel lengthening are given as though they are prefixes or suffixes depending on their position, but it should be understood that they are not actually separate from the  vowel segments that host them.
One example is the vowel lengthening allomorph \fm{μ-} of the stative prefix \fm{ya-} \~\ \fm{i-} as in \fm{x̱aatéen} ‘I can see him/her/it’ versus \fm{x̱at yatéen} ‘s/he/it can see me’.
In the form \fm{x̱aatéen} ‘I can see him/her/it’ the stative prefix appears as lengthening of the vowel in the \fm{x̱aa} syllable which would otherwise be short \fm{x̱a} as in \fm{x̱ax̱á} ‘I am eating him/her/it’.
Compare \fm{x̱a-μ-\rt[²]{tin}-μμH} for \fm{x̱aatéen} ‘I can see him/her/it’ with the \fm{μ-} stative prefix versus \fm{x̱a-\rt[²]{x̱a}-μH} for \fm{x̱ax̱á} ‘I am eating him/her/it’ without the \fm{μ-} stative prefix.

Beyond single morphemes like affixes and clitics, this catalogue also includes common combinations of morphemes as they appear within a verb word.
These combinations are given without a hyphen or equals sign because they are not single morphemes; instead their definition is a sequence of separate morphemes and the reader should refer to each of the individual elements separately.
As well as the lack of a hyphen or equals sign in their listing, all entries of combinations of morphemes can be identified by the “≡ \fm{a-b-c-…}” in the beginning of their definition.
This means that the combination is equivalent to “≡” the sequence of morphemes given.
For example, the form \fm{dli} ≡ \fm{d-l-i-} is the combination of the \fm{d-} voice prefix, the \fm{l-} valency prefix, and the \fm{i-} stative prefix.
Having identified \fm{dli} as a combination of morphemes, the reader can then look at the individual entries for \fm{d-}, \fm{l-}, and \fm{i-}.
This is illustrated by the following example entry:

\begin{morphdesc}
\item[dli]
	≡ \fm{d-l-i-}
	combination of \fm{d-} voice prefix,
		\fm{l-} (or \fm{lˢ-}) valency prefix,
		and \fm{i-} stative prefix
	\begin{itemize}
	\item	\fm{wutulitlʼíx} (pfv; tr, \fm{∅}, ach) ‘we made it dirty’ with \fm{l-i-}
		\versus{\fm{sh wutudlitlʼíx} (pfv) ‘we made ourselves dirty’ with \fm{d-l-i-}}
	\end{itemize}
\end{morphdesc}

\subsection{Lack of null morphemes}\label{sec:intro-null}

Null (zero) morphemes have been omitted from this catalogue.
The purpose of this catalogue is to provide a quick reference for identifying morphemes based on their phonological forms and since null morphemes have no phonological form they do not fit within the purpose of this document.
Null morphemes are really just artificial ‘bookkeeping’ elements used for linguistic analysis and do not actually exist in the language so it is questionable to ascribe meanings to them.
In addition, it is possible in principle to have a contrast between every overt morpheme and a corresponding null morpheme representing the absence of the overt morpheme.
Compare the following two analyses of the same verb form \fm{toox̱á} ‘we are eating him/her/it’ in (\ref{ex:intro-null-morphemes}) where (\ref{ex:intro-null-morphemes-nonull}) has only overt morphemes and (\ref{ex:intro-null-morphemes-lotsanull}) includes null morphemes.
The analysis in (\ref{ex:intro-null-morphemes-lotsanull}) has seven additional null affixes for the lack of an overt object, lack of overt irrealis, lack of overt aspect/conjugation, lack of overt modality, lack of a repetitive suffix, lack of overt past tense, and lack of overt clause type.

\pex\label{ex:intro-null-morphemes}%
\a\label{ex:intro-null-morphemes-nonull}%
\begingl
	\gla	\rlap{Toox̱á.} @ {} @ {} //
	\glb	too- \rt[²]{x̱a} -μH //
	\glc	\xx{1pl·s}- \rt[²]{eat} -\xx{var} //
	\gld	\rlap{\xx{impfv}.we.eat} {} {} //
	\glft	‘We are eating him/her/it.’
		//
\endgl
\a\label{ex:intro-null-morphemes-lotsanull}%
\begingl
	\gla	\rlap{Toox̱á.} @ {} @ {} @ {} @ {} @ {} @ {} @ {} @ {} @ {} //
	\glb	∅- ∅- ∅- ∅- too- \rt[²]{x̱a} -μH -∅ -∅ -∅ //
	\glc	\xx{3·o}- \xx{real}- \xx{impfv}- \xx{nmod}- \xx{1pl·s}-
		\rt[²]{eat} -\xx{var} -\xx{nrep} -\xx{npast} -\xx{nsub} //
	\gld	\rlap{\xx{impfv}.we.eat} {} {} {} {} {} {} {} {} {} //
	\glft	‘We are eating him/her/it.’
		//
\endgl
\xe

Although null morphology like in (\ref{ex:intro-null-morphemes-lotsanull}) can be useful for certain kinds of analysis, for most purposes it is needlessly complex.
In addition, listing all of the possible null morphemes in Tlingit verbs here would at least double the size of this document, making it much less useful as a quick reference.
The sole exception to the avoidance of null morphemes is the use of the symbol \fm{∅} to represent the distinct conjugation class that is characterized by the absence of one of the \fm{n-}, \fm{g̱-}, or \fm{g-} prefixes.
This \fm{∅} is not actually a morpheme but rather a label for the set of verbs that lack \fm{n-}, \fm{g̱-}, or \fm{g-} in certain forms where one would otherwise be expected (e.g.\ imperative forms like \fm{gashí!} ‘sing it!’
 with \fm{ga-} versus \fm{x̱á!} ‘eat it!’ with nothing).

\subsection{Verb root representation}\label{sec:intro-root}

Verb roots are represented in a similar way to the representation in \cite{crippen:2019}, but with one significant difference.
Specifically, the unpredictable tone patterns are now given by a superscript capital letter ᴴ or ᴸ rather than the less transparent ʼ or ʰ.
A root represented as \rt{CVʰ} in \cite{crippen:2019} is here given as \rt{CVᴸ} and a root represented as \rt{CVʼC} in \cite{crippen:2019} is here given as \rt{CVᴴC}.
This representation is iconically closer to the actual tone patterns in Northern Tlingit varieties.

For the case of \rt{CVʼC} → \rt{CVᴴC}, compare the two roots \fm{\rt{xit}} ‘scratch’ and \fm{\rt{siᴴt}} ‘braid’ (formerly \fm{\rt{siʼt}}).
The affirmative perfective forms in (\ref{ex:intro-root-xit-uH}) and (\ref{ex:intro-root-sit-uH}) both have a \fm{-μH} stem with a short vowel and high tone.
The negative perfective form for \fm{\rt{xit}} in (\ref{ex:intro-root-xit-uuL}) has the predicted \fm{-μμL} stem with a long vowel and low tone.
But the stem of \fm{\rt{siᴴt}} with \fm{-μμH} in (\ref{ex:intro-root-sit-uuH}) has high tone rather than low tone.
This deviation from the expected low tone is reflected by the representation of the root as \fm{\rt{siᴴt}} with ᴴ which replaces the earlier \fm{\rt{siʼt}}.

\begin{multicols}{2}
\pex\label{ex:intro-root-xit}%
\a\label{ex:intro-root-xit-uH}%
\begingl
	\gla	\rlap{Kax̱wlixít.} @ {} @ {} @ {} @ {} @ {} @ {} //
	\glb	ka- w- x̱- l- i- \rt[²]{xit} -μH //
	\glc	\xx{qual}- \xx{pfv}- \xx{1sg·s}- \xx{xtn}- \xx{stv}- \rt[²]{scratch} -\xx{var} //
	\gld	\rlap{\xx{pfv}.I.scratch} {} {} {} {} {} {} //
	\glft	‘I scratched, furrowed it.’
		//
\endgl
\a\label{ex:intro-root-xit-uuL}%
\begingl
	\gla	Tléil \rlap{kax̱wlax\gm{ee}t.} @ {} @ {} @ {} @ {} @ {} //
	\glb	tléil ka- w- x̱- l- \rt[²]{xit} -μμL //
	\glc	\xx{neg} \xx{qual}- \xx{pfv}- \xx{1sg·s}- \xx{xtn}- \rt[²]{scratch} -\xx{var} //
	\gld	not \rlap{\xx{pfv}.I.scratch} {} {} {} {} {} //
	\glft	‘I didn’t scratch, furrow it.’
		//
\endgl
\xe

\pex\label{ex:intro-root-sit}%
\a\label{ex:intro-root-sit-uH}%
\begingl
	\gla	\rlap{Kax̱wlisít.} @ {} @ {} @ {} @ {} @ {} @ {} //
	\glb	ka- w- x̱- l- i- \rt[²]{siᴴt} -μH //
	\glc	\xx{qual}- \xx{pfv}- \xx{1sg·s}- \xx{xtn}- \xx{stv}- \rt[²]{braid} -\xx{var} //
	\gld	\rlap{\xx{pfv}.I.scratch} {} {} {} {} {} {} //
	\glft	‘I braided it.’
		//
\endgl
\a\label{ex:intro-root-sit-uuH}%
\begingl
	\gla	Tléil \rlap{kax̱wlas\gm{ée}t.} @ {} @ {} @ {} @ {} @ {} //
	\glb	tléil ka- w- x̱- l- \rt[²]{siᴴt} -μμH //
	\glc	\xx{neg} \xx{qual}- \xx{pfv}- \xx{1sg·s}- \xx{xtn}- \rt[²]{braid} -\xx{var} //
	\gld	not \rlap{\xx{pfv}.I.scratch} {} {} {} {} {} //
	\glft	‘I didn’t braid it.’
		//
\endgl
\xe
\end{multicols}

For the case of \fm{\rt{CVʰ}} → \fm{\rt{CVᴸ}}, compare the two roots \fm{\rt{ta}} ‘boil’ and \fm{\rt{taᴸ}} ‘sg.\ sleep’ (formerly \fm{\rt{taʰ}}).
The prospective forms in (\ref{ex:intro-root-ta-uuH}) and (\ref{ex:intro-root-taL-uuH}) both have a stem with a long vowel and high tone \fm{-μμH}.
The repetitive imperfective form for \fm{\rt{ta}} in (\ref{ex:intro-root-ta-ueuH}) has the predicted stem with a long vowel, ablaut, and high tone \fm{-μₑμH}.
But the stem of \fm{\rt{taᴸ}} with \fm{-μₑμL} in (\ref{ex:intro-root-taL-ueuL}) has low tone rather than high tone.
This deviation from the expected tone is reflected by the representation of the root as \fm{\rt{taᴸ}} with ᴸ which replaces the earlier \fm{\rt{taʰ}}.

\begin{multicols}{2}
\pex\label{ex:intro-root-ta}%
\a\label{ex:intro-root-ta-uuH}%
\begingl
	\gla	\rlap{Kuḵasatáa.} @ {} @ {} @ {} @ {} @ {} @ {} //
	\glb	g- u- g̱- x̱a- sa- \rt[¹]{ta} -μμH //
	\glc	\xx{gcnj}- \xx{irr}- \xx{mod}- \xx{1sg·s}- \xx{csv}- \rt[¹]{boil} -\xx{var} //
	\gld	\rlap{\xx{prosp}.I.make.boil} {} {} {} {} {} {} //
	\glft	‘I will boil it.’
		//
\endgl
\a\label{ex:intro-root-ta-ueuH}%
\begingl
	\gla	\rlap{X̱asat\gm{éi}x̱.} @ {} @ {} @ {} @ {} //
	\glb	x̱a- sa- \rt[¹]{ta} -μₑμH -x̱ //
	\glc	\xx{1sg·s}- \xx{csv}- \rt[¹]{boil} -\xx{var} -\xx{rep} //
	\gld	\rlap{\xx{impfv}.I.make.boil.\xx{rep}} {} {} {} {} //
	\glft	‘I repeatedly boil it.’
		//
\endgl
\xe

\pex\label{ex:intro-root-taL}%
\a\label{ex:intro-root-taL-uuH}%
\begingl
	\gla	\rlap{Kuḵasatáa.} @ {} @ {} @ {} @ {} @ {} @ {} //
	\glb	g- u- g̱- x̱a- sa- \rt[¹]{taᴸ} -μμH //
	\glc	\xx{gcnj}- \xx{irr}- \xx{mod}- \xx{1sg·s}- \xx{csv}- \rt[¹]{sleep·\xx{sg}} -\xx{var} //
	\gld	\rlap{\xx{prosp}.I.make.sleep·\xx{sg}} {} {} {} {} {} {} //
	\glft	‘I will make him/her/it sleep.’
		//
\endgl
\a\label{ex:intro-root-taL-ueuL}%
\begingl
	\gla	\rlap{X̱asat\gm{ei}x̱.} @ {} @ {} @ {} @ {} //
	\glb	x̱a- sa- \rt[¹]{taᴸ} -μₑμL -x̱ //
	\glc	\xx{1sg·s}- \xx{csv}- \rt[¹]{sleep·\xx{sg}} -\xx{var} -\xx{rep} //
	\gld	\rlap{\xx{impfv}.I.make.sleep·\xx{sg}.\xx{rep}} {} {} {} {} //
	\glft	‘I repeatedly make him/her/it sleep.’
		//
\endgl
\xe
\end{multicols}

\subsection{Abbreviations and verb lexical information}\label{sec:intro-abbrev}

Abbreviations have been generally avoided in favour of complete names in English.
The only major exception is the representation of verb lexical information.
A verb used in an example is given with its grammatical and lexical information in parentheses before the English translation.
The first use of a verb in an example is followed in parentheses by the grammatical aspect, then a semicolon, and then most of the important lexical information separated by commas. The second use of the verb (usually in a form contrasting with the first form) has only the grammatical aspect in parentheses and the rest of the information is implicitly the same.
Consider the following example:

\begin{itemize}
\item	\fm{at wutusixook} (pfv; tr, \fm{g̱}, \fm{-μμL} act) ‘we dried something’
	\versus{\fm{wutusixook} (pfv) ‘we dried him/her/it’}
\end{itemize}

Immediately following the form “\fm{at wutusixook}” is the parenthesized list “(pfv; tr, \fm{g̱}, \fm*{-μμL} act)”.
This list has two parts: before the semicolon “;” there is the grammatical aspect of the verb form and then after the semicolon there is the lexical information of the verb.
The grammatical aspect “pfv” indicates that the form is perfective aspect.
The “tr” indicates that the verb is transitive (takes both subject and object arguments).
The \fm{g̱} means that this verb is a member of the \fm{g̱} conjugation class and so for instance will have the prefix \fm{g̱-} in an imperative mood form and the preverb \fm{yei=} in a progressive aspect form.
The “\fm{-μμL} act” indicates the verb’s eventuality class: the verb is an activity verb and its imperfective aspect forms will normally have \fm{-μμL} stem variation with a long vowel and low tone.
The second form \fm{wutusixook} is based on the same verb lexical item with the same lexical information, so only its grammatical aspect is given in parentheses.
Contrast this with the next example that has two different verb lexical items:

\begin{itemize}
\item	\fm{at wutusixook} (pfv; tr, \fm{g̱}, \fm{-μμL} act) ‘we dried something’
	\versus{\fm{at wutuwax̱áa} (pfv; tr, \fm{∅}, \fm{-μH} act) ‘we ate something’}
\end{itemize}

Here the second form “\fm{at wutuwax̱áa}” is a different verb as can be seen by the stem \fm{x̱áa} (thus root \fm{\rt[²]{x̱a}} ‘eat’) which is unrelated to the stem \fm{xook} (root \fm{\rt[¹]{xuk}} ‘dried’).
Since this second form is a different verb it is given with its full list of verb lexical information.
Comparing the two we see for example that where the verb of “\fm{at wutusixook}” belongs to the \fm{g̱} conjugation class, the verb of “\fm{at wutuwax̱áa}” belongs to the \fm{∅} conjugation class.
In addition, although both are activity verbs given “act”, the verb of “\fm{at wutusixook}” has \fm{-μμL} long high tone stem variation in its imperfective aspect form (e.g.\ \fm{at tusaxook} ‘we are drying something’) where instead the verb of “\fm{at wutuwax̱áa}” has \fm{-μH} short high tone stem variation in its imperfective aspect form (e.g.\ \fm{at toox̱á} ‘we are eating something’).

The following lists give the abbreviations used for the different parts of verb form information.
These lists are meant to be exhaustive so if there is an abbreviation used in the document that is not listed here, please report it so it can be corrected.

\vspace{\baselineskip}
\noindent
List of grammatical aspect, mood, tense, and clause type abbreviations:
\begin{description}[font={\normalfont}, style=sameline, labelindent=\parindent, labelwidth=3em, leftmargin=!]
\item[admon]	admonitive mood
\item[cmpv]	comparative derivation
\item[cond]	conditional mood
\item[csec]	consecutive aspect
\item[ctng]	contingent mood
\item[hab]	habitual aspect
\item[hort]	hortative mood
\item[imp]	imperative mood
\item[impfv]	imperfective aspect
\item[past]	past tense
\item[pfv]	perfective aspect
\item[pot]	potential mood
\item[prog]	progressive aspect
\item[prosp]	prospective (‘future’) aspect
\item[rel]	relative clause type
\item[rep]	repetitive iterativity
\item[rlzn]	realizational aspect
\item[sub]	subordinate clause type
\end{description}

\vspace{\baselineskip}
\noindent
List of transitivity abbreviations:
\begin{description}[font={\normalfont}, style=sameline, labelindent=\parindent, labelwidth=4em, leftmargin=!]
\item[impers]		impersonal: no subject and no object
\item[subj intr]	subject intransitive: subject but no object (≡ unergative)
\item[obj intr]		object intransitive: object but no subject (≡ unaccusative)
\item[tr]		transitive: both subject and object
\end{description}

\vspace{\baselineskip}
\noindent
List of conjugation class symbols:
\begin{description}[font={\normalfont}, style=sameline, labelindent=\parindent, labelwidth=3em, leftmargin=!]
\item[\fm{n}]	\fm{n} conjugation class\newline
		reflected by \fm{n-} prefix in imperative mood
\item[\fm{g̱}]	\fm{g̱} conjugation class\newline
		reflected by \fm{g̱-} prefix in imperative mood\newline
		and \fm{yei=} preverb in prospective aspect
\item[\fm{g}]	\fm{g} conjugation class\newline
		reflected by \fm{g-} prefix in imperative mood\newline
		and \fm{kei=} preverb in prospective aspect
\item[\fm{∅}]	\fm{∅} conjugation class\newline
		reflected by absence of prefix in imperative mood
\end{description}

\vspace{\baselineskip}
\noindent
List of eventuality class abbreviations:
\begin{description}[font={\normalfont}, style=sameline, labelindent=\parindent, labelwidth=3em, leftmargin=!]
\item[act]	activity\newline
		has an imperfective aspect form without stative prefix
		and without repetitive suffix, does not require motion derivation
\item[state]	state\newline
		has an imperfective aspect form with stative prefix
		and without repetitive suffix, does not require motion derivation
\item[ach]	achievement\newline
		does not have imperfective aspect form without repetitive suffix,
		does not require motion derivation
\item[mot]	motion\newline
		does not have imperfective aspect form without repetitive suffix
		and requires motion derivation
\end{description}
