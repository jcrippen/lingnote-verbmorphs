%!TEX root = ../lingnote-verbmorphs.tex

\subsection{E}\label{sec:alphalist-e}
\begin{morphdesc}[resume*=alphalist]
\item[ee]\label{m:ee}
	≡ \fm{a-i-}
	combination of argument marking \X{a-}
		and second person singular subject \X[i-2sg]{i-}

\item[ee-]
	allomorph of second person singular subject \fm{i-}

\item[ee=]
	allomorph of second person singular object \fm{i-}

\item[-ee]\label{m:-ee-sub}
	allomorph of subordinate clause suffix \X[-í-sub]{-í}

\item[-ée]\label{m:-ée-sub}
	allomorph of subordinate clause suffix \X[-í-sub]{-í}

\item[éeg̱i=]\label{m:éeg̱i=}
	Variant form of the locational preverb \X{éig̱i=} ‘beach’.
	Attested in Transitional Northern varieties, but expected to also occur in Southern varieties.
	Derived from the noun \fm{éeḵ} \~\ \fm{éiḵ} ‘beach’
		with the special locative postposition allomorph
		\X[-i-loc]{-i} \~\ \X[-í-loc]{-í} ‘at’
		(instead of the regular allomorphs \fm{-xʼ} and \X[-μ-loc]{-μ}).
	As with \fm{éig̱i=}, \fm{éeg̱i=} can be either segmented into the noun \fm{éeḵ} ‘beach’
		and the postposition \fm{-i} ‘locative’
		or instead presented as a single unit \fm{éeg̱i}
		that may be glossed as either ‘on beach’ or simply ‘beach’.
	The examples below illustrate both strategies.
	Compare \X{ÿeiḵ=} and its variant forms from the same noun without the locative postposition.
	\begin{itemize}
	\item	\vbform{yú éeg̱i yan aa uhaanch}{hab}[subj intr, \fm{∅}, mot]{one would stand out on the beach}
		\parencite[64.31]{dauenhauer-dauenhauer:1987}
			\vbmorph{yú&\gm{éeḵ}&\gm{-i=}&yan=&aa=&u-&\rt[¹]{han}&-μμL&-ch}
				{\xx{dist}&beach&\·\xx{loc}&\xx{term}&\xx{part}&\xx{zpfv}&\rt[¹]{stand.\xx{sg}}&\·\xx{var}&\·\xx{rep}}
	\item	\vbform{du tláa éeg̱i daxáash}{impfv}[subj intr, \fm{n}, \fm{-μμH} act]{his mother is cutting (fish) on the beach}
		\parencite[315.14]{swanton:1909}
			\vbmorph{du&tláa&\gm{éeg̱i=}&da-&\rt[²]{xash}&-μμH}
				{\xx{3h.psr}&mother&on.beach&\xx{apsv}&\rt[²]{cut}&\·\xx{var}}
	\end{itemize}

\item[eeḵ=]\label{m:eeḵ=}
	Variant form of the directional preverb \X{ÿeiḵ=} ‘beach’ used in Southern Tlingit varieties,
		derived from the noun \fm{éeḵ} \~\ \fm{éiḵ} ‘beach’.
	Parallels the Tongass form \X{eèḵ=} [\ipa{ʔiːʰq}].
	Poorly attested with only one instance identified.
	See \X{ÿeeḵ=} for an alternative form that is reported but not attested.
	Compare \X{éeg̱i=} \~\ \X{éig̱i=} from the same noun
	with the special locative postposition allomorph
		\X[-i-loc]{-i} \~\ \X[-í-loc]{-í} ‘at’.
	\begin{itemize}
	\item	\vbform{eeḵ wudi̥yáa}{pfv}[subj intr, \fm{∅}, mot]{he packed it down}
		\parencite[17]{leer:1975g}
			\vbmorph{\gm{eeḵ=}&wu-&d-&i-&\rt[²]{ÿa}&-μμH}
				{beach&\xx{pfv}&\xx{apsv}&\xx{stv}&\rt[²]{backpack}&\·\xx{var}}
		\newline
		The original translation has an object ‘it’ suggesting a referential object
			but the form appears to be antipassive and as such a closer translation
			might be either ‘he packed stuff down’ with an indefinite object
			or ‘he packed down’ without an object.
	\end{itemize}

\item[eèḵ=]\label{m:eèḵ=}
	Variant form of the directional preverb \X{ÿeiḵ=} ‘beach’ used in Tongass Tlingit varieties,
		derived from the noun \fm{éeḵ} \~\ \fm{éiḵ} ‘beach’, Tongass \fm{eeḵ} [\ipa{ʔiːq}].
	Parallels the Southern form \X{eeḵ=} and the Taku Inland Northern form \X{eiḵ=}.
	Compare \X{éeg̱i=} \~\ \X{éig̱i=} from the same noun
		with the special locative postposition allomorph
		\X[-i-loc]{-i} \~\ \X[-í-loc]{-í} ‘at’.
	\begin{itemize}
	\item	\vbform{aàn eèḵ wutu̥wa.at}{pfv}[subj intr, \fm{∅}, mot]{we went down with them}
		\parencite[66.233]{leer:1978}
			\vbmorph{aà&-n&\gm{eèḵ=}&wu-&tu-&wa-&\rt[¹]{.at}&-μ}
				{\xx{3n}&\·\xx{instr}&beach&\xx{pfv}&\xx{1pl.s}&\xx{stv}&\rt[¹]{go.\xx{pl}}&\·\xx{var}}
	\item \vbform{eèḵ nagut}{prog}{she was walking down to the beach}
		\parencite[104.13]{leer:1978}
			\vbmorph{\gm{eèḵ=}&na-&\rt[¹]{gut}&-μ}
				{beach&\xx{ncnj}&\rt[¹]{go.\xx{sg}}&\·\xx{var}}		
	\end{itemize}

\item[eeÿa]\label{m:eeÿa-a-i-ÿa}
	≡ \fm{a-i-ÿa-}
	combination of argument marking \X{a-}
		and second person singular subject \X[i-2sg]{i-}
		and stative \X[ÿa-stv]{ÿa-};
	same form as \X[eeÿa-a-ʷ-i-ÿa]{eeÿa} ≡ \fm{a-ʷ-i-ÿa-}
		which has perfective \X[ʷ-pfv]{ʷ-}

\item[eeÿa]\label{m:eeÿa-a-ʷ-i-ÿa}
	≡ \fm{a-ʷ-i-ÿa-}
	combination of argument marking \X{a-}
		and perfective \X[ʷ-pfv]{ʷ-}
		and second person singular subject \X[i-2sg]{i-}
		and stative \X[ÿa-stv]{ÿa-};
	same form as \X[eeÿa-a-i-ÿa]{eeÿa} ≡ \fm{a-i-ÿa-}
		which does not have perfective \X[ʷ-pfv]{ʷ-}

\item[-éi]\label{m:-éi}
	≡ \fm{-i yéi}
	combination of relative clause suffix \X[-i-rel]{-i}
		and noun \fm{yéi} \~\ \fm{yé} ‘place, manner, way’;
	associated with certain verbs where it is particularly common, but can occur in general with
		any verb so this is not lexically specified or restricted;
	apparently optional in all varieties where it occurs, so that speakers produce and accept
		forms with uncontracted \fm{…i yéi} as well as contracted \fm{…éi}
	\begin{enumerate}
	\item	the ordinary case of contraction \fm{…i yéi} → \fm{…éi} which can occur with any
			suitable relative clause form of a verb;
		often found with similative \fm{yáx̱} ‘like’ as in \fm{…éi yáx̱ yatee} ‘like the way that …’
			though occurs in other contexts as well
		\cite{story-naish:1973} explicitly indicate several examples of these as “…-éi”
			with a hyphen setting off the contraction from the verb stem;
		\begin{itemize}
		\item	\vbform{kadlichʼáchʼx̱éi}{rep impfv}[obj intr, \fm{∅}, \fm{-x̱} state]{the way that she/he/it is spotted}
			\parencite[206.2877]{story-naish:1973}
				\vbmorph{ka-&d-&lˢ-&i-&\rt{chʼachʼ}&-μH&-x̱&-i&yéi}
					{\xx{qual}&\xx{mid}&\xx{xtn}&\xx{stv}&\rt{spotted}&\·\xx{var}&\·\xx{rep}&\·\xx{rel}&way}
			\versus \vbform{kawdichʼáchʼ}{pfv}{she/he/it has gotten spots}
			\parencites[206.2876]{story-naish:1973}[10/240]{leer:1973}
				\vbmorph{ka-&w-&d-&i-&\rt[¹]{chʼachʼ}&-μH}
					{\xx{hsfc}&\xx{pfv}&\xx{mid}&\xx{stv}&\rt[¹]{spotted}&\·\xx{var}}
		\item	\vbform{naaliyéi}{impfv}[obj intr, \fm{n}, \fm{-μμH} state]{the way that she/he/it is far away}
			\parencite[20.210]{nyman-leer:1993}
				\vbmorph{na-&μ-&\rt[¹]{li}&-μL&-ÿi&yéi}
					{\xx{ncnj}&\xx{stv}&\rt[¹]{distant}&\·\xx{var}&\·\xx{rel}&way}
			\versus \vbform{naaléeyi yéi}{impfv}{the way that she/he/it is far away}
				\vbmorph{na-&μ-&\rt[¹]{li}&-μμH&-ÿi&yéi}
					{\xx{ncnj}&\xx{stv}&\rt[¹]{distant}&\·\xx{var}&\·\xx{rel}&way}
			\versus \vbform{naalée}{impfv}{she/he/it is far away}
				\vbmorph{na-&μ-&\rt[¹]{li}&-μμH}
					{\xx{ncnj}&\xx{stv}&\rt[¹]{distant}&\·\xx{var}}
		\end{itemize}
	\item	possibly in a handful of nouns that may be derived from relative clauses
		\begin{itemize}
		\item	\fm{ḵʼaakanéi} ‘large grease dish’
		\item	\fm{naag̱asʼéi} ‘fox’
			from \fm{\rt{g̱asʼ}} ‘scratch’
				\vbmorph{na-&μ-&\rt{g̱asʼ}&-μL&-i&yéi}
					{\xx{ncnj}&\xx{stv}&\rt{scratch}&\·\xx{var}&\·\xx{rel}&way}
			\versus \vbform{akag̱áasʼ}{impfv}[tr, \fm{n}/\fm{∅}, act]{she/he/it scratched him/her/it}
				\vbmorph{a-&ka-&\rt[²]{g̱asʼ}&-μμH}
					{\xx{3>3}&\xx{qual}&\rt[²]{scratch}&\·\xx{var}}
		\item	\fm{naatsilanéi} ‘blowfly larva’
		\item	\fm{taatlʼeeshdéi} ‘red winged blackbird’
		\item	\fm{tlaag̱eendéi} \~\ \fm{tlaaḵʼeendéi} ‘leech’
		\item	\fm{tsʼeig̱eenéi} ‘magpie’				
		\item	\fm{yaanashg̱wanéi} ‘nighthawk; green-winged teal’
			from \fm{\rt{g̱wan}} ‘go spread-limbed’
				\vbmorph{ÿaa=&na-&d-&sh-&\rt{g̱wan}&-μL&-i&yéi}
					{along&\xx{ncnj}&\xx{mid}&\xx{pej}&\rt{go.spread}&\·\xx{var}&\·\xx{rel}&way}
			\versus \vbform{wujig̱waan}{pfv}[subj intr, \fm{n}, mot]{it (frog) went spread-legged; it (bird) went spread-winged}
				\parencite[853]{leer:1976}
				\vbmorph{wu-&d-&sh-&i-&\rt{g̱wan}&-μμL}
					{\xx{pfv}&\xx{mid}&\xx{pej}&\xx{stv}&\rt{go.spread}&\·\xx{var}}
			\newline
			compare \vbform{awlig̱wán}{pfv}[tr, \fm{∅}, ach]{she/he/it tied him/her/it in a bow}
				and \fm{lag̱wán} ‘bow (tied)’
		\end{itemize}
	\end{enumerate}

\item[éig̱i=]\label{m:éig̱i=}
	Locational preverb ‘beach’ indicating location on land above the shoreline of a body of water.
	Derived from the noun \fm{éeḵ} \~\ \fm{éiḵ} ‘beach’
		with the special locative postposition allomorph
		\X[-i-loc]{-i} \~\ \X[-í-loc]{-í} ‘at’
		(instead of the regular allomorphs \fm{-xʼ} and \X[-μ-loc]{-μ}).
	This locative allomorph is unique in that it only occurs with a handful of preverbs,
		for which see the detailed entry of \X[-í-loc]{-í}.
	Variation between forms with \fm{ei} [\ipa{è}ː] and forms with \fm{ee} [\ipa{ìː}]
		reflects the more general pattern of uvular lowering where \fm{i} /\ipa{i}/
		is lowered to \fm{e} /\ipa{e}/ next to a uvular sound, seen for example in
		\fm{g̱eey} [\ipa{qìːj}] \~\ \fm{g̱eiy} [\ipa{qèːj}] ‘bay’ as well as in the
		pair of \fm{éeḵ} \~\ \fm{éiḵ} ‘beach’.
	As with other preverbs including \X[-i-loc]{-i}, \fm{éig̱i=} can be represented either
		fully segmented as \fm{éig̱-i=}
		or unsegmented as \fm{éig̱i=}.
	\newline
	Variant forms:
	\begin{allolist}
	\item[{\X{éeg̱i=}}]	Form in Transitional Northern and Southern varieties without uvular lowering
	\end{allolist}
	\begin{enumerate}
	\item\label{item:éig̱i=motderiv}
		Motion verbs with the motion derivation
			\motderiv{éig̱i=}{∅, \fm{-x̱} rep}{on beach}.
		This motion derivation is not listed by \textcite[301]{leer:1991}
			but it is part of the same set as \fm{gági=} ‘into open’,
			\fm{dáag̱i=} ‘inland’,
			and \fm{héeni=} ‘in water’,
			all of which assign \fm{∅} conjugation class and \fm{-x̱} repetitive.
	\item\label{item:éig̱i=motderiv-multi}
		Motion verbs with multiple motion derivations,
			one of which adds \fm{éig̱i=}.
		\begin{itemize}
		\item	\vbform{éig̱i yux̱ wutuwax̱óotʼ}{pfv}[tr, \fm{n}, mot]{we dragged her outside on the beach}
			\parencite[285.12]{swanton:1909}
				\vbmorph{\gm{éig̱i=}&yux̱=&wu-&tu-&wa-&\rt[²]{x̱utʼ}&-μμH}
					{beach&out&\xx{pfv}&\xx{1sg.s}&\xx{stv}&\rt[²]{drag}&\·\xx{var}}
		\item	\vbform{éig̱i aan ÿeiḵ uwagút}{pfv}[subj intr, \fm{∅}, mot]{he went down to the beach with it}
			\parencite[263.4]{swanton:1909}
				\vbmorph{\gm{éig̱i=}&aa&-n&ÿeiḵ=&u-&wa-&\rt[¹]{gut}&-μH}
					{beach&\xx{3n}&\xx{instr}&beach&\xx{zpfv}&\xx{stv}&\rt[¹]{go.\xx{sg}}&\·\xx{var}}
			\newline
			This form is notable because it includes both \fm{éig̱i=} and \X{ÿeiḵ=}.
		\end{itemize}
	\end{enumerate}

\item[eiḵ=]\label{m:eiḵ=}
	Variant form of directional preverb \X{ÿeeḵ=} \~\ \X{ÿeiḵ=} ‘beach’.
	Attested only from one speaker (\fm{Seidayaa} E.\ Nyman)
		but this form is plausibly underdocumented and more widespread.
	Derived from the same noun \fm{éeḵ} \~\ \fm{éiḵ} ‘beach’ as the other forms of this preverb.
	Compare the Tongass form \X[eeḵ=]{eèḵ=}.
	\begin{itemize}
	\item	\vbform{yáaxʼ eiḵ u.aatch}{hab}[subj intr, \fm{∅}, mot]{they come down to the beach here}
		\parencite[158.1239]{nyman-leer:1993}
			\vbmorph{yáa&-xʼ&\gm{eiḵ=}&u-&\rt[¹]{.at}&-μμL&-ch}
				{\xx{prox}&\·\xx{loc}&beach&\xx{zpfv}&\rt[¹]{go.\xx{pl}}&\·\xx{var}&\·\xx{rep}}
	\item	\vbform{eiḵ nashxʼílʼ}{prog}[obj intr, \fm{∅}, mot]{it is sliding down to the beach}
		\parencite[178.219]{nyman-leer:1993}
			\vbmorph{\gm{eiḵ=}&na-&sh-&\rt[¹]{x̱ʼilʼ}&-μH}
				{beach&\xx{ncnj}&\xx{pej}&\rt[¹]{slide}&\·\xx{var}}
	\item	\vbform{áx̱ eiḵ ayawdlitsáḵ}{pfv}[tr, \fm{∅}, mot]{he poked his way with her down to the beach there}
		\parencite[191.470]{nyman-leer:1993}
			\vbmorph{á&-x̱&\gm{eiḵ=}&a-&ÿa-&w-&d-&lˢ-&i-&\rt[²]{tsaḵ}&-μH}
				{\xx{3n}&\·\xx{pert}&beach&\xx{3>3}&\xx{qual}&\xx{pfv}&\xx{mid}&\xx{xtn}&\xx{stv}&\rt[²]{poke}&\·\xx{var}}
	\end{itemize}
		
\end{morphdesc}
