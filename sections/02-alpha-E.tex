%!TEX root = ../lingnote-verbmorphs.tex

\subsection{E}\label{sec:alphalist-e}
\begin{morphdesc}[resume*=alphalist]
\item[ee]\label{m:ee}
	≡ \fm{a-i-}
	combination of argument marking \X{a-}
		and second person singular subject \X[i-2sg]{i-}

\item[ee-]
	allomorph of second person singular subject \fm{i-}

\item[ee=]
	allomorph of second person singular object \fm{i-}

\item[éeg̱i=]
	directional noun ‘beach’ with special locative postposition \fm{-í} \~\ \fm{-i} ‘at’,
	variant form \fm{éig̱i=};
	derived from noun \fm{éeḵ} \~\ \fm{éiḵ} ‘beach’;
	compare \fm{ÿeeḵ=} \~\ \fm{ÿeiḵ=} \~\ \fm{eèḵ=}

\item[eèḵ=]
	variant form of directional preverb \fm{ÿeeḵ=} \~\ \fm{ÿeiḵ=} used in Tongass Tlingit

\item[eeÿa]\label{m:eeÿa-a-i-ÿa}
	≡ \fm{a-i-ÿa-}
	combination of argument marking \X{a-}
		and second person singular subject \X[i-2sg]{i-}
		and stative \X[ÿa-stv]{ÿa-};
	same form as \X[eeÿa-a-ʷ-i-ÿa]{eeÿa} ≡ \fm{a-ʷ-i-ÿa-}
		which has perfective \X[ʷ-pfv]{ʷ-}

\item[eeÿa]\label{m:eeÿa-a-ʷ-i-ÿa}
	≡ \fm{a-ʷ-i-ÿa-}
	combination of argument marking \X{a-}
		and perfective \X[ʷ-pfv]{ʷ-}
		and second person singular subject \X[i-2sg]{i-}
		and stative \X[ÿa-stv]{ÿa-};
	same form as \X[eeÿa-a-i-ÿa]{eeÿa} ≡ \fm{a-i-ÿa-}
		which does not have perfective \X[ʷ-pfv]{ʷ-}

\item[éig̱i=]
	variant form of directional noun \fm{éeg̱i=} ‘beach’ used in some Northern varieties;
	arises from uvular lowering of \fm{ée} to \fm{éi}
\end{morphdesc}
