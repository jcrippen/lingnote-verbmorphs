%!TEX root = ../lingnote-verbmorphs.tex

\subsection{C}\label{sec:alphalist-c}

\begin{morphdesc}[resume*=alphalist]
\item[-ch]\label{m:-ch}
	repetitive suffix indicating iteration of an eventuality;
	occurs in several different contexts:
		(items \ref{item:-ch-conjclass}, \ref{item:-ch-motderiv}, \ref{item:-ch-oddrep})
		in repetitive imperfective forms,
		(\ref{item:-ch-hear}) in several verbs based on \fm{\rt[²]{.ax̱}} ‘hear’,
		(\ref{item:-ch-hab}) in habitual forms,
		(\ref{item:-ch-combo}) in combinations with other repetitive or derivational suffixes,
		(\ref{item:-ch-roots}) and perhaps frozen in some stems and roots;
	the widespread distribution of \fm{-ch}	suggests it may be a
		relatively abstract or generic kind of iterativity
		in contrast with more specific kinds expressed by other
		repetitive suffixes
	\newline
	allomorphs:
	\begin{allolist}
	\item[-ch]	basic form
	\item[\X{-j}]	orthographic variant used when followed by a vowel
	\end{allolist}
	\begin{enumerate}
	\item	\label{item:-ch-conjclass}
		repetitive suffix in repetitive imperfective forms
		predicted by lexically specified \fm{g̱} or \fm{g} conjugation class
		\begin{itemize}
		\item[\fm{g̱}:]	\X{yei=}…\fm{-ch} (\ref{item:-ch-conjclass-g̱})
		\item[\fm{g}:]	\X{kei=}…\fm{-ch} (\ref{item:-ch-conjclass-g})
		\end{itemize}
		\begin{enumerate}
		\item	\label{item:-ch-conjclass-g̱}
			\fm{g̱} conjugation class repetitive imperfectives
				with \X{yei=} ‘down’ and \fm{-ch};
			these overlap in form with the \fm{∅} conjugation class
				\fm{yei=…-ch} repetitive imperfectives
				from the motion derivation
				\motderiv{yei=}{\fm{∅}, \fm{-ch} rep}{down, disembarking, exiting}
				(see \ref{item:-ch-motderiv-zdir} below)
				but the conjugation class is different with consequent differences
				in other forms like imperatives
			\begin{itemize}
			\item	\fm{–níx̱ch}
				from \fm{\rt[¹]{nix̱}} \~\ \fm{\rt[¹]{nex̱}} ‘safe, rescued’ in
				\newline
				\vbform{yei asníx̱ch}{rep impfv}[tr, \fm{g̱}, ach]{she/he/it repeatedly rescues him/her/it}
					\vbmorph{\gm{yei=}&a-&s-&\rt[¹]{nix̱}&-μH&\gm{-ch}}
						{down&\xx{3>3}&\xx{csv}&\rt[¹]{safe}&\·\xx{var}&\·\xx{rep}}
				\versus \vbform{g̱asneex̱!}{imp}{save him/her/it!}
					\vbmorph{g̱a-&s-&\rt[¹]{nix̱}&-μμL}
						{\xx{g̱cnj}&\xx{csv}&\rt[¹]{safe}&\·\xx{var}}
			\item	\fm{–sʼélʼch}
				from \fm{\rt[²]{sʼelʼ}} ‘rip, tear, peel’ in
				\newline
				\vbform{yei akasʼélʼch}{rep impfv}[tr, \fm{g̱}, ach]{she/he/it repeatedly tears, peels off him/her/it}
					\vbmorph{\gm{yei=}&a-&ka-&\rt[²]{sʼelʼ}&-μH&\gm{-ch}}
						{down&\xx{3>3}&\xx{hsfc}&\rt[²]{tear}&\·\xx{var}&\·\xx{rep}}
				\versus \vbform{kag̱asʼéilʼ!}{imp}{tear, peel off him/her/it!}
					\vbmorph{ka-&g̱a-&\rt[²]{sʼelʼ}&-μμH}
						{\xx{hsfc}&\xx{g̱cnj}&\rt[²]{tear}&\·\xx{var}}
			\item	\fm{–shátch}
				from \fm{\rt[²]{shaᴴt}} ‘grab’ in
				\newline
				\vbform{yei alshátch}{rep impfv}[tr, \fm{g̱}, ach]{she/he/it repeatedly captures, holds down him/her/it}
					\vbmorph{\gm{yei=}&a-&lˢ-&\rt[²]{shaᴴt}&-μH&\gm{-ch}}
						{down&\xx{3>3}&\xx{xtn}&\rt[²]{grab}&\·\xx{var}&\·\xx{rep}}
				\versus \vbform{g̱alsháat!}{imp}{capture, hold down him/her/it!}
					\vbmorph{g̱a-&lˢ-&\rt[²]{shaᴴt}&-μμH}
						{\xx{g̱cnj}&\xx{xtn}&\rt[²]{grab}&\·\xx{var}}
			\item	\fm{–ḵéech}
				from \fm{\rt[¹]{ḵi}} ‘plural sit’ in
				\newline
				\vbform{yei has ḵéech}{rep impfv}[subj intr, \fm{g̱}, ach]{they repeatedly sit down}
					\vbmorph{\gm{yei=}&has=&\rt[¹]{ḵi}&-μμH&\gm{-ch}}
						{down&\xx{plh}&\rt[¹]{sit.\xx{pl}}&\·\xx{var}&\·\xx{rep}}
				\versus \vbform{g̱ayḵí!}{imp}{you guys sit down!}
					\vbmorph{g̱a-&ÿ-&\rt[¹]{ḵi}&-μH}
						{\xx{g̱cnj}&\xx{2pl.s}&\rt[¹]{sit.\xx{pl}}&\·\xx{var}}
			\item	\fm{–teech}
				from \fm{\rt[¹]{tiᴸ}} ‘be, exist’ in
				\newline
				\vbform{yei ḵusteech}{rep impfv}[subj intr, \fm{g̱}, \fm{-μμL} state]{she/he/it repeatedly comes into existence, is born}
					\vbmorph{\gm{yei=}&ḵu-&d-&s-&\rt[¹]{tiᴸ}&-μμL&\gm{-ch}}
						{down&\xx{areal}&\xx{mid}&\xx{xtn}&\rt[¹]{be}&\·\xx{var}&\·\xx{rep}}
				\versus \vbform{ḵug̱astí!}{imp}{come into existence, be born!}
					\vbmorph{ḵu-&g̱a-&d-&s-&\rt[¹]{tiᴸ}&-μH}
						{\xx{areal}&\xx{g̱cnj}&\xx{mid}&\xx{xtn}&\rt[¹]{be}&\·\xx{var}}
			\end{itemize}
		\item	\label{item:-ch-conjclass-g}
			\fm{g} conjugation class repetitive imperfectives
				with \X{kei=} ‘up’ and \fm{-ch};
			these overlap in form with the \fm{∅} conjugation class
				\fm{kei=…-ch} repetitive imperfectives
				from the motion derivation
				\motderiv{kei=}{\fm{∅}, \fm{-ch} rep}{up}
				(see \ref{item:-ch-motderiv-zdir} below)
				but the conjugation class is different with consequent differences
				in other forms like imperatives
			\begin{itemize}
			\item	\fm{–yíḵch}
				from \fm{\rt[²]{ÿiḵ}} \~\ \fm{\rt[²]{ÿeḵ}} ‘bite, hold in mouth’ in
				\newline
				\vbform{kei ayíḵch}{rep impfv}[tr, \fm{g}, ach]{she/he/it repeatedly bites, carries in mouth him/her/it}
					\vbmorph{\gm{kei=}&a-&\rt[²]{ÿiḵ}&-μH&\gm{-ch}}
						{up&\xx{3>3}&\rt[²]{bite}&\·\xx{var}&\·\xx{rep}}
				\versus \vbform{gayeeḵ!}{imp}{bite, carry in mouth him/her/it!}
					\vbmorph{ga-&\rt[²]{ÿiḵ}&-μμL}
						{\xx{gcnj}&\rt[²]{bite}&\·\xx{var}}
			\item	\fm{–tsʼíkʼch}
				from \fm{\rt[²]{tsʼikʼ}} \~\ \fm{\rt[²]{tsʼikʼw}} ‘pinch’ in
				\newline
				\vbform{kei altsʼíkʼch}{rep impfv}[tr, \fm{g}, \fm{-μμH} act]{she/he/it repeatedly pinches him/her/it}
					\vbmorph{\gm{kei=}&a-&lˢ-&\rt[²]{tsʼikʼ}&-μH&\gm{-ch}}
						{up&\xx{3>3}&\xx{xtn}&\rt[²]{pinch}&\·\xx{var}&\·\xx{rep}}
				\versus \vbform{galatsʼéekʼ!}{imp}{pinch him/her/it!}
					\vbmorph{ga-&lˢa-&\rt[²]{tsʼikʼ}&-μμH}
						{\xx{gcnj}&\xx{xtn}&\rt[²]{pinch}&\·\xx{var}}
			\item	\fm{–shátch}
				from \fm{\rt[²]{shaᴴt}} ‘grab’ in
				\newline
				\vbform{kei ashátch}{rep impfv}[tr, \fm{g}, ach]{she/he/it repeatedly catches, grabs up him/her/it}
					\vbmorph{\gm{kei=}&a-&\rt[²]{shaᴴt}&-μH&\gm{-ch}}
						{up&\xx{3>3}&\rt[²]{grab}&\·\xx{var}&\·\xx{rep}}
				\versus \vbform{gasháat!}{imp}{catch, grab up him/her/it!}
					\vbmorph{ga-&\rt[²]{shaᴴt}&-μμH}
						{\xx{gcnj}&\rt[²]{grab}&\·\xx{var}}
			\item	\fm{–shéich}
				from \fm{\rt[²]{sha}} ‘bark, yap at’ in
				\newline
				\vbform{kei ashéich}{rep impfv}[subj intr, \fm{g}, \fm{-μμH} act]{she/he/it repeatedly barks}
					\vbmorph{\gm{kei=}&a-&\rt[²]{sha}&-μᵉμH&\gm{-ch}}
						{up&\xx{xpl}&\rt[²]{bark}&\·\xx{var}&\·\xx{rep}}
				\versus \vbform{agashá!}{imp}{bark!}
					\vbmorph{a-&ga-&\rt[²]{sha}&-μH}
						{\xx{xpl}&\xx{gcnj}&\rt[²]{bark}&\·\xx{var}}
			\item	\fm{–sheech}
				from \fm{\rt[²]{shiᴸ}} ‘sing’ in
				\newline
				\vbform{kei asheech}{rep impfv}[tr, \fm{g}, \fm{-μH} act]{she/he/it repeatedly sings it}
					\vbmorph{\gm{kei=}&a-&\rt[²]{shiᴸ}&-μμL&\gm{-ch}}
						{up&\xx{3>3}&\rt[²]{sing}&\·\xx{var}&\·\xx{rep}}
				\versus \vbform{gashí!}{imp}{sing it!}
					\vbmorph{ga-&\rt[²]{shiᴸ}&-μH}
						{\xx{gcnj}&\rt[²]{sing}&\·\xx{var}}
			\end{itemize}
		\end{enumerate}
	\item	\label{item:-ch-motderiv}
		repetitive suffix in repetitive imperfective forms arising from motion derivations
		that assign \fm{g̱}, \fm{g}, or \fm{∅} conjugation class
		\begin{itemize}
		\item[\fm{g̱}:] \X{yei=}…\fm{-ch} (\ref{item:-ch-motderiv-g̱})
		\item[\fm{g}:] \X{kei=}…\fm{-ch} (\ref{item:-ch-motderiv-g})
		\item[\fm{∅}:]	\fm{\xx{dir}=}…\fm{-ch} (\ref{item:-ch-motderiv-zdir})
			where \fm{\xx{dir}=} is one of
				\begin{inlinelist}
				\item	\X{yei=}
				\item	\X{kei=}
				\item	\X{daaḵ=}
				\item	\X{daak=}
				\item	\X{ÿeiḵ=}
				\item	\X{ḵux̱=}
				\end{inlinelist}
		\item[\fm{∅}:]	\X[ÿaa=along]{ÿaa=} \~\ \X[ÿa-face]{ÿa-}\X[u-irr]{u-}…\fm{-ch}
			(\ref{item:-ch-motderiv-zalong})
		\item[\fm{∅}:]	\X[ÿaa=along]{ÿaa=} \~\ \X[sha-head]{sha-}\X[ÿa-face]{ÿa-}\X[u-irr]{u-}…\fm{-ch}
			(\ref{item:-ch-motderiv-zhead})
		\end{itemize}
		\begin{enumerate}
		\item	\label{item:-ch-motderiv-g̱}
			\fm{g̱} conjugation class repetitive imperfectives
				with \X{yei=} ‘down’ and \fm{-ch};
			these are identical to the \fm{yei=…-ch} repetitive imperfectives
				from lexically specified \fm{g̱} conjugation class
				(see \ref{item:-ch-conjclass-g̱} above),
				with the only difference being that they arise
				from motion derivations rather than
				lexically specified conjugation class;
			these are distinct from the similar \fm{yei=…-ch}
				repetitive imperfectives from the motion derivation
				\motderiv{yei=}{\fm{∅}, \fm{-ch} rep}{down, disembarking, exiting}
				(see \ref{item:-ch-motderiv-zdir} below)
				because the conjugation class is different
				with consequent differences in other forms like imperatives
				\begin{itemize}
				\item	\fm{–xíxch}
					from \fm{\rt[¹]{xix}} ‘fall’ 
					\newline
					using \motderiv{NP ká-xʼ}{\fm{g̱}, \fm{yei=…-ch} rep}{down onto NP} in
					\newline
					\vbform{a káa yei xíxch}{rep impfv}[obj intr, \fm{g̱}, mot]{it repeatedly falls on it}
						\vbmorph{a&ká&-μ&\gm{yei=}&\rt[¹]{xix}&-μH&\gm{-ch}}
							{\xx{3n}&\xx{hsfc}&\·\xx{loc}&down&\rt[¹]{fall}&\·\xx{var}&\·\xx{rep}}
					\versus \vbform{a káa g̱aag̱axeex}{hort}{let it fall on it}
						\vbmorph{a&ká&-μ&g̱aa-&g̱a-&\rt[¹]{xix}&-μμL}
							{\xx{3n}&\xx{hsfc}&\·\xx{loc}&\xx{g̱cnj}&\xx{mod}&\rt[¹]{fall}&\·\xx{var}}
				\item	\fm{–yéich}
					from \fm{\rt[²]{ÿa}} ‘lower, spread out’
					\newline
					using \motderiv{NP-x̱}{\fm{g̱}, \fm{yei=…-ch} rep}{down along NP} in
					\newline
					\vbform{áx̱ yei akayéich}{rep impfv}[tr, \fm{g̱}, mot]{she/he/it repeatedly spreads it out on it}
						\vbmorph{á&-x̱&\gm{yei=}&a-&ka-&\rt[²]{ÿa}&-μᵉμH&\gm{-ch}}
							{\xx{3n}&\·\xx{pert}&down&\xx{3>3}&\xx{hsfc}&\rt[²]{spread}&\·\xx{var}&\·\xx{rep}}
					\versus \vbform{áx̱ kag̱ayá!}{imp}{spread it out on it!}
						\vbmorph{á&-x̱&ka-&g̱a-&\rt[²]{ÿa}&-μH}
							{\xx{3n}&\·\xx{pert}&\xx{hsfc}&\xx{g̱cnj}&\rt[²]{spread}&\·\xx{var}}
				\end{itemize}
		\item	\label{item:-ch-motderiv-g}
			\fm{g} conjugation class repetitive imperfectives
				with \X{kei=} ‘up’ and \fm{-ch};
			these are identical to the \fm{kei=…-ch} repetitive imperfectives
				from lexically specified \fm{g} conjugation class
				(see \ref{item:-ch-conjclass-g} above),
				with the only difference being that they arise
				from motion derivations rather than
				lexically specified conjugation class;
			these are distinct from the similar \fm{kei=…-ch}
				repetitive imperfectives from the motion derivation
				\motderiv{kei=}{\fm{∅}, \fm{-ch} rep}{up}
				(see \ref{item:-ch-motderiv-zdir} below)
				because the conjugation class is different
				with consequent differences in other forms like imperatives
				\begin{itemize}
				\item	\fm{–ḵúx̱}
					from \fm{\rt[²]{ḵux̱}} ‘go by boat, other vehicle’
					\newline
					using \motderiv{ḵut=}{\fm{g}, \fm{kei=…-ch} rep}{lost} in
					\newline
					\vbform{ḵut kei ḵúx̱ch}{rep impfv}[subj intr, \fm{g}, mot]{she/he/it repeatedly gets lost boating/driving}
						\vbmorph{ḵut=&\gm{kei=}&\rt[¹]{ḵux̱}&-μH&\gm{-ch}}
							{\xx{err}&up&\rt[¹]{go.boat}&\·\xx{var}&\·\xx{rep}}
					\versus \vbform{ḵut gag̱aḵoox̱}{hort}{let him/her/it get lost boating/driving}
						\vbmorph{ḵut=&ga-&g̱a-&\rt[¹]{ḵux̱}&-μμL}
							{\xx{err}&\xx{gcnj}&\xx{mod}&\rt[¹]{go.boat}&\·\xx{var}}
				\item	\fm{–téech}
					from \fm{\rt[²]{ti}} ‘handle’
					\newline
					using \motderiv{NP-dáx̱}{\fm{g}, \fm{kei=…-ch} rep}{upward, starting from NP} in
					\newline
					\vbform{aadáx̱ kei atéech}{rep impfv}[tr, \fm{g}, mot]{she/he/it repeatedly picks it up from there}
						\vbmorph{aa&-dáx̱&\gm{kei=}&a-&\rt[²]{ti}&-μμH&\gm{-ch}}
							{\xx{3n}&\·\xx{abl}&up&\xx{3>3}&\rt[²]{handle}&\·\xx{var}&\·\xx{rep}}
					\versus \vbform{aadáx̱ gatí!}{imp}{pick it up from there!}
						\vbmorph{aa&-dáx̱&ga-&\rt[²]{ti}&-μH}
							{\xx{3n}&\·\xx{abl}&\xx{gcnj}&\rt[²]{handle}&\·\xx{var}}
				\end{itemize}
		\item	\label{item:-ch-motderiv-zdir}
			\fm{∅} conjugation class repetitive imperfectives
				with \X{yei=} ‘down’,
				\X{kei=} ‘up’,
				\X{daaḵ=} ‘inland’,
				\X{daak=} ‘out to sea’,
				\X{ÿeiḵ=} ‘beachward’,
				and \X{ḵux̱=} ‘back’;
			the forms with \fm{yei=} ‘down’ and \fm{kei=} ‘up’ overlap with
				the \fm{yei=…-ch} and \fm{kei=…-ch} repetitive imperfectives
				that arise either
				from lexically specified \fm{g̱}/\fm{g} conjugation class
				(\ref{item:-ch-conjclass-g̱} and \ref{item:-ch-conjclass-g} above)
				or from \fm{g̱}/\fm{g} motion derivations
				(\ref{item:-ch-motderiv-g̱} and \ref{item:-ch-motderiv-g} above),
				but conjugation class differs
				and consequently other forms like imperatives are different
				(meanings may also differ but this is unclear)
			\begin{itemize}
			\item	\fm{−x̱ʼílʼch}
				from \fm{\rt[¹]{x̱ʼilʼ}} ‘slip, slide’
				\newline
				using \motderiv{yei=}{\fm{∅}, \fm{-ch} rep}{down, disembarking, exiting} in
				\newline
				\vbform{yei shax̱ʼílʼch}{rep impfv}[obj intr, \fm{∅}, mot]{she/he/it repeatedly slips and falls down}
					\vbmorph{\gm{yei=}&sha-&\rt[¹]{x̱ʼilʼ}&-μH&\gm{-ch}}
						{down&\xx{pej}&\rt[¹]{slip}&\·\xx{var}&\·\xx{rep}}
				\versus \vbform{yei x̱shax̱ʼéelʼ}{hort}{let him/her/it slip and fall}
					\vbmorph{yei=&x̱-&sha-&\rt[¹]{x̱ʼilʼ}&-μμH}
						{down&\xx{mod}&\xx{pej}&\rt[¹]{slip}&\·\xx{var}}
			\item	\fm{–.átch}
				from \fm{\rt[¹]{.at}} ‘plural go’
				\newline
				using \motderiv{kei=}{∅, \fm{-ch} rep}{up} in
				\newline
				\vbform{kei has .átch}{rep impfv}[subj intr, \fm{∅}, mot]{they repeatedly go up}
					\vbmorph{\gm{kei=}&has=&\rt[¹]{.at}&-μH&\gm{-ch}}
						{up&\xx{plh}&\rt[¹]{go.\xx{pl}}&\·\xx{var}&\·\xx{rep}}
				\versus \vbform{kei yi.á!}{imp}{you guys go up!}
					\vbmorph{kei=&ÿi-&\rt[¹]{.at}&-⊗}
						{up&\xx{2pl.s}&\rt[¹]{go.\xx{pl}}&\·\xx{var}}
			\item	\fm{–téech}
				from \fm{\rt[²]{ti}} ‘handle’
				\newline
				using \motderiv{daaḵ=}{\fm{∅}, \fm{-ch} rep}{inland, back from open, off fire} in
				\newline
				\vbform{daaḵ atéech}{rep impfv}[tr, \fm{∅}, mot]{she/he/it repeatedly takes it inland}
					\vbmorph{\gm{daaḵ=}&a-&\rt[²]{ti}&-μμH&\gm{-ch}}
						{inland&\xx{3>3}&\rt[¹]{handle}&\·\xx{var}&\·\xx{rep}}
				\versus \vbform{daaḵ tí!}{imp}{take it inland!}
					\vbmorph{daaḵ=&\rt[²]{ti}&-μH}
						{inland&\rt[¹]{handle}&\·\xx{var}}
			\item	\fm{–háshch}
				from \fm{\rt[¹]{hash}} ‘float’
				\newline
				using \motderiv{daak=}{\fm{∅}, \fm{-ch} rep}{out to sea; into open; onto fire; precipitate} in
				\newline
				\vbform{daak laháshch}{rep impfv}[obj intr, \fm{∅}, mot]{she/he/it repeatedly floats out to sea}
					\vbmorph{\gm{daak=}&lˢa-&\rt[¹]{hash}&-μH&\gm{-ch}}
						{out.sea&\xx{xtn}&\rt[¹]{float}&\·\xx{var}&\·\xx{rep}}
				\versus \vbform{daak g̱alahaash}{hort}{let him/her/it float out to sea}
					\vbmorph{daak=&g̱a-&lˢa-&\rt[¹]{hash}&-μμL}
						{out.sea&\xx{mod}&\xx{xtn}&\rt[¹]{float}&\·\xx{var}}
			\item	\fm{–gúḵch}
				from \fm{\rt[¹]{guḵ}} ‘push; plural run’
				\newline
				using \motderiv{ÿeiḵ=}{\fm{∅}, \fm{-ch} rep}{beachward, to the beach} in
				\newline
				\vbform{yeiḵ has lugúḵch}{rep impfv}[obj intr, \fm{∅}, mot]{they repeatedly run to the beach}
					\vbmorph{\gm{ÿeiḵ=}&has=&lu-&\rt[¹]{guḵ}&-μH&\gm{-ch}}
						{beach&\xx{plh}&nose&\rt[¹]{push}&\·\xx{var}&\·\xx{rep}}
				\versus \vbform{yeiḵ yee lugúḵ!}{imp}{you guys run to the beach!}
					\vbmorph{ÿeiḵ=&ÿee=&lu-&\rt[¹]{guḵ}&-μH}
						{beach&\xx{2pl.o}&nose&\rt[¹]{push}&\·\xx{var}}
			\item	\fm{–gútch}
				from \fm{\rt[¹]{gut}} ‘singular go’
				\newline
				using \motderiv{ḵux̱=/ḵúx̱-de= d-}{\fm{∅}, \fm{-ch} rep}{back, returning} in
				\newline
				\vbform{ḵux̱ dagútch}{rep impfv}[subj intr, \fm{∅}, mot]{she/he/it repeatedly goes back}
					\vbmorph{\gm{ḵux̱=}&da-&\rt[¹]{gut}&-μH&\gm{-ch}}
						{\xx{rev}&\xx{mid}&\rt[¹]{go.\xx{sg}}&\·\xx{var}&\·\xx{rep}}
				\versus \vbform{ḵux̱ idagú!}{imp}{go back!}
					\vbmorph{ḵux̱=&i-&da-&\rt[¹]{gut}&-⊗}
						{\xx{rev}&\xx{2sg.s}&\xx{mid}&\rt[¹]{go.\xx{sg}}&\·\xx{var}}
			\end{itemize}
		\item	\label{item:-ch-motderiv-zalong}
			\fm{∅} conjugation class repetitive imperfectives
				with \X[ÿaa=along]{ÿaa=} ‘along’ \~\ \X[ÿa-face]{ÿa-} ‘face’
				+ irrealis \X[u-irr]{u-};
			these have \fm{ÿaa=} in repetitive imperfective forms
				and \fm{ÿa-u-} (usually appearing as \fm{wu}) in other forms
				\parencites[609–611]{crippen:2019}[608]{edwards:2009}[304]{leer:1991}
			\begin{itemize}
			\item	\fm{–ḵúx̱ch}
				from \fm{\rt[¹]{ḵux̱}} ‘go by boat, other vehicle’
				\newline
				using \motderiv{NP daa-x̱ ÿaa= \~\ ÿa-u-}{\fm{∅}, \fm{-ch} rep}{circling NP} in
				\newline
				\vbform{a daax̱ yaa ḵúx̱ch}{rep impfv}[subj intr, \fm{∅}, mot]{she/he/it boats around it}
					\vbmorph{a&daa&-x̱&ÿaa=&\rt[¹]{ḵux̱}&-μH&\gm{-ch}}
						{\xx{3n}&around&\·\xx{pert}&along&\rt[¹]{go.boat}&\·\xx{var}&\·\xx{rep}}
				\versus \vbform{a daax̱ wuḵúx̱!}{imp}{boat around it!}
					\vbmorph{a&daa&-x̱&ÿ-&u-&\rt[¹]{ḵux̱}&-μH}
						{\xx{3n}&around&-\xx{pert}&face&\xx{irr}&\rt[¹]{go.boat}&\·\xx{var}}
			\item	\fm{–.ésch}
				from \fm{\rt[¹]{.es}} ‘stagger’
				\newline
				using \motderiv{ÿaa= \~\ ÿa-u-}{\fm{∅}, \fm{-ch} rep}{obliquely, circuitously} in
				\newline
				\vbform{yaa yakakla.ésch}{rep impfv}[obj intr, \fm{∅}, mot]{she/he/it repeatedly staggers}
				\parencite[209.2911]{story-naish:1973}
					\vbmorph{ÿaa=&ÿa-&ka-&k-&lˢa-&\rt[¹]{.es}&-μH&\gm{-ch}}
						{along&face&\xx{hsfc}&\xx{qual}&\xx{xtn}&\rt[¹]{stagger}&\·\xx{var}&\·\xx{rep}}
				\versus \vbform{yaa yakanal.és}{prog}[obj intr, \fm{∅}, mot]{she/he/it is staggering along}
				\parencite[209.2910]{story-naish:1973}
					\vbmorph{ÿaa=&ÿa-&ka-&na-&lˢ-&\rt[¹]{.es}&-μH}
						{along&face&\xx{hsfc}&\xx{ncnj}&\xx{xtn}&\rt[¹]{stagger}&\·\xx{var}}
			\end{itemize}
		\item	\label{item:-ch-motderiv-zhead}
			\fm{∅} conjugation class repetitive imperfectives
				with \X[sha-head]{sha-} ‘head’ + \X[ÿa-face]{ÿa-} ‘face’
				+ irrealis \X[u-irr]{u-};
			like those above in \ref{item:-ch-motderiv-zalong}, these verbs have
				\fm{ÿaa=} in repetitive imperfective forms
				and \fm{ÿa-u-} in other forms
				\parencites[611]{crippen:2019}[120, 134, 314]{leer:1991}
			\begin{itemize}
			\item	\fm{–téech}
				from \fm{\rt[²]{ti}} ‘handle’
				\newline
				using \motderiv{ÿax̱= ÿaa= \~\ sha-ÿa-u-}{\fm{∅}, \fm{-ch} rep}{hanging up} in
				\newline
				\vbform{ÿax̱ ÿaa shax̱atéech}{rep impfv}[tr, \fm{∅}, mot]{I repeatedly hang it up}
					\parencite[315]{leer:1991}
					\vbmorph{ÿax̱=&ÿaa=&sha-&x̱a-&\rt[²]{ti}&-μμH&\gm{-ch}}
						{facing&along&head&\xx{1sg.s}&\rt[²]{handle}&\·\xx{var}&\·\xx{rep}}
				\versus \vbform{ÿax̱ shawootí!}{imp}{hang it up!}
					\parencite[315]{leer:1991}
					\vbmorph{ÿax̱=&sha-&ÿa-&u-&\rt[²]{ti}&-μH}
						{facing&head&face&\xx{irr}&\rt[²]{handle}&\·\xx{var}}
			\end{itemize}
		\end{enumerate}
	\item	\label{item:-ch-oddrep}
		repetitive suffix in unexpected repetitive imperfective forms;
		these are cases where a verb has a repetitive imperfective form
			with \fm{-ch} that is not predicted from other grammatical
			properties like conjugation class membership
			or the application of a motion derivation
		\begin{enumerate}
		\item	\label{item:-ch-oddrep-event}
			achievement verbs that have a repetitive imperfective with \fm{-ch};
			\citeauthor{leer:1991} considers these to be “primary imperfectives” 
				(thus non-repetitive activities rather than achievements)
				despite the presence of the repetitive
					suffix \fm{-ch} \parencite[245]{leer:1991}
				and so \citeauthor{leer:1991} often represents these as 
					“A-ch” \parencite[iii]{leer:1978b} or the like
					with “A” standing for ‘activity’;
				these seem to have repetitive interpretations so it is simpler
					to treat them as irregular repetitive imperfective forms
					than as non-repetitives with a meaningless \fm{-ch} suffix
			\begin{itemize}
			\item	\fm{–léich}
				from \fm{\rt[¹]{la}} ‘yell, cheer’ in
				\newline
				\vbform{daléich}{rep impfv}[subj intr, \fm{∅}, ach, \fm{-ch} rep]{she/he/it yells, cheers}
					\parencites[251.3598]{story-naish:1973}[439]{leer:1976}
					\vbmorph{da-&\rt[¹]{la}&-μᵉμH&\gm{-ch}}
						{\xx{mid}&\rt[¹]{yell}&\·\xx{var}&\·\xx{rep}}
				\versus \vbform{wudiláa}{pfv}{she/he/it yelled, shouted, cheered}
					\vbmorph{wu-&d-&i-&\rt[¹]{la}&-μμH}
						{\xx{pfv}&\xx{mid}&\xx{stv}&\rt[¹]{shout}&\·\xx{var}}
				\exand \vbform{kei x̱tudaláa}{hort}{let’s yell/cheer}
					\parencite[251.3599]{story-naish:1973}
					\vbmorph{kei=&x̱-&tu-&da-&\rt[¹]{la}&-μμH}
						{up&\xx{mod}&\xx{1pl.s}&\xx{mid}&\rt[¹]{yell}&\·\xx{var}}
				\newline
				forms with \X{kei=} ‘up’ may reflect the motion derivation
					\motderiv{kei=}{\fm{∅}, \fm{-ch} rep}{up}
					(see \ref{item:-ch-motderiv-zdir}),
					but forms without \fm{kei=} are still unexpected
					given \fm{∅} conjugation class implied by perfective \fm{-μμH}
			\item	\fm{–g̱átch}
				from \fm{\rt[¹]{g̱aᴴt}} ‘pl.\ fall scattered/apart’
				\newline
				\vbform{kadásʼ kadag̱átch}{rep impfv}[obj intr, \fm{g̱}, ach, \fm{-ch} rep]{hail repeatedly falls}
				\parencite[85.1058]{story-naish:1973}
					\vbmorph{ka-&dásʼ&ka-&da-&\rt[¹]{g̱aᴴt}&-μH&\gm{-ch}}
						{\xx{sro}&hail&\xx{sro}&\xx{mid}&\rt[¹]{scattered}&\·\xx{var}&\·\xx{rep}}
				\versus \vbform{tléiḵw de kawdig̱áat}{pfv}{berries have already fallen}
				\parencite[85.1057]{story-naish:1973}
					\vbmorph{tléiḵw&de&ka-&w-&d-&i-&\rt[¹]{g̱aᴴt}&-μμH}
						{berry&already&\xx{sro}&\xx{pfv}&\xx{mid}&\xx{stv}&\rt[¹]{scattered}&\·\xx{var}}
				\newline
				this is irregular because the verb is \fm{g̱} conjugation class
				but the repetitive imperfective form lacks \X{yei=}
				(\ref{item:-ch-conjclass-g̱} and \ref{item:-ch-motderiv-g̱});
				there is also a causative 
				\vbform{alg̱átch}{rep impfv}{she/he/it scatters them}
				\parencite[823]{leer:1976};
				the root \fm{\rt[¹]{g̱aᴴt}} ‘pl.\ fall scattered/apart’
				appears in a number of secondary aspectual derivations 
				(Leer’s “epiaspect”) using \fm{-ch} such as
				\vbform{wudig̱átch}{rep pfv}{they fell scattered}
				and also occurs with some other irregular repetitives such as
				\vbform{aklag̱átkw}{rep impfv}{she/he/it is sifting it}
				and \vbform{kadzig̱átk}{rep impfv}{they (berries) fall easily from branch}
				\parencite[all][823]{leer:1976}
			\end{itemize}
		\item	\label{item:-ch-oddrep-state}
			state imperfectives with \X[i-stv]{i-} \~\ \X[ÿa-stv]{ÿa-} and \fm{-ch};
			the only regular cases are with the root \fm{\rt[²]{.ax̱}} ‘hear’,
				for which see \ref{item:-ch-hear};
			state imperfectives with the stems \fm{–.ítʼch} ‘sparkle’
				and \fm{–núkch} ‘helpless, undependable’
				probably have frozen \fm{-ch}
				for which see \ref{item:-ch-roots-CVCC}
		\end{enumerate}
	\item	\label{item:-ch-hear}
		repetitive suffix in a number of verbs based on \fm{\rt[²]{.ax̱}} ‘hear’,
			dealt with separately here because of the complex relations between them;
		many verbs with this root have repetitive imperfective forms with \fm{-ch},
			some of which seem to be predictable from conjugation class
			but others which are not predictable;
		some verbs based on this root have \fm{-ch} in all forms regardless of aspect
			suggesting that they are secondary aspectual derivations from the
			repetitive imperfectives, but the mechanisms for this are unclear
			and there are unexpected gaps and overlaps
			as well as grammatical details that still need to be described;
		verbs based on \fm{\rt[²]{.ax̱}} ‘hear’ with \fm{-ch} include states,
			activities, and achievements,
			occur with at least \fm{∅}, \fm{n}, and \fm{g̱} conjugation classes,
			and have stem forms \fm{–.áx̱ch} and \fm{–.áax̱ch}
			(the latter phonologically unexpected)
		\begin{enumerate}
		\item	verbs based on \fm{\rt[²]{.ax̱}} ‘hear’
			that have state imperfectives with \fm{-ch}
			\begin{itemize}
			\item	\fm{–.áx̱ch} ‘able to hear’
				from \fm{\rt[²]{.ax̱}} ‘hear’ in
				\newline
				\vbform{aya.áx̱ch}{rep impfv}[tr, \fm{g}, \fm{-ch} state]{she/he/it can hear him/her/it}
					\vbmorph{a-&ÿa-&\rt[²]{.ax̱}&-μH&\gm{-ch}}
						{\xx{3>3}&\xx{stv}&\rt[²]{hear}&\·\xx{var}&\·\xx{rep}}
				\versus \vbform{aawa.áx̱}{pfv}[tr, \fm{∅}, ach]{she/he/it heard him/her/it}
					\vbmorph{a-&μʷ-&wa-&\rt[²]{.ax̱}&-μH}
						{\xx{3>3}&\xx{pfv}&\xx{stv}&\rt[²]{hear}&\·\xx{var}}
				\newline
				there are several aspect forms listed below that are documented
					for this verb which have \fm{-ch}
					(thus appearing to be secondary aspectual derivations),
					but other aspects (perfective, progressive, etc.)\
					are replaced by the \fm{∅} conjugation class verb
					\vbform{aawa.áx̱}{pfv}{she/he/it heard him/her/it}
					without \fm{-ch} that is given above
					\parencite[10933]{eggleston:2017};
				compare the similar
				\vbform{ayatéen}{impfv}[tr, \fm{g}, \fm{-μμH} state]{she/he/it can see him/her/it}
					which also lacks some aspect forms that are suppleted by
					\vbform{awsiteen}{pfv}[tr, \fm{g̱}, ach]{she/he/it saw him/her/it}
				\begin{itemize}
				\item	\vbform{kei akg̱wa.áx̱ch}{prosp}{she/he/it will be able to hear him/her/it}
					\newline
					prospective aspect (expected \fm{–.áax̱})
				\item	\vbform{aga.áx̱chni}{cond}{if she/he/it can hear him/her/it}
					\newline
					conditional mood (expected \fm{–.áx̱ni})
				\item	\vbform{a.áx̱ji nooch}{hab impfv}{she/he/it always can hear him/her/it}
					\newline
					habitual auxiliary with imperfective aspect
				\item	\vbform{líl ee.áx̱jiḵ!}{phib impfv}{don’t you be able to hear it!}
					\newline
					prohibitive mood with imperfective aspect
				\item	\vbform{a.áx̱jin}{past impfv}{she/he/it used to be able to hear him/her/it}
					\newline
					past tense with imperfective aspect
				\end{itemize}
				forms of this verb with indefinite human arguments have specialized meanings
				\begin{itemize}
				\item	indefinite human subject \X{du-} in
					\vbform{duwa.áx̱ch}{rep impfv}{someone/people can hear him/her/it}
					also means ‘she/he/it is noisy, loud’;
					\vbform{tléixʼ wooch yáx̱ duwa.áx̱ch}{rep impfv}{they are pronounced together}
					\parencite[02/159]{leer:1973}
				\item	indefinite human object \X[ḵu-indef]{ḵu-} in
					\vbform{ḵuwa.áx̱ch}{rep impfv}{she/he/it can hear someone/people}
					also means ‘she/he/it can hear, has hearing, hears well’
				\item	negative with extensional \X{lˢ-}
					and indefinite human object \X[ḵu-indef]{ḵu-} in
					\vbform{tléil ḵool.áx̱ch}{rep impfv}{she/he/it cannot hear someone/people}
					also means ‘she/he/it is deaf’, for example
					\vbform{tléil kei ḵukux̱la.áx̱ch}{neg prosp}{he will be deaf}
					\parencite[02/158]{leer:1973}
				\end{itemize}
			\item	\X{se-} \~\ \X[sa-voice]{sa-} ‘voice’ + \fm{–.áx̱ch} ‘hear voice’
				from \fm{\rt[²]{.ax̱}} ‘hear’ in
				\newline
				\vbform{asaya.áx̱ch}{rep impfv}[tr, \fm{∅}?, \fm{-ch} state]{she/he/it hears voice of him/her/it}
					\vbmorph{a-&se-&ÿa-&\rt[²]{.ax̱}&-μH&\gm{-ch}}
						{\xx{3>3}&voice&\xx{stv}&\rt[²]{hear}&\·\xx{var}&\·\xx{rep}}
				\versus \vbform{aseiwa.áx̱}{pfv}{she/he/it heard voice of him/her/it}
					\vbmorph{a-&se-&μʷ-&wa-&\rt[²]{.ax̱}&-μH}
						{\xx{3>3}&voice&\xx{pfv}&\xx{stv}&\rt[²]{hear}&\·\xx{var}}
			\item	\X{x̱ʼe-} \~\ \X{x̱ʼa-} ‘mouth’ + \fm{–.áx̱ch} ‘understand’
				from \fm{\rt[²]{.ax̱}} ‘hear’ in
				\newline
				\vbform{ax̱ʼaya.áx̱ch}{rep impfv}[tr, \fm{∅}, \fm{-ch} state]{she/he/it understands him/her/it}
					\vbmorph{a-&x̱ʼe-&ÿa-&\rt[²]{.ax̱}&-μH&\gm{-ch}}
						{\xx{3>3}&mouth&\xx{stv}&\rt[²]{hear}&\·\xx{var}&\·\xx{rep}}
				\versus \vbform{ax̱ʼeiwa.áx̱}{pfv}{she/he/it understood him/her/it}
					\vbmorph{a-&x̱ʼe-&μʷ-&wa-&\rt[²]{.ax̱}&-μH}
						{\xx{3>3}&mouth&\xx{pfv}&\xx{stv}&\rt[²]{hear}&\·\xx{var}}
				\newline
				the negative form with a second person object is an important phrase
					for language learners
				\newline
				\vbform{tléil ix̱ʼeix̱a.áx̱ch}{neg rep impfv}{I don’t understand you}
				\parencite[02/170]{leer:1973}
					\vbmorph{tléil&i-&x̱ʼe-&u-&x̱a-&\rt[²]{.ax̱}&-μH&\gm{-ch}}
						{not&\xx{2sg.o}&mouth&\xx{irr}&\xx{1sg.s}&\rt[²]{hear}&\·\xx{var}&\·\xx{rep}}
				\newline
				this verb notably has a habitual aspect form with unexpected
					\X{-μμL} stem variation 
					(see \ref{item:-ch-hab-zother-CVC-μμL} below
					for similar cases with other verbs)
				\newline
				\vbform{ax̱ʼei.aax̱ch}{hab}{she/he/it always understands him/her/it}
					\vbmorph{a-&x̱ʼe-&u-&\rt[²]{.ax̱}&-μμL&\gm{-ch}}
						{\xx{3>3}&mouth&\xx{zpfv}&\rt[²]{hear}&\·\xx{var}&\·\xx{rep}}
				\newline
				\textcite[120]{leer:1976} gives a \fm{g} conjugation class form
					with progressive aspect which seems to be
					a secondary aspectual derivation
				\newline
				\vbform{kei ax̱ʼana.áx̱ch}{prog}[tr, \fm{g}, \fm{-ch} state]{she/he/it is starting to understand his/her/its speech}
					\vbmorph{kei=&a-&x̱ʼe-&na-&\rt[²]{.ax̱}&-μH&\gm{-ch}}
						{up=&\xx{3>3}&mouth&\xx{ncnj}&\rt[²]{hear}&\·\xx{var}&\·\xx{rep}}
				\newline
				\textcite[120]{leer:1976} also pairs this with a 
					\fm{g̱} conjugation class verb that has the stem \fm{–.áax̱ch}
					but this is probably the ‘learn speech’ verb listed below
			\item	\X{lˢ-} extensional + \fm{–.áx̱ch} ‘play instrument’
				from \fm{\rt[²]{.ax̱}} ‘hear’ in
				\newline
				\vbform{ali.áx̱ch}{rep impfv}[tr, \fm{n}, \fm{-ch} state]{she/he/it plays it (musical instrument)}
					\vbmorph{a-&lˢ-&i-&\rt[²]{.ax̱}&-μH&\gm{-ch}}
						{\xx{3>3}&\xx{xtn}&\xx{stv}&\rt[²]{hear}&\·\xx{var}&\·\xx{rep}}
				\versus \vbform{awli.aax̱}{pfv}{she/he/it played it}
					\vbmorph{a-&w-&lˢ-&i-&\rt[²]{.ax̱}&-μμL}
						{\xx{3>3}&\xx{pfv}&\xx{xtn}&\xx{stv}&\rt[²]{hear}&\·\xx{var}}
				\newline
				also some derived verbs
				\begin{itemize}
				\item	\vbform{s dli.áx̱ch}{rep impfv}[subj intr]{they are playing}
					\parencite[02/157]{leer:1973}
					(\cite[115]{leer:1976} “also capable of doing it”)
					with antipassive \X{d-};
					also \vbform{wudli.áx̱}{pfv}{she/he/it played}
					\parencite[02/157]{leer:1973}
				\item	\vbform{yánde yaa sh nal.áx̱}{prog}{it (music) is coming to an end}
					\parencite[02/157]{leer:1973}
					with terminative \motderiv{yan= \~\ yax̱= \~\ yánde}{∅, \fm{-μμL} rep}{ending, finishing, terminating}
				\end{itemize}
			\end{itemize}
		\item	verbs based on \fm{\rt[²]{.ax̱}} ‘hear’
			that have \fm{-ch} in aspect forms other than the
				repetitive imperfective or habitual;
			these are probably from secondary aspectual derivations
				but the details are still unclear
			\begin{itemize}
			\item	\fm{–.áx̱ch} ‘hear’
				from \fm{\rt[²]{.ax̱}} ‘hear’ in
				\newline
				\vbform{kḵwa.áx̱ch}{rep prosp}[tr?, conj?, ach?]{I will be hearing [him/her/it?]}
				\parencite[02/159]{leer:1973}
					\vbmorph{k-&ḵ-&w-&x̱a-&\rt[²]{.ax̱}&-μH&\gm{-ch}}
						{\xx{gcnj}&\xx{mod}&\xx{irr}&\xx{1sg.s}&\rt[²]{hear}&\·\xx{var}&\·\xx{rep}}
				\newline
				possibly from \vbform{aya.áx̱ch}{rep impfv}[tr, \fm{g}, \fm{-ch} state]{she/he/it can hear him/her/it}
					discussed above but \fm{g} conjugation class would predict
					\X{kei=} ‘up’ in the prospective form which is absent here
			\item	\X{ḵut=} errative ‘lost’ + \fm{–.áx̱ch} ‘mishear’
				from \fm{\rt[²]{.ax̱}} ‘hear’ in
				\newline
				?\vbform{ḵut kei kax̱wla.áx̱ch}{rep pfv}[tr, \fm{g}, ach?]{I missed (what was said)}
				\parencite[107.1390]{story-naish:1973}
					\vbmorph{ḵut=&kei=&ka-&w-&x̱-&lˢa-&\rt[²]{.ax̱}&-μH&\gm{-ch}}
						{\xx{err}&up&\xx{qual}&\xx{pfv}&\xx{1sg.s}&\xx{xtn}&\rt[²]{hear}&\·\xx{var}&\·\xx{rep}}
				\newline
				\textcite[02/168]{leer:1973} suggests that this is actually
					\fm{ḵut kei kax̱la.áx̱ch} without \X[w-pfv]{w-}
					and so is a repetitive imperfective
					rather than a repetitive perfective,
					otherwise we would expect \fm{ḵut kei kax̱wli.áx̱ch}
					with stative \X[i-stv]{i-};
				if this is in fact a repetitive imperfective then it may
					be plausibly related to the \X{lˢ-} + \fm{\rt[²]{.ax̱}}
					verb meaning ‘play instrument’ discussed above
					and is not an example of secondary aspect derivation
			\end{itemize}
		\item	verbs based on \fm{\rt[²]{.ax̱}} ‘hear’
				that have \fm{-ch} in other aspect forms
				along with an unexplained \X{-μμH} stem
				(thus \fm{–.áax̱ch});
			these seem to be secondary aspect derivations
				but the stem form is unexpected and there is
				no obvious mechanism that would normally produce this stem
			\begin{itemize}
			\item	\X{x̱ʼe-} \~\ \X{x̱ʼa-} ‘mouth’ + \fm{–.áax̱ch} ‘learn speech’
				from \fm{\rt[²]{.ax̱}} ‘hear’ in
				\newline
				\vbform{ax̱ʼaya.áax̱ch}{rep impfv}[tr, \fm{g̱}, \fm{-ch} state]{she/he/it learns it (language) easily/quickly}
					\vbmorph{a-&x̱ʼe-&ÿa-&\rt[²]{.ax̱}&-μμH&\gm{-ch}}
						{\xx{3>3}&mouth&\xx{stv}&\rt[²]{hear}&\·\xx{var}&\·\xx{rep}}
				\versus \vbform{yei ax̱ʼana.áax̱ch}{prog}{she/he/it is learning his/her/its speech}
				\parencite[120]{leer:1976}
					\vbmorph{yei=&a-&x̱ʼe-&na-&\rt[²]{.ax̱}&-μμH&\gm{-ch}}
						{down&\xx{3>3}&mouth&\xx{ncnj}&\rt[²]{hear}&\·\xx{var}&\·\xx{rep}}
				\exalso \vbform{ax̱ʼeiwa.áax̱ch}{pfv}{she/he/it learned his/her/its speech}
				\parencites[02/170]{leer:1973}[116]{leer:1976}
					\vbmorph{a-&x̱ʼe-&μʷ-&wa-&\rt[²]{.ax̱}&-μμH&\gm{-ch}}
						{\xx{3>3}&mouth&\xx{pfv}&\xx{stv}&\rt[²]{hear}&\·\xx{var}&\·\xx{rep}}
				\newline
				\cite{leer:1973} does not distinguish forms with
					\fm{–.áx̱ch} or \fm{–.áx̱} that mean ‘understand’
					from forms with \fm{–.áax̱ch} that mean ‘learned’
					but these are distinct in \cite{leer:1976};
				there are a few examples of verbs without \X{x̱ʼe-} \~\ \X{x̱ʼa-}
					which suggest a basic meaning of ‘learn by repeated hearing’:
				\begin{itemize}
				\item	\vbform{aawa.áax̱ch}{pfv}{he learned it to the letter (by hearing over and over)}
					\parencites[02/170]{leer:1973}[116]{leer:1976}
				\item	\vbform{yei nx̱a.áax̱ch}{prog}{I’m catching on (to language)}
					\parencite[02/170]{leer:1973}
				\end{itemize}
			\item	\fm{sh tóo} ‘at inside of self’
					+ \X[ka-qual]{ka-} qualifier
					+ \X{lˢ-} applicative
					+ \fm{–.áax̱ch} ‘rehearse song’
				from \fm{\rt[²]{.ax̱}} ‘hear’ in
				\newline
				\vbform{sh tóo yei at kagax̱dul.áax̱ch}{rep prosp}[tr, \fm{g}, ach?]{they will be rehearsing things (songs)}
				\parencites[169.2321]{story-naish:1973}[02/171]{leer:1973}
					\vbmorph{sh&tú&-μ&yei=&at=&ka-&ga-&x̱-&du-&d-&lˢ-&\rt[²]{.ax̱}&-μμH&\gm{-ch}}
						{\xx{rflx}&mind&\·\xx{loc}&down&\xx{ind.n.o}&\xx{qual}&\xx{gcnj}&\xx{mod}&\xx{ind.h.s}&\xx{mid}&\xx{appl}&\rt[²]{hear}&\·\xx{var}&\·\xx{rep}}
				\newline
				this verb is also reported with the stems 
					\fm{–.áx̱ch} with \X{-μH} + \fm{-ch}, 
					\fm{–.áax̱} with \X{-μμH}, 
					and \fm{–.aax̱} with \X{-μμL}
					\parencites[02/171]{leer:1973}[116]{leer:1976}
					which probably reflects some kind of dialect
					or community variation
			\end{itemize}
		\end{enumerate}
	\item	\label{item:-ch-hab}
		repetitive suffix in all habitual aspect forms,
		either with \fm{\xx{cnj}-} … \fm{-ch}
			for \fm{n}, \fm{g̱}, and \fm{g} conjugation class verbs
			or with \X[u-pfv]{u-} … \fm{-ch} 
			for \fm{∅} conjugation class verbs;
		there is some complexity in stem variation among habitual forms which is
			not entirely predictable by the phonology of \fm{-ch}
			but instead depends also on conjugation class and other unclear factors,
			so all known patterns with variable stem paradigm verbs are given here
			\begin{itemize}
			\item	\fm{n}/\fm{g̱}/\fm{g} conjugation class patterns
				\begin{itemize}
				\item	\fm{\xx{cnj}-}…\X{-μH}\fm{-ch}
					with \fm{\rt{CVC}}, \fm{\rt{CVCʼ}}, \fm{\rt{CVᴴC}}
					(\ref{item:-ch-hab-ng̱g-CVC})
				\item	\fm{\xx{cnj}-}…\X{-μμH}/\X{-μᵉμH}\fm{-ch}
					with \fm{\rt{CV}}
					(\ref{item:-ch-hab-ng̱g-CV})
				\item	\fm{\xx{cnj}-}…\X{-μμL}/\X{-μᵉμL}\fm{-ch}
					with \fm{\rt{CVᴸ}}
					(\ref{item:-ch-hab-ng̱g-CVL})
				\end{itemize}
			\item	\fm{∅} conjugation class patterns
				\begin{itemize}
				\item	\fm{\xx{dir}=}\X[u-pfv]{u-}…\X{-μμL}/\X{-μμH}\fm{-ch}
					with \fm{\rt{CVC}}, \fm{\rt{CVCʼ}}, \fm{\rt{CVᴴC}}
					(\ref{item:-ch-hab-zdir-CVC})
				\item	\X[u-pfv]{u-}…\X{-μH}\fm{-ch}
					with \fm{\rt{CVC}}, \fm{\rt{CVCʼ}}, \fm{\rt{CVᴴC}}
					(\ref{item:-ch-hab-zother-CVC-μH})
				\item	\X[u-pfv]{u-}…\X{-μμL}/\X{-μμH}\fm{-ch}
					with \fm{\rt{CVC}}, \fm{\rt{CVCʼ}}, \fm{\rt{CVᴴC}}
					(\ref{item:-ch-hab-zother-CVC-μμL})
				\item	\X[u-pfv]{u-}…\X{-μμH}\X{-ÿ}\fm{-ch}
					with \fm{\rt{CV}} or \fm{\rt{CVᴸ}}
					(\ref{item:-ch-hab-z-CV})
				\end{itemize}
			\end{itemize}
		\begin{enumerate}
		\item	\label{item:-ch-hab-ng̱g}
			habitual forms of \fm{n}, \fm{g̱}, and \fm{g} conjugation class verbs
			have the \X{n-}, \X[g̱-conj]{g̱-}, or \X[g-conj]{g-} conjugation prefix
			and \fm{-ch}
			with stem variation depending on root shape
			if not an invariable stem
			\begin{enumerate}
			\item	\label{item:-ch-hab-ng̱g-CVC}
				habitual forms of \fm{n}/\fm{g̱}/\fm{g} conjugation class verbs
				with \fm{\rt{CVC}}, \fm{\rt{CVCʼ}}, or \fm{\rt{CVᴴC}} roots
				have \X{-μH}
				and \fm{-ch}
				\begin{itemize}
				\item	\fm{–húnch}
					from \fm{\rt[²]{hun}} ‘sell’ in
					\newline
					\vbform{anahúnch}{hab}[tr, \fm{n}, \fm{-μμH} act]{she/he/it always sells him/her/it}
						\vbmorph{a-&na-&\rt[²]{hun}&\gm{-μH}&\gm{-ch}}
							{\xx{3>3}&\xx{ncnj}&\rt[²]{sell}&\·\xx{var}&\·\xx{rep}}
					\versus \vbform{nahoon!}{imp}{sell it!}
						\vbmorph{na-&\rt[²]{hun}&-μμL}
							{\xx{ncnj}&\rt[²]{sell}&\·\xx{var}}
				\item	\fm{–.úsʼch}
					from \fm{\rt[²]{.usʼ}} ‘wash, scrub’ in
					\newline
					\vbform{ana.úsʼch}{hab}[tr, \fm{n}, ach, \fm{-k} rep]{she/he/it always washes him/her/it}
						\vbmorph{a-&na-&\rt[²]{.usʼ}&\gm{-μH}&\gm{-ch}}
							{\xx{3>3}&\xx{ncnj}&\rt[²]{wash}&\·\xx{var}&\·\xx{rep}}
					\versus \vbform{na.óosʼ!}{imp}{wash it!}
						\vbmorph{na-&\rt[²]{.usʼ}&-μμH}
							{\xx{ncnj}&\rt[²]{wash}&\·\xx{var}}
				\item	\fm{–.úxch}
					from \fm{\rt[²]{.uᴴx}} ‘blow’ in
					\newline
					\vbform{ana.úxch}{hab}[tr, \fm{n}, ach, \fm{-sʼ} rep]{she/he/it always blows on him/her/it}
						\vbmorph{a-&na-&\rt[²]{.uᴴx}&\gm{-μH}&\gm{-ch}}
							{\xx{3>3}&\xx{ncnj}&\rt[²]{blow}&\·\xx{var}&\·\xx{rep}}
					\versus \vbform{na.óox!}{imp}{blow on it!}
						\vbmorph{na-&\rt[²]{.uᴴx}&-μμH}
							{\xx{ncnj}&\rt[²]{blow}&\·\xx{var}}
				\end{itemize}
			\item	\label{item:-ch-hab-ng̱g-CV}
				habitual forms of \fm{n}/\fm{g̱}/\fm{g} conjugation class verbs
				with \fm{\rt{CV}} roots
				have \X{-μμH}
				or ablaut \X{-μᵉμH}
				and \fm{-ch}
				\begin{itemize}
				\item	\fm{–ḵéech}
					from \fm{\rt[¹]{ḵi}} ‘plural sit’ in
					\newline
					\vbform{has g̱aḵéech}{hab}[subj intr, \fm{g̱}, ach]{they always sit down}
						\vbmorph{has=&g̱a-&\rt[¹]{ḵi}&\gm{-μμH}&\gm{-ch}}
							{\xx{plh}&\xx{g̱cnj}&\rt[¹]{sit.\xx{pl}}&\·\xx{var}&\·\xx{rep}}
					\versus \vbform{g̱ayḵí!}{imp}{you pl.\ sit down!}
						\vbmorph{g̱a-&ÿ-&\rt[¹]{ḵi}&-μH}
							{\xx{g̱cnj}&\xx{2pl.s}&\rt[¹]{sit.\xx{pl}}&\·\xx{var}}
				\item	\fm{–géich}
					from \fm{\rt[²]{ga}} ‘predict’ in
					\newline
					\vbform{ashunasgéich}{hab}[tr, \fm{n}, ach, \fm{-tʼ} rep]{she/he/it always predicts him/her/it}
					\parencite[22097]{eggleston:2017}
						\vbmorph{a-&shu-&na-&s-&\rt[²]{ga}&\gm{-μᵉμH}&\gm{-ch}}
							{\xx{3>3}&end&\xx{ncnj}&\xx{xtn}&\rt[²]{predict}&\·\xx{var}&\·\xx{rep}}
					\versus \vbform{shunasgá!}{imp}{predict him/her/it!}
						\vbmorph{shu-&na-&s-&\rt[²]{ga}&-μH}
							{end&\xx{ncnj}&\xx{xtn}&\rt[²]{predict}&\·\xx{var}}
				\end{itemize}
			\item	\label{item:-ch-hab-ng̱g-CVL}
				habitual forms of \fm{n}/\fm{g̱}/\fm{g} conjugation class verbs
				with \fm{\rt{CVᴸ}} roots
				have \X{-μμL}
				or ablaut \X{-μᵉμL}
				and \fm{-ch}
				\begin{itemize}
				\item	\fm{–tʼeech}
					from \fm{\rt[²]{tʼiᴸ}} ‘find’ in
					\newline
					\vbform{agatʼeech}{hab}[tr, \fm{g}, ach]{she/he/it always finds him/her/it}
						\vbmorph{a-&ga-&\rt[²]{tʼiᴸ}&\gm{-μμL}&\gm{-ch}}
							{\xx{3>3}&\xx{gcnj}&\rt[²]{find}&\·\xx{var}&\·\xx{rep}}
					\versus \vbform{gatʼee!}{imp}{find it!}
						\vbmorph{ga-&\rt[²]{tʼiᴸ}&-μμL}
							{\xx{gcnj}&\rt[²]{find}&\·\xx{var}}
				\item	\fm{–teich}
					from \fm{\rt[¹]{taᴸ}} ‘sg.\ sleep’ in
					\newline
					\vbform{nateich}{hab}[tr, \fm{n}, \fm{-μH} act]{she/he/it always sleeps}
						\vbmorph{na-&\rt[¹]{taᴸ}&\gm{-μᵉμL}&\gm{-ch}}
							{\xx{ncnj}&\rt[¹]{sleep.\xx{sg}}&\·\xx{var}&\·\xx{rep}}
					\versus \vbform{natá!}{imp}{sleep!}
						\vbmorph{na-&\rt[¹]{taᴸ}&-μH}
							{\xx{ncnj}&\rt[¹]{sleep.\xx{sg}}&\·\xx{var}}
				\end{itemize}
			\end{enumerate}
		\item	\label{item:-ch-hab-z}
			habitual forms of \fm{∅} conjugation class verbs
			have perfective \X[u-pfv]{u-} and \fm{-ch}
			with stem variation that depends on both root shape
			and motion derivation;
			these stem variation patterns are fully documented
			but not all exceptions are reliably described
			and explanations for the complexity of patterns are still wanting
			\begin{enumerate}
			\item	\label{item:-ch-hab-zdir-CVC}
				habitual forms of verbs
				with \fm{\rt{CVC}}, \fm{\rt{CVCʼ}}, or \fm{\rt{CVᴴC}} roots
				and a \fm{∅} conjugation class motion derivation
				using a group D preverb
				(\X{kei=}, \X{yei=}, \X{ÿeiḵ=}, \X{daaḵ=}, \X{daak=})
				have stems with either \X{-μμL} (\fm{\rt{CVC}}) 
				or \X{-μμH} (\fm{\rt{CVCʼ}}, \fm{\rt{CVᴴC}})
				and \fm{-ch}
				\begin{itemize}
				\item	\fm{–.aatch}
					from \fm{\rt[¹]{.at}} ‘plural go’
					using \motderiv{kei=}{\fm{∅}, \fm{-ch} rep}{up} in
					\newline
					\vbform{kei has u.aatch}{hab}[subj intr, \fm{∅}, mot]{they always go up}
						\vbmorph{kei=&has=&u-&\rt[¹]{.at}&\gm{-μμL}&\gm{-ch}}
							{up&\xx{plh}&\xx{zpfv}&\rt[¹]{go.\xx{pl}}&\·\xx{var}&\·\xx{rep}}
					\versus \vbform{kei yi.á!}{imp}{you guys go up!}
						\vbmorph{kei=&ÿi-&\rt[¹]{.at}&-⊗}
							{up&\xx{2pl.s}&\rt[¹]{go.\xx{pl}}&\·\xx{var}}
				\item	\fm{–g̱éexʼch}
					from \fm{\rt[²]{g̱ixʼ}} ‘throw’
					using \motderiv{kei=}{\fm{∅}, \fm{-ch} rep}{up} in
					\newline
					\vbform{kei akoog̱éexʼch}{hab}[tr, \fm{∅}, mot]{she/he/it always throws it upward}
						\vbmorph{kei=&a-&ka-&u-&\rt[²]{g̱ixʼ}&\gm{-μμH}&\gm{-ch}}
							{up&\xx{3>3}&\xx{sro}&\xx{zpfv}&\rt[²]{throw}&\·\xx{var}&\·\xx{rep}}
					\versus \vbform{kei kag̱éexʼ!}{imp}{throw it upward!}
						\vbmorph{kei=&ka-&\rt[²]{throw}&-μμH}
							{up&\xx{sro}&\rt[²]{throw}&\·\xx{var}}
				\item	\fm{–gwáatlch}
					from \fm{\rt[²]{gwaᴴtl}} ‘roll, rock’
					using \motderiv{kei=}{\fm{∅}, \fm{-ch} rep}{up} in
					\newline
					\vbform{kei akoolgwáatlch}{hab}[tr, \fm{∅}, mot]{she/he/it always rolls it upward}
						\vbmorph{kei=&a-&ka-&u-&l-&\rt[²]{gwaᴴtl}&\gm{-μμH}&\gm{-ch}}
							{up&\xx{3>3}&\xx{sro}&\xx{zpfv}&\xx{csv}&\rt[²]{roll}&\·\xx{var}&\·\xx{rep}}
					\versus \vbform{kei kalagwáatl!}{imp}{roll it upward!}
						\vbmorph{kei=&ka-&la-&\rt[²]{gwaᴴtl}&-μμH}
							{up&\xx{sro}&\xx{csv}&\rt[²]{roll}&\·\xx{var}}
				\end{itemize}
			\item	\label{item:-ch-hab-zother-CVC-μH}
				most other habitual forms of \fm{∅} conjugation class verbs
				with \fm{\rt{CVC}}, \fm{\rt{CVCʼ}}, and \fm{\rt{CVᴴC}} roots
				have \X{-μH} and \fm{-ch}
				\begin{itemize}
				\item	\fm{–.ínch}
					from \fm{\rt[²]{.in}} ‘gather, pick berries’ in
					\newline
					\vbform{oo.ínch}{hab}[tr, \fm{∅}, \fm{-μμH} act]{she/he/it always picks them (berries)}
						\vbmorph{a-&u-&\rt[²]{.in}&\gm{-μH}&\gm{-ch}}
							{\xx{3>3}&\xx{zpfv}&\rt[²]{pick.berry}&\·\xx{var}&\·\xx{rep}}
					\versus \vbform{.ín!}{imp}{pick them!}
						\vbmorph{\rt[²]{.in}&-μH}
							{\rt[²]{pick.berry}&\·\xx{var}}
				\item	\fm{–xásʼch}
					from \fm{\rt[²]{xasʼ}} ‘scrape’ in
					\newline
					\vbform{ooxásʼch}{hab}[tr, \fm{∅⁺}, \fm{-μμH} act]{she/he/it always scrapes him/her/it}
						\vbmorph{a-&u-&\rt[²]{xasʼ}&\gm{-μH}&\gm{-ch}}
							{\xx{3>3}&\xx{zpfv}&\rt[²]{scrape}&\·\xx{var}&\·\xx{rep}}
					\versus \vbform{xásʼ!}{imp}{scrape it!}
						\vbmorph{\rt[²]{xasʼ}&-μH}
							{\rt[²]{scrape}&\·\xx{var}}
				\item	\fm{–.útlch}
					from \fm{\rt[²]{.uᴴtl}} ‘boil fish’
					\newline
					\vbform{oosh.útlch}{hab}[tr, \fm{∅}, ach]{she/he/it always boils him/her/it}
						\vbmorph{a-&u-&sh-&\rt[²]{.uᴴtl}&\gm{-μH}&\gm{-ch}}
							{\xx{3>3}&\xx{zpfv}&\xx{pej}&\rt[²]{boil.fish}&\·\xx{var}&\·\xx{rep}}
					\versus \vbform{sha.útl!}{imp}{boil it!}
						\vbmorph{sha-&\rt[²]{.uᴴtl}&-μH}
							{\xx{pej}&\rt[²]{boil.fish}&\·\xx{var}}
				\end{itemize}
			\item	\label{item:-ch-hab-zother-CVC-μμL}
				but some habitual forms of \fm{∅} conjugation class verbs
				with \fm{\rt{CVC}}, \fm{\rt{CVCʼ}}, and \fm{\rt{CVᴴC}} roots
				instead have either \X{-μμL} (\fm{\rt{CVC}})
				or \X{-μμH} (\fm{\rt{CVCʼ}}, \fm{\rt{CVᴴC}})
				and \fm{-ch}
				for unclear reasons
				\parencites[124–126]{crippen:2019}[99–100]{eggleston:2013a}[204]{leer:1991};
				this exception is independent of \fm{∅} versus \fm{∅⁺} conjugation class
				\parencite[155–156, 160, 163]{crippen:2019}
				\begin{itemize}
				\item	\fm{–shooḵch}
					from \fm{\rt[²]{shuḵ}} ‘laugh at’ in
					\newline
					\vbform{ooshooḵch}{hab}[tr, \fm{∅}, \fm{-μμL} act]{she/he/it always laughs at him/her/it}
						\vbmorph{a-&u-&\rt[²]{shuḵ}&\gm{-μμL}&\gm{-ch}}
							{\xx{3>3}&\xx{zpfv}&\rt[²]{laugh.at}&\·\xx{var}&\·\xx{rep}}
					\versus \vbform{shúḵ!}{imp}{laugh at him/her/it!}
						\vbmorph{\rt[²]{shuḵ}&-μH}
							{\rt[²]{laugh.at}&\·\xx{var}}
				\item	\fm{–chʼéix̱ʼch}
					from \fm{\rt[²]{chʼex̱ʼ}} ‘point at’ in
					\newline
					\vbform{oochʼéix̱ʼch}{hab}[tr, \fm{∅}, ach, \fm{-t}]{she/he/it always points at him/her/it}
						\vbmorph{a-&u-&\rt[²]{chʼex̱ʼ}&\gm{-μμH}&\gm{-ch}}
							{\xx{3>3}&\xx{zpfv}&\rt[²]{point.at}&\·\xx{var}&\·\xx{rep}}
					\versus \vbform{chʼéx̱ʼ!}{imp}{point at it!}
						\vbmorph{\rt[²]{chʼex̱ʼ}&-μH}
							{\rt[²]{point.at}&\·\xx{var}}
				\item	\fm{–táawch}
					from \fm{\rt[²]{taᴴw}} ‘steal’ in
					\newline
					\vbform{ootáawch}{hab}[tr, \fm{∅}, \fm{-μμH} act]{she/he/it always steals him/her/it}
						\vbmorph{a-&u-&\rt[²]{taᴴw}&\gm{-μμH}&\gm{-ch}}
							{\xx{3>3}&\xx{zpfv}&\rt[²]{steal}&\·\xx{var}&\·\xx{rep}}
					\versus \vbform{táw!}{imp}{steal it!}
						\vbmorph{\rt[²]{taᴴw}&-μH}
							{\rt[²]{steal}&\·\xx{var}}
				\end{itemize}
			\item	\label{item:-ch-hab-z-CV}
				habitual forms of \fm{∅} conjugation class verbs
				with \fm{\rt{CV}} and \fm{\rt{CVᴸ}} roots
				have \X{-μμH} with \X{-ÿ} or \X{-w} and \fm{-ch};
				the presence of \fm{-ÿ} or \fm{-w} between the stem
					and \fm{-ch} apparently blocks the ablaut
					that would otherwise be expected from \fm{-ch}
					so /\ipa{a}/ and /\ipa{u}/ do not become [\ipa{e}]
				\begin{itemize}
				\item	\fm{–x̱áaÿch}
					from \fm{\rt[²]{x̱a}} ‘eat’ in
					\newline
					\vbform{oox̱áaÿch}{hab}[tr, \fm{∅}, \fm{-μH} act]{she/he/it always eats him/her/it}
						\vbmorph{a-&u-&\rt[²]{x̱a}&\gm{-μμH}&\gm{-ÿ}&\gm{-ch}}
							{\xx{3>3}&\xx{zpfv}&\rt[²]{eat}&\·\xx{var}&\·\xx{ÿsfx}&\·\xx{rep}}
					\versus \vbform{x̱á!}{imp}{eat it!}
						\vbmorph{\rt[²]{x̱a}&-μH}
							{\rt[²]{eat}&\·\xx{var}}
				\item	\fm{–kóowch}
					from \fm{\rt[²]{kuᴸ}} ‘come to know’ in
					\newline
					\vbform{ooskóowch}{hab}[tr, \fm{∅⁺}, ach]{she/he/it always comes to know him/her/it}
						\vbmorph{a-&u-&s-&\rt[²]{kuᴸ}&\gm{-μμH}&\gm{-w}&\gm{-ch}}
							{\xx{3>3}&\xx{zpfv}&\xx{xtn}&\rt[²]{know}&\·\xx{var}&\·\xx{ÿsfx}&\·\xx{rep}}
					\versus \vbform{sakóo!}{imp}{know it!}
						\vbmorph{sa-&\rt[²]{kuᴸ}&-μH}
							{\xx{xtn}&\rt[²]{know}&\·\xx{var}}
				\end{itemize}
			\end{enumerate}
		\end{enumerate}
	\item	\label{item:-ch-combo}
		in combination with other repetitive or derivational suffixes
		\begin{enumerate}
		\item	with \X{-aa} as \X{-jaa},
			also with \X{-áa} as \X{-jáa};
			see \X{-jaa} for more detail
		\item	with \X{-áḵw} as \fm{-jáḵw};
			attested only with the root \fm{\rt{xʼwan}} ‘boot’
			where it is plausibly repetitive in denoting the events of
			removing one boot and then the other; predicts the possibility of
			other repetitive removal or repetitive deprivation events
			with \fm{-ch} and \fm{-áḵw}, but none are attested
			\begin{itemize}
			\item	\fm{–xʼwánjáḵw} ‘remove boots’
				from \fm{\rt{xʼwan}} ‘boot’ (noun \fm{xʼwán}) in
				\newline
				\vbform{kawdlixʼwánjáḵw}{pfv}[subj intr?, \fm{∅}?, ach?]{she/he/it removed boots}
				\parencite[f04/77]{leer:1973}
					\vbmorph{ka-&w-&d-&lˢ-&i-&\rt{xʼwan}&-μH&\gm{-ch}&\gm{-áḵw}}
						{\xx{qual}&\xx{pfv}&\xx{mid}&\xx{tr}&\xx{stv}&\rt{boot}&\·\xx{var}&\·\xx{rep}&\·\xx{dprv}}
			\end{itemize}
		\end{enumerate}
	\item	\label{item:-ch-roots}
		potentially identifiable as a frozen suffix in some stems and roots
		\begin{enumerate}
		\item	\label{item:-ch-roots-CVCC}
			CVCC stems with possible frozen \fm{-ch}
			\begin{itemize}
			\item	\fm{g̱ánch} ‘tobacco’
				from unknown \fm{\rt{g̱an}}
			\item	\fm{g̱úḵch} \~\ \fm{g̱áḵwch} ‘hooligan trap’
				from unknown \fm{\rt{g̱uḵ}} \~\ \fm{\rt{g̱aḵw}}
			\item	\fm{–.ítʼch} ‘sparkle, glint, reflect light’
				from noun \fm{ítʼch} ‘glass’
				probably from \fm{\rt[¹]{.itʼ}} ‘soaked’ in
				\newline
				\vbform{kadli.ítʼch}{impfv}[obj intr, \fm{g}, inv state]{it sparkles, reflects light}
					\vbmorph{ka-&d-&lˢ-&i-&\rt[¹]{.itʼ}&-μH&\gm{-ch}}
						{\xx{hsfc}&\xx{mid}&\xx{intr}&\xx{stv}&\rt[¹]{sparkle}&\·\xx{var}&\·\xx{rep}}
			\item	\fm{–núkch} ‘be helpless, undependable’
				perhaps from \fm{\rt[¹]{nuk}} ‘singular sit’
				or from \fm{\rt[¹]{nikw}} ‘sick’ in
				\newline
				\vbform{yanúkch}{impfv}[obj intr, \fm{g}, inv state]{she/he/it is helpless, undependable}
				\parencites[04/215]{leer:1973}[309]{leer:1976}[19]{leer:1978b}
					\vbmorph{ÿa-&\rt[¹]{nuk}&-μH&\gm{-ch}}
						{\xx{stv}&\rt[¹]{helpless}&\·\xx{var}&\·\xx{rep}}
				\newline
				there is only one distinct attestation of this form
					\parencite[04/215]{leer:1973}
					so it is difficult to say more about the verb,
					but \citeauthor{leer:1973} cryptically suggests
					an alternate form \fm{yanúkts} \parencite[309]{leer:1976}
					or \fm{yanúkt} \parencite[19]{leer:1978b}
					which would have either \X{-ts} or \X{-t};
				other possibly related roots include
				\begin{inlinelist}
				\item	\fm{\rt[¹]{naᴴk}} ‘numb’
				\item	\fm{\rt[¹]{nikw}} \~\ \fm{\rt[¹]{nuk}}‘feel’
				\item	\fm{núkt} ‘blue grouse’
				\item	\fm{–núkts} ‘sweet, good tasting’
					(see \X{-ts} for more)
				\item	\fm{nóosh} ‘dead salmon’
				\item	\fm{\rt{nutlʼ}} ‘balk, refuse to cooperate’
				\item	\fm{\rt{nuts}} ‘spoiled, soured’
				\end{inlinelist}
			\end{itemize}
		\item	\label{item:-ch-roots-CVC}
			CVC roots with possible frozen \fm{-ch}
			\begin{itemize}
			\item	\fm{\rt{chech}} ‘dust off’;
				compare 
				\begin{inlinelist}
				\item	\fm{\rt{cha}} ‘strain out’
				\item	\fm{\rt{chux}} ‘rub, massage’
				\item	\fm{\rt{chuᴴk}} ‘rub to soften’
				\item	\fm{\rt{chuxʼ}} ’graze, touch lightly’
				\end{inlinelist}
			\item	\fm{\rt{hich}} ‘spy on, suspect, accuse’;
				compare 
				\begin{inlinelist}
				\item	\fm{\rt{hi}} ‘be evil’
				\item	\fm{\rt{hin}} ‘cranky’ (noun \fm{ḵukahín})
				\item	\fm{\rt{hixw}} ‘bewitch’ (noun \fm{héexw} ‘witchcraft’)
				\item	\fm{héix̱waa} ‘magic, charm’
				\item	perhaps also \fm{\rt{hi}} \~\ \fm{\rt{he}} ‘pay shaman’
				\end{inlinelist}
			\item	\fm{\rt{shich}} ‘female’
				(prenominal adjective \fm{shéech} ‘female’);
				compare \fm{\rt{shaʷ}} ‘woman; marry’
					(noun \fm{sháa} ‘woman’):
				\fm{shéech} could perhaps derive from \fm[*]{shéich}
					based on \fm{\rt{shaʷ}}
					with ablaut \X{-μᵉμH} and \fm{-ch}
			\item	\fm{\rt{tʼach}} ‘clap, slap; swim’
				compare
				\begin{inlinelist}
				\item	\fm{\rt{tʼak}} ‘dent; press into cake’
				\item	\fm{\rt{tʼakw}} ‘slap tail; hit hard’
				\item	\fm{\rt{tʼakw}} ‘temple of head’
				\item	\fm{\rt{tʼalʼ}} ‘press flat’
				\item	\fm{\rt{tʼasʼ}} ‘look angrily’
				\item	\fm{\rt{tʼatʼ}} ‘pat in hands’
				\item	\fm{\rt{tʼatl}} ‘splash water’
				\item	\fm{\rt{tʼaxʼ}} ‘cut into bits; flick with finger’
				\end{inlinelist}
			\item	\fm{\rt{tʼuch}} ‘sting, smart’
				compare \fm{\rt{tʼuᴴk}} ‘shoot with arrow’
			\end{itemize}
		\end{enumerate}
	\end{enumerate}

\item[-chʼ]\label{m:-chʼ}
	suffix with unknown meaning;
	discussed by \textcite[56]{story:1966},
		appears to be derivational (present in all forms)
		and not inflectional (present only in some forms);
	may be part of unknown \X{-chʼáḵw} with deprivative \X{-áḵw} ‘lacking’,
		part of unknown \X{-chʼálʼ} with repetitive \X{-álʼ},
		and part of intensifier \X{-chʼán} with restorative \X{-án};
	otherwise documented only in one verb as \X{-áchʼ} (which see);
	because the meaning of \fm{-chʼ} \~\ \fm{-áchʼ} is unknown it is glossed as \xx{unkn}
	\newline
	allomorphs:
	\begin{allolist}
	\item[-chʼ]	single consonant form
	\item[\X{-áchʼ}]	form with epenthetic (filler) vowel \fm{á}
	\item[\X{-sh}]	form after ejective consonant in \X{-chʼán} \~\ \X{-shán}
	\end{allolist}
	\begin{enumerate}
	\item	in \fm{g̱eeg̱áchʼ} \~\ \fm{g̱eig̱áchʼ} ‘hammock, swing for baby’
		and related forms
		see \X{-áchʼ}
	\item	in \fm{kawdudliséewchʼáḵw} ‘it (berry) is full of rain (and so tasteless)’
		see \X{-chʼáḵw}
	\item	in \fm{ḵéichʼálʼ} \~\ \fm{ḵéechʼálʼ} ‘seam’
		see \X{-chʼálʼ}
	\item	in \fm{kuli.áax̱chʼán} ‘she/he/it is fascinating, interesting to hear’
		and related forms
		see \X{-chʼán}
	\item	possibly identifiable as the final consonant in some CVC verb roots:
		\begin{inlinelist}
		\item	\fm{\rt{.achʼ}} ‘insufficient’
		\item	\fm{\rt{chʼachʼ}} ‘spotted, polka-dotted’
			(compare \fm{\rt{chʼalʼ}} ‘pale, spotted’, \fm{chʼáalʼ} ‘willow’,
			\fm{tlʼáatlʼ} ‘yellow salmonberry’,
			also perhaps \fm{cháasʼ} ‘humpy or pink salmon’)
		\item	\fm{\rt{duchʼ}} ‘tweak, pinch and twist’
			(compare \fm{\rt{dutlʼ}} ‘roll up’ and \fm{\rt{tuchʼ}} ‘rub with hand’)
		\item	\fm{\rt{duchʼ}} ‘cut into chunks’
			(also noun \fm{g̱áaxʼw kadóochʼi} ‘herring eggs crumbled into chunks’)
		\item	\fm{\rt{g̱wachʼ}} \~\ \fm{\rt{g̱uchʼ}} ‘wrap in blanket’
		\item	\fm{\rt{hachʼ}} ‘shameful’
			(compare \fm{\rt{hasʼ}} ‘vomit’ and \fm{\rt{hatlʼ}} ‘crap’)
		\item	\fm{\rt{kuchʼ}} ‘curly’
			(also noun \fm{kakóochʼi} ‘fiddlehead (of fern)’)
		\item	\fm{\rt{kuchʼ}} ‘fart noiselessly’
			(also noun \fm{kóochʼ} ‘noiseless fart’)
		\item	\fm{\rt{ḵichʼ}} ‘watch covertly, spy’
		\item	\fm{\rt{ḵʼichʼ}} ‘scabby’
			(also noun \fm{ḵʼéechʼ} ‘scar with scab’)
		\item	\fm{\rt{tuchʼ}} ‘rub with hand’
			(compare \fm{\rt{duchʼ}} ‘tweak’ and \fm{\rt{dutlʼ}} ‘roll up’)
		\item	\fm{\rt{tʼuchʼ}} ‘char’
			(also noun \fm{tʼoochʼ} ‘charcoal’,
			\fm{waḵlitaaktʼoochʼí} ‘iris of eye’,
			compare \fm{Duktʼootlʼ} ‘Black Skin’)
		\item	\fm{\rt{wuchʼ}} ‘murky’
			(compare \fm{\rt{wusʼ}} ‘murky’)
		\item	\fm{\rt{wuchʼ}} ‘stubborn’
			(compare \fm{\rt{wutlʼ}} ‘stubborn’ and \fm{\rt{wusʼ}} ‘tough’)
		\item	\fm{\rt{ÿachʼ}} ‘too short’
			(compare \fm{\rt{ÿatʼ}} ‘long’ and \fm{\rt{ÿatlʼ}} ‘short’,
			also in noun \fm{sʼuḵkulayáachʼi} ‘lowest, shortest rib’)
		\end{inlinelist}
	\item	possibly identifiable as the final consonant in some CVCC nouns:
		\begin{inlinelist}
		\item	\fm{ḵʼánchʼ} ‘unidentified species of seaweed’
			(compare \fm{ḵʼaan} ‘dolphin’ and \fm{néiḵʼán} ‘stone weir’)
		\item	\fm{tláxchʼ} ‘dead branches (for tinder)’
			(compare \fm{\rt{tlax̱}} ‘mouldy’)
		\item	\fm{xíxchʼ} ‘frog’
			(compare \fm{\rt{xix}} ‘fall, move through space’)
		\item	\fm{yáxwchʼ} \~\ \fm{yúxchʼ} ‘sea otter’
			(compare \fm{\rt{yux̱ʼ}} ‘waterlogged’)
		\end{inlinelist}
	\item	possibly identifiable as the final consonant in some CVVC nouns:
		\begin{inlinelist}
		\item	\fm{éechʼ} ‘boulder’
		\item	\fm{iḵnáachʼ} \~\ \fm{eḵnáachʼ} ‘brass’
			(\fm{eeḵ} \~\ \fm{eiḵ} ‘copper’ with unidentified \fm{náachʼ},
			compare \fm{\rt{natlʼ}} \~\ \fm{\rt{natsʼ}}
				‘pruned, wrinkled from water; clumsy’)
		\item	\fm{ḵáachʼ} ‘red seaweed’
		\item	\fm{kʼóochʼ} ‘rear projection (snowshoe, spear)’
			(compare \fm{kʼóolʼ} ‘tailbone’)
		\item	\fm{ḵʼéechʼ} ‘scar with scab’
			(also verb \fm{\rt{ḵʼichʼ}} ‘scabby’)
		\item	\fm{lakʼéechʼ} ‘nape of neck’
			(\fm{la-} \~\ \fm{le-} ‘neck’ with unidentified \fm{kʼéechʼ})
		\item	\fm{lugóochʼ} ‘lobe of nostril, nasal alae’
			(\fm{lú} ‘nose’ with unidentified \fm{góochʼ},
			compare \fm{gangóosh} ‘two-eared headdress’)
		\item	\fm{lugwáachʼ} ‘rhinocerous auklet’
			(\fm{lú} ‘nose’ with unidentified \fm{gwáachʼ},
			compare \fm{\rt{gwatlʼ}} ‘fold, bend’,
			\fm{\rt{gwalʼ}} ‘curl’,
			and \fm{\rt{gwaᴴsh}} ‘hop on one leg’)
		\item	\fm{nóochʼi} ‘fish heart; gill bone’
		\item	\fm{sháachʼ} ‘smelt, sardine, young herring’
		\item	\fm{sháachʼ} ‘unidentified lichen; licorice’
			(also in \fm{ashayakiksháachʼi} ‘unidentified lichen’,
			\fm{asxʼaansháachʼi} ‘unidentified bird species’)
		\item	\fm{tʼoochʼ} ‘charcoal’
			(also in \fm{waḵlitaaktʼoochʼí} ‘iris of eye’,
			verb \fm{\rt{tʼuchʼ}} ‘char’,
			compare \fm{Duktʼootlʼ} ‘Black Skin’)
		\item	\fm{yéichʼ} ‘canoe carving depth peg’
		\end{inlinelist}
	\end{enumerate}

\item[-chʼáḵw]\label{m:-chʼáḵw}
	suffix with unknown meaning, but possibly a combination of unknown \X{-chʼ}
		and deprivative \X{-áḵw} ‘lacking’;
	attested only in one verb derived from the noun \fm{séew} ‘rain’
	\begin{itemize}
	\item	\vbform{kawdudliséewchʼáḵw}{pfv}[obj intr, conj?, ach?]{it (berry) is full of rain (and so tasteless)}
		\parencite[56]{story:1966}
			\vbmorph{ka-&w-&du-&d-&l-&i-&\rt{siw}&-μμH&\gm{-chʼ}&\gm{-áḵw}}
				{\xx{sro}&\xx{pfv}&\xx{xpl}&\xx{mid}&\xx{intr}&\xx{stv}&\rt{rain}&\·\xx{var}&\·\xx{unkn}&\·\xx{dprv}}
	\end{itemize}

\item[-chʼálʼ]\label{m:-chʼálʼ}
	suffix with unknown meaning, but possibly a combination of unknown \X{-chʼ}
		and repetitive \X{-álʼ};
	attested only in one noun derived from the root \fm{\rt[²]{ḵa}} ‘stitch, sew’;
	compare the stem \fm{–ḵéilʼútʼ} ‘lick seam’ with \X{-lʼútʼ}
	\begin{itemize}
	\item	\fm{ḵéichʼálʼ} \~\ \fm{ḵéechʼálʼ} ‘seam’
		\parencite[f01/23]{leer:1973}
		\vbmorph{\rt{ḵa}&-μᵉμH&\gm{-chʼ}&\gm{-álʼ}}
			{\rt{stitch}&\·\xx{var}&\·\xx{unkn}&\·\xx{rep}}
		\newline
		the form \fm{ḵéichʼálʼ} has \X{-μᵉμH} stem variation
			with ablaut of /\ipa{a}/ → [\ipa{e}] as is normal with a consonant suffix
			after a /\ipa{Ca}/ or /\ipa{Cu}/ root,
		but the form \fm{ḵéechʼálʼ} is unexpected and suggests reanalysis
			with interdialectal reversal of uvular lowering
	\end{itemize}

\item[-chʼán]\label{m:-chʼán}
	intensifier suffix in verbs with the ‘extraordinary state’ derivation
		made up of:
		qualifier \X[ka-qual]{ka-}
		+ irrealis \X[u-irr]{u-}
		+ extensional \X{s-}/\X{lˢ-}
		+ state \X[i-stv]{i-}
		+ intensifier \X{-chʼán} \~\ \fm{-shán}
		with \fm{g} conjugation class
		\parencite[655]{crippen:2019};
	discussed by \textcite[56]{story:1966}
		who demonstrates an alternation between basic \fm{-chʼán}
		versus \X{-shán} after an ejective consonant as a kind of dissimilation
		\parencite[878]{crippen:2019};
	the exact meaning is unclear but this suffix is associated with experience of
		an intense “pleasurable or fearful reaction” \parencite[56]{story:1966}
		to a situation, thus the label ‘intensifier’;
	\fm{-chʼán} is possibly formed by combination of
		unknown \X{-chʼ} and restorative \X{-án}
		and \fm{-shán} by unknown \X{-sh} and restorative \X{-án},
		but the composition of meaning is unclear;
	if treated as a single suffix \fm{-chʼán} \~\ \fm{-shán}
		it is glossed as \xx{intns},
		otherwise \fm{-chʼ} \xx{unkn} or \fm{-sh} \xx{unkn}
		+ \fm{-án} \xx{rest}
	\newline
	allomorphs:
	\begin{allolist}
	\item[-chʼán]	basic form
	\item[\X{-shán}]	form used after an ejective consonant
	\end{allolist}
	\begin{enumerate}
	\item	suffix in verbs with the extraordinary state derivation outlined above
		\begin{itemize}
		\item	\fm{–.áax̱chʼán} ‘fascinating to hear’
			from \fm{\rt[²]{.ax̱}} ‘hear’ in
			\newline
			\vbform{kuli.áax̱chʼán}{impfv}[obj intr, \fm{g}, inv state]{she/he/it is fascinating, interesting to hear}
			\parencites[02/172]{leer:1973}[120]{leer:1976}
				\vbmorph{ka-&ʷ-&l-&i-&\rt{.ax̱}&-μμH&\gm{-chʼán}}
					{\xx{qual}&\xx{irr}&\xx{xtn}&\xx{stv}&\rt{hear}&\·\xx{var}&\·\xx{intns}}
			\versus \vbform{aawa.áx̱}{pfv}[tr, \fm{∅}, ach]{she/he/it heard him/her/it}
				\vbmorph{a-&μʷ-&wa-&\rt[²]{.ax̱}&-μH}
					{\xx{3>3}&\xx{pfv}&\xx{stv}&\rt[²]{hear}&\·\xx{var}}
		\item	\fm{–jáaḵwchʼán} ‘like to beat up’
			from \fm{\rt[²]{jaḵw}} ‘beat up’ in
			\newline
			\vbform{alijáaḵwchʼán}{impfv}[tr, \fm{g}?, inv state]{she/he/it likes to beat up, fight with him/her/it}
			\parencite[132.223]{dauenhauer-dauenhauer:1987}
				\vbmorph{a-&lˢ-&i-&\rt[²]{jaḵw}&-μμH&\gm{-chʼán}}
					{\xx{3>3}&\xx{xtn}&\xx{stv}&\rt[²]{beat.up}&\·\xx{var}&\xx{intns}}
			\versus \vbform{ajáaḵw}{impfv}[tr, \fm{n}, \fm{-μμH} act]{she/he/it beats up him/her/it}
				\vbmorph{a-&\rt[²]{jaḵw}&-μμH}
					{\xx{3>3}&\rt[²]{beat.up}&\·\xx{var}}
		\item	\fm{–nóokchʼán} ‘look delicious’
			from \fm{\rt[⁰]{nuk}} ‘sweet’ (usually as \fm{–núkts}) in
			\newline
			\vbform{x̱ʼalinóokchʼán}{impfv}[obj intr, \fm{g}?, inv state]{she/he/it looks delicious}
			\parencite[04/219]{leer:1973}
				\vbmorph{x̱ʼe-&l-&i-&\rt[⁰]{nuk}&-μμH&\gm{-chʼán}}
					{mouth&\xx{intr}&\xx{stv}&\rt{sweet}&\·\xx{var}&\·\xx{intns}}
			\versus \vbform{linúkts}{impfv}[obj intr, \fm{g}, inv state]{she/he/it is sweet (tasting)}
				\vbmorph{lˢ-&i-&\rt[⁰]{nuk}&-μH&-ts}
					{\xx{intr}&\xx{stv}&\rt[⁰]{sweet}&\·\xx{var}&\·\xx{unkn}}
		\item	\fm{-chʼán} is specifically attested with the roots:
			\begin{inlinelist}
			\item	\fm{\rt{.ax̱}} ‘hear’
			\item	\fm{\rt{jaḵw}} ‘beat up’
			\item	\fm{\rt{na}} ‘damp, oil’
				(\fm{–néisʼ} = \fm{\rt{na}} + \X{-μᵉμH} + \X{-sʼ})
			\item	\fm{\rt{nik}} ‘tell’
			\item	\fm{\rt{nitl}} \~\ \fm{\rt{netl}} ‘fat (human)’
			\item	\fm{\rt{nuk}} ‘sweet, delicious’
				(\fm{–núkts} = \fm{\rt{nuk}} + \X{-μH} + \X{-ts};
				loss of \fm{-ts} with \fm{-chʼán})
			\item	\fm{\rt{tul}} ‘spin, drill’
			\item	\fm{\rt{tʼaᴸ}} ‘hot’
				(\fm{–tʼáaÿ} = \fm{\rt{tʼaᴸ}} + \X{-μμH} + \X{-ÿ})
			\item	\fm{\rt{.ush}} ‘pout’
			\item	\fm{\rt{.uᴴw}} ‘buy’
			\item	\fm{\rt{was}} ‘roast’
			\item	\fm{\rt{wash}} ‘yawn’;
			\end{inlinelist}
			see \X{-shán} for others
		\end{itemize}
	\item	suffix in one noun, without the other morphology of the extraordinary state derivation
		\begin{itemize}
		\item	\fm{toolchʼán} ‘top, spinning toy’
			from \fm{\rt[²]{tul}} ‘spin, drill’
			\vbmorph{\rt[²]{tul}&-μμL&-chʼ&-án}
				{\rt[²]{drill}&\·\xx{var}&\·\xx{unkn}&\·\xx{rest}}
		\end{itemize}
	\end{enumerate}

\item[chush=]\label{m:chush=}
	allomorph of reflexive object \X{sh=}

\end{morphdesc}
