%!TEX root = ../lingnote-verbmorphs.tex

\subsection{Sh}\label{sec:alphalist-sh}
\begin{morphdesc}[resume*=alphalist]
\item[sh-]\label{m:sh-}
	valency prefix of classifier
	\begin{enumerate}
	\item	pejorative
		\begin{enumerate}
		\item	pejorative entity
		\item	pejorative eventuality
		\end{enumerate}
	\item	negative
	\item	unclear meaning
	\end{enumerate}

\item[sh=]\label{m:sh=}
	reflexive object;
	requires middle voice \fm{d-}

\item[-sh]\label{m:-sh}
	unknown suffix with very limited distribution
	\begin{enumerate}
	\item	allomorph of \X{-chʼ} in \X{-chʼán} \~\ \X{-shán};
		the form \fm{-shán} occurs instead of intensive \fm{-chʼán}
			when immediately following an ejective consonant;
		since \fm{-chʼán} might be analyzed as \X{-chʼ} + \X{-án},
		the \fm{-shán} form implies \fm{-sh} + \fm{-án};
		see \X{-shán} for more detail
	\item	unknown suffix which occurs only in the stem 
			\fm{–gwéinsh} ‘tinker, fiddle, mess with’
			which is analyzed as \fm{\rt{gwen}-μμH-sh};
		the underlying root is unknown but could plausibly
			arise from either \fm{\rt{gu}} ‘joy, enjoy, fun’
			or \fm{\rt{gu}} ‘poke, stab; sea mammals go in group’
			with unknown \X{-n}
			and predicted stem \X{-μᵉμH};
		this \fm{-sh} could plausibly arise from repetitive \X{-ch}
		\begin{itemize}
		\item	\vbform{algwéinsh}{impfv}[tr?, conj?, act]{she/he/it fiddles with him/her/it}
				\vbmorph{a-&l-&\rt{gwen}&-μμH&\gm{-sh}}
					{\xx{3>3}&\xx{xtn}&\rt{fiddle?}&\·\xx{var}&\·\xx{unkn}}
		\end{itemize}
	\end{enumerate}
	the \fm{-sh} suffix might also be identified in a few nouns:
	\begin{itemize}
	\item	\fm{gúksh} ‘corner’
		(\fm{gúk} ‘ear’?; compare \fm{gukshutú} ‘corner’)
	\item	\fm{gutguníksh} ‘kind of owl’
		(possibly onomatopoiea)
	\item	\fm{sʼíksh} ‘false hellebore, skookum root’
		(unknown root; compare \fm{sʼeek} ‘black bear’,
		\fm{\rt{sʼixw}} ‘sour’,
		\fm{\rt{sʼikw}} ‘crisp’)
	\item	\fm{wéiksh} ‘ulu’
	\end{itemize}

\item[…sh]\label{m:…sh}
	≡ \fm{d-sh-}
	combination of voice \X{d-}
		and valency \X{sh-},
	appears only as a coda consonant and so requires a preceding vowel
	\begin{itemize}
	\item	\fm{yaa sh kanx̱ashxʼáḵw} (prog; tr, \fm{n}, ach) ‘I am making myself comfortable’
			with \fm{d-sh-}\newline
		versus \fm{sh kax̱wjixʼaaḵw} (pfv) ‘I made myself comfortable’
			with \fm{d-sh-i-}
	\end{itemize}

\item[sha-]\label{m:sha-val}
	allomorph of valency \X{sh-}

\item[sha-]\label{m:sha-head}
	incorporated noun indicating head or hair of the head;
	derived from relational noun \fm{shá} ‘head’

\item[shakux=]
	incorporated noun ‘thirst’,
	saturates object argument;
	derived from \fm{shá} ‘head’ and \fm{\rt[¹]{kux}} ‘dry’
		in verb \fm{x̱at shaawakúx} (pfv; obj intr, \fm{∅}, ach) ‘I got thirsty’
		suggesting a nominalization \fm{shakoox} ‘thirsting’
	\begin{itemize}
	\item	\fm{ax̱ éet shakux uwaháa} (pfv; obj intr, \fm{∅}, mot) ‘thirst appeared to me’,
		i.e.\ ‘I got thirsty’
		\parencite[01/11]{leer:1973}
	\end{itemize}

\item[-shán]\label{m:-shán}
	allomorph of intensifier \X{-chʼán},
	used when immediately following an ejective consonant;
	occurs as part of the ‘extraordinary state’ derivation
		made up of:
		qualifier \X[ka-qual]{ka-}
		+ irrealis \X[u-irr]{u-}
		+ valency \X{s-}/\X{lˢ-}
		+ state \X[i-stv]{i-}
		+ intensifier \X{-chʼán} \~\ \fm{-shán}
		with \fm{g} conjugation class
		\parencite[655]{crippen:2019};
	possibly formed by combination of unknown \X{-sh}
		and restorative \X{-án};
	see \X{-chʼán} for more discussion and examples
	\begin{itemize}
	\item	\fm{–téesʼshán} ‘fascinating to watch’
		from \fm{\rt[¹]{tisʼ}} ‘stare; look out’ in
		\newline
		\vbform{kulitéesʼshán}{impfv}[obj intr, \fm{g}, inv state]{she/he/it is fascinating to watch}
		\parencite[87.1085]{story-naish:1973}
			\vbmorph{k-&u-&lˢ-&i-&\rt[¹]{tisʼ}&-μμH&\gm{-shán}}
				{\xx{qual}&\xx{irr}&\xx{xtn}&\xx{stv}&\rt[²]{stare}&\·\xx{var}&\·\xx{intns}}
		\versus \vbform{kaawatísʼ}{pfv}[subj intr, \fm{∅}, ach]{she/he/it stared}
			\parencite[06/243]{leer:1973}
			\vbmorph{ka-&μʷ-&wa-&\rt[¹]{tisʼ}&-μH}
				{\xx{qual}&\xx{pfv}&\xx{stv}&\rt[¹]{stare}&\·\xx{var}}
	\item	\fm{–x̱éetlʼshán} \~\ \fm{–x̱éitlʼshán} ‘frightening’
		from \fm{\rt{x̱itlʼ}} \~\ \fm{\rt{x̱etlʼ}} ‘fear’ in
		\newline
		\vbform{kulix̱éetlʼshán}{impfv}[obj intr, \fm{g}, inv state]{she/he/it is frightening}
		\parencite[63.732]{story-naish:1973}
			\vbmorph{k-&u-&lˢ-&i-&\rt[²]{x̱itlʼ}&-μμH&\gm{-shán}}
				{\xx{qual}&\xx{irr}&\xx{xtn}&\xx{stv}&\rt[²]{fear}&\·\xx{var}&\·\xx{intns}}
		\versus \vbform{áxʼ akoox̱dlix̱éetlʼ}{impfv}[subj intr, \fm{g}, \fm{-μμH} state]{I am afraid of it}
			\vbmorph{á&-xʼ&a-&k-&oo-&x̱-&d-&lˢ-&i-&\rt[²]{x̱itlʼ}&-μμH}
				{\xx{3n}&\·\xx{loc}&\xx{xpl}&\xx{qual}&\xx{irr}&\xx{1sg.s}&\xx{mid}&\xx{xtn}&\xx{stv}&\rt[²]{fear}&\·\xx{var}}
	\item	\fm{-shán} is specifically attested with the roots:
		\begin{inlinelist}
		\item	\fm{\rt{.etsʼ}} ‘handle gingerly’
		\item	\fm{\rt{tisʼ}} ‘stare’
		\item	\fm{\rt{.usʼ}} ‘wash’
		\item	\fm{\rt{walʼ}} ‘break’
		\item	\fm{\rt{wasʼ}} \~\ \fm{\rt{wusʼ}} ‘ask’
		\item	\fm{\rt{x̱itlʼ}} \~\ \fm{\rt{x̱etlʼ}} ‘fear’
		\item	\fm{\rt{x̱ʼwalʼ}} ‘fluff, down’
		\item	\fm{\rt{yeᴴn}} ‘wave’ (probably from \fm[*]{yen̓} \~\ \fm[*]{yenʔ})
		\item	\fm{\rt{yux̱ʼ}} ‘gnaw’
		\end{inlinelist}
	\end{itemize}

\item[shi]
	≡ \fm{sh-i-}
	combination of valency \X{sh-}
		and stative \X[i-stv]{i-}
\end{morphdesc}
