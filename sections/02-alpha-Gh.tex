%!TEX root = ../lingnote-verbmorphs.tex

\subsection{G̱}\label{sec:alphalist-gh}
\begin{morphdesc}[resume*=alphalist]
\item[g̱-]\label{m:g̱-conj}
	\fm{g̱} conjugation prefix, downward spatial orientation;
	can occur together with modal \fm{g̱-}

\item[g̱a-]\label{m:g̱a-conj}
	allomorph of \fm{g̱} conjugation prefix \X[g̱-conj]{g̱-}

\item[g̱a-]\label{m:g̱a-mod}
	allomorph of modal \X[g̱-mod]{g̱-} with epenthetic (filler) vowel \fm{a}

\item[g̱-]\label{m:g̱-mod}
	modal prefix in prospective aspect,
		hortative mood,
		potential mood,
		and contingent mood;
	can occur together with \fm{g̱-} conjugation class prefix in hortative, potential, contingent;
	\newline
	allomorphs:
	\begin{allolist}
	\item[{\X[g̱a-mod]{g̱a-}}]
			form with epenthetic (filler) vowel \fm{a}
	\item[{\X[x̱-g̱cnj]{x̱-}}]
			form occurring as consonant in coda of a syllable
	\end{allolist}
	combinations:
	\begin{allolist}
	\item[ḵa]	≡ \fm{g̱-x̱a-} with first person singular subject \fm{x̱a-}
	\end{allolist}
	\begin{enumerate}
	\item	prospective aspect:
			conjugation \fm{g-}
			+ irrealis \fm{u-}
			+ modal \fm{g̱-}
		\begin{itemize}
		\item	\vbform{gug̱atáa}{prosp}[subj intr, \fm{n}, \fm{-μH} act]{she/he/it will sleep}
			\vbmorph{g-&u-&\gm{g̱a-}&\rt[¹]{taᴸ}&-μμH}
				{\xx{gcnj}&\xx{irr}&\xx{mod}&\rt[¹]{sleep.\xx{sg}}&\·\xx{var}}
			\versus \vbform{wootaa}{pfv}{she/he/it slept}
			\vbmorph{wu-&μ-&\rt[¹]{taᴸ}&-μμL}
				{\xx{pfv}&\xx{stv}&\rt[¹]{sleep.\xx{sg}}&\·\xx{var}}
		\end{itemize}
	\item	hortative mood:
			conjugation \fm{n-}/\fm{g̱-}/\fm{g-}
			+  modal \fm{g̱-}
		\begin{itemize}
		\item	\vbform{nag̱ataa}{hort}[subj intr, \fm{n}, \fm{-μH} act]{let him/her/it sleep}
			\vbmorph{na-&\gm{g̱a-}&\rt[¹]{taᴸ}&-μμL}
				{\xx{ncnj}&\xx{mod}&\rt[¹]{sleep.\xx{sg}}&\·\xx{var}}
		\end{itemize}
	\item	potential mood:
			irrealis \fm{u-}
			+ conjugation \fm{n-}/\fm{g̱-}/\fm{g-}
			+ modal \fm{g̱-}
	\item	contingent mood:
			conjugation \fm{n-}/\fm{g̱-}/\fm{g-}
			+ modal \fm{g̱-}
			(+ \fm{-n} + \fm{-ín})
	\end{enumerate}

\item[g̱ax̱=]\label{m:g̱ax̱=}
	incorporated noun ‘crying’, saturates object;
	derived from \fm{\rt[¹]{g̱ax̱}} ‘cry’
	\begin{itemize}
	\item	\vbform{gax̱satí}{impfv}[subj intr, \fm{g}, \fm{-μH} act]{they cry}
		\vbmorph{\gm{gax̱=}&sa-&\rt[¹]{tiᴸ}&-μH}
			{cry&\xx{appl}&\rt[¹]{be}&\·\xx{var}}
		\versus \vbform{g̱áax̱}{impfv}[subj intr, \fm{g}, \fm{-μμH} act]{she/he/it cries}
		\vbmorph{\rt[¹]{g̱ax̱}&-μμH}
			{\rt[¹]{cry}&\·\xx{var}}
	\item	\vbform{kei gax̱ gax̱yisatée}{prosp}[subj intr, \fm{g}, \fm{-μμH} act]{you pl.\ will cry}
		\parencite[60.683]{story-naish:1973}
		\vbmorph{kei=&\gm{gax̱=}&ga-&w̸-&x̱-&sa-&\rt[¹]{tiᴸ}&-μμH}
			{up=&cry&\xx{gcnj}&\xx{irr}&\xx{mod}&\xx{appl}&\rt[¹]{be}&\·\xx{var}}
	\end{itemize}

\item[g̱unayéi=]\label{m:g̱unayéi=}
	inceptive manner preverb ‘starting, beginning’ indicating the initiation of motion or the
		beginning of an eventuality;
	derived from the noun \fm{g̱unayéi} ‘elsewhere, different place’ which is itself
		from the modifier \fm{g̱una} ‘different, other’
		and the noun \fm{yé} \~\ \fm{yéi} ‘place, way’;
	the initial use as a motion derivation ‘going elsewhere’ > ‘starting off’
		with motion verbs has been extended to apply to other verbs that do not normally
		denote motion such as activities and states;
	some speakers only use the \X{g̱unéi=} form as a preverb and reserve \fm{g̱unayéi} only as
		a noun or postposition phrase because the inceptive ‘starting’ meaning no longer
		has an obvious connection in meaning to ‘elsewhere, different place’;
	can be glossed \xx{incep} for ‘inceptive’ or more transparently as ‘start’ or ‘begin’
		although these glosses can be misleading since they are verbs in English
	\newline
	allomorphs:
	\begin{allolist}
	\item[\X{g̱unéi=}]	variant form used by some speakers
	\item[\X{g̱unyéi=}]	variant form used by some speakers
	\end{allolist}
	\begin{enumerate}
	\item	initiation of motion in the motion derivation
			\motderiv{g̱unayéi \~\ g̱unéi}{∅, \fm{-x̱} rep}{starting off, setting out, beginning}
		\begin{itemize}
		\item	\vbform{g̱unayéi wtuwa.át}{pfv}[subj intr, \fm{∅}, mot, \fm{-x̱} rep]{we started going}
				\vbmorph{\gm{g̱unayéi}=&w-&tu-&wa-&\rt[¹]{.at}&-μH}
					{\xx{incep}&\xx{pfv}&\xx{1pl.s}&\rt[¹]{go.\xx{pl}}&\·\xx{var}}
			\versus \vbform{wutuwa.aat}{pfv}[subj intr, \fm{n}, \fm{yoo=i-…-k} rep]{we went}
				\vbmorph{wu-&tu-&wa-&\rt[¹]{.at}&-μμL}
					{\xx{pfv}&\xx{1pl.s}&\rt[¹]{go.\xx{pl}}&\·\xx{var}}
		\item	\vbform{g̱unayéi ushx̱ʼéelʼch}{hab}[obj intr, \fm{∅}, mot, \fm{-x̱} rep]{it always starts sliding}
			\parencite[176.182]{nyman-leer:1993}
				\vbmorph{\gm{g̱unayéi}=&u-&sh-&\rt[¹]{x̱ʼilʼ}&-μμH&-ch}
					{\xx{incep}&\xx{zpfv}&\xx{pej}&\rt[¹]{slide}&\·\xx{var}&\·\xx{rep}}
			\versus \vbform{át x̱at nashx̱ʼílʼch}{hab}[obj intr, \fm{n}, mot, \fm{yoo=…i-…-k} rep]{I always slide around there}
			\parencite[136041]{eggleston:2017}
				\vbmorph{á&-t&x̱at=&na-&sh-&\rt[¹]{x̱ʼilʼ}&-μH&-ch}
					{\xx{3n}&\·\xx{pnct}&\xx{1sg.o}&\xx{ncnj}&\xx{pej}&\rt[¹]{slide}&\·\xx{var}&\·\xx{rep}}
		\end{itemize}
		contrast this usage of \fm{g̱unayéi} as part of a motion derivation with its use as an
			ordinary noun in a postposition phrase where it does not indicate initiation
			but instead refers to ‘elsewhere, different place’, in which case the
			conjugation class and/or repetitive imperfective form of the verb may be
			different from the motion derivation
		\begin{itemize}
		\item	\fm{Aag̱áa chʼa \gm{g̱unayéi}de áwé s woo.aat.}
			“Then they moved to a different place.”
			\parencite[172.131]{dauenhauer-dauenhauer:1987}
		\item	\fm{Sʼíxʼ dei \gm{g̱unayéi}xʼ kax̱waa.étsʼ.}
			“I’ve already moved the dish carefully to another place.”
			\parencite[137.1848]{story-naish:1973}
		\end{itemize}
		also contrast with the use of \fm{g̱unayéi} as a noun modifier in which case it precedes
			another noun and so does not occur immediately before the verb or other preverbs
		\begin{itemize}
		\item	\fm{\gm{G̱unayéi} ḵwáan haa woogaaḵ.} “Folk from other localities are visiting us.”
			\parencite[240.3399]{story-naish:1973}
		\end{itemize}
	\item	initiation of any eventuality in the eventuality derivation
			\motderiv{g̱unayéi \~\ g̱unéi}{∅, \fm{-x̱} rep}{beginning, starting, initiating};
		application of this derivation shifts the eventuality class of the verb to an 
			achievement thus prohibiting a non-repetitive imperfective aspect form
			(like a motion verb);
		also note the change of conjugation class and the addition of the \fm{-x̱} repetitive
			imperfective
		\begin{itemize}
		\item	\vbform{g̱unayéi aawax̱áa}{pfv}[tr, \fm{∅}, ach]{she/he/it started eating him/her/it}
				\vbmorph{\gm{g̱unayéi}=&a-&μʷ-&wa-&\rt[²]{x̱a}&-μμH}
					{\xx{incep}&\xx{3>3}&\xx{pfv}&\xx{stv}&\rt[²]{eat}&\·\xx{var}}
			\versus	\vbform{aawax̱áa}{pfv}[tr, \fm{∅}, \fm{-μH} act]{she/he/it ate him/her/it}
				\vbmorph{a-&μʷ-&wa-&\rt[²]{x̱a}&-μμH}
					{\xx{3>3}&\xx{pfv}&\xx{stv}&\rt[²]{eat}&\·\xx{var}}
		\item	\vbform{g̱unayéi ḵuwtuwashée}{pfv}[subj intr, \fm{∅}, ach]{we began searching}
			\parencite[68.132]{dauenhauer-dauenhauer:1987}
				\vbmorph{\gm{g̱unayéi}=&ḵu-&w-&tu-&wa-&\rt[¹]{shiᴸ}&-μμH}
					{\xx{incep}&\xx{areal}&\xx{pfv}&\xx{1pl.s}&\xx{stv}&\rt[¹]{reach.out}&\·\xx{var}}
			\versus \vbform{ḵoowashee}{pfv}[subj intr, \fm{n}, \fm{-μμH} act]{she/he/it searched}
			\parencite[546]{leer:1976}
				\vbmorph{ḵu-&μʷ-&wa-&\rt[¹]{shiᴸ}&-μμH}
					{\xx{areal}&\xx{pfv}&\xx{stv}&\rt[¹]{reach.out}&\·\xx{var}}
		\end{itemize}
	\end{enumerate}

\item[g̱unéi=]\label{m:g̱unéi=}
	variant form of inceptive manner preverb \X{g̱unayéi=} ‘starting, beginning’
		indicating the initiating of motion or the beginning of an eventuality;
	this form arises by contraction of the final two syllables of \fm{g̱unayéi}
		[\ipa{qù.nà.ˈjéː}] into \fm{g̱unéi} [\ipa{qù.ˈnéː}] by eliding the
		the vowel \fm{a} [\ipa{à}] and the approximant \fm{y} [\ipa{j}];
	see the \X{g̱unayéi=} entry for more details on meaning and etymology;
	for some speakers the \fm{g̱unéi=} form is in free variation with \fm{g̱unayéi=},
		with the choice of either presumably depending on phonological phrasing
		and other pronunciation factors;
	some speakers however use only \fm{g̱unéi=} with verbs and \fm{g̱unayéi} only as a noun,
		probably because the noun meaning of \fm{g̱unayéi} ‘elsewhere, other place’
		is no longer obviously connected to the inceptive preverb meaning of
		‘starting, beginning’;
	the pattern of only \fm{g̱unéi=} for verbs is attested at least in Tongass Tlingit and in
		Chilkat Northern Tlingit but it is likely to occur elsewhere
	\begin{enumerate}
	\item	initiation of motion in the motion derivation
			\motderiv{g̱unayéi \~\ g̱unéi}{∅, \fm{-x̱} rep}{starting off, setting out, beginning}
		\begin{itemize}
		\item	\vbform{g̱unéi s uwaḵúx̱}{pfv}[subj intr, \fm{∅}, mot, \fm{-x̱} rep]{they started boating}
			\parencite[86.70]{dauenhauer-dauenhauer:1987}
				\vbmorph{\gm{g̱unéi}=&s=&u-&wa-&\rt[¹]{ḵux̱}&-μH}
					{\xx{incep}&\xx{plh}&\xx{zpfv}&\xx{stv}&\rt[¹]{go.boat}&\·\xx{var}}
			\versus \vbform{g̱unayéi has uḵoox̱ch}{hab}{they always started boating}
			\parencite[96.306]{dauenhauer-dauenhauer:1987}
				\vbmorph{g̱unayéi=&has=&u-&\rt[¹]{ḵux̱}&-μμL&-ch}
					{\xx{incep}&\xx{plh}&\xx{zpfv}&\rt[¹]{go.boat}&\·\xx{var}&\·\xx{rep}}
			\newline
			these two examples are from the same speaker, showing variation between
				\fm{g̱unéi=} and \fm{g̱unayéi=}
		\end{itemize}
	\item	initiation of any eventuality in the eventuality derivation
			\motderiv{g̱unayéi \~\ g̱unéi}{∅, \fm{-x̱} rep}{beginning, starting, initiating}
		\begin{itemize}
		\item	\vbform{g̱unéi s wulisʼís}{pfv}[obj intr, \fm{∅}, ach]{they began to be windblown}
			\parencite[90.157]{dauenhauer-dauenhauer:1987}
				\vbmorph{\gm{g̱unéi}=&s=&wu-&lˢ-&i-&\rt[¹]{sʼiᴴs}&-μH}
					{\xx{incep}&\xx{plh}&\xx{pfv}&\xx{xtn}&\xx{stv}&\rt[¹]{windblown}&\·\xx{var}}
			\versus \vbform{wulisʼées}{pfv}[obj intr, \fm{n}, ach]{she/he/it was blown by wind}
				\vbmorph{wu-&lˢ-&i-&\rt[¹]{sʼiᴴs}&-μμH}
					{\xx{pfv}&\xx{xtn}&\xx{stv}&\rt[¹]{windblown}&\·\xx{var}}
		\end{itemize}
	\end{enumerate}

\item[g̱unyéi=]\label{m:g̱unyéi=}
	variant form of inceptive manner preverb \X{g̱unayéi=} ‘starting, beginning’
		indicating the initiating of motion or the beginning of an eventuality;
	this form arises by contraction of the final two syllables of \fm{g̱unayéi}
		[\ipa{qù.nà.ˈjéː}] into \fm{g̱unéi} [\ipa{qùn.ˈjéː}] by eliding the
		the vowel \fm{a} [\ipa{à}];
	this form of \X{g̱unayéi=} is only attested in transcription from one source
		but it likely occurs among other speakers and needs more documentation
	\begin{enumerate}
	\item	initiation of motion in the motion derivation
			\motderiv{g̱unayéi \~\ g̱unéi}{∅, \fm{-x̱} rep}{starting off, setting out, beginning}
		\begin{itemize}
		\item	\vbform{g̱unyéi uwagút}{pfv}[subj intr, \fm{∅}, mot, \fm{-x̱} rep]{she/he/it started going}
			\parencite[142.21]{naish:1966}
				\vbmorph{\gm{g̱unyéi}=&u-&wa-&\rt[¹]{gut}&-μH}
					{\xx{incep}&\xx{zpfv}&\xx{stv}&\rt[¹]{go.\xx{sg}}&\·\xx{var}}
		\end{itemize}
	\item	initiation of any eventuality in the eventuality derivation
			\motderiv{g̱unayéi \~\ g̱unéi}{∅, \fm{-x̱} rep}{beginning, starting, initiating}
		\begin{itemize}
		\item	\vbform{g̱unyéi x̱waayáa}{pfv}[tr, \fm{∅}, ach]{I started packing it}
			\parencite[140.10]{naish:1966}
				\vbmorph{\gm{g̱unyéi}=&ʷ-&x̱a-&μ-&\rt[¹]{ya}&-μμH}
					{\xx{incep}&\xx{pfv}&\xx{1sg}&\xx{stv}&\rt[¹]{backpack}&\·\xx{var}}
			\versus \vbform{ayáa}{impfv}[tr, \fm{g}, \fm{-μμH} act]{she/he/it is packing him/her/it}
				\vbmorph{a-&\rt[²]{ya}&-μμH}
					{\xx{3>3}&\rt[²]{backpack}&\·\xx{var}}
		\end{itemize}
	\end{enumerate}	
\end{morphdesc}
