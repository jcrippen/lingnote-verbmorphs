%!TEX root = ../lingnote-verbmorphs.tex

\subsection{G̱}\label{sec:alphalist-gh}
\begin{morphdesc}[resume*=alphalist]
\item[g̱-]\label{m:g̱-conj}
	\fm{g̱} conjugation prefix, downward spatial orientation;
	can occur together with modal \fm{g̱-}

\item[g̱a-]\label{m:g̱a-conj}
	allomorph of \fm{g̱} conjugation prefix \X[g̱-conj]{g̱-}

\item[g̱a-]\label{m:g̱a-mod}
	allomorph of modal \X[g̱-mod]{g̱-} with epenthetic (filler) vowel \fm{a}

\item[g̱-]\label{m:g̱-mod}
	modal prefix in prospective aspect,
		hortative mood,
		potential mood,
		and contingent mood;
	can occur together with \fm{g̱-} conjugation class prefix in hortative, potential, contingent;
	\newline
	allomorphs:
	\begin{allolist}
	\item[{\X[g̱a-mod]{g̱a-}}]
			form with epenthetic (filler) vowel \fm{a}
	\item[{\X[x̱-g̱cnj]{x̱-}}]
			form occurring as consonant in coda of a syllable
	\end{allolist}
	combinations:
	\begin{allolist}
	\item[ḵa]	≡ \fm{g̱-x̱a-} with first person singular subject \fm{x̱a-}
	\end{allolist}
	\begin{enumerate}
	\item	prospective aspect:
			conjugation \fm{g-}
			+ irrealis \fm{u-}
			+ modal \fm{g̱-}
		\begin{itemize}
		\item	\vbform{gug̱atáa}{prosp}[subj intr, \fm{n}, \fm{-μH} act]{she/he/it will sleep}
			\vbmorph{g-&u-&\gm{g̱a-}&\rt[¹]{taᴸ}&-μμH}
				{\xx{gcnj}&\xx{irr}&\xx{mod}&\rt[¹]{sleep.\xx{sg}}&\·\xx{var}}
			\versus \vbform{wootaa}{pfv}{she/he/it slept}
			\vbmorph{wu-&μ-&\rt[¹]{taᴸ}&-μμL}
				{\xx{pfv}&\xx{stv}&\rt[¹]{sleep.\xx{sg}}&\·\xx{var}}
		\end{itemize}
	\item	hortative mood:
			conjugation \fm{n-}/\fm{g̱-}/\fm{g-}
			+  modal \fm{g̱-}
		\begin{itemize}
		\item	\vbform{nag̱ataa}{hort}[subj intr, \fm{n}, \fm{-μH} act]{let him/her/it sleep}
			\vbmorph{na-&\gm{g̱a-}&\rt[¹]{taᴸ}&-μμL}
				{\xx{ncnj}&\xx{mod}&\rt[¹]{sleep.\xx{sg}}&\·\xx{var}}
		\end{itemize}
	\item	potential mood:
			irrealis \fm{u-}
			+ conjugation \fm{n-}/\fm{g̱-}/\fm{g-}
			+ modal \fm{g̱-}
	\item	contingent mood:
			conjugation \fm{n-}/\fm{g̱-}/\fm{g-}
			+ modal \fm{g̱-}
			(+ \fm{-n} + \fm{-ín})
	\end{enumerate}

\item[g̱ax̱=]\label{m:g̱ax̱=}
	Incorporated noun ‘crying’
		derived from the root \fm{\rt[¹]{g̱ax̱}} ‘cry’
		and the (alienable) noun \fm{g̱aax̱} ‘crying’.
	Only known from one verb (see below) which is used as a plural subject counterpart
		to the verb based on \fm{\rt[¹]{g̱ax̱}} ‘cry’ that is only used for
		singular subjects.
	This verb seems to be based on a single argument root \fm{\rt{ti}}
		with an object added by applicative \fm{s-}
		and then that object saturated by the incorporated \fm{g̱ax̱=}
		leaving the subject to refer to the entities that cry.
	The root \fm{\rt{ti}} in this verb is not clearly identified, but may be the same as in
		\vbform{du yaadé yaḵatí}{pfv}[subj intr, \fm{∅}, \fm{-μH} act]{she/he curses, scolds his/her face}
		with incorporated noun \X{ÿaḵa-} from \fm{ÿaḵá} ‘speech, saying’ (\fm{\rt{ḵa}} ‘say’)
	It is somehow connected to the larger network of verbs based on the roots
		\fm{\rt{ti}} ‘handle’, \fm{\rt{tiᴸ}} ‘be, exist’, and \fm{\rt{ti}} ‘imitate’
		\parencite[383–396]{leer:1976} which need further explanation.
	The analyses below give the root as \fm{\rt[¹]{tiᴸ}} ‘be, exist’
		following \cite[387]{leer:1976} 
		(also implied by \cite[24]{leer:1978b} “(n, A3) make oral noise”
			listed under “-ti*ˋ (n, I3b) be”)
		thus meaning something like ‘exist as/through crying’,
		but it is unclear why this should apply only to plural arguments
		and why the verb is an activity rather than a state.
	\begin{itemize}
	\item	\vbform{gax̱satí}{impfv}[subj intr, \fm{g}, \fm{-μH} act]{they cry}
			\vbmorph{\gm{g̱ax̱=}&sa-&\rt[¹]{tiᴸ}&-μH}
				{cry&\xx{appl}&\rt[¹]{be}&\·\xx{var}}
		\versus \vbform{g̱áax̱}{impfv}[subj intr, \fm{g}, \fm{-μμH} act]{she/he/it cries}
			\vbmorph{\rt[¹]{g̱ax̱}&-μμH}
				{\rt[¹]{cry}&\·\xx{var}}
			\andnot{	\vbform[*]{has g̱áax̱}{impfv}{they cry}
				using \fm{has=} for plural}
		\versus \fm{g̱aax̱} (noun) ‘crying’
			\vbmorph{\rt{g̱ax̱}&-μμL}
				{\rt{cry}&\·\xx{var}}
	\item	\vbform{kei g̱ax̱ gax̱yisatée}{prosp}{you pl.\ will cry}
		\parencite[60.683]{story-naish:1973}
			\vbmorph{kei=&\gm{gax̱=}&ga-&ʷ-&x̱-&ÿi-&sa-&\rt[¹]{tiᴸ}&-μμH}
				{up=&cry&\xx{gcnj}&\xx{irr}&\xx{mod}&\xx{2pl.s}&\xx{appl}&\rt[¹]{be}&\·\xx{var}}
		\newline
		note the difference between \fm{g̱ax̱} [\ipa{qàχ}] and \fm{gax̱} [\ipa{kàχ}]
	\item	\vbform{g̱ax̱wusitee}{pfv}{they cried, started crying}
		\parencite[387]{leer:1976}
			\vbmorph{\gm{g̱ax̱=}&wu-&s-&i-&\rt[¹]{tiᴸ}&-μμL}
				{cry&\xx{pfv}&\xx{appl}&\xx{stv}&\rt[¹]{be}&\·\xx{var}}
	\item	\vbform{g̱ax̱sitíkwt}{rep impfv}{they cry easily, tend to cry}
		\parencite[387]{leer:1976}
			\vbmorph{\gm{g̱ax̱=}&s-&i-&\rt[¹]{tiᴸ}&-μH&-kw&-t}
				{cry&\xx{appl}&\xx{stv}&\rt[¹]{be}&\·\xx{var}&\·\xx{rep}&\·\xx{pnct}}
	\item	\vbform{has g̱ax̱satée nooch}{hab impfv}{they would weep}
		\parencite[128.102]{dauenhauer-dauenhauer:1987}
			\vbmorph{has=&\gm{g̱ax̱=}&sa-&\rt[¹]{tiᴸ}&-μμH&=nooch}
				{\xx{plh}&cry&\xx{appl}&\rt[¹]{be}&\·\xx{var}&\·\xx{hab.aux}}
	\end{itemize}

\item[g̱unayéi=]\label{m:g̱unayéi=}
	Inceptive manner preverb ‘starting, beginning’ indicating the initiation of motion or the
		beginning of an eventuality.
	Derived from the noun \fm{g̱unayéi} ‘elsewhere, different place’ which is itself
		from the modifier \fm{g̱una} ‘different, other’
		and the noun \fm{yé} \~\ \fm{yéi} ‘place, way’ (see \X{yéi=}).
	The original meaning with motion verbs seems to have been ‘going somewhere else’
		but it has instead come to mean ‘starting off, beginning’.
	The motion derivation used with motion verbs has been extended to apply to
		other verbs that do not normally denote motion such as activities and states,
		describing more generally the initiation of any eventuality rather than just
		an event involving change of location.
	Some speakers only use the \X{g̱unéi=} form as a preverb and reserve \fm{g̱unayéi} only as
		a noun or postposition phrase because the inceptive ‘starting’ meaning no longer
		has an obvious connection in meaning to ‘elsewhere, different place’.
	Can be glossed \xx{incep} for ‘inceptive’ or more transparently as ‘start’ or ‘begin’
		although these glosses can be misleading since they are verbs in English.
	\newline
	allomorphs:
	\begin{allolist}
	\item[\X{g̱unéi=}]	variant form used by some speakers
	\item[\X{g̱unyéi=}]	variant form used by some speakers
	\end{allolist}
	\begin{enumerate}
	\item	Initiation of motion in the motion derivation
			\motderiv{g̱unayéi \~\ g̱unéi}{∅, \fm{-x̱} rep}{starting off, setting out, beginning}.
		\begin{itemize}
		\item	\vbform{g̱unayéi wtuwa.át}{pfv}[subj intr, \fm{∅}, mot, \fm{-x̱} rep]{we started going}
				\vbmorph{\gm{g̱unayéi}=&w-&tu-&wa-&\rt[¹]{.at}&-μH}
					{\xx{incep}&\xx{pfv}&\xx{1pl.s}&\rt[¹]{go.\xx{pl}}&\·\xx{var}}
			\versus \vbform{wutuwa.aat}{pfv}[subj intr, \fm{n}, \fm{yoo=i-…-k} rep]{we went}
				\vbmorph{wu-&tu-&wa-&\rt[¹]{.at}&-μμL}
					{\xx{pfv}&\xx{1pl.s}&\rt[¹]{go.\xx{pl}}&\·\xx{var}}
		\item	\vbform{g̱unayéi ushx̱ʼéelʼch}{hab}[obj intr, \fm{∅}, mot, \fm{-x̱} rep]{it always starts sliding}
			\parencite[176.182]{nyman-leer:1993}
				\vbmorph{\gm{g̱unayéi}=&u-&sh-&\rt[¹]{x̱ʼilʼ}&-μμH&-ch}
					{\xx{incep}&\xx{zpfv}&\xx{pej}&\rt[¹]{slide}&\·\xx{var}&\·\xx{rep}}
			\versus \vbform{át x̱at nashx̱ʼílʼch}{hab}[obj intr, \fm{n}, mot, \fm{yoo=…i-…-k} rep]{I always slide around there}
			\parencite[136041]{eggleston:2017}
				\vbmorph{á&-t&x̱at=&na-&sh-&\rt[¹]{x̱ʼilʼ}&-μH&-ch}
					{\xx{3n}&\·\xx{pnct}&\xx{1sg.o}&\xx{ncnj}&\xx{pej}&\rt[¹]{slide}&\·\xx{var}&\·\xx{rep}}
		\end{itemize}
		Contrast this usage of \fm{g̱unayéi} as part of a motion derivation with its use as an
			ordinary noun in a postposition phrase where it does not indicate initiation
			but instead refers to ‘elsewhere, different place’, in which case the
			conjugation class and/or repetitive imperfective form of the verb may be
			different from the motion derivation.
		\begin{itemize}
		\item	\fm{Aag̱áa chʼa \gm{g̱unayéi}de áwé s woo.aat.}
			“Then they moved to a different place.”
			\parencite[172.131]{dauenhauer-dauenhauer:1987}
		\item	\fm{Sʼíxʼ dei \gm{g̱unayéi}xʼ kax̱waa.étsʼ.}
			“I’ve already moved the dish carefully to another place.”
			\parencite[137.1848]{story-naish:1973}
		\end{itemize}
		Also contrast with the use of \fm{g̱unayéi} as a noun modifier in which case it precedes
			another noun and so does not occur immediately before the verb or other preverbs.
		\begin{itemize}
		\item	\fm{\gm{G̱unayéi} ḵwáan haa woogaaḵ.} “Folk from other localities are visiting us.”
			\parencite[240.3399]{story-naish:1973}
		\end{itemize}
	\item	Initiation of any eventuality in the eventuality derivation
			\motderiv{g̱unayéi \~\ g̱unéi}{∅, \fm{-x̱} rep}{beginning, starting, initiating}.
		Application of this derivation shifts the eventuality class of the verb to an 
			achievement thus prohibiting a non-repetitive imperfective aspect form
			(like a motion verb).
		Also note the change of conjugation class and the addition of the \fm{-x̱} repetitive
			imperfective.
		\begin{itemize}
		\item	\vbform{g̱unayéi aawax̱áa}{pfv}[tr, \fm{∅}, ach]{she/he/it started eating him/her/it}
				\vbmorph{\gm{g̱unayéi}=&a-&μʷ-&wa-&\rt[²]{x̱a}&-μμH}
					{\xx{incep}&\xx{3>3}&\xx{pfv}&\xx{stv}&\rt[²]{eat}&\·\xx{var}}
			\versus	\vbform{aawax̱áa}{pfv}[tr, \fm{∅}, \fm{-μH} act]{she/he/it ate him/her/it}
				\vbmorph{a-&μʷ-&wa-&\rt[²]{x̱a}&-μμH}
					{\xx{3>3}&\xx{pfv}&\xx{stv}&\rt[²]{eat}&\·\xx{var}}
		\item	\vbform{g̱unayéi ḵuwtuwashée}{pfv}[subj intr, \fm{∅}, ach]{we began searching}
			\parencite[68.132]{dauenhauer-dauenhauer:1987}
				\vbmorph{\gm{g̱unayéi}=&ḵu-&w-&tu-&wa-&\rt[¹]{shiᴸ}&-μμH}
					{\xx{incep}&\xx{areal}&\xx{pfv}&\xx{1pl.s}&\xx{stv}&\rt[¹]{reach.out}&\·\xx{var}}
			\versus \vbform{ḵoowashee}{pfv}[subj intr, \fm{n}, \fm{-μμH} act]{she/he/it searched}
			\parencite[546]{leer:1976}
				\vbmorph{ḵu-&μʷ-&wa-&\rt[¹]{shiᴸ}&-μμH}
					{\xx{areal}&\xx{pfv}&\xx{stv}&\rt[¹]{reach.out}&\·\xx{var}}
		\end{itemize}
	\end{enumerate}

\item[g̱unéi=]\label{m:g̱unéi=}
	Variant form of inceptive manner preverb \X{g̱unayéi=} ‘starting, beginning’
		indicating the initiating of motion or the beginning of an eventuality.
	This form arises by contraction of the final two syllables of \fm{g̱unayéi}
		[\ipa{qù.nà.ˈjéː}] into \fm{g̱unéi} [\ipa{qù.ˈnéː}] by eliding the
		the vowel \fm{a} [\ipa{à}] and the approximant \fm{y} [\ipa{j}].
	See the \X{g̱unayéi=} entry for details on meaning and etymology.
	For some speakers the \fm{g̱unéi=} form is in free variation with \fm{g̱unayéi=},
		with the choice of either presumably depending on phonological phrasing
		and other pronunciation factors.
	Some speakers however use only \fm{g̱unéi=} with verbs and \fm{g̱unayéi} only as a noun,
		probably because the noun meaning of \fm{g̱unayéi} ‘elsewhere, other place’
		is no longer obviously connected to the inceptive preverb meaning of
		‘starting, beginning’.
	The pattern of only \fm{g̱unéi=} for verbs is attested at least in Tongass Tlingit and in
		Chilkat Northern Tlingit but it is likely to occur elsewhere.
	\begin{enumerate}
	\item	Initiation of motion in the motion derivation
			\motderiv{g̱unayéi \~\ g̱unéi}{∅, \fm{-x̱} rep}{starting off, setting out, beginning}.
		\begin{itemize}
		\item	\vbform{g̱unéi s uwaḵúx̱}{pfv}[subj intr, \fm{∅}, mot, \fm{-x̱} rep]{they started boating}
			\parencite[86.70]{dauenhauer-dauenhauer:1987}
				\vbmorph{\gm{g̱unéi}=&s=&u-&wa-&\rt[¹]{ḵux̱}&-μH}
					{\xx{incep}&\xx{plh}&\xx{zpfv}&\xx{stv}&\rt[¹]{go.boat}&\·\xx{var}}
			\versus \vbform{g̱unayéi has uḵoox̱ch}{hab}{they always started boating}
			\parencite[96.306]{dauenhauer-dauenhauer:1987}
				\vbmorph{g̱unayéi=&has=&u-&\rt[¹]{ḵux̱}&-μμL&-ch}
					{\xx{incep}&\xx{plh}&\xx{zpfv}&\rt[¹]{go.boat}&\·\xx{var}&\·\xx{rep}}
			\newline
			These two examples are from the same speaker, showing variation between
				\fm{g̱unéi=} and \fm{g̱unayéi=}.
		\end{itemize}
	\item	Initiation of any eventuality in the eventuality derivation
			\motderiv{g̱unayéi \~\ g̱unéi}{∅, \fm{-x̱} rep}{beginning, starting, initiating}.
		\begin{itemize}
		\item	\vbform{g̱unéi s wulisʼís}{pfv}[obj intr, \fm{∅}, ach]{they began to be windblown}
			\parencite[90.157]{dauenhauer-dauenhauer:1987}
				\vbmorph{\gm{g̱unéi}=&s=&wu-&lˢ-&i-&\rt[¹]{sʼiᴴs}&-μH}
					{\xx{incep}&\xx{plh}&\xx{pfv}&\xx{xtn}&\xx{stv}&\rt[¹]{windblown}&\·\xx{var}}
			\versus \vbform{wulisʼées}{pfv}[obj intr, \fm{n}, ach]{she/he/it was blown by wind}
				\vbmorph{wu-&lˢ-&i-&\rt[¹]{sʼiᴴs}&-μμH}
					{\xx{pfv}&\xx{xtn}&\xx{stv}&\rt[¹]{windblown}&\·\xx{var}}
		\end{itemize}
	\end{enumerate}

\item[g̱unyéi=]\label{m:g̱unyéi=}
	Variant form of inceptive manner preverb \X{g̱unayéi=} ‘starting, beginning’
		indicating the initiating of motion or the beginning of an eventuality.
	This form arises by contraction of the final two syllables of \fm{g̱unayéi}
		[\ipa{qù.nà.ˈjéː}] into \fm{g̱unyéi} [\ipa{qùn.ˈjéː}] by eliding the
		the vowel \fm{a} [\ipa{à}].
	This form of \X{g̱unayéi=} is only attested in transcription from one source
		but it likely occurs among other speakers and needs more documentation.
	\begin{enumerate}
	\item	Initiation of motion in the motion derivation
			\motderiv{g̱unayéi \~\ g̱unéi}{∅, \fm{-x̱} rep}{starting off, setting out, beginning}.
		\begin{itemize}
		\item	\vbform{g̱unyéi uwagút}{pfv}[subj intr, \fm{∅}, mot, \fm{-x̱} rep]{she/he/it started going}
			\parencite[142.21]{naish:1966}
				\vbmorph{\gm{g̱unyéi}=&u-&wa-&\rt[¹]{gut}&-μH}
					{\xx{incep}&\xx{zpfv}&\xx{stv}&\rt[¹]{go.\xx{sg}}&\·\xx{var}}
		\end{itemize}
	\item	Initiation of any eventuality in the eventuality derivation
			\motderiv{g̱unayéi \~\ g̱unéi}{∅, \fm{-x̱} rep}{beginning, starting, initiating}.
		\begin{itemize}
		\item	\vbform{g̱unyéi x̱waayáa}{pfv}[tr, \fm{∅}, ach]{I started packing it}
			\parencite[140.10]{naish:1966}
				\vbmorph{\gm{g̱unyéi}=&ʷ-&x̱a-&μ-&\rt[¹]{ya}&-μμH}
					{\xx{incep}&\xx{pfv}&\xx{1sg}&\xx{stv}&\rt[¹]{backpack}&\·\xx{var}}
			\versus \vbform{ayáa}{impfv}[tr, \fm{g}, \fm{-μμH} act]{she/he/it is packing him/her/it}
				\vbmorph{a-&\rt[²]{ya}&-μμH}
					{\xx{3>3}&\rt[²]{backpack}&\·\xx{var}}
		\end{itemize}
	\end{enumerate}	
\end{morphdesc}
