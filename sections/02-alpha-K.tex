%!TEX root = ../lingnote-verbmorphs.tex

\subsection{K}\label{sec:alphalist-k}
\begin{morphdesc}[resume*=alphalist]
\item[ka-]\label{m:ka-hsfc}
	incorporated noun ‘horizontal surface’,
	derived from relational noun \fm{ká} ‘horizontal surface, flat top of’;
	can occur together with one of
		qualifier \fm{ka-} ‘small and round’
		or qualifier \fm{ka-} of unknown meaning
		or comparative \fm{ka-}

\item[ka-]\label{m:ka-sro}
	qualifier ‘small and round’;
	can occur together with one of
		incorporated noun \fm{ka-} ‘horizontal surface’
		or qualifier \fm{ka-} of unknown meaning
		or comparative \fm{ka-}

\item[ka-]\label{m:ka-qual}
	qualifier of unknown meaning;
	can occur together with one of
		qualifier \fm{ka-} ‘small and round’ 
		or incorporated noun \fm{ka-} ‘horizontal surface’
		or comparative \fm{ka-}

\item[ka-]\label{m:ka-cmpv}
	comparative prefix, used along with irrealis \fm{u-} \~\ \fm{oo-} \~\ \fm{w-};
	required in comparative forms of state verbs that denote dimensions (e.g.\ short, heavy);
	can occur together with one of
		qualifier \fm{ka-} ‘small and round’
		or incorporated noun \fm{ka-} ‘horizontal surface’
		or qualifier \fm{ka-} of unknown meaning
	\begin{itemize}
	\item	\vbform{ḵúdáx̱ koodáal}{impfv}[obj intr, \fm{n}, \fm{-μμH} cmpv state]{she/he/it is too heavy}
			\vbmorph{ḵúdáx̱&k-&u-&μ-&\rt[¹]{dal}&-μμH}
				{too.much&\xx{cmpv}&\xx{irr}&\xx{stv}&\rt[¹]{heavy}&\·\xx{var}}
		\versus \vbform{yadál}{impfv}[obj intr, \fm{n}, \fm{-μH} state]{she/he/it is heavy}
			\vbmorph{ÿa-&\rt[¹]{dal}&-μH}
				{\xx{stv}&\rt[¹]{heavy}&\·\xx{var}}
	\end{itemize}

\item[-k]\label{m:-k}
	repetitive suffix;
	allomorph \fm{-kw} with labialization

\item[-kʼ]\label{m:-kʼ}
	diminutive suffix;
	allomorph \fm{-kʼw} with labialization

\item[kaawa]
	≡ \fm{ka-μʷ-wa-}
	combination of any prefix of the form \fm{ka-},
		perfective \fm{μʷ-},
		and stative \fm{wa-}
	\begin{itemize}
	\item	\vbform{kaawagaan}{pfv}[obj intr, \fm{g̱}, ach]{she/he/it burned}
			\vbmorph{\gm{ka-}&\gm{μʷ-}&\gm{wa-}&\rt[¹]{gan}&-μμL}
				{\xx{hsfc}&\xx{pfv}&\xx{stv}&\rt[¹]{burn}&\·\xx{var}}
		\versus \vbform{woogaan}{pfv}[obj intr, \fm{g̱}, ach]{she/he/it burned}
			\vbmorph{wu-&μ-&\rt[¹]{gan}&-μμL}
				{\xx{pfv}&\xx{stv}&\rt[¹]{burn}&\·\xx{var}}
	\end{itemize}

\item[keey-]
	inalienable incorporated noun \fm{keey} ‘knee’
	\begin{itemize}
	\item	\vbform{yan x̱at keeyshakawdligásʼ}{pfv}[obj intr, \fm{∅}, mot]{I fell down and skidded on my knees}
		\parencite[193.2689]{story-naish:1973}
			\vbmorph{yan=&x̱at=&\gm{keey-}&sha-&ka-&w-&d-&l-&i-&\rt[¹]{gasʼ}&-μH}
				{ground&\xx{1sg.o}&knee&head&\xx{hsfc}&\xx{pfv}&\xx{mid}&\xx{xtn}&\xx{stv}&\rt[¹]{end·fall}&\·\xx{var}}
	\end{itemize}

\item[keeya]
	≡ \fm{ka-μʷ-i-ya-}
	combination of any prefix of the form \fm{ka-},
		perfective \fm{μʷ-},
		second person singular subject \fm{i-},
		and stative \fm{ÿa-}
	\begin{itemize}
	\item	\vbform{keeyayúk}{pfv}[tr, \fm{∅}, ach]{you sg.\ shook him/her/it}
			\vbmorph{\gm{ka-}&\gm{μʷ-}&\gm{i-}&\gm{ya-}&\rt[²]{yuᴴk}&-μH}
				{\xx{qual}&\xx{pfv}&\xx{2sg.s}&\xx{stv}&\rt[²]{shake}&\·\xx{var}}
		\versus \vbform{akaawayúk}{pfv}[tr, \fm{∅}, ach]{she/he/it shook him/her/it}
			\vbmorph{a-&ka-&μʷ-&wa-&\rt[²]{yuᴴk}&-μH}
				{\xx{3>3}&\xx{qual}&\xx{pfv}&\xx{stv}&\rt[²]{shake}&\·\xx{var}}
	\end{itemize}

\item[kei=]\label{m:kei=}
	direction preverb ‘up’

\item[kḵwa]
	≡ \fm{g-u-g̱-x̱a-}
	combination of conjugation \fm{g-},
		irrealis \fm{u-},
		and modal \fm{g̱a-},
			together indicating prospective (‘future’) aspect,
		along with first person singular subject \fm{x̱-} / \fm{x̱a-};
	this form occurs when there is an
		immediately preceding vowel (incorporated noun, object prefix, preverb, etc.);
	\fm{kuḵa} occurs instead if there is no preceding vowel
	\begin{itemize}
	\item	\fm{yee kḵwax̱áa} (prosp) ‘I will eat you (pl.)’
			with \fm{ÿee=g-u-g̱-x̱a-}\newline
		versus \fm{at kuḵax̱áa} (prosp; \fm{∅}, \fm{-μμH} act) ‘I will eat something’
			with \fm{at=g-u-g̱-x̱a-}
	\end{itemize}

\item[-kt]\label{m:-kt}
	orthographic variant of \X{-kwt} without overt labialization;
	occurs after roots with /\ipa{u}/ where the orthography omits labialization
		because it predictably occurs on consonants after the labial vowel;
	actually a combination of repetitive \X{-kw} and repetitive \X{-t}
		that behaves like a distinct suffix;
	see \X{-kwt} for details
	\begin{itemize}
	\item	\fm{–húkt} ‘characteristically wades’
		from \fm{\rt[¹]{hu}} ‘wade’ in
		\newline
		\vbform{jidihúkt}{impfv}[subj intr?, conj?, state]{she/he/it characteristically wades}
		\parencites[01/173]{leer:1973}[63]{leer:1976}
			\vbmorph{ji-&d-&i-&\rt[¹]{hu}&-μH&-kw&-t}
				{hand&\xx{mid}&\xx{stv}&\rt[¹]{wade}&\·\xx{var}&\·\xx{rep}&\·\xx{ict}}
		\versus \vbform{át jeewdihoo}{pfv}[subj intr?, \fm{n}, mot]{she/he/it waded around there}
		\parencites[240.3409]{story-naish:1973}[01/173]{leer:1973}[63]{leer:1976}
			\vbmorph{á&-t&ji-&μw-&d-&i-&\rt[¹]{hu}&-μμL}
				{\xx{3n}&\·\xx{pnct}&hand&\xx{pfv}&\xx{mid}&\xx{stv}&\rt[¹]{wade}&\·\xx{var}}
	\end{itemize}

\item[kuḵa]
	≡ \fm{g-u-g̱-x̱a-}
	combination of conjugation \fm{g-},
		irrealis \fm{u-} prefix,
		and modal \fm{g̱a-},
			together indicating prospective (‘future’) aspect,
		along with first person singular subject \fm{x̱-} / \fm{x̱a-};
	this form occurs when there is no
		immediately preceding vowel (incorporated noun, object prefix, preverb, etc.);
	\fm{kḵwa} occurs instead if there is a preceding vowel
	\begin{itemize}
	\item	\fm{at kuḵax̱áa} (prosp; \fm{∅}, \fm{-μμH} act) ‘I will eat something’
			with \fm{at=g-u-g̱-x̱a-}\newline
		versus \fm{yee kḵwax̱áa} (prosp) ‘I will eat you (pl.)’
			with \fm{ÿee=g-u-g̱-x̱a-}
	\end{itemize}

\item[kuḵwa]
	variant of \fm{kuḵa}
	
\item[-kw]\label{m:-kw}
	allomorph of repetitive \X{-k} with labialization

\item[-kʼw]\label{m:-kʼw}
	allomorph of diminutive \X{-kʼ} with labialization

\item[-kwt]\label{m:-kwt}
	≡ \fm{-kw-t}
	combination of repetitive \X{-kw}
		and ictive repetitive \X{-t}
		but acts like a distinct suffix
		\parencite[153]{leer:1991};
	occurs only with open syllable CV roots
		in multipositional repetitive state imperfectives
		and in tendency state imperfectives
		(\X{-k} \~\ \X{-kw} is used instead
		with closed syllable CVC roots)
	the compositional meaning of this combination is unclear,
		but both cases are relatively well attested;
	forms with \X{-kt} are based on \X{-kw} and not \X{-k},
		reflecting labialization not indicated in the orthography
		when immediately following a labial vowel /\ipa{u}/,
		so this is orthographic variation and not actual allomorphy
	\newline
	allomorphs:
	\begin{allolist}
	\item[\X{-kt}]	form without overt labialization
	\item[-kwt]	form with labialization
	\end{allolist}
	\begin{enumerate}
	\item	repetitive suffix combination as part of multipositional repetitive
		state imperfectives \parencite[540]{crippen:2019}, indicating
		entities distributed at multiple positions along some path in space
		\begin{itemize}
		\item	\fm{–dákwt} ‘bodies of water lie’
			from \fm{\rt[¹]{daᴸ}} ‘flow’ in
			\newline
			\vbform{áx̱ naadákwt}{impfv}[obj intr, \fm{n}, state]{they (bodies of water) lie here and there along it}
			\parencite[328]{leer:1991}
				\vbmorph{á&-x̱&na-&μ-&\rt[¹]{daᴸ}&-μH&-kw&\gm{-t}}
					{\xx{3n}&\·\xx{pert}&\xx{ncnj}&\xx{stv}&\rt[¹]{flow}&\·\xx{var}&\·\xx{rep}&\·\xx{ict}}
			\versus \vbform{át déin}{pos impfv}{it (body of water) flows, lies there}
			\parencites[05/2]{leer:1973}[313]{leer:1976}
				\vbmorph{á&-t&\rt[¹]{daᴸ}&-μμᵉH&-n}
					{\xx{3n}&\·\xx{pnct}&\rt[¹]{flow}&\·\xx{var}&\·\xx{nsfx}}
			\exand \vbform{héen naadaa}{impfv}[obj impfv, \fm{n}, \fm{-μμL} state]{the river flows}
			\parencites[94.1193]{story-naish:1973}[05/3]{leer:1973}[313]{leer:1976}
				\vbmorph{héen&na-&μ-&\rt[¹]{daᴸ}&-μμL}
					{river&\xx{ncnj}&\xx{stv}&\rt[¹]{flow}&\·\xx{var}}
			\exand \vbform{woodaa}{pfv}{it flowed}
			\parencites[94.1192]{story-naish:1973}[05/2–3]{leer:1973}[313]{leer:1976}
				\vbmorph{wu-&μ-&\rt[¹]{daᴸ}&-μμL}
					{\xx{pfv}&\xx{stv}&\rt[¹]{flow}&\·\xx{var}}
			\newline
			not related to \fm{dákwtasi} ‘fertilizer’ (from \fm{\rt{dakw}} ‘render’)
			or to \fm{nadáakw} ‘table’ (from Chinook Jargon \fm{latáp} from French \fm{la table})
		\end{itemize}
	\item	repetitive suffix combination in tendency state imperfectives,
		indicating a tendency for an event to occur as a characteristic
		property of some entity
		\begin{itemize}
		\item	\fm{–.íkwt} ‘cooks quickly, easily’
			from \fm{\rt[¹]{.i}} ‘cooked’
		\end{itemize}
	\end{enumerate}

\item[kwḵa]
	variant of \fm{kḵwa} used primarily in \cite{story-naish:1973}
\end{morphdesc}
