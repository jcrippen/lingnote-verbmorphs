%!TEX root = ../lingnote-verbmorphs.tex

\subsection{S}\label{sec:alphalist-s}
\begin{morphdesc}[resume*=alphalist]
\item[s-]\label{m:s-}
	valency prefix of classifier
	\begin{enumerate}
	\item	argument addition
		\begin{enumerate}
		\item	lone argument of intransitive
		\item	causative
		\item	applicative
		\end{enumerate}
	\item	spatial extension
		\begin{enumerate}
		\item	extended entity
		\item	extended eventuality
		\end{enumerate}
	\end{enumerate}

\item[…s]\label{m:…s}
	≡ \fm{d-s-}
	combination of voice \X{d-}
		and valency \X{s-},
	appears only as a coda consonant and so requires a preceding vowel
	\begin{itemize}
	\item	\fm{yax̱ sh x̱asnei} (rep impfv; tr, \fm{∅}, ach) ‘I repeatedly dress myself’
			with \fm{d-s-}\newline
		versus \fm{yan sh x̱wadzinéi} (pfv) ‘I dressed myself’
			with \fm{d-s-i-}
	\end{itemize}

\item[s=]
	allomorph of human pluralizer \fm{has=} for third person subject or object

\item[-s]\label{m:-s}
	unknown suffix that occurs only with the root \fm{\rt[²]{ḵe}} \~\ \fm{\rt[²]{ḵi}} ‘pay’;
	this form is documented only in Tongass Tlingit, with the allomorph \X{-ás}
		documented elsewhere;
	the meaning of this suffix is unknown because it is only attested with a single root,
		although it could potentially be identified among CVC roots with coda /\ipa{s}/;
	this suffix is distinct from and apparently unrelated to \X{-sʼ}
	\newline
	allomorphs:
	\begin{allolist}
	\item[-s]	basic form
	\item[\X{-ás}]	with epenthetic (filler) vowel \fm{á}
	\end{allolist}
	\begin{itemize}
	\item	\vbform{awli̥ḵeís}{pfv}[tr, \fm{n}?, ach?]{she/he/it gave him/her/it expecting something in return}
		(Tongass dialect) \parencite[f01/68]{leer:1973}
			\vbmorph{a-&w-&lˢ-&i-&\rt[²]{ḵe}&-μμˀ&\gm{-s}}
				{\xx{3>3}&\xx{pfv}&\xx{xtn}&\xx{stv}&\rt[²]{pay}&\·\xx{var}&\·\xx{unkn}}
		\versus \vbform{aawaḵei}{pfv}[tr, \fm{n}, ach]{she/he/it paid him/her/it}
		(Tongass dialect) \parencite[f01/66]{leer:1973}
			\vbmorph{a-&μʷ-&wa-&\rt[²]{ḵe}&-μμ}
				{\xx{3>3}&\xx{pfv}&\xx{stv}&\rt[²]{pay}&\·\xx{var}}
	\end{itemize}

\item[-sʼ]\label{m:-sʼ}
	repetitive suffix only occurring with specific verbs;
	unlike the \X{-ch}, \X{-k}, and \X{-x̱} repetitive suffixes,
		this suffix is never specified by conjugation class
		or as part of a motion derivation;
	for some verbs \fm{-sʼ} occurs instead of
		the conjugation class-specific repetitive suffix (\fm{-ch}, \fm{-k}, \fm{-x̱})
		but other verbs can use either the conjugation class-specific suffix
		or \fm{-sʼ};
	the relatively large number of verbs attested with this suffix
		suggests that it has a productive meaning
		but the details remain unclear;
	the repetitive suffix \X{-lʼ} is historically related,
		similar to the relationship between \X{s-} and \X{l-};
	there may be a broader historical relationship
		between \X{-lʼ}, \X{-sʼ} \X{-tʼ}, and \X{-xʼ}
		since all of these are ejective
	\newline
	allomorphs:
	\begin{allolist}
	\item[-sʼ]	basic form
	\item[\X{-ásʼ}]	form with epenthetic (filler) vowel \fm{á}
	\end{allolist}
	\begin{enumerate}
	\item	repetitive suffix with a wide variety of roots,
		appearing in a repetitive imperfective form
		or in forms derived from the repetitive imperfective
		(secondary aspectual derivations, Leer’s “epiaspect”)
		\begin{itemize}
		\item	\fm{–ḵéisʼ} from \fm{\rt{ḵa}} ‘stitch, sew’ in
			\newline
			\vbform{aḵéisʼ}{rep impfv}[tr, \fm{∅}, ach]{she/he/it is stitching, sewing him/her/it}
				\vbmorph{a-&\rt[²]{ḵa}&-μμᵉH&\gm{-sʼ}}
					{\xx{3>3}&\rt[²]{stitch}&\·\xx{var}&\·\xx{rep}}
			\versus \vbform{aawaḵáa}{pfv}{she/he/it stitched, sewed him/her/it}
				\vbmorph{a-&μʷ-&wa-&\rt[²]{ḵa}&-μμH}
					{\xx{3>3}&\xx{pfv}&\xx{stv}&\rt[²]{stitch}&\·\xx{var}}
			\newline
			compare
			\begin{inlinelist}
			\item	\fm{kax̱duḵéisʼáḵw} \~\ \fm{katḵéisʼáḵw} ‘quilt’ with \X{-áḵw}
			\item	\fm{ḵéichʼálʼ} ‘seam’ with \X{-chʼálʼ}
			\item	\fm{–ḵéilʼútʼ} ‘lick seam’ with \X{-lʼútʼ}
			\item	\fm{ḵéinaa} ‘awl’ with \X{-n} and \X{-aa}
			\end{inlinelist}
		\item	roots attested with \fm{-sʼ} include
			\parencite[533]{crippen:2019}:
			\begin{inlinelist}
			\item	\fm{\rt{cha}} ‘strain, sift’ (\fm{–chéisʼ})
			\item	\fm{\rt{chuk}} ‘rub soft’
			\item	\fm{\rt{chux}} ‘knead’
			\item	\fm{\rt{geᴴÿ}} ‘pay debt’
			\item	\fm{\rt{gish}} ‘soak’
			\item	\fm{\rt{gwal}} ‘beat’
			\item	\fm{\rt{g̱uk}} ‘squeeze’
			\item	\fm{\rt{hin}} ‘water’
			\item	\fm{\rt{jaᴸ}} ‘advise’ (\fm{–jeisʼ})
			\item	\fm{\rt{kik}} ‘shake out’
			\item	\fm{\rt{kel}} ‘soak’
			\item	\fm{\rt{ḵa}} ‘stitch, sew’ (\fm{–ḵéisʼ})
			\item	\fm{\rt{lʼikw}} \~\ \fm{\rt{lʼuk}} ‘blink’ (in \fm{xeitl lʼíkwsʼi} ‘lightning’)
			\item	\fm{\rt{lʼixw}} \~\ \fm{\rt{lʼux}} ‘close eye’
			\item	\fm{\rt{naᴸ}} ‘damp; oil’ (\fm{–neisʼ})
			\item	\fm{\rt{naḵw}} ‘bait, octopus’
			\item	\fm{\rt{nal}} ‘blow nose, steam’
			\item	\fm{\rt{sʼiḵ}} \~\ \fm{\rt{sʼeḵ}} ‘suck; smoke’
			\item	\fm{\rt{taxʼ}} ‘bite’
			\item	\fm{\rt{tiy}} ‘patch’
			\item	\fm{\rt{tuk}} ‘pop’
			\item	\fm{\rt{tʼaᴸ}} ‘hot’ (\fm{–tʼeisʼ})
			\item	\fm{\rt{tʼak}} ‘dent’
			\item	\fm{\rt{tʼixʼ}} ‘hard’
			\item	\fm{\rt{tsik}} ‘roast’
			\item	\fm{\rt{wu}} ‘send for’ (\fm{–wéisʼ})
			\item	\fm{\rt{xa}} ‘pour’ (\fm{–xéisʼ})
			\item	\fm{\rt{xwach}} ‘scrape’
			\item	\fm{\rt{x̱eḵ}} ‘insomnia, wake early’ (see also \X[-ḵ-dprv]{-ḵ})
			\item	\fm{\rt{x̱ik}} ‘flap wings’
			\item	\fm{\rt{x̱ux̱}} ‘summon; compose song’
			\item	\fm{\rt{x̱ʼeᴴÿ}} ‘encourage’
			\item	\fm{\rt{yiḵ}} ‘mark; pull’
			\item	\fm{\rt{yuk}} ‘shake’
			\end{inlinelist}
		\end{itemize}
	\item	unidentified suffix in some nouns with complex codas;
		/\ipa{sʼ}/ appearing as the second consonant in a complex coda suggests that
		these cases are suffixes, but the root and morphology are unidentified
		\begin{itemize}
		\item	\fm{g̱áḵsʼi} ‘fish roasted whole by tail’
			from unknown \fm{\rt{g̱aḵ}},
			compare \fm{g̱aaḵ} ‘lynx’,
			\fm{\rt{g̱aḵ}} ‘dog/raven makes noise’,
			\fm{\rt{g̱aḵ}} ‘toss quoits, gambling sticks’
		\item	\fm{ḵáx̱sʼi} ‘coho salmon tied to bush’
			from unknown \fm{\rt{ḵax̱}}
		\item	\fm{náksʼ} ‘cold sore on tongue’
			from unknown \fm{\rt{nak}}
		\item	\fm{x̱ikshakahánsʼi} ‘shoulder fringe on jacket’
			from \fm{\rt{han}} ‘cut into strips’
		\item	\fm{x̱ʼwánsʼ} ‘crumbly substance’,
			\fm{attux̱ʼúnsʼi} ‘buckshot’,
			\fm{attux̱ʼúnsʼi náakw} ‘black pepper’,
			and \fm{tux̱ʼwánsʼi} ‘pellets’
			from \fm{\rt{x̱ʼwan}} \~\ \fm{\rt{x̱ʼun}} ‘(wood) rot to powder’,
			(noun \fm{x̱ʼoon} ‘dry rotten wood for tanning’)
		\end{itemize}
	\item	possibly frozen in some CVC roots with coda /\ipa{sʼ}/;
		some of these cases are unlikely because they lack expected ablaut
		(/\ipa{a}, \ipa{u}/ → [\ipa{e}] with \X{-μᵉμL} or \X{-μᵉμH}),
		but they could descend from
		a historical stage before ablaut was active
		\begin{itemize}
		\item	\fm{\rt{dasʼ}} in \fm{kadásʼ} ‘hail’
			possibly related to \fm{\rt{tats}} \~\ \fm{\rt{dats}}
			‘knock berries off branch’
			or perhaps \fm{\rt{daᴴn}} ‘snow fall heavy; dust’
		\item	\fm{\rt{dusʼ}} ‘slosh; tsunami’
			perhaps related to \fm{\rt{dutlʼ}} ‘roll up’,
			\fm{\rt{duchʼ}} ‘tweak, pinch and twist’
		\item	\fm{\rt{dusʼ}} ‘sooty’
			(noun \fm{dúsʼ} ‘soot’)
		\item	\fm{\rt{gusʼ}} ‘cloudy’
			(noun \fm{góosʼ} ‘cloud’)
			perhaps related to \fm{\rt{gu}} ‘poke, stab’ or \fm{\rt{gu}} ‘base’
		\item	\fm{\rt{g̱asʼ}} ‘scratch with nail, claw’
			(noun \fm{daag̱ásʼaa} ‘body scraper‘,
			\fm{shakikalg̱áasʼ} ‘white-crowned sparrow’)
		\item	\fm{\rt{hisʼ}} ‘borrow’
			perhaps related to \fm{\rt{hiᴸ}} \~\ \fm{\rt{heᴸ}} ‘pay shaman’
		\item	\fm{\rt{kasʼ}} ‘algae’
			(noun \fm{káasʼ} ‘algae’)
		\item	\fm{\rt{ḵasʼ}} ‘crack, splinter, split’
			(noun \fm{ḵáasʼ} ‘splinter, stick’)
		\item	\fm{\rt{ḵisʼ}} ‘flood’
			perhaps related to \fm{\rt{ḵi}} ‘plural sit’
		\item	\fm{\rt{lesʼ}} in \fm{hintaklaleisʼí} ‘first-run king salmon’
			probably related to \fm{\rt{la}} ‘melt, thaw’
		\item	\fm{\rt{nesʼ}} in \fm{tlax̱aneisʼ} ‘belted kingfisher’
			(\textit{Megaceryle alcyon})
			perhaps related to \fm{\rt{naᴸ}} ‘damp, oil’
			(noun \fm{neisʼ} ‘oil, liniment’)
		\item	\fm{\rt{nisʼ}} ‘sea urchin’
			(noun \fm{néesʼ} ‘sea urchin’)
		\item	\fm{\rt{sʼasʼ}} ‘sway dance’
			(noun \fm{sʼáasʼ} ‘waltz’, \fm{sʼáasʼ} ‘warbler, goldfinch’)
			perhaps related to \fm{\rt{sʼatʼ}} ‘left (side)’
		\item	\fm{\rt{sʼusʼ}} ‘thread on stick’
			(noun \fm{sʼóosʼ} ‘stick for drying’, \fm{sʼóosʼani} ‘conifer cone’,
				\fm{sʼúsʼ} ‘harlequin duck’)
			probably related to \fm{\rt{sʼu}} ‘thin branch; twist to limber’
		\item	\fm{\rt{tesʼ}} ‘limp, flabby’
			(noun \fm{téisʼ} ‘flab’)
			probably related to \fm{\rt{ta}} ‘fat’ (noun \fm{taaÿ} ‘fat’),
			compare \fm{\rt{tetlʼ}} ‘fat (nonhuman)’,
			\fm{téelʼ} ‘dog salmon’
		\item	\fm{\rt{tisʼ}} ‘stare’
			probably related to \fm{\rt{tin}} ‘see’
		\item	\fm{\rt{wesʼ}} in \fm{wéisʼ} ‘louse’
			perhaps related to \fm{\rt{waᴴn}} ‘maggoty’
			and \fm{woon} ‘maggot’
		\item	\fm{\rt{xisʼ}} in \fm{xéesʼ} ‘louse nit’
			compare \fm{xéen} ‘fly (insect)’ and \fm{wéisʼ} ‘louse’
		\item	\fm{\rt{x̱ʼasʼ}} in \fm{x̱ʼásʼ} ‘jaw, mandible’
			probably related to \fm{\rt{x̱ʼe}} ‘mouth’,
			perhaps via earlier \fm[*]{\rt{x̱ʼa-y}}
			(compare \fm{x̱ʼa-}, \fm{\rt{x̱ʼeᴴÿ}} ‘encourage’)
		\item	\fm{\rt{x̱ʼwasʼ}} ‘shed hair, go bald; cheap’
			(also \fm{shax̱ʼwáasʼ} ‘bald spot’)
			probably related to \fm{\rt{x̱ʼwalʼ}} ‘down, fluff’,
			also compare \fm{x̱ʼwánsʼ} ‘crumbly substance’ above
		\item	\fm{\rt{ÿesʼ}} ‘dark, discoloured’
			(noun \fm{ÿéisʼ} ‘obsidian’, \fm{yétsʼ} ‘black dye’)
			perhaps related to \fm{\rt{ÿe}} ‘strange’,
			\fm{\rt{ÿa}} ‘resemble’,
			or \fm{\rt{wu}} ‘pale’
		\end{itemize}
	\end{enumerate}


\item[sa-]\label{m:sa-val}
	allomorph of valency \X{s-}

\item[sa-]\label{m:sa-voice}
	allomorph of incorporated noun \X{se-} ‘voice’

\item[se-]\label{m:se-}
	incorporated noun indicating voice or vocalization

\item[si]
	≡ \fm{s-i-}
	combination of valency \X{s-}
		and stative \X[i-stv]{i-}
\end{morphdesc}
