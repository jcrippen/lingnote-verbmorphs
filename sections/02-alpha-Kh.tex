%!TEX root = ../lingnote-verbmorphs.tex

\subsection{Ḵ}\label{sec:alphalist-kh}
\begin{morphdesc}[resume*=alphalist]
\item[ḵ, ḵa]
	≡ \fm{g̱-x̱-}
	combination of first person singular subject \fm{x̱-} / \fm{x̱a-} with either one of
		conjugation \fm{g̱-}
		or modal \fm{g̱-}
	\begin{enumerate}
	\item	with conjugation \fm{g̱-}: \fm{g̱-x̱a-} → \fm{ḵa}
	\item	with modal\fm{g̱-}: \fm{g̱-x̱a-} → \fm{ḵa}
	\end{enumerate}

\item[-ḵ]\label{m:-ḵ-optphib}
	optative/prohibitive modality suffix;
	\newline
	allomorphs:
	\begin{allolist}
	\item[{\X[-ḵw-optphib]{-ḵw}}]
		 	with labialization,
	\item[-íḵ \~\ -iḵ] with epenthetic (filler) vowel \fm{í} \~\ \fm{i}
	\item[-úḵ \~\ -uḵ] with epenthetic (filler) vowel \fm{í} \~\ \fm{i} and labialization
	\end{allolist}
	always occurs with a preceding particle (see below) and usually with irrealis \fm{u-},
	in contrast with deprivative \fm{-ḵ} which does not require a particle and irrealis;
	both the optative/prohibitive and the deprivative are related to 
		the Eyak negative \fm{-ɢ}
		and so presumably descend from a Proto-Na-Dene \fm[*]{-ɢ} 
		that was somehow related to negation
		\parencites{leer:2000b}[872, 876]{crippen:2019}
	\begin{enumerate}
	\item	with optative particle \fm{gu.aal} ‘hopefully’
			and usually also \fm{gushí} \~\ \fm{kwshé} ‘maybe’
	\item	with prohibitive particle \fm{líl} ‘don’t’
			or alternatively with negative particle \fm{tléil} ‘not’
	\end{enumerate}

\item[-ḵ]\label{m:-ḵ-dprv}
	deprivative suffix \fm{-ḵ} \~\ \fm{-ḵw} ‘lacking, removed’;
	generally describes a situation where something is deprived, lacking, or removed,
		but also occurs in some cases with unclear meaning;
	phonologically similar to the optative/prohibitive \X[-ḵ-optphib]{-ḵ}
		but the epenthesized allomorph of the deprivative is \X{-áḵw}
		rather than \X{-iḵ} \~\ \X{-íḵ} or \X{-uḵ} \~\ \X{-úḵ} or the like
		for the optative/prohibitive, showing that they are distinct suffixes;
	also, the optative/prohibitive is always accompanied by a particle
		(\fm{gu.aal}, \fm{líl}, etc.)
		but the deprivative does not require a particle;
	see optative/prohibitive \X[-ḵ-optphib]{-ḵ} for more background and discussion
	\newline
	allomorphs:
	\begin{allolist}
	\item[{\X[-ḵw-dprv]{-ḵw}}]
			with labialization
	\item[\X{-áḵw}]	with epenthetic (filler) vowel \fm{á}
	\end{allolist}
	\begin{enumerate}
	\item	lacking, without, removed;
		some cases are adverbs suggesting otherwise undocumented verbs derived from nouns,
			probably statives with \X[ka-qual]{ka-}
			+ \X{s-}/\X{lˢ-}
			+ \X{d-}
			+ \X[i-stv]{i-}
			+ noun
			+ \fm{-ḵ};
		a couple of cases involve unknown or puzzling roots
		\begin{itemize}
		\item	\fm{kalsʼáaxwḵ} (adverb) ‘hatless’ from \fm{sʼáaxw} ‘hat’ (also \fm{\rt{sʼaxw}} ‘stack’)
			\vbmorph{ka-&d-&lˢ-&\rt{sʼaxw}&-μμH&\gm{-ḵ}}
				{\xx{qual}&\xx{mid}&\xx{intr}&\rt{hat}&\·\xx{var}&\·\xx{dprv}}
		\item	\vbform{lishísʼḵ}{impfv}[obj intr, \fm{g}, inv state]{she/he/it is raw} 
			\vbmorph{lˢ-&i-&\rt{shisʼ}&-μH&\gm{-ḵ}}
				{\xx{intr}&\xx{stv}&\rt{raw?}&\·\xx{var}&\·\xx{dprv}}
			\newline
			also noun \fm{shísʼḵ} ‘raw flesh; sapling wood’;
			some dialects have \fm{shásʼḵ} instead;
			possibly related to \fm{\rt{shisʼ}} ‘squeeze out guts’ 
				as noted by \textcite[59]{story:1966}
			or less likely to \fm{\rt{shi}} ‘reach out’ + repetitive \X{-sʼ};
			\textcite[59]{story:1966} erroneously gives “łɩšɩ́sʼg” = \fm[*]{lishísʼk}
		\item	\fm{kaltéelḵ} (adverb) ‘shoeless’ from \fm{téel} ‘shoe’
			\vbmorph{ka-&d-&lˢ-&\rt{til}&-μμH&\gm{-ḵ}}
				{\xx{qual}&\xx{mid}&\xx{intr}&\rt{shoe}&\·\xx{var}&\·\xx{dprv}}
		\item	\fm{kaltsáaxʼḵ} (adverb) ‘mittenless’ from \fm{tsáaxʼ} ‘mitten, glove’
			\vbmorph{ka-&d-&lˢ-&\rt{tsaxʼ}&-μμH&\gm{-ḵ}}
				{\xx{qual}&\xx{mid}&\xx{intr}&\rt{mitten}&\·\xx{var}&\·\xx{dprv}}
		\end{itemize}
	\item	lexically specified with certain roots, originally deprivative
		but now frozen and noncompositional		
		\begin{itemize}
		\item	\vbform{lixʼwásʼḵ}{impfv}[obj intr, \fm{g}, inv state]{she/he/it is numb}
			\parencites[f04/80]{leer:1973}[762]{leer:1976} from unknown \fm{\rt{xʼwasʼ}}
			\vbmorph{lˢ-&i-&\rt[⁰]{xʼwasʼ}&-μH&\gm{-ḵ}}
				{\xx{intr}&\xx{stv}&\rt[⁰]{feeling?}&\·\xx{var}&\·\xx{dprv}}
			\newline
			also part of \fm{hintakxʼwásʼg̱i} ‘bufflehead’ (\textit{Bucephala albeola} L.)
				\parencite[f04/79]{leer:1973}
				\vbmorph{héen&táak&\rt{xʼwasʼ}&-μH&\gm{-ḵ}&-i}
					{water&below&\rt{\xx{unkn}}&\·\xx{var}&\·\xx{unkn}&\·\xx{poss}}
			\newline
			which might literally mean ‘numb below the surface of the water’
				but the reasoning behind this meaning is unclear;
			the noun \fm{xʼúsʼ} \~\ \fm{xʼwásʼ} ‘club’ is not obviously related
		\item	\vbform{diyáshḵ}{impfv}[obj intr, \fm{n}?, inv state]{she/he/it is scarce, rare}
			\parencites[03/167]{leer:1973}[199]{leer:1976} from unknown \fm{\rt{ÿash}}
			\vbmorph{d-&i-&\rt{ÿash}&-μH&\gm{-ḵ}}
				{\xx{mid}&\xx{stv}&\rt{plenty?}&\·\xx{var}&\xx{dprv}}
			\newline
			perhaps related to \fm{Daasʼadiyáash} ‘Dezadeash Lake’
				and/or \fm{chʼiyaash} ‘flat-bottomed canoe (of Yakutat)’
				both with unknown etymology;
			less likely \fm{\rt{ÿatlʼ}} \~\ \fm{\rt{ÿachʼ}} ‘short’
				and hence \fm{\rt{ÿatʼ}} ‘long’;
			even more speculatively \fm{\rt{ÿat}} ‘child’
				and \fm{\rt{wat}} ‘mature’;
			probably no connection to \fm{\rt{ÿaᴴsh}} ‘platform’ (noun \fm{kaÿáash})
				or \fm{\rt{ÿasʼ}} ‘smooth’
		\end{itemize}
	\item	unclear meaning in the suffix combination \X{-ḵxʼ} ≡ \fm{-ḵ} + \X{-xʼ}
			with plural/repetitive \X{-xʼ};
		associated with an increase in degree
			but the composition of meaning is unclear;
		see \X{-ḵxʼ} for more details
	\item	possibly identifiable as the final consonant in a couple of roots,
		originally deprivative but now frozen and noncompositional
		\begin{itemize}
		\item	\fm{\rt{naḵ}} ‘abandoning, leaving behind’ in verbs and as a postposition;
			the hypothetical \fm[*]{\rt{na}} root that would combine with \fm{-ḵ}
				to give \fm{\rt{naḵ}} has not been identified
				and may no longer exist independently
			\begin{enumerate}[label=\alph*.]
			\item	as a verb root \fm{\rt[²]{naḵ}} ‘give up, let go’ in
				\begin{itemize}[label=•]
				\item	\vbform{ajeewanáḵ}{pfv}[tr, \fm{∅}, ach]{she/he/it let him/her/it go}
					\vbmorph{a-&ji-&μʷ-&wa-&\rt[²]{naḵ}&-μH}
						{\xx{3>3}&hand&\xx{pfv}&\xx{stv}&\rt[²]{let.go}&\xx{var}}
				\item	\vbform{ax̱ʼeiwanáḵ}{pfv}[tr, \fm{∅}, ach]{she/he/it gave up it (drinking, smoking)}
					\vbmorph{a-&x̱ʼe-&μʷ-&wa-&\rt[²]{naḵ}&-μH}
						{\xx{3>3}&mouth&\xx{pfv}&\xx{stv}&\rt[²]{let.go}&\xx{var}}
				\item	\vbform{ax̱ʼawsináḵ}{pfv}[tr, \fm{∅}, ach]{she/he/it silenced him/her/it}
					\vbmorph{a-&x̱ʼe-&w-&s-&i-&\rt[²]{naḵ}&-μH}
						{\xx{3>3}&mouth&\xx{pfv}&\xx{csv}&\xx{stv}&\rt[²]{let.go}&\xx{var}}
				\end{itemize}
			\item	as a postposition \fm{náḵ} ‘away from, leaving behind’ in
				\begin{itemize}[label=•]
				\item	\vbform{a náḵ neil oo.aatch}{hab}[subj intr, \fm{∅}, mot]{they would go home from it}
					\parencite[66.64]{dauenhauer-dauenhauer:1987}
					\vbmorph{a&náḵ&neil=&a-&u-&\rt[¹]{.at}&-μμL&-ch}
						{\xx{3n}&away&home&\xx{ind.h.s}&\xx{zpfv}&\rt[¹]{go.\xx{pl}}&\·\xx{var}&\·\xx{rep}}
				\item	\vbform{a waḵnáḵ awlisín}{pfv}[tr, \fm{∅}, ach]{she/he/it hid him/her/it away from his/her/its eyes}
					\parencite[04/77]{leer:1973}
					\vbmorph{a&waaḵ&-náḵ&a-&w-&lˢ-&i-&\rt{sin}&-μH}
						{\xx{3n.psr}&eye&\·away&\xx{3>3}&\xx{pfv}&\xx{csv}&\xx{stv}&\rt{hide}&\·\xx{var}}
				\end{itemize}
			\end{enumerate}
		\item	\fm{sayeiḵ} \~\ \fm{siyeiḵ} ‘next day, day after’
				(also \fm{seig̱ánín} \~\ \fm{seig̱án} ‘tomorrow’
				< \fm[*]{sayeiḵ-án-ín});
			the \fm{sa-} is probably a frozen form of \X{s-}, and the stem \fm{–yeiḵ}
			possibly derives from the noun \fm{ÿee} ‘time’ (see \X[ÿee=time]{ÿee=}) 
			and deprivative \fm{-ḵ}
		\item	\fm{\rt{x̱iḵ}} \~\ \fm{\rt{x̱eḵ}} ‘lack sleep’ in verbs and a noun;
			originally from \fm{\rt{x̱i}} \~\ \fm{\rt{x̱e}} ‘stay overnight’
				and deprivative \fm{-ḵ}
				but reanalyzed as a new root with distinct stem variation,
				compare \fm{\rt{x̱exʼw}} ‘plural sleep’ with plural \X{-xʼw}
			\begin{enumerate}[label=\alph*.]
			\item	as a verb root \fm{\rt{x̱eḵ}} \~\ \fm{\rt{x̱iḵ}} ‘insomnia’ in
				\begin{itemize}[label=•]
				\item	\vbform{wudix̱éḵ}{pfv}[obj intr, \fm{∅}, ach]{she/he/it had insomnia}
					\parencites[f02/23]{leer:1973}[794]{leer:1976}
						\vbmorph{wu-&d-&i-&\rt[¹]{x̱eḵ}&-μH}
							{\xx{pfv}&\xx{mid}&\xx{stv}&\rt[¹]{insomnia}&\·\xx{var}}
				\item	\vbform{awsix̱éḵ}{pfv}[tr, \fm{∅}, ach]{she/he/it woke him/her/it early}
					\vbmorph{a-&w-&s-&i-&\rt{x̱eḵ}&-μH}
						{\xx{3>3}&\xx{pfv}&\xx{csv}&\xx{stv}&\rt[¹]{insomnia}&\·\xx{var}}
				\item	\vbform{ÿax̱éḵsʼ}{impfv}{she/he/it gets up early}
						\vbmorph{ÿa-&\rt[¹]{x̱eḵ}&-μH&-sʼ}
							{\xx{stv}&\rt[¹]{insomnia}&\·\xx{var}&\·\xx{rep}}
				\end{itemize}
			\item	as a noun \fm{x̱eeḵ} \~\ \fm{x̱eiḵ} ‘insomnia; tiredness’
				especially in the phrase
				\vbform{x̱eiḵch uwajáḵ}{pfv}[tr, \fm{∅}, ach]{she/he/it suffered bout of insomnia}
				(literally ‘insomnia killed him/her/it’)
					\vbmorph{x̱eiḵ&-ch&ⱥ-&u-&wa-&\rt[²]{jaḵ}&-μH}
						{insomnia&\·\xx{erg}&\xx{3>3}&\xx{zpfv}&\xx{stv}&\rt[²]{kill}&\·\xx{var}}
			\end{enumerate}
		\end{itemize}
	\end{enumerate}

\item[ḵaa=]
	allomorph of indefinite human object \fm{ḵu-} ‘someone, people, one, them’;
	possibly like \fm{ax̱=} used only as possessor of incorporated noun
		(compare \fm{ḵaa keidlí áwé} ‘it’s someone’s dog’)
	\begin{itemize}
	\item	\fm{ḵaa seiwa.áx̱} (pfv; tr, \fm{∅}, ach) ‘s/he/it heard someone’s voice’
	\end{itemize}

\item[ḵu-]\label{m:ḵu-indef}
	indefinite human object ‘someone, people, one, them’;
	allomorph \fm{ḵaa=} as possessor of incorporated noun;
	compare PP pronoun \fm{ḵú-} ‘someone, people, one, them’
	\begin{itemize}
	\item	\fm{ḵuwsiteen} (pfv; tr, \fm{g̱}, ach) ‘s/he/it saw someone/people’
	\end{itemize}

\item[ḵu-]\label{m:ḵu-areal}
	areal prefix indicating space, area, extent, or weather;
	compare \fm{ḵúx̱(-de)=} ‘back, returning’, \fm{ḵut=} ‘lost’
	\begin{itemize}
	\item	\fm{ḵuwakʼéi} (impfv; impers, \fm{g}, \fm{-μμH} state) ‘it is good weather’\newline
		versus \fm{yakʼéi} (impfv; obj intr) ‘it is good’
	\end{itemize}

\item[ḵut=]\label{m:ḵut=}
	Preverb indicating that eventuality involves becoming lost either literally
		or metaphorically.
	Can be glossed as ‘astray’ or ‘lost’, and may be abbreviated \xx{err}
		for ‘errative’ although this can be confused with \fm{ḵwáaḵ}
		‘wrongly’ and the amissive \X{-x̱aa} ‘miss target’.
	Occurs as part of the motion derivation
		\motderiv{ḵut=}{g}{going astray, getting lost}
		so that verbs are shifted to the \fm{g} conjugation class and
		become achievement-like without a basic imperfective form;
		note that this derivation does not supply a repetitive imperfective form.
	Probably evolved from areal \fm{ḵú} (see \X[ḵu-areal]{ḵu-}) and the punctual postposition
		\fm{-t}, perhaps with an original meaning of ‘to somewhere’ that came to be
		interpreted more specifically as ‘to somewhere unknown’ and thence ‘astray, lost’.
	\begin{enumerate}
	\item	Motion that goes astray, usually with the result that the moving entity
			becomes lost, although may instead mean that the entity is no
			longer in view or no longer exists.
		\begin{itemize}
		\item	\vbform{ḵut woogoot}{pfv}[subj intr, \fm{g}, mot]{she/he/it got lost}
			\parencite[126.565]{nyman-leer:1993}
				\vbmorph{\gm{ḵut=}&wu-&μ-&\rt[¹]{gut}&-μμL}
					{\xx{err}&\xx{pfv}&\xx{stv}&\rt[¹]{go.\xx{sg}}&\·\xx{var}}
		\item	\vbform{góosʼ ḵut kawdlisʼées}{pfv}[obj intr, \fm{g}, mot]{the clouds have blown away}
			\parencite[32.248]{story-naish:1973}
				\vbmorph{góosʼ&\gm{ḵut=}&ka-&w-&d-&lˢ-&i-&\rt[¹]{sʼiᴴs}&-μμH}
					{cloud&\xx{err}&\xx{sro}&\xx{pfv}&\xx{mid}&\xx{xtn}&\xx{stv}&\rt[¹]{windblown}&\·\xx{var}}
			\versus \vbform{káast át wulisʼées}{pfv}[obj intr, \fm{n}, mot]{the barrel was blown around}
			\parencite[32.246]{story-naish:1973}
				\vbmorph{káast&á&-t&wu-&lˢ-&i-&\rt[¹]{sʼiᴴs}&-μμH}
					{barrel&\xx{3n}&\·\xx{pnct}&\xx{pfv}&\xx{xtn}&\xx{stv}&\rt[¹]{windblown}&\·\xx{var}}
		\item	\vbform{ax̱ lítayi ḵut x̱waag̱éexʼ}{pfv}[tr, \fm{g}, mot]{I lost my knife}
			\parencite[129.1719]{story-naish:1973}
				\vbmorph{ax̱&líta&-yi&\gm{ḵut=}&ʷ-&x̱a-&μ-&\rt[²]{g̱ixʼ}&-μμH}
					{\xx{1sg.psr}&knife&\·\xx{poss}&\xx{err}&\xx{pfv}&\xx{1sg.s}&\xx{stv}&\rt[²]{throw}&\·\xx{var}}
			\versus \vbform{aawag̱éexʼ}{pfv}[tr, \fm{n}, mot]{she/he/it threw it}
			\parencite[845]{leer:1976}
				\vbmorph{a-&μʷ-&wa-&\rt[²]{g̱ixʼ}&-μμH}
					{\xx{3>3}&\xx{pfv}&\xx{stv}&\rt[²]{throw}&\·\xx{var}}
		\end{itemize}
	\item	Event where agent becomes metaphorically lost in the situation so that they
			remain engaged perhaps to excess.
		\begin{itemize}
		\item	\vbform{ḵut at x̱waax̱aa}{pfv}[tr, \fm{g}, ach]{I got carried away eating things}
			\parencite[220 \#8c]{leer:1991}
				\vbmorph{\gm{ḵut=}&at=&ʷ-&x̱a-&μ-&\rt[²]{x̱a}&-μμL}
					{\xx{err}&\xx{ind.n.o}&\xx{pfv}&\xx{1sg.s}&\xx{stv}&\rt[²]{eat}&\·\xx{var}}
			\versus \vbform{at x̱waax̱áa}{pfv}[tr, \fm{∅}, \fm{-μH} act]{I ate things}
				\vbmorph{at=&ʷ-&x̱a-&μ-&\rt[²]{x̱a}&-μμH}
					{\xx{ind.n.o}&\xx{pfv}&\xx{1sg.s}&\xx{stv}&\rt[²]{eat}&\·\xx{var}}
		\end{itemize}
	\item	Event of deception where the deceived becomes metaphorically lost due to the trickery
			of the agent.
		\begin{itemize}
		\item	\vbform{ḵut x̱at wuliyeil}{pfv}[tr, \fm{g}, ach]{she/he/it cheated, tricked, deceived me}
			\parencite[46.479]{story-naish:1973}
				\vbmorph{\gm{ḵut=}&x̱at=&wu-&lˢ-&i-&\rt[²]{yel}&-μμL}
					{\xx{err}&\xx{1sg.o}&\xx{pfv}&\xx{xtn}&\xx{stv}&\rt[²]{fake}&\·\xx{var}}
			\versus \vbform{awliyél}{pfv}[tr, \fm{∅}, ach]{she/he/it faked it, pretended to do it}
			\parencite[158.2161]{story-naish:1973}
				\vbmorph{a-&w-&lˢ-&i-&\rt[²]{yel}&-μH}
					{\xx{3>3}&\xx{pfv}&\xx{xtn}&\xx{stv}&\rt[²]{fake}&\·\xx{var}}
		\end{itemize}
	\end{enumerate}

\item[ḵux̱=]\label{m:ḵux̱=}

\item[-ḵw]\label{m:-ḵw-optphib}
	allomorph of optative/prohibitive \X[-ḵ-optphib]{-ḵ} with labialization

\item[-ḵw]\label{m:-ḵw-dprv}
	allomorph of deprivative \X[-ḵ-dprv]{-ḵ} with labialization

\item[-ḵxʼ]\label{m:-ḵxʼ}
	≡ \fm{-ḵ-xʼ}
	combination of deprivative \X[-ḵ-dprv]{-ḵ}
		and plural/repetitive \X{-xʼ};
	associated with an increase in degree of some quality
		but the composition of meaning is unclear;
	only two cases are attested, both reported by \textcite[59]{story:1966},
		which are notably not listed by \textcites{leer:1973}{leer:1976};
	these could plausibly be mishearings of repetitive \X{-k}
		instead of deprivative \fm{-ḵ} but there is no direct evidence
		to support this hypothesis;
	alternatively these could originally have used \fm{-k} but changed
		to \fm{-ḵ} as a kind of place dissimilation since
		\fm{-ḵ} is uvular and \fm{-xʼ} is velar
	\begin{itemize}
	\item	\fm{–tsínḵxʼ} ‘plural very expensive’
		from \fm{\rt{tsin}} ‘alive, strong’ in
		\newline
		\vbform{x̱ʼalitsínḵxʼ}{impfv}[obj intr, \fm{g}, inv state]{they are all very expensive}
		\parencite[59]{story:1966} 
			\vbmorph{x̱ʼe-&lˢ-&i-&\rt[¹]{tsin}&-μH&\gm{-ḵ}&-xʼ}
				{mouth&\xx{xtn}&\xx{stv}&\rt[¹]{strong}&\·\xx{var}&\·\xx{dprv}&\·\xx{pl}}
		\versus \vbform{dax̱ x̱ʼadlitsínxʼ}{impfv}{they are each expensive}
			\parencite[154.1173]{nyman-leer:1993}
			\vbmorph{dax̱=&x̱ʼe-&d-&lˢ-&i-&\rt[¹]{tsin}&-μH&-xʼ}
				{\xx{distb}&mouth&\xx{mid}&\xx{xtn}&\xx{stv}&\rt[¹]{strong}&\·\xx{var}&\·\xx{pl}}
		\versus \vbform{x̱ʼalitseen}{impfv}{she/he/it is expensive}
			\vbmorph{x̱ʼe-&lˢ-&i-&\rt[¹]{tsin}&-μμL}
				{mouth&\xx{xtn}&\xx{stv}&\rt[¹]{strong}&\·\xx{var}}
	\item	\fm{–ÿátʼḵxʼ} ‘plural very long’
		from \fm{\rt{ÿatʼ}} ‘long’ in
		\newline
		\vbform{diyátʼḵxʼ}{impfv}[obj intr, \fm{g}, inv state]{they are very long}
		\parencite[59]{story:1966}
			\vbmorph{d-&i-&\rt[¹]{ÿatʼ}&-μH&\gm{-ḵ}&-xʼ}
				{\xx{mid}&\xx{stv}&\rt[¹]{long}&\·\xx{var}&\·\xx{dprv}&\·\xx{pl}}
		\versus \vbform{diyátʼxʼ}{impfv}{they are long}
			\vbmorph{d-&i-&\rt[¹]{ÿatʼ}&-μH&-xʼ}
				{\xx{mid}&\xx{stv}&\rt[¹]{long}&\·\xx{var}&\·\xx{pl}}
		\versus \vbform{yayátʼ}{impfv}{she/he/it is long}
			\vbmorph{ÿa-&\rt[¹]{ÿatʼ}&-μH}
				{\xx{stv}&\rt[¹]{long}&\·\xx{var}&}
	\end{itemize}

\end{morphdesc}
