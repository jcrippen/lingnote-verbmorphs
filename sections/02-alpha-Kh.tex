%!TEX root = ../lingnote-verbmorphs.tex

\subsection{Ḵ}\label{sec:alphalist-kh}
\begin{morphdesc}[resume*=alphalist]
\item[ḵ, ḵa]
	≡ \fm{g̱-x̱-}
	combination of first person singular subject \fm{x̱-} / \fm{x̱a-} with either one of
		conjugation \fm{g̱-}
		or modal \fm{g̱-}
	\begin{enumerate}
	\item	with conjugation \fm{g̱-}: \fm{g̱-x̱a-} → \fm{ḵa}
	\item	with modal\fm{g̱-}: \fm{g̱-x̱a-} → \fm{ḵa}
	\end{enumerate}

\item[-ḵ]\label{m:-ḵ-optphib}
	optative/prohibitive modality suffix;
	\newline
	allomorphs:
	\begin{allolist}
	\item[{\X[-ḵw-optphib]{-ḵw}}]
		 	with labialization,
	\item[-íḵ \~\ -iḵ] with epenthetic (filler) vowel \fm{í} \~\ \fm{i}
	\item[-úḵ \~\ -uḵ] with epenthetic (filler) vowel \fm{í} \~\ \fm{i} and labialization
	\end{allolist}
	always occurs with a preceding particle (see below) and usually with irrealis \fm{u-},
	in contrast with deprivative \fm{-ḵ} which does not require a particle and irrealis;
	both the optative/prohibitive and the deprivative are related to 
		the Eyak negative \fm{-ɢ}
		and so presumably descend from a Proto-Na-Dene \fm[*]{-ɢ} 
		that was somehow related to negation
		\parencites{leer:2000b}[872, 876]{crippen:2019}
	\begin{enumerate}
	\item	with optative particle \fm{gu.aal} ‘hopefully’
			and usually also \fm{gushí} \~\ \fm{kwshé} ‘maybe’
	\item	with prohibitive particle \fm{líl} ‘don’t’
			or alternatively with negative particle \fm{tléil} ‘not’
	\end{enumerate}

\item[-ḵ]\label{m:-ḵ-dprv}
	Deprivative suffix \fm{-ḵ} \~\ \fm{-ḵw} ‘lacking, removed’,
		generally describing a situation where something is deprived, lacking, removed,
		or otherwise missing from its usual environment,
		but also found in some cases with unclear meaning.
	Phonologically similar to the optative/prohibitive \X[-ḵ-optphib]{-ḵ}
		but the epenthesized allomorph of the deprivative is \X{-áḵw}
		rather than \X{-iḵ} \~\ \X{-íḵ} or \X{-uḵ} \~\ \X{-úḵ} or the like
		for the optative/prohibitive, implying that they are distinct suffixes
		with different phonologies.
	Also, the optative/prohibitive is always accompanied by a particle
		(\fm{gu.aal}, \fm{líl}, etc.)
		but the deprivative does not require a particle,
		thus further implying that they are distinct.
	See the entry for optative/prohibitive \X[-ḵ-optphib]{-ḵ}
		for more background and discussion.
	\newline
	allomorphs:
	\begin{allolist}
	\item[{\X[-ḵw-dprv]{-ḵw}}]
			with labialization
	\item[\X{-áḵw}]	with epenthetic (filler) vowel \fm{á}
	\end{allolist}
	\begin{enumerate}
	\item	In forms meaning ‘lacking’, ‘without’, or ‘removed’.
		Some of these cases are adverbs which suggest otherwise undocumented verbs
			derived from nouns.
		These are probably derived from stative verbs with \X[ka-qual]{ka-}
			+ \X{s-}/\X{lˢ-}
			+ \X{d-}
			+ \X[i-stv]{i-}
			+ noun
			+ \fm{-ḵ}.
		Can usually be translated with English ‘-less’.
		\begin{itemize}
		\item	\fm{kalsʼáaxwḵ} (adverb) ‘hatless’ from \fm{sʼáaxw} ‘hat’
			(verb root \fm{\rt{sʼaxw}} ‘stack’)
			\vbmorph{ka-&d-&lˢ-&\rt{sʼaxw}&-μμH&\gm{-ḵ}}
				{\xx{qual}&\xx{mid}&\xx{intr}&\rt{hat}&\·\xx{var}&\·\xx{dprv}}
		\item	\fm{kaltéelḵ} (adverb) ‘shoeless’ from \fm{téel} ‘shoe’
			\vbmorph{ka-&d-&lˢ-&\rt{til}&-μμH&\gm{-ḵ}}
				{\xx{qual}&\xx{mid}&\xx{intr}&\rt{shoe}&\·\xx{var}&\·\xx{dprv}}
		\item	\fm{kaltsáaxʼḵ} (adverb) ‘mittenless’ from \fm{tsáaxʼ} ‘mitten, glove’
			\vbmorph{ka-&d-&lˢ-&\rt{tsaxʼ}&-μμH&\gm{-ḵ}}
				{\xx{qual}&\xx{mid}&\xx{intr}&\rt{mitten}&\·\xx{var}&\·\xx{dprv}}
		\end{itemize}
	\item	Lexically required with certain roots.
		This probably was originally deprivative but has become fossilized and
			so now has no obvious contribution to meaning.
		\begin{itemize}
		\item	\vbform{lishísʼḵ}{impfv}[obj intr, \fm{g}, inv state]{she/he/it is raw}
			\parencites[10/80–81]{leer:1973}[550]{leer:1976}
			from unknown \fm{\rt{shisʼ}}
			\vbmorph{lˢ-&i-&\rt{shisʼ}&-μH&\gm{-ḵ}}
				{\xx{intr}&\xx{stv}&\rt{raw?}&\·\xx{var}&\·\xx{dprv}}
			\newline
			Also noun \fm{shísʼḵ} ‘raw flesh; sapling wood’;
				some dialects have \fm{shásʼḵ} instead.
			The root is not attested in other contexts so its meaning is unclear.
			Possibly related to \fm{\rt{shisʼ}} ‘squeeze out guts’ 
				as noted by \textcite[59]{story:1966}
				or less likely to \fm{\rt{shi}} ‘reach out’ + repetitive \X{-sʼ}.
			\textcite[59]{story:1966} erroneously gives “łɩšɩ́sʼg” = \fm[*]{lishísʼk}
				which is presumably a mishearing of \fm{lishísʼḵ}.
		\item	\vbform{lixʼwásʼḵ}{impfv}[obj intr, \fm{g}, inv state]{she/he/it is numb}
			\parencites[f04/80]{leer:1973}[762]{leer:1976}
			from unknown \fm{\rt{xʼwasʼ}}
			\vbmorph{lˢ-&i-&\rt[⁰]{xʼwasʼ}&-μH&\gm{-ḵ}}
				{\xx{intr}&\xx{stv}&\rt[⁰]{feeling?}&\·\xx{var}&\·\xx{dprv}}
			\newline
			Also part of \fm{hintakxʼwásʼg̱i} ‘bufflehead’ (\textit{Bucephala albeola} L.)
				\parencite[f04/79]{leer:1973}
				\vbmorph{héen&táak&\rt{xʼwasʼ}&-μH&\gm{-ḵ}&-i}
					{water&below&\rt{\xx{unkn}}&\·\xx{var}&\·\xx{unkn}&\·\xx{poss}}
			\newline
			which might literally mean ‘numb below the surface of the water’
				but the reasoning behind this meaning is unclear.
			The noun \fm{xʼúsʼ} \~\ \fm{xʼwásʼ} ‘club’ is not obviously related.
		\item	\vbform{diyáshḵ}{impfv}[obj intr, \fm{n}?, inv state]{she/he/it is scarce, rare}
			\parencites[03/167]{leer:1973}[199]{leer:1976} from unknown \fm{\rt{ÿash}}
			\vbmorph{d-&i-&\rt{ÿash}&-μH&\gm{-ḵ}}
				{\xx{mid}&\xx{stv}&\rt{plenty?}&\·\xx{var}&\xx{dprv}}
			\newline
			Perhaps related to \fm{Daasʼadiyáash} ‘Dezadeash Lake’
				and/or \fm{chʼiyaash} ‘flat-bottomed canoe (of Yakutat)’,
				both with unclear etymology.
			Less likely related to \fm{\rt{ÿatlʼ}} \~\ \fm{\rt{ÿachʼ}} ‘short’
				and hence \fm{\rt{ÿatʼ}} ‘long’.
			Even more speculatively related to \fm{\rt{ÿat}} ‘child’
				and \fm{\rt{wat}} ‘mature’.
			Probably no connection to \fm{\rt{ÿaᴴsh}} ‘platform’ (noun \fm{kaÿáash})
				or \fm{\rt{ÿasʼ}} ‘smooth’.
		\end{itemize}
	\item	Unclear meaning in the suffix combination \X{-ḵxʼ} ≡ \fm{-ḵ} + \X{-xʼ}
			with plural/repetitive \X{-xʼ}.
		This is associated with an increase in degree
			but the composition of meaning is unclear.
		See \X{-ḵxʼ} for more details.
	\item	Possibly identifiable as the final consonant in a handful of roots,
			perhaps originally deprivative but now fossilized.
		\begin{itemize}
		\item	In the root \fm{\rt{naḵ}} ‘abandoning, leaving behind’
				in verbs and as a postposition.
			The hypothetical \fm[*]{\rt{na}} root that would combine with \fm{-ḵ}
				to give \fm{\rt{naḵ}} has not been identified.
			\begin{enumerate}[label=\alph*.]
			\item	As a verb root \fm{\rt[²]{naḵ}} ‘give up, let go’ in
				\begin{itemize}[label=•]
				\item	\vbform{ajeewanáḵ}{pfv}[tr, \fm{∅}, ach]{she/he/it let him/her/it go}
					\vbmorph{a-&ji-&μʷ-&wa-&\gm{\rt[²]{naḵ}}&-μH}
						{\xx{3>3}&hand&\xx{pfv}&\xx{stv}&\rt[²]{let.go}&\xx{var}}
				\item	\vbform{ax̱ʼeiwanáḵ}{pfv}[tr, \fm{∅}, ach]{she/he/it gave up it (drinking, smoking)}
					\vbmorph{a-&x̱ʼe-&μʷ-&wa-&\gm{\rt[²]{naḵ}}&-μH}
						{\xx{3>3}&mouth&\xx{pfv}&\xx{stv}&\rt[²]{let.go}&\xx{var}}
				\item	\vbform{ax̱ʼawsináḵ}{pfv}[tr, \fm{∅}, ach]{she/he/it silenced him/her/it}
					\vbmorph{a-&x̱ʼe-&w-&s-&i-&\gm{\rt[²]{naḵ}}&-μH}
						{\xx{3>3}&mouth&\xx{pfv}&\xx{csv}&\xx{stv}&\rt[²]{let.go}&\xx{var}}
				\end{itemize}
			\item	As a postposition \fm{náḵ} ‘away from, leaving behind’ in
				\begin{itemize}[label=•]
				\item	\vbform{a náḵ neil oo.aatch}{hab}[subj intr, \fm{∅}, mot]{they would go home from it}
					\parencite[66.64]{dauenhauer-dauenhauer:1987}
					\vbmorph{a&\gm{náḵ}&neil=&a-&u-&\rt[¹]{.at}&-μμL&-ch}
						{\xx{3n}&away&home&\xx{ind.h.s}&\xx{zpfv}&\rt[¹]{go.\xx{pl}}&\·\xx{var}&\·\xx{rep}}
				\item	\vbform{a waḵnáḵ awlisín}{pfv}[tr, \fm{∅}, ach]{she/he/it hid him/her/it away from his/her/its eyes}
					\parencite[04/77]{leer:1973}
					\vbmorph{a&waaḵ&\gm{-náḵ}&a-&w-&lˢ-&i-&\rt{sin}&-μH}
						{\xx{3n.psr}&eye&\·away&\xx{3>3}&\xx{pfv}&\xx{csv}&\xx{stv}&\rt{hide}&\·\xx{var}}
				\end{itemize}
			\end{enumerate}
		\item	The noun \fm{sayeiḵ} \~\ \fm{siyeiḵ} ‘next day, day after’
				and the noun \fm{seig̱ánín} \~\ \fm{seig̱án} ‘tomorrow’
				(from \fm[*]{sayeiḵ-án-ín}).
			The \fm{sa-} is probably a frozen form of \X{s-}
				and the stem \fm{–yeiḵ} possibly derives
				from the noun \fm{ÿee} ‘time’ (see \X[ÿee=time]{ÿee=}) 
				with deprivative \fm{-ḵ}.
		\item	The root \fm{\rt{x̱iḵ}} \~\ \fm{\rt{x̱eḵ}} ‘lack sleep’ in verbs and a noun.
			Originally from \fm{\rt{x̱i}} \~\ \fm{\rt{x̱e}} ‘stay overnight’
				with deprivative \fm{-ḵ}
				but reanalyzed as a new root with distinct stem variation.
			Compare \fm{\rt{x̱exʼw}} ‘plural sleep’ derived from 
				\fm{\rt{x̱i}} \~\ \fm{\rt{x̱e}} ‘stay overnight’
				with plural \X{-xʼw}.
			\begin{enumerate}[label=\alph*.]
			\item	As a verb root \fm{\rt{x̱eḵ}} \~\ \fm{\rt{x̱iḵ}} ‘insomnia’ in
				\begin{itemize}[label=•]
				\item	\vbform{wudix̱éḵ}{pfv}[obj intr, \fm{∅}, ach]{she/he/it had insomnia}
					\parencites[f02/23]{leer:1973}[794]{leer:1976}
						\vbmorph{wu-&d-&i-&\gm{\rt[¹]{x̱eḵ}}&-μH}
							{\xx{pfv}&\xx{mid}&\xx{stv}&\rt[¹]{insomnia}&\·\xx{var}}
				\item	\vbform{awsix̱éḵ}{pfv}[tr, \fm{∅}, ach]{she/he/it woke him/her/it early}
					\vbmorph{a-&w-&s-&i-&\gm{\rt{x̱eḵ}}&-μH}
						{\xx{3>3}&\xx{pfv}&\xx{csv}&\xx{stv}&\rt[¹]{insomnia}&\·\xx{var}}
				\item	\vbform{ÿax̱éḵsʼ}{impfv}{she/he/it gets up early}
						\vbmorph{ÿa-&\gm{\rt[¹]{x̱eḵ}}&-μH&-sʼ}
							{\xx{stv}&\rt[¹]{insomnia}&\·\xx{var}&\·\xx{rep}}
				\end{itemize}
			\item	As a noun \fm{x̱eeḵ} \~\ \fm{x̱eiḵ} ‘insomnia; tiredness’
				especially in the phrase
				\vbform{x̱eiḵch uwajáḵ}{pfv}[tr, \fm{∅}, ach]{she/he/it suffered bout of insomnia}
				(literally ‘insomnia killed him/her/it’)
					\vbmorph{\gm{x̱eiḵ}&-ch&ⱥ-&u-&wa-&\rt[²]{jaḵ}&-μH}
						{insomnia&\·\xx{erg}&\xx{3>3}&\xx{zpfv}&\xx{stv}&\rt[²]{kill}&\·\xx{var}}
			\end{enumerate}
		\end{itemize}
	\end{enumerate}

\item[ḵaa=]
	allomorph of indefinite human object \fm{ḵu-} ‘someone, people, one, them’;
	possibly like \fm{ax̱=} used only as possessor of incorporated noun
		(compare \fm{ḵaa keidlí áwé} ‘it’s someone’s dog’)
	\begin{itemize}
	\item	\fm{ḵaa seiwa.áx̱} (pfv; tr, \fm{∅}, ach) ‘s/he/it heard someone’s voice’
	\end{itemize}

\item[ḵáaḵw=]\label{m:ḵáaḵw=}
	Variant form of manner preverb \X{ḵwáaḵ=} ‘wrongly, mistakenly’
		that is attested among Juneau, Inland, and Transitional Northern Tlingit varieties.
	Allomorphs \X{ḵáaḵwt=} and \X{ḵáaḵwx̱=} are attested in parallel with the allomorphs
		for \X{ḵwáaḵ=}, but there is no attested instance of predicted \fm{ḵáaḵwde=}.
	See the entry for \X{ḵwáaḵ=} for more examples and discussion.

	This variant form arises by metathesis of labialization
		from the initial consonant to the final consonant:
		the original initial [\ipa{qʰʷ}] loses labialization to become the new initial [\ipa{qʰ}]
		and the original final [\ipa{q}] gains labialization to become the new final [\ipa{qʷ}].
	We might expect a form where labialization occurs on both the initial and final consonants
		as \fm{ḵwáaḵw} [\ipa{qʰʷáːqʷ}], suggesting the spread of labialization across the syllable,
		but this form is unattested.
	\newline
	Allomorphs:
	\begin{allolist}
	\item[ḵáaḵwde=]		predicted but unattested form
				with allative postposition \fm{-de} \~\ \fm{-dé}
	\item[\X{ḵáaḵwt=}]	form with punctual postposition \fm{-t} ‘at, to, around’
	\item[\X{ḵáaḵwx̱=}]	form with pertingent postposition \fm{-x̱} ‘at, of, contacting’
	\end{allolist}
	\begin{enumerate}
	\item	Forms with the punctual postposition \fm{-t} ‘at, to, around’.
		\begin{itemize}
		\item	\vbform{ḵáaḵwt x̱at wuneiyídáx̱}{sub pfv}{after me happening wrongly}
			in \fm{Yeisú tlél jinkaat a yáanáx̱ woox̱éeych ḵáaḵwt x̱at wuneiyídáx̱.}
			“Ten days havenʼt passed since my accident yet.”
			\parencite[132127]{eggleston:2017}
				\vbmorph{\gm{ḵáaḵw}&\gm{-t=}&x̱at=&wu-&\rt[¹]{ne}&-μμL&-ÿí&-dáx̱}
					{wrong&\·\xx{pnct}&\xx{1sg.o}&\xx{pfv}&\rt[¹]{happen}&\·\xx{var}&\·\xx{sub}&\·\xx{abl}}
		\end{itemize}
	\item	Forms with the pertingent postposition \fm{-x̱} ‘of, contacting’.
		\begin{itemize}
		\item	\vbform{ḵáaḵwx̱ daak uwagúdi yáx̱}{sub pfv}[subj intr, \fm{∅}, mot]{like he went into danger}
			\parencite[220.37]{dauenhauer-dauenhauer:1987}
				\vbmorph{\gm{ḵáaḵw}&\gm{-x̱=}&daak=&u-&wa-&\rt[¹]{gut}&-μH&-i&yáx̱}
					{wrong&\·\xx{pert}&out&\xx{zpfv}&\xx{stv}&\rt[¹]{go.\xx{sg}}&\·\xx{var}&\·\xx{sub}&like}
		\item	\vbform{Líl xʼwán ḵáaḵwx̱ x̱at wuneeḵ léelkʼw}{pfv prohib}[obj intr, \fm{∅}, ach]{don’t let anything bad happen to me, grandfather}
			\parencite[6.108]{nyman-leer:1993}
				\vbmorph{líl&xʼwán&\gm{ḵáaḵw}&\gm{-x̱=}&x̱at=&wu-&\rt[¹]{ni}&-μμL&-ḵ}
					{\xx{phib}&\xx{pcl}&wrong&\·\xx{pert}&\xx{1sg.o}&\xx{pfv}&\rt[¹]{happen}&\·\xx{var}&\·\xx{phib}}
		\end{itemize}
	\end{enumerate}

\item[ḵáaḵwt=]\label{m:ḵáaḵwt=}
	Allomorph of variant form \X{ḵáaḵw=} of manner prevberb \X{ḵwáaḵ=} ‘wrongly, mistakenly’
		with punctual postposition \fm{-t} ‘at, to, around’.
	See \X{ḵáaḵw=} and \X{ḵwáaḵ=} for details.

\item[ḵáaḵwx̱=]\label{m:ḵáaḵwx̱=}
	Allomorph of variant form \X{ḵáaḵw=} of manner preverb \X{ḵwáaḵ=} ‘wrongly, mistakenly’
		with pertingent postposition \fm{-x̱} ‘of, contacting’.
	See \X{ḵáaḵw=} and \X{ḵwáaḵ=} for details.

\item[ḵu-]\label{m:ḵu-indef}
	indefinite human object ‘someone, people, one, them’;
	allomorph \fm{ḵaa=} as possessor of incorporated noun;
	compare PP pronoun \fm{ḵú-} ‘someone, people, one, them’
	\begin{itemize}
	\item	\fm{ḵuwsiteen} (pfv; tr, \fm{g̱}, ach) ‘s/he/it saw someone/people’
	\end{itemize}

\item[ḵu-]\label{m:ḵu-areal}
	areal prefix indicating space, area, extent, or weather;
	compare \fm{ḵúx̱(-de)=} ‘back, returning’, \fm{ḵut=} ‘lost’
	\begin{itemize}
	\item	\fm{ḵuwakʼéi} (impfv; impers, \fm{g}, \fm{-μμH} state) ‘it is good weather’\newline
		versus \fm{yakʼéi} (impfv; obj intr) ‘it is good’
	\end{itemize}

\item[ḵut=]\label{m:ḵut=}
	Preverb indicating that eventuality involves becoming lost either literally
		or metaphorically.
	Can be glossed as ‘astray’ or ‘lost’, and may be abbreviated \xx{err}
		for ‘errative’ although this can be confused
		with the manner preverb \X{ḵwáaḵ=} ‘wrongly, mistakenly’
		and the amissive suffix \X{-x̱aa} ‘miss target’.
	Occurs as part of the motion derivation
		\motderiv{ḵut=}{g}{going astray, getting lost}
		so that verbs are shifted to the \fm{g} conjugation class and
		become achievement-like without a basic imperfective form;
		note that this derivation does not supply a repetitive imperfective form.
	Probably evolved from areal \fm{ḵú} (see \X[ḵu-areal]{ḵu-}) and the punctual postposition
		\fm{-t}, perhaps with an original meaning of ‘to somewhere’ that came to be
		interpreted more specifically as ‘to somewhere unknown’ and thence ‘astray, lost’.
	\begin{enumerate}
	\item	Motion that goes astray, usually with the result that the moving entity
			becomes lost, although may instead mean that the entity is no
			longer in view or no longer exists.
		\begin{itemize}
		\item	\vbform{ḵut woogoot}{pfv}[subj intr, \fm{g}, mot]{she/he/it got lost}
			\parencite[126.565]{nyman-leer:1993}
				\vbmorph{\gm{ḵut=}&wu-&μ-&\rt[¹]{gut}&-μμL}
					{\xx{err}&\xx{pfv}&\xx{stv}&\rt[¹]{go.\xx{sg}}&\·\xx{var}}
		\item	\vbform{góosʼ ḵut kawdlisʼées}{pfv}[obj intr, \fm{g}, mot]{the clouds have blown away}
			\parencite[32.248]{story-naish:1973}
				\vbmorph{góosʼ&\gm{ḵut=}&ka-&w-&d-&lˢ-&i-&\rt[¹]{sʼiᴴs}&-μμH}
					{cloud&\xx{err}&\xx{sro}&\xx{pfv}&\xx{mid}&\xx{xtn}&\xx{stv}&\rt[¹]{windblown}&\·\xx{var}}
			\versus \vbform{káast át wulisʼées}{pfv}[obj intr, \fm{n}, mot]{the barrel was blown around}
			\parencite[32.246]{story-naish:1973}
				\vbmorph{káast&á&-t&wu-&lˢ-&i-&\rt[¹]{sʼiᴴs}&-μμH}
					{barrel&\xx{3n}&\·\xx{pnct}&\xx{pfv}&\xx{xtn}&\xx{stv}&\rt[¹]{windblown}&\·\xx{var}}
		\item	\vbform{ax̱ lítayi ḵut x̱waag̱éexʼ}{pfv}[tr, \fm{g}, mot]{I lost my knife}
			\parencite[129.1719]{story-naish:1973}
				\vbmorph{ax̱&líta&-yi&\gm{ḵut=}&ʷ-&x̱a-&μ-&\rt[²]{g̱ixʼ}&-μμH}
					{\xx{1sg.psr}&knife&\·\xx{poss}&\xx{err}&\xx{pfv}&\xx{1sg.s}&\xx{stv}&\rt[²]{throw}&\·\xx{var}}
			\versus \vbform{aawag̱éexʼ}{pfv}[tr, \fm{n}, mot]{she/he/it threw it}
			\parencite[845]{leer:1976}
				\vbmorph{a-&μʷ-&wa-&\rt[²]{g̱ixʼ}&-μμH}
					{\xx{3>3}&\xx{pfv}&\xx{stv}&\rt[²]{throw}&\·\xx{var}}
		\end{itemize}
	\item	Event where agent becomes metaphorically lost in the situation so that they
			remain engaged perhaps to excess.
		\begin{itemize}
		\item	\vbform{ḵut at x̱waax̱aa}{pfv}[tr, \fm{g}, ach]{I got carried away eating things}
			\parencite[220 \#8c]{leer:1991}
				\vbmorph{\gm{ḵut=}&at=&ʷ-&x̱a-&μ-&\rt[²]{x̱a}&-μμL}
					{\xx{err}&\xx{ind.n.o}&\xx{pfv}&\xx{1sg.s}&\xx{stv}&\rt[²]{eat}&\·\xx{var}}
			\versus \vbform{at x̱waax̱áa}{pfv}[tr, \fm{∅}, \fm{-μH} act]{I ate things}
				\vbmorph{at=&ʷ-&x̱a-&μ-&\rt[²]{x̱a}&-μμH}
					{\xx{ind.n.o}&\xx{pfv}&\xx{1sg.s}&\xx{stv}&\rt[²]{eat}&\·\xx{var}}
		\end{itemize}
	\item	Event of deception where the deceived becomes metaphorically lost due to the trickery
			of the agent.
		\begin{itemize}
		\item	\vbform{ḵut x̱at wuliyeil}{pfv}[tr, \fm{g}, ach]{she/he/it cheated, tricked, deceived me}
			\parencite[46.479]{story-naish:1973}
				\vbmorph{\gm{ḵut=}&x̱at=&wu-&lˢ-&i-&\rt[²]{yel}&-μμL}
					{\xx{err}&\xx{1sg.o}&\xx{pfv}&\xx{xtn}&\xx{stv}&\rt[²]{fake}&\·\xx{var}}
			\versus \vbform{awliyél}{pfv}[tr, \fm{∅}, ach]{she/he/it faked it, pretended to do it}
			\parencite[158.2161]{story-naish:1973}
				\vbmorph{a-&w-&lˢ-&i-&\rt[²]{yel}&-μH}
					{\xx{3>3}&\xx{pfv}&\xx{xtn}&\xx{stv}&\rt[²]{fake}&\·\xx{var}}
		\end{itemize}
	\end{enumerate}

\item[ḵux̱=]\label{m:ḵux̱=}
	Directional preverb meaning ‘back, reverse, returning’.
	Can be glossed as ‘back’ but can also be \xx{rev} for ‘revertive’ or ‘reverse’.
	This preverb occurs as part of a motion derivation based on
		\motderiv{NP-t/x̱/dé}{∅, \fm{-μμL} rep}{arriving at NP}
		with \fm{ḵux̱=} in the place of the NP.
	Its appearance in this motion derivation is irregular
		with \fm{ḵux̱=} instead of predicted but unattested \fm[*]{ḵux̱t=} 
			that would have punctual \fm{-t}
		and also \fm{ḵux̱=} instead of predicted but unattested \fm[*]{ḵux̱x̱=}
			that would have pertingent \fm{-x̱},
		although predicted \fm{ḵúx̱de=} with allative \fm{-de} is attested.
	The irregular \fm{ḵux̱=} form seems likely to have arisen by contraction
		of both \fm{ḵux̱t=} and \fm{ḵux̱x̱=}, but this is only speculation.
	\begin{enumerate}
	\item	With motion verbs using the motion derivation
			\motderiv{ḵux̱= / ḵúx̱de=}{∅, \fm{-μμL} rep}{going back, reverse, returning}.
		\begin{enumerate}
		\item	Subject intransitive motion verbs where the \X{d-} middle voice prefix
			occurs along with \fm{ḵux̱=} or \fm{ḵúx̱de=}.
		\item	Object intransitive motion verbs where the \X{d-} middle voice prefix
			occurs along with \fm{ḵux̱=} or \fm{ḵúx̱de=}.
		\item	Transitive motion verbs where \fm{ḵux̱=} or \fm{ḵúx̱de=} occur
			but there is no middle voice marking.
		\end{enumerate}
	\item	With other verbs using the motion derivation
			\motderiv{ḵux̱= / ḵúx̱de=}{∅, \fm{-μμL} rep}{going back, reverse, returning}.
	\item	As a part of some nouns and adverbs.
		\begin{itemize}
		\item	\fm{ḵúx̱de dís} 
				(Tongass \fm{ḵux̱de dis} [\ipa{qʰʷuχ.teːʰ tis}])
				‘April (month)’,
				literally ‘returning/back month’
				\parencite[f01/122]{leer:1973}.
		\item	\fm{ḵux̱dakʼóolʼin} 
				(Tongass \fm{ḵux̱dakʼoólʼin} [\ipa{qʰʷuχ.ta.kʼuːˀ.ɬʼin}])
				‘backward’
				\parencite[f01/122]{leer:1973},
				from \fm{kʼóolʼ} (T.\ \fm{kʼoólʼ} [\ipa{kʼuːˀɬʼ}]) ‘hindquarters’
				\parencite[f04/135]{leer:1973}.
			The \fm{da-} may be a reduced form of the allative postposition
				\fm{-dé} \~\ \fm{-de} ‘to, toward’.
			The \fm{-in} is unidentified but it could be related to either the
				instrumental postposition \fm{een} (T.\ \fm{eèn} [\ipa{ʔiːʰn}]) ‘with’
				or perhaps to the past tense \X[-ín-past]{-ín} that has some other
				puzzling functions that are not obviously related to past tense.
		\item	\fm{ḵux̱deinwú} \~\ \fm{ḵux̱deinú} \~\ \fm{ḵux̱deinóo}
				(Tongass \fm{ḵux̱deìnu} [\ipa{qʰuχ.teːʰ.nu}])
				 ‘eddy, whirlpool’
				 \parencites[04/192]{leer:1973}[19]{leer:1978b}[M·91]{leer-hitch-ritter:2001}.
			Of uncertain etymology; \textcites{leer:1973}{leer:1978b} lists this under a
				unique root \fm{\rt{nu}}
				but that gives no insight into its structure
				and does not account for the \fm{ḵux̱deinwú} form.
			The initial portion is certainly \fm{ḵux̱=}
				and might further be \fm{ḵúx̱-de=}
				depending on whether the \fm{…de…} is part of a following stem.
			The \fm{…dein…} portion could be from \fm{\rt{da}} ‘flow’
				\parencites[05/1]{leer:1973}[313]{leer:1976}
				with \X{-n} and thus \X{-μᵉμL} stem variation
				(or \X{-μᵉμH} with \fm{H}→\fm{L} after suffixation)
				but this leaves \fm{…wú} unaccounted for.
			The \fm{…wú} might be \fm{\rt{wu}} ‘float on water or air’
				\parencites[03/299]{leer:1973}[235]{leer:1976}
				but this does not fit well with other verbs based on the same root.
		\item	\fm{ḵux̱shusxéexi} ‘ascending colon’
		\end{itemize}
	\end{enumerate}


\item[\llap{*}ḵux̱x̱=]\label{m:ḵux̱x̱=}
	Predicted but unattested allomorph of \X{ḵux̱=} ‘back, reverse, returning’.
	In cases where \fm[*]{ḵux̱x̱=} would be expected, \fm{ḵux̱=} occurs instead.
	See \X{ḵux̱=} for examples and more discussion.

\item[ḵúx̱de=]\label{m:ḵúx̱de=}
	Allomorph of \X{ḵux̱=} ‘back, reverse, returning’
		with the allative postposition \fm{-dé} ‘to, toward’.
	Derived from \fm{ḵux̱=} and the motion derivation
		\motderiv{NP-t/x̱/dé}{∅, \fm{-μμL} rep}{arriving at NP}
		with the allative postposition \fm{-de}.

\item[-ḵw]\label{m:-ḵw-optphib}
	allomorph of optative/prohibitive \X[-ḵ-optphib]{-ḵ} with labialization

\item[-ḵw]\label{m:-ḵw-dprv}
	allomorph of deprivative \X[-ḵ-dprv]{-ḵ} with labialization

\item[ḵwáaḵ=]\label{m:ḵwáaḵ=}
	Manner preverb ‘wrongly, mistakenly’ indicating that an event was performed incorrectly
		or that it happened in an unfortunate or undesirable way.
	\textcite[134, 297]{leer:1991} implies that this preverb
		only occurs with the pertingent postposition \fm{-x̱}
		and also that it only occurs in combination with the preverb \X{daaḵ=},
		but both points are mistaken: forms with \fm{-t} are attested
		and forms without \fm{daaḵ=} are also attested.
	
	The origin of this preverb is unclear as there is no corresponding noun or noun phrase
		that resembles it, nor are there any other parts of speech that seem to be related.
	It plausibly derives from the areal morpheme \fm{ḵu-} in some way, but
		the details of its evolution are obscure; the final \fm{-ḵ} might be related to
		the deprivative \X[-ḵ-dprv]{-ḵ} and optative-prohibitive \X[-ḵ-optphib]{-ḵ}
		but this connection is speculative.
	
	There are variant forms \X{ḵáaḵw=} / \X{ḵáaḵwt=} / \X{ḵáaḵwx̱=} which are attested among
		Juneau, Inland, and Transitional Tlingit varieties; see their entries for details.
	They do not appear to differ from the \fm{ḵwáaḵ} forms in grammar or meaning.
	\newline
	Allomorphs:
	\begin{allolist}
	\item[\X{ḵwáaḵde=}]	form with allative postposition \fm{-de} \~\ \fm{-dé} ‘to, toward’
	\item[\X{ḵwáaḵt=}]	form with punctual postposition \fm{-t} ‘at, to, around’
	\item[\X{ḵwáaḵx̱=}]	form with pertingent postposition \fm{-x̱} ‘at, of, contacting’
	\end{allolist}
	\begin{enumerate}
	\item	Forms with the punctual postposition \fm{-t} ‘at, to, around’.
		\begin{itemize}
		\item	\vbform{ḵwáaḵt uwanée}{pfv}[obj intr, \fm{∅}, ach]{an accident happened}
			\parencite[133]{naish:1966}
				\vbmorph{\gm{ḵwáaḵ}&\gm{-t=}&u-&wa-&\rt[¹]{ni}&-μμH}
					{wrong&\·\xx{pnct}&\xx{zpfv}&\xx{stv}&\rt[¹]{happen}&\·\xx{var}}
			\versus \vbform{ḵwáaḵt yee wuneiyí}{sub pfv}{if an accident happens to you}
			\parencite[31.230]{story-naish:1973}
				\vbmorph{\gm{ḵwáaḵ}&\gm{-t=}&yee=&wu-&\rt[¹]{ne}&-μμL&-ÿí}
					{wrong&\·\xx{pnct}&\xx{1pl.o}&\xx{pfv}&\rt[¹]{happen}&\·\xx{var}&\·\xx{sub}}
			\versus \vbform{ḵwáaḵt tsé aanéi xʼwán yeewháan}{admon}{let it not be that things happen wrongly for you}
			\parencites[124 \#44]{leer:1991}[164.47]{dauenhauer-dauenhauer:1990}
				\vbmorph{\gm{ḵwáaḵ}&\gm{-t=}&tsé&aa=&u-&&\rt[¹]{ne}&-μμH}
					{wrong&\·\xx{pnct}&\xx{pcl}&\xx{part}&\xx{irr}&\xx{zcnj}&\rt[¹]{happen}&\·\xx{var}}
			\versus \vbform{ḵáaḵwt x̱at wuneiyídáx̱}{sub pfv}{after me happening wrongly}
			in \fm{Yeisú tlél jinkaat a yáanáx̱ woox̱éeych ḵáaḵwt x̱at wuneiyídáx̱.}
			“Ten days havenʼt passed since my accident yet.”
			\parencite[132127]{eggleston:2017}
				\vbmorph{\gm{ḵáaḵw}&\gm{-t=}&x̱at=&wu-&\rt[¹]{ne}&-μμL&-ÿí&-dáx̱}
					{wrong&\·\xx{pnct}&\xx{1sg.o}&\xx{pfv}&\rt[¹]{happen}&\·\xx{var}&\·\xx{sub}&\·\xx{abl}}
		\item	\vbform{ḵwáaḵt wudzigít}{pfv}[subj intr, \fm{∅}, ach]{he made a mistake}
			\parencites[24]{leer:1963}[658]{leer:1976}
				\vbmorph{\gm{ḵwáaḵ}&\gm{-t=}&wu-&d-&s-&i-&\rt[¹]{git}&-μH}
					{wrong&\·\xx{pnct}&\xx{pfv}&\xx{apsv}&\xx{csv}&\xx{stv}&\rt[¹]{fall.anim}&\·\xx{var}}
			\versus \vbform{áxʼ ḵwáaḵt x̱wadzigidi yé}{rel pfv}{place where I had an accident}
			\parencite[25]{leer:1963}
				\vbmorph{á&-xʼ&\gm{ḵwáaḵ}&\gm{-t=}&ʷ-&x̱a-&d-&s-&i-&\rt[¹]{git}&-μL&-i&yé}
					{\xx{3n}&\·\xx{loc}&wrong&\·\xx{pnct}&\xx{pfv}&\xx{1sg.s}&\xx{apsv}&\xx{csv}&\xx{stv}&\rt[¹]{fall.anim}&\·\xx{var}&\·\xx{rel}&place}
			\versus \vbform{ḵwáaḵt x̱wasgeetch}{hab}{I behave badly}
			\parencite[173.2380]{story-naish:1973}
				\vbmorph{\gm{ḵwáaḵ}&\gm{-t=}&ʷ-&x̱a-&d-&s-&\rt[¹]{git}&-μμL&-ch}
					{wrong&\·\xx{pnct}&\xx{zpfv}&\xx{1sg.s}&\xx{apsv}&\xx{csv}&\rt[¹]{fall.anim}&\·\xx{var}&\·\xx{rep}}
		\item	\vbform{ḵwáaḵt yaawaḵáa}{pfv}[subj intr, \fm{∅}, \fm{x̱ʼa-…-μH} act]{she/he spoke wrong}
			\parencite[857]{leer:1976}
				\vbmorph{\gm{ḵwáaḵ}&\gm{-t=}&ÿa-&μʷ-&wa-&\rt[¹]{ḵa}&-μμH}
					{wrong&\·\xx{pnct}&\xx{qual}&\xx{pfv}&\xx{stv}&\rt[¹]{say}&\·\xx{var}}
			\versus \vbform{ḵwáaḵt ash yawsiḵáa}{pfv}[tr, \fm{∅}, \fm{x̱ʼa-…-μH} act]{she/he spoke wrongly to him/her}
			\parencite[857]{leer:1976}
				\vbmorph{\gm{ḵwáaḵ}&\gm{-t=}&ash=&ÿa-&w-&s-&i-&\rt[¹]{ḵa}&-μμH}
					{wrong&\·\xx{pnct}&\xx{3prx.o}&\xx{qual}&\xx{pfv}&\xx{csv}&\xx{stv}&\rt[¹]{say}&\·\xx{var}}
		\end{itemize}
	\item	Forms with the pertingent postposition \fm{-x̱} ‘of, contacting’.
		\begin{itemize}
		\item	\vbform{ḵáaḵwx̱ daak uwagúdi yáx̱}{sub pfv}[subj intr, \fm{∅}, mot]{like he went into danger}
			\parencite[220.37]{dauenhauer-dauenhauer:1987}
				\vbmorph{\gm{ḵáaḵw}&\gm{-x̱=}&daak=&u-&wa-&\rt[¹]{gut}&-μH&-i&yáx̱}
					{wrong&\·\xx{pert}&out&\xx{zpfv}&\xx{stv}&\rt[¹]{go.\xx{sg}}&\·\xx{var}&\·\xx{sub}&like}
		\item	\vbform{ḵwáaḵx̱ wookwaan}{pfv}[subj intr, \fm{n}, mot]{he swam wrongly}
			\parencite[311.8/16]{swanton:1909}
				\vbmorph{\gm{ḵwáaḵ}&\gm{-x̱=}&wu-&μ-&\rt[¹]{kwan}&-μμL}
					{wrong&\·\xx{pert}&\xx{pfv}&\xx{stv}&\rt[¹]{swim.sfc}&\·\xx{var}}
			\newline
			This form surprisingly has \fm{-μμL} suggesting \fm{n} conjugation class
				rather than \fm{∅} conjugation class
				which would have \fm{-μH} instead.
			The original text is \orth{\!qâkx wukwā′n\!} where the \orth{ā}
				strongly supports \fm{-μμL} over \fm{-μH}.
		\item	\vbform{Líl xʼwán ḵáaḵwx̱ x̱at wuneeḵ léelkʼw}{pfv prohib}[obj intr, \fm{∅}, ach]{don’t let anything bad happen to me, grandfather}
			\parencite[6.108]{nyman-leer:1993}
				\vbmorph{líl&xʼwán&\gm{ḵáaḵw}&\gm{-x̱=}&x̱at=&wu-&\rt[¹]{ni}&-μμL&-ḵ}
					{\xx{phib}&\xx{pcl}&wrong&\·\xx{pert}&\xx{1sg.o}&\xx{pfv}&\rt[¹]{happen}&\·\xx{var}&\·\xx{phib}}
		\end{itemize}
	\item	Forms with the allative postposition \fm{-de} ‘to, toward’.
		These are poorly documented but expected from the motion derivation
			\motderiv{NP-t/x̱/dé}{∅, \fm{-μμL} rep}{arriving at NP}
			in contexts with prospective or progressive aspect.
		\begin{itemize}
		\item	\vbform{ḵwáaḵde yaa yanx̱aḵéin}{prog}[subj intr, \fm{∅}, \fm{x̱ʼa-…-μH} act]{I’m making a mistake in speaking}
			\parencite[133]{naish:1966}
				\vbmorph{\gm{ḵwáaḵ}&\gm{-de=}&ÿaa=&ÿa-&n-&x̱a-&\rt[¹]{ḵa}&-μμᵉH&-n}
					{wrong&\·\xx{all}&along&\xx{qual}&\xx{ncnj}&\xx{1sg.s}&\rt[¹]{speak}&\·\xx{var}&\·\xx{nsfx}}
		\end{itemize}
	\item	As part of an adverb derived via an unknown process.
		This is attested by only one form so very little is known about it.
		\begin{itemize}
		\item	\vbform{ḵáaḵwx̱dagán yáx̱ awsinei}{pfv}[tr, \fm{n}, ach]{she/he did it the wrong way accidentally}
			\parencite[275]{leer:1973}
				\vbmorph{\gm{ḵáaḵw}&\gm{-x̱=}&da-&\rt[¹]{gan}&-μH&yáx̱&a-&w-&s-&i-&\rt[¹]{ne}&-μμL}
					{wrong&\·\xx{pert}&\xx{mid}?&\rt[¹]{burn}?&\·\xx{var}&like&\xx{3>3}&\xx{pfv}&\xx{csv}&\xx{stv}&\rt[¹]{happen}&\·\xx{var}}
		\end{itemize}
	\item	As a noun modifier attested in one sentence from \textcite{swanton:1909}.
		If this is not a one-off innovation then it suggests the \fm{ḵwáaḵ} element formerly
			had a wider distribution outside of verb derivations.
		\begin{itemize}
		\item	\vbform{ḵwáaḵ ÿéináx̱ ax̱ taÿeet eeÿagút}{pfv}[subj intr, \fm{∅}, mot]{you came below me through the wrong way}
			\parencite[269.11]{swanton:1909}
				\vbmorph{\gm{ḵwáaḵ}&ÿéi&-náx̱&ax̱&taÿee&-t&u-&i-&ÿa-&\rt[¹]{gut}&-μH}
					{\gm{wrong}&way&\·\xx{perl}&\xx{1sg.psr}&below&\·\xx{pnct}&\xx{zpfv}&\xx{2sg.s}&\xx{stv}&\rt[¹]{go.\xx{sg}}&\·\xx{var}}
		\end{itemize}
	\end{enumerate}

\item[ḵwáaḵde=]\label{m:ḵwáaḵde=}
	Allomorph of manner preverb \X{ḵwáaḵ=} ‘wrongly, mistakenly’
		with allative postposition \fm{-dé} ‘to, toward’.
	This form with \fm{-dé} is attested only rarely but is expected from the motion derivation
		\motderiv{NP-t/x̱/dé}{∅, \fm{-μμL} rep}{arriving at NP}
		in contexts with prospective or progressive aspect.
	See the entry for \X{ḵwáaḵ=} for more details.
	\begin{itemize}
	\item	\vbform{ḵwáaḵde yaa yanx̱aḵéin}{prog}[subj intr, \fm{∅}, \fm{x̱ʼa-…-μH} act]{I’m making a mistake in speaking}
		\parencite[133]{naish:1966}
			\vbmorph{\gm{ḵwáaḵ}&\gm{-de=}&ÿaa=&ÿa-&n-&x̱a-&\rt[¹]{ḵa}&-μμᵉH&-n}
				{wrong&\·\xx{all}&along&\xx{qual}&\xx{ncnj}&\xx{1sg.s}&\rt[¹]{speak}&\·\xx{var}&\·\xx{nsfx}}
	\end{itemize}

\item[ḵwáaḵt=]\label{m:ḵwáaḵt=}
	Allomorph of manner preverb \X{ḵwáaḵ=} ‘wrongly, mistakenly ’
		with punctual postposition \fm{-t} ‘at, to, around’.
	See the entry for \X{ḵwáaḵ=} for more details.
	\begin{itemize}
	\item	\vbform{ḵwáaḵt uwanée}{pfv}[obj intr, \fm{∅}, ach]{an accident happened}
		\parencite[133]{naish:1966}
			\vbmorph{\gm{ḵwáaḵ}&\gm{-t=}&u-&wa-&\rt[¹]{ni}&-μμH}
				{wrong&\·\xx{pnct}&\xx{zpfv}&\xx{stv}&\rt[¹]{happen}&\·\xx{var}}
	\end{itemize}

\item[ḵwáaḵx̱=]\label{m:ḵwáaḵx̱=}
	Allomorph of manner preverb \X{ḵwáaḵ=} ‘wrongly, mistakenly’
		with pertingent postposition \fm{-x̱} ‘of, contacting’.
	See the entry for \X{ḵwáaḵ=} for more details.
	\begin{itemize}
	\item	\vbform{ḵwáaḵx̱ wookwaan}{pfv}[subj intr, \fm{n}, mot]{he swam wrongly}
			\parencite[311.8/16]{swanton:1909}
				\vbmorph{\gm{ḵwáaḵ}&\gm{-x̱=}&wu-&μ-&\rt[¹]{kwan}&-μμL}
					{wrong&\·\xx{pert}&\xx{pfv}&\xx{stv}&\rt[¹]{swim.sfc}&\·\xx{var}}
	\end{itemize}

\item[-ḵxʼ]\label{m:-ḵxʼ}
	≡ \fm{-ḵ-xʼ}
	combination of deprivative \X[-ḵ-dprv]{-ḵ}
		and plural/repetitive \X{-xʼ};
	associated with an increase in degree of some quality
		but the composition of meaning is unclear;
	only two cases are attested, both reported by \textcite[59]{story:1966},
		which are notably not listed by \textcites{leer:1973}{leer:1976};
	these could plausibly be mishearings of repetitive \X{-k}
		instead of deprivative \fm{-ḵ} but there is no direct evidence
		to support this hypothesis;
	alternatively these could originally have used \fm{-k} but changed
		to \fm{-ḵ} as a kind of place dissimilation since
		\fm{-ḵ} is uvular and \fm{-xʼ} is velar
	\begin{itemize}
	\item	\fm{–tsínḵxʼ} ‘plural very expensive’
		from \fm{\rt{tsin}} ‘alive, strong’ in
		\newline
		\vbform{x̱ʼalitsínḵxʼ}{impfv}[obj intr, \fm{g}, inv state]{they are all very expensive}
		\parencite[59]{story:1966} 
			\vbmorph{x̱ʼe-&lˢ-&i-&\rt[¹]{tsin}&-μH&\gm{-ḵ}&-xʼ}
				{mouth&\xx{xtn}&\xx{stv}&\rt[¹]{strong}&\·\xx{var}&\·\xx{dprv}&\·\xx{pl}}
		\versus \vbform{dax̱ x̱ʼadlitsínxʼ}{impfv}{they are each expensive}
			\parencite[154.1173]{nyman-leer:1993}
			\vbmorph{dax̱=&x̱ʼe-&d-&lˢ-&i-&\rt[¹]{tsin}&-μH&-xʼ}
				{\xx{distb}&mouth&\xx{mid}&\xx{xtn}&\xx{stv}&\rt[¹]{strong}&\·\xx{var}&\·\xx{pl}}
		\versus \vbform{x̱ʼalitseen}{impfv}{she/he/it is expensive}
			\vbmorph{x̱ʼe-&lˢ-&i-&\rt[¹]{tsin}&-μμL}
				{mouth&\xx{xtn}&\xx{stv}&\rt[¹]{strong}&\·\xx{var}}
	\item	\fm{–ÿátʼḵxʼ} ‘plural very long’
		from \fm{\rt{ÿatʼ}} ‘long’ in
		\newline
		\vbform{diyátʼḵxʼ}{impfv}[obj intr, \fm{g}, inv state]{they are very long}
		\parencite[59]{story:1966}
			\vbmorph{d-&i-&\rt[¹]{ÿatʼ}&-μH&\gm{-ḵ}&-xʼ}
				{\xx{mid}&\xx{stv}&\rt[¹]{long}&\·\xx{var}&\·\xx{dprv}&\·\xx{pl}}
		\versus \vbform{diyátʼxʼ}{impfv}{they are long}
			\vbmorph{d-&i-&\rt[¹]{ÿatʼ}&-μH&-xʼ}
				{\xx{mid}&\xx{stv}&\rt[¹]{long}&\·\xx{var}&\·\xx{pl}}
		\versus \vbform{yayátʼ}{impfv}{she/he/it is long}
			\vbmorph{ÿa-&\rt[¹]{ÿatʼ}&-μH}
				{\xx{stv}&\rt[¹]{long}&\·\xx{var}&}
	\end{itemize}
\end{morphdesc}
