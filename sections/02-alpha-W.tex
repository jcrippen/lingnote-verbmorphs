%!TEX root = ../lingnote-verbmorphs.tex

\subsection{W}\label{sec:alphalist-w}
\begin{morphdesc}[resume*=alphalist]
\item[ʷ-]\label{m:ʷ-pfv}
	allomorph of perfective \fm{wu-} when followed by
	first person singular subject \fm{x̱-} \~\ \fm{x̱a-},
	labializing its fricative to form \fm{x̱w} or \fm{x̱wa}
	(phonetically [\ipa{χʷ}] or [\ipa{χʷa}])

\item[ʷ-]\label{m:ʷ-irr}
	allomorph of irrealis \fm{u-}

\item[w-]\label{m:w-pfv}
	allomorph of perfective \fm{wu-} in coda of a syllable;
	in Inland Tlingit \fm{m-} is used instead, may also occur elsewhere in older Tlingit
		(e.g.\ song lyrics);
	19th century Tlingit occasionally has full \fm{wu-} rather than \fm{w-},
		e.g.\ \fm{awusikóo du éesh hídi} ‘she knew her father’s house’
		\parencite[255.7]{swanton:1909}
	\begin{itemize}
	\item	\fm{awsiteen} (pfv; tr, \fm{g̱}, ach) ‘s/he/it caught sight of (saw) him/her/it’\newline
		versus \fm{x̱at wusiteen} (pfv) ‘s/he/it caught sight of (saw) me’
	\end{itemize}

\item[w-]\label{m:w-irr}
	allomorph of irrealis \fm{u-}

\item[w̃-]\label{m:w̃-}
	variant form of perfective \fm{w-} allomorph in coda of a syllable;
	occurs in some varieties where \fm{m-} formerly occurred, compare \fm{m-} still used
	in Teslin and Carcross/Tagish Tlingit varieties

\item[-w]\label{m:-w}
	allomorph of \X{-ÿ} sonorant suffix of unknown meaning
	when preceded by labialized (round) sound

\item[wa-]\label{m:wa-}
	allomorph of stative \fm{ÿa-} when preceded by labialized (round) sound

\item[wu-]\label{m:wu-}
	perfective prefix used in most perfective aspect forms;
	see also \fm{∅} conjugation class perfective \fm{u-};
	the perfective prefix has a very complex phonology with many different patterns that depend
	on both	preceding and following prefixes;
	see individual entries for specific details and verb prefix charts for comprehensive patterns;
	reconstructed as \fm[*]{ŋu-} \~\ \fm[*]{ŋʷ-} cognate with Proto-Dene \fm[*]{ŋi-} perfective
	\newline
	allomorphs:
	\begin{allolist}
	\item[\X{m-}]	coda consonant following a vowel, variant form of \fm{w-} used in Teslin
			and Carcross/Tagish varieties
	\item[{\X[u-pfv]{u-}}]
			special allomorph only used with \fm{∅} conjugation class verbs
	\item[{\X[ʷ-pfv]{ʷ-}}]
			labialization of a consonant, with first person singular subject \fm{x̱-} \~\ \fm{x̱a-}
	\item[{\X[w-pfv]{w-}}]
			coda consonant following a vowel (i.e.\ \fm{u} of \fm{wu-} deleted)
	\item[\X{w̃-}]	coda consonant following a vowel, retaining the nasalization of former \fm{m-}
	\item[{\X[ÿ-pfv]{ÿ-}}]
			delabialized \fm{y} or \fm{ÿ} preceding or merged with a front vowel
	\item[\X{ÿu-}]	abstract representation reflecting labialized and delabialized forms
	\item[\X{μʷ-}]	lengthening of preceding vowel with labialization spread to other prefixes
	\item[\X{μw-}]	lengthening of preceding vowel with labial consonant coda
	\item[\X{μm-}]	lengthening of preceding vowel with \fm{m} consonant coda
	\end{allolist}
	combinations:
	\begin{allolist}
	\item[x̱w]	≡ \fm{ʷ-x̱-} with first person singular subject \fm{x̱-}
	\item[x̱wa]	≡ \fm{ʷ-x̱a-} with first person singular subject \fm{x̱a-}
	\item[ÿ]	≡ \fm{ÿ-i-} with second singular subject \X[i-2sg]{i-}
	\item[ÿee]	≡ \fm{ÿ-i-μ-} with second singular subject \X[i-2sg]{i-}
			and stative \X{μ-}
	\item[ÿeeÿ]	≡ \fm{ÿu-ÿi-} with second plural subject \X{ÿi-}
	\item[ÿeeÿ]	≡ \fm{ÿu-ÿi-μ-} with second plural subject \X{ÿi-}
			and stative \X{μ-}
	\item[ÿeeÿCi]	≡ \fm{ÿu-ÿi-C-i-} with second plural subject \X{ÿi-}
			and valency \X{s-} or \X{l-}/\X{lˢ-} or \X{sh-}
			and stative \X[i-stv]{i-}
	\item[ÿi]	≡ \fm{ÿ-i-} with second singular subject \X[i-2sg]{i-}
	\end{allolist}
	\begin{itemize}
	\item	\vbform{at wuduwax̱áa}{pfv}[tr, \fm{∅}, \fm{-μH} act]{someone/people ate something}
			\vbmorph{at=&wu-&du-&wa-&\rt[²]{x̱a}&-μμH}
				{\xx{ind.n.o}&\xx{pfv}&\xx{ind.h.s}&\xx{stv}&\rt[²]{eat}&\·\xx{var}}
		\versus \vbform{at dux̱á}{impfv}{someone/people is/are eating something}
			\vbmorph{at=&du-&\rt[²]{x̱a}&-μH}
				{\xx{ind.n.o}&\xx{ind.h.s}&\rt[²]{eat}&\·\xx{var}}
	\end{itemize}

\item[wush=]\label{m:wush=}
	allomorph of reciprocal object \fm{woosh=}

\item[wooch=]\label{m:wooch=}
	allomorph of reciprocal object \fm{woosh=}

\item[woosh=]\label{m:woosh=}
	reciprocal object
\end{morphdesc}
