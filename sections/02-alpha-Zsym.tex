%!TEX root = ../lingnote-verbmorphs.tex

\subsection{Symbols}\label{sec:alphalist-sym}
\begin{morphdesc}[resume*=alphalist]
\item[μ-]\label{m:μ-}
	allomorph of stative \fm{ÿa-} \~\ \fm{i-}
	\begin{itemize}
	\item	\fm{x̱at yatéen} (impfv; tr, \fm{∅}, \fm{-μμH} state) ‘s/he/it can see me’
			with \fm{ÿa-}\newline
		versus \fm{x̱aatéen} (impfv) ‘I can see him/her/it’
			with \fm{μ-}\newline
		(not normally \fm[*]{x̱ayatéen}, although some speakers also permit this form)
	\end{itemize}

\item[μʷ-]\label{m:μʷ-}
	allomorph of perfective \fm{wu-} when preceded by CV and followed by stative \fm{ÿa-}
	\begin{itemize}
	\item	\fm{aawajáḵ} (pfv; \fm{∅}, ach) ‘s/he/it killed him/her/it’
			with \fm{a-μʷ-wa-\rt[¹]{jaḵ}-μH}\newline
		versus \fm{wutuwajáḵ} (pfv) ‘we killed him/her/it’
			with \fm{wu-tu-wa-\rt[¹]{jaḵ}-μH}
	\end{itemize}

\item[μw-]\label{m:μw-}
	allomorph of perfective \fm{wu-}

\item[μm-]\label{m:μm-}
	allomorph of perfective \fm{wu-}

\item[-μ]\label{m:-μ-loc}
	Allomorph of the locative postposition \fm{-xʼ} which occurs after a noun or pronoun
		that ends with a /\ipa{CV́}/ syllable, resulting in lengthening of the syllable
		to [\ipa{CV́ː}].
	This allomorph is identical in meaning to the ordinary locative \fm{-xʼ}
		but it is only permitted immediately before the verb.
	Because of this, \fm{-μ} is part of several preverbs that contain the locative postposition
		and in some cases is no longer perceived as a distinct suffix.
	Derives from debuccalization of [\ipa{xʼ}] leaving only the laryngeal gesture of the ejective
		which was reinterpreted as glottalization on the vowel: [\ipa{CVxʼ}] → [\ipa{CVːˀ}].
	This is still reflected by the Tongass allomorph \fm{-ʼ} as in
		\fm{aá yeì ÿateè} ‘she/he/it is there’.
	Glottalized vowels then became long vowels with H or HL tone in the dialects that
		underwent tonogenesis.

\item[-μL]\label{m:-μL}
	stem variation: short vowel (μ) with low tone (L) so […\ipa{V̀}…]
	\begin{itemize}
	\item	\fm{neil uwagudi ḵáa} (pfv rel; subj intr, \fm{∅}, mot) ‘man who went home’
		with \fm{\rt[¹]{gut}} ‘sg.\ go’ and \fm{-μL}
		in \fm{∅} conjugation class perfective aspect relative clause
	\end{itemize}

\item[-μH]\label{m:-μH}
	stem variation: short vowel (μ) with high tone (H) so […\ipa{V́}…]
	\begin{itemize}
	\item	\fm{neil uwagút} (pfv; subj intr, \fm{∅}, mot) ‘s/he/it went home’
		with \fm{\rt[¹]{gut}} ‘sg.\ go’ and \fm{-μH}
		in \fm{∅} conjugation class perfective aspect main clause
	\end{itemize}

\item[-μμL]\label{m:-μμL}
	stem variation: long vowel (μμ) with low tone (L) so […\ipa{V̀ː}…]
	\begin{itemize}
	\item	\fm{neildé woogoot} (pfv; subj intr, \fm{n}, mot) ‘s/he/it went homeward’
		with \fm{\rt[¹]{gut}} ‘sg.\ go’ and \fm{-μμL}
		in \fm{n} conjugation class perfective aspect main clause
	\end{itemize}

\item[-μμH]\label{m:-μμH}
	stem variation: long vowel (μμ) with high tone (H) so […\ipa{V́ː}…]
	\begin{itemize}
	\item	\fm{neildé gug̱agóot} (prosp; subj intr, \fm{∅}/\fm{n}, mot) ‘s/he/it will go home’
		with \fm{\rt[¹]{gut}} ‘sg.\ go’ and \fm{-μμH}
		in prospective aspect main clause
	\end{itemize}

\item[-μμHL]\label{m:-μμHL}
	stem variation: long vowel (μμ) with falling tone (HL) so […\ipa{V̂ː}…];
	only in Southern Tlingit dialects (Sanya, Henya) that have phonologically
		contrastive falling tone

\item[-μᵉμL]\label{m:-μᵉμL}
	stem variation: ablaut (/\ipa{a, u}/ → [\ipa{e}]) long vowel (μμ)
		with low tone (L) so […\ipa{èː}…];
	normally occurs only with \fm{\rt{CVᴸ}} roots
	\begin{itemize}
	\item	\fm{x̱ateix̱} (rep impfv; subj intr, \fm{n}, \fm{-μH} act) ‘I repeatedly sleep’
			with \fm{\rt[¹]{taᴸ}} ‘sg.\ sleep’ and \fm{-μᵉμL-x̱}\newline
		versus
		\fm{x̱atá} (impfv) ‘I am sleeping’
			with \fm{-μH}\newline
		but \fm{yaa nx̱atéin} (prog) ‘I am falling asleep’
			with \fm{-μᵉμH-n}
	\end{itemize}

\item[-μᵉμH]\label{m:-μᵉμH}
	stem variation: ablaut (/\ipa{a, u}/ → [\ipa{e}]) long vowel (μμ)
		with high tone (H) so […\ipa{éː}…];
	normally occurs only with \fm{\rt{CV}} or \fm{\rt{CVᴸ}} roots
	\begin{itemize}
	\item	\fm{x̱ax̱éix̱} (rep impfv; tr, \fm{∅}, \fm{-μH} act) ‘I repeatedly eat it’
			with \fm{\rt[²]{x̱a}} ‘eat’ and \fm*{-μᵉμH-x̱}\newline
		versus
		\fm{x̱ax̱á} (impfv) ‘I am eating it’
			with \fm{-μH}
	\end{itemize}

\item[-μᵉμHL]\label{m:-μᵉμHL}
	stem variation: ablaut (/\ipa{a, u}/ → [\ipa{e}]) long vowel (μμ)
		with falling tone (HL) so […\ipa{êː}…];
	only in Southern Tlingit dialects (Sanya, Henya) that have phonologically
		contrastive falling tone;
	normally occurs only with \fm{\rt{CV}} or \fm{\rt{CVᴸ}} roots

\item[-⊗]\label{m:-DEL}
	irregular deletion (⊗) of final consonant
		resulting in a short vowel with high tone (μH)
		so […\ipa{V́}]
		(no tone […\ipa{V}] in Tongass Tlingit);
	only occurs in imperatives with
		\begin{inlinelist}
		\item	\fm{\rt[¹]{gut}} ‘sg go’
		\item	\fm{\rt[¹]{.at}} ‘pl go’
		\item	\fm{\rt[¹]{nuk}} ‘sg sit’
		\end{inlinelist}
	\begin{itemize}
	\item	\vbform{neildé nagú!}{imp}[subj intr, \fm{n}, mot]{(you sg.)\ go home!}
			\vbmorph{neil&-dé&na-&\rt[ˢ]{gu\gm{t}}&\gm{-⊗}}
				{home&\·\xx{all}&\xx{ncnj}&\rt[ˢ]{go·\xx{sg}}&\·\xx{var}}
		\versus \vbform{neildé yeegoot}{pfv}{you sg.\ went home}
			\vbmorph{neil&-dé&ÿ-&i-&μ-&\rt[ˢ]{gut}&-μμL}
				{home&\·\xx{all}&\xx{pfv}&\xx{2sg.s}&\xx{stv}&\rt[ˢ]{go·\xx{sg}}&\·\xx{var}}
	\item	\vbform{neildé nay.á!}{imp}[subj intr, \fm{n}, mot]{you guys go home!}
			\vbmorph{neil&-dé&na-&ÿ-&\rt[ˢ]{.a\gm{t}}&\gm{-⊗}}
				{home&\·\xx{all}&\xx{ncnj}&\xx{2pl.s}&\rt[ˢ]{go·\xx{pl}}&\·\xx{var}}
		\versus \vbform{neildé yeey.aat}{pfv}{you guys went home}
			\vbmorph{neil&-dé&ÿ-&ÿ-&μ-&\rt[ˢ]{.at}&-μμL}
				{home&\·\xx{all}&\xx{pfv}&\xx{2pl.s}&\xx{stv}&\rt[ˢ]{go·\xx{pl}}&\·\xx{var}}
	\item	\vbform{g̱anú!}{imp}[subj intr, \fm{g̱}, mot]{(you sg.)\ sit down!}
			\vbmorph{g̱a-&\rt[ˢ]{nu\gm{k}}&\gm{-⊗}}
				{\xx{g̱cnj}&\rt[ˢ]{sit·\xx{sg}}&\·\xx{var}}
		\versus \vbform{yeenook}{pfv}{you sg.\ sat down}
			\vbmorph{ÿ-&i-&μ-&\rt[ˢ]{nuk}&-μμL}
				{\xx{pfv}&\xx{2sg.s}&\xx{stv}&\rt[ˢ]{sit·\xx{sg}}&\·\xx{var}}
	\end{itemize}
\end{morphdesc}
