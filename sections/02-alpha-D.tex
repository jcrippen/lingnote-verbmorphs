%!TEX root = ../lingnote-verbmorphs.tex

\subsection{D}\label{sec:alphalist-d}
\begin{morphdesc}[resume*=alphalist]
\item[d-]\label{m:d-}
	voice prefix, traditionally analyzed as part of the classifier;
	suppresses an argument (passive, antipassive)
	or reduces the scope of its reference (middle);
	\newline
	allomorphs:
	\begin{allolist}
	\item[d-]	basic form
	\item[\X{da-}]	with epenthetic (filler) vowel \fm{a}
			(no \fm{s-}/\fm{l-}/\fm{lˢ-}/\fm{sh-} and no \fm{i-})
	\end{allolist}
	combinations:
	\begin{allolist}
	\item[\X{di}]	≡ \fm{d-i-} with stative \X[i-stv]{i-}
	\item[\X{dli}]	≡ \fm{d-l-i-} with valency \X{l-}/\X{lˢ-} and stative \X[i-stv]{i-}
	\item[\X{dzi}]	≡ \fm{d-s-i-} with valency \X{s-} and stative \X[i-stv]{i-}
	\item[\X{ji}]	≡ \fm{d-sh-i-} with valency \X{sh-} and stative \X[i-stv]{i-}
	\item[\X{…l}]	≡ \fm{d-s-} with valency \X{l-}/\X{lˢ-} (no \fm{i-})
	\item[\X{…s}]	≡ \fm{d-s-} with valency \X{s-} (no \fm{i-})
	\item[\X{…sh}]	≡ \fm{d-sh-} with valency \X{sh-} (no \fm{i-})
	\end{allolist}
	the \fm{…s} / \fm{…l} / \fm{…sh} forms must be preceded by a vowel,
		with epenthetic (filler) \fm{i} inserted if no preceding prefixes provide a vowel,
		see those entries for examples
%		(the hypothetical \fm[*]{dza} / \fm[*]{dla} / \fm[*]{ja}
%			and \fm[*]{…dz} / \fm[*]{…dl} / \fm[*]{…j} do not occur
%			and are instead \fm{…s} / \fm{…l} / \fm{…sh})
	\begin{enumerate}
	\item	middle voice
	\item	passive voice
	\item	antipassive voice
	\end{enumerate}

\item[da-]\label{m:da-}
	allomorph of voice prefix \X{d-} with epenthetic (filler) vowel;
	occurs only whenever there is no stative \X[i-stv]{i-}
		and no valency \X{s-}, \X{l-}/\X{lˢ-}, or \X{sh-}
	\begin{itemize}
	\item	\vbform{tléil yan sh wudax̱eech}{neg pfv}[tr, \fm{∅}, mot]{he did not throw himself down}
			\vbmorph{tléil&yan=&sh=&wu-&\gm{da-}&\rt[²]{x̱ich}&-μμL}
				{\xx{neg}&ground&\xx{rflx.o}&\xx{pfv}&\xx{mid}&\rt[²]{throw}&\·\xx{var}}
		\versus \vbform{yan sh wudix̱ích}{pfv}{he threw himself down}
		\parencite[227.3217]{story-naish:1973}
			\vbmorph{yan=&sh=&wu-&d-&i-&\rt[²]{x̱ich}&-μH}
				{ground&\xx{rflx.o}&\xx{pfv}&\xx{mid}&\xx{stv}&\rt[²]{throw}&\·\xx{var}}
	\end{itemize}

\item[daa-]
	inalienable incorporated noun \fm{daa} ‘around, about, surrounding’

\item[daak=]\label{m:daak=}
	directional preverb ‘out to sea (away from land)’;
	derived from directional noun \fm{dáak} ‘out at sea’
		(compare \fm{dákde=})

\item[dáag̱i=]
	Locational preverb ‘inland’ indicating location on land away from a body of water.
	Derived from the noun \fm{dáaḵ} ‘inland’
		with the special locative postposition allomorph
		\X[-i-loc]{-i} \~\ \X[-í-loc]{-í} ‘at’
		(instead of the regular allomorphs \fm{-xʼ} and \fm{-μ}).
	This locative allomorph is unique in that it only occurs with a handful of preverbs,
		for which see the detailed entry of \X[-í-loc]{-í}.
	Compare the directional preverb \X{daaḵ=} / \X{dáḵde=} which is based on the same noun.
	\begin{enumerate}
	\item	In combination with the directional preverb \X{daaḵ=} / \X{dáḵde=} ‘inland’.
		It is not entirely clear what the combination of \fm{dáag̱i=} and \fm{daaḵ=}
			mean together since from a first approximation they appear to convey
			the same information.
		\begin{itemize}
		\item	\vbform{dáag̱i daaḵ uwalʼíxʼ}{pfv}[obj intr, \fm{∅}, ach]{it was broken all the way up}
			\parencite[18.172]{nyman-leer:1993}
				\vbmorph{\gm{dáaḵ}&\gm{-i=}&daaḵ=&u-&wa-&\rt{lʼixʼ}&-μH}
					{inland&\·\xx{loc}&inland&\xx{zpfv}&\xx{stv}&\rt{break}&\·\xx{var}}
		\item	\vbform{yóot dáag̱i daaḵ ishéex}{csec}[subj intr, \fm{∅}, mot]{having run off up inland}
			\parencite[275.7]{swanton:1909}
				\vbmorph{yóo&-t&\gm{dáaḵ}&\gm{-i=}&daaḵ=&d-&sh-&\rt[¹]{xix}&-μμH}
					{\xx{dist}&\·\xx{pnct}&inland&\·\xx{loc}&inland&\xx{mid}&\xx{pej}&\rt[¹]{fall}&\·\xx{var}}
		\end{itemize}
	\item	In forms where it occurs alone or with some other postposition phrase.
		\begin{itemize}
		\item	\vbform{dáag̱i ḵoon aawayeesh}{pfv}[tr, \fm{n}, \fm{-μμH} act]{he pulled it inland with people}
			\parencite[144.140]{dauenhauer-dauenhauer:1987}
				\vbmorph{\gm{dáaḵ}&\gm{-i=}&ḵoo&-n&a-&μʷ-&wa-&\rt[²]{ÿish}&-μμL}
					{inland&\·\xx{loc}&\xx{ind.h}&\·\xx{instr}&\xx{3>3}&\xx{pfv}&\xx{stv}&\rt[²]{pull}&\·\xx{var}}
		\end{itemize}
	\end{enumerate}

\item[daaḵ=]\label{m:daaḵ=}
	directional preverb ‘inland (away from water body)’;
	derived from directional noun \fm{dáaḵ} ‘inland’
		(compare \fm{dáag̱i=}, \fm{dáḵde=}; noun \fm{daḵká} ‘on inland’)

\item[daa.it-]
	inalienable incorporated noun \fm{daa.ít} ‘joint’;
	this noun only rarely occurs independently and instead is more often part of a
		compound noun or an incorporate in a verb;
	the noun \fm{daa.ít} appears to be a compound of \fm{daa} ‘around’
		and \fm{ít} ‘following, after’
		but the current meaning of ‘joint’ is not compositional
		so \fm{daa.ít} is now an indivisible unit;
	the obscure Tongass form \fm{du daá.eèt} “his joints”
		\parencite[02/244]{leer:1973} suggests a different etymology than \fm{ít} ‘after’,
		perhaps the same source as \fm{eetí} ‘remains of, space of’, but this may be confused
		with the very similar \fm{ḵaa daa.eedí} “personal effects used to cast spell by witch”
		\parencite[02/247]{leer:1973}, though also compare the obscure
		\fm{ḵaankak.eetxʼ} “in public” \parencite[02/246]{leer:1973}
	\begin{enumerate}
	\item	as an incorporated noun in a couple of verbs
		\begin{itemize}
		\item	\vbform{tlél daa.itḵooshgóok}{impfv}[subj intr?, \fm{g}?, state?]{he is uncoordinated, awkward}
			\parencite[138.14]{dauenhauer-dauenhauer:1987}
				\vbmorph{tlél&\gm{daa.it}-&ḵu-&u-&sh-&\rt{guᴴk}&-μμH}
					{\xx{neg}&joint&\xx{areal}&\xx{irr}&\xx{pej}&\rt{know.how}&\·\xx{var}}
			\versus \vbform{ashigóok}{impfv}[tr, \fm{g}, \fm{-μμH} state]{she/he/it knows how to do it}
				\vbmorph{a-&sh-&i-&\rt{guᴴk}&-μμH}
					{\xx{3>3}&\xx{pej}&\xx{stv}&\rt{know.how}&\·\xx{var}}
		\item	\vbform{ax̱ daa.ittuwdixwétl}{pfv}[obj intr, \fm{∅}, ach]{my joints got tired}
			\parencite[192.2673]{story-naish:1973}
				\vbmorph{ax̱&\gm{daa.it}-&tu-&w-&d-&i-&\rt{xwetl}&-μH}
					{\xx{1sg.o}&joint&inside&\xx{pfv}&\xx{mid}&\xx{stv}&\rt{tired}&\·\xx{var}}
			\versus \vbform{x̱at wudixwétl}{pfv}[obj intr, \fm{∅}, ach]{I got tired}
				\vbmorph{x̱at=&wu-&d-&i-&\rt{xwetl}&-μH}
					{\xx{1sg.o}&\xx{pfv}&\xx{mid}&\xx{stv}&\rt{tired}&\·\xx{var}}
		\end{itemize}
	\item	in a few compound nouns, some of which could be derived from verbs
		\begin{itemize}
		\item	\fm{daa.ittunéekw} ‘arthritis’
				\vbmorph{\gm{daa.it}-&tu-&\rt{nikw}&-μμH}
					{joint&inside&\rt{sick}&\·\xx{var}}
		\item	\fm{du daa.itnagóowu} ‘his/her behaviour, actions, mannerisms’
				\vbmorph{du&\gm{daa.it}-&na-&\rt{gu}&-μμH&-wu}
					{\xx{3h.poss}&joint&\xx{ncnj}&\rt{stab}&\·\xx{var}&\·\xx{poss}}
			\versus \vbform{awsigoo}{pfv}[tr, \fm{n}, mot]{she/he/it poked, stabbed him/her/it}
				\vbmorph{a-&w-&s-&i-&\rt{gu}&-μμL}
					{\xx{3>3}&\xx{pfv}&\xx{xtn}&\xx{stv}&\rt{stab}&\·\xx{var}}
		\item	\fm{du daa.ittutseení} ‘his/her bodily strength’
				\vbmorph{du&\gm{daa.it}-&tu-&\rt{tsin}&-μμL&-í}
					{\xx{3h.poss}&joint&inside&\rt{strong}&\·\xx{var}&\·\xx{poss}}
		\item	\fm{du daa.itwéisʼi} ‘his/her gland’
				\vbmorph{du&\gm{daa.it}-&\rt{wesʼ}&-μμH&-i}
					{\xx{3h.poss}&joint&\rt{louse}&\·\xx{var}&\·\xx{poss}}
		\item	\fm{du daa.itx̱ʼáak} ‘his/her interarticular space, space between joints’
				\vbmorph{du&\gm{daa.it}-&\rt{x̱ʼak}&-μμH}
					{\xx{3h.poss}&joint&\rt{between}&\·\xx{var}}
		\item	\fm{hít daa.itdeidí} ‘house timbers’
				\vbmorph{hít&\gm{daa.it}-&daa-&.eit&-i}
					{house&joint&around&thing&\·\xx{poss}}
		\end{itemize}
	\end{enumerate}

\item[dag̱a-]
	allomorph of distributive or non-human pluralizer \fm{dax̱=};
	position of this allomorph is uncertain as it is only attested in forms without
	argument or aspectual prefixes

\item[dákde=]
	directional preverb ‘out to sea (away from land)’
		with allative postposition \fm{-dé} \~\ \fm{-de} ‘toward’
	derived from directional noun \fm{dáak} ‘out at sea’
		(compare \fm{daak=})
\item[dáḵde=]
	directional noun \fm{dáaḵ} ‘inland (away from water body)’
		with allative postposition \fm{-dé} \~\ \fm{-de} ‘toward’
	derived from directional noun \fm{dáaḵ} ‘inland’
		(compare \fm{dáag̱i=}, \fm{daaḵ=}; noun \fm{daḵká} ‘on inland’)

\item[dax̱=]
	distributive pluralizer or non-human pluralizer;
	can occur before human pluralizer \fm{has=} but not after

\item[deik=]
	variant form of preverb \fm{daak=} ‘out to sea’
		used in Southern and Transitional Northern communities;
	the reason for using \fm{deik=} versus \fm{daak=} is still unclear;
	compare similar \fm{deiḵ=} versus \fm{daaḵ=}
	\begin{itemize}
	\item	\vbform{deik ḵoowatín}{pfv}[subj intr, \fm{∅}, ach+mot]{he has gotten vision}
		(Southern dialect) \parencite[06/212]{leer:1973}
			\vbmorph{\gm{deik=}&ḵu-&μʷ-&wa-&\rt[²]{tin}&-μH}
				{out&\xx{areal}&\xx{pfv}&\xx{stv}&\rt[²]{see}&\·\xx{var}}
		\versus \vbform{daak ḵoowatín}{pfv}{he has gotten vision} (Northern dialect)
			\vbmorph{daak=&ḵu-&μʷ-&wa-&\rt[²]{tin}&-μH}
				{out&\xx{areal}&\xx{pfv}&\xx{stv}&\rt[²]{see}&\·\xx{var}}
	\end{itemize}

\item[deiḵ=]
	variant form of preverb \fm{daaḵ=} ‘inland’ 
		used in Southern and Transitional Northern communities;
	the reason for using \fm{deiḵ=} versus \fm{daaḵ=} is still unclear;
	compare similar \fm{deik=} versus \fm{daak=}
	\begin{itemize}
	\item	\vbform{i chkáx̱ deiḵ tí}{imp}[tr, \fm{∅}, mot]{put it (glove) on your hand}
		(Southern dialect) \parencite[05/79]{leer:1973}
			\vbmorph{i&ji-&ká&-x̱&\gm{deiḵ=}&\rt[²]{ti}&-μH}
				{\xx{2sg.psr}&hand&\xx{hsfc}&\·\xx{pert}&on&\rt[²]{handle}&\·\xx{var}}
		\versus \vbform{i jikáx̱ daaḵ tí}{imp}{put it (glove) on your hand} (Northern dialect)
			\vbmorph{i&ji-&ká&-x̱&daaḵ=&\rt[²]{ti}&-μH}
				{\xx{2sg.psr}&hand&\xx{hsfc}&\·\xx{pert}&on&\rt[²]{handle}&\·\xx{var}}
	\end{itemize}

\item[di]\label{m:di}
	≡ \fm{d-i-}
	combination of voice \X{d-}
		and stative \X[i-stv]{i-}
	\begin{itemize}
	\item	\vbform{sh tuditéen}{impfv}[tr, \fm{∅}, \fm{-μμH} state]{we can see ourselves}
			\vbmorph{sh=&tu-&\gm{d-}&\gm{i-}&\rt[²]{tin}&-μμH}
				{\xx{rflx.o}&\xx{1pl.s}&\xx{mid}&\xx{stv}&\rt[²]{see}&\·\xx{var}}
		\versus \vbform{tuwatéen}{impfv}{we can see him/her/it}
			\vbmorph{tu-&wa-&\rt[²]{tin}&-μμH}
				{\xx{1pl.s}&\xx{stv}&\rt[²]{see}&\·\xx{var}}
	\end{itemize}

\item[dji]\label{m:dji}
	 = \X{ji} ≡ \fm{d-sh-i-}
		combination of voice \X{d-}
			and valency \X{sh-}
			and stative \X[i-stv]{i-};
	the form \fm{dji} is occasionally used instead of \X{ji}
		to explicitly indicate the presence of both \fm{d-} and \fm{sh-}
		which are otherwise represented as a single letter \fm{j} [\ipa{tʃ}];
	compare \X{dli} and \X{dzi} where the affricates [\ipa{tɬ}] and [\ipa{ts}]
		are written with two letters;
	this practice is rare and only occurs in linguistic and teaching contexts

\item[dli]\label{m:dli}
	≡ \fm{d-l-i-}
	combination of voice  \X{d-},
		valency \X{l-}/\X{lˢ-},
		and stative \X[i-stv]{i-}
	\begin{itemize}
	\item	\vbform{sh wutudlitlʼíx}{pfv}[tr, \fm{∅}, ach]{we made ourselves dirty}
			\vbmorph{sh=&wu-&tu-&\gm{d-}&\gm{l-}&\gm{i-}&\rt[¹]{tlʼix}&-μH}
				{\xx{rflx.o}&\xx{pfv}&\xx{1pl.s}&\xx{mid}&\xx{csv}&\xx{stv}&\rt[¹]{dirt}&\·\xx{var}}
		\versus \vbform{wutulitlʼíx}{pfv}{we made him/her/it dirty}
			\vbmorph{wu-&tu-&l-&i-&\rt[¹]{tlʼix}&-μH}
				{\xx{pfv}&\xx{1pl.s}&\xx{csv}&\xx{stv}&\rt[¹]{dirt}&\·\xx{var}}
	\end{itemize}

\item[du-]\label{m:du-}
	\begin{enumerate}
	\item	indefinite human subject of transitive verbs;
		see \fm{ḵaa=} and \fm{ḵu-} for indefinite human object,
		and see \fm{a-} for indefinite human subject of subject intransitive verbs
		\begin{itemize}
		\item	\vbform{x̱at wuduwax̱oox̱}{pfv}[tr, \fm{g̱}, ach]{someone/people summoned me}
				\vbmorph{x̱at=&wu-&\gm{du-}&wa-&\rt[²]{x̱ux̱}&-μμL}
					{\xx{1sg.o}&\xx{pfv}&\xx{ind.h.s}&\xx{stv}&\rt[²]{summon}&\·\xx{var}}
			\versus \vbform{x̱at woox̱oox̱}{pfv}{she/he/it summoned me}
				\vbmorph{x̱at=&wu-&μ-&\rt[²]{x̱ux̱}&-μμL}
					{\xx{1sg.o}&\xx{pfv}&\xx{stv}&\rt[²]{summon}&\·\xx{var}}
		\item	\vbform{x̱at wududziteen}{pfv}[tr, \fm{g̱}, ach]{someone/people saw me}
				\vbmorph{x̱at=&wu-&\gm{du-}&d-&s-&i-&\rt[²]{tin}&-μμL}
					{\xx{1sg.o}&\xx{pfv}&\xx{ind.h.s}&\xx{mid}&\xx{xtn}&\xx{stv}&\rt[²]{see}&\·\xx{var}}
			\versus \vbform{x̱at wusiteen}{pfv}{she/he/it saw me}
				\vbmorph{x̱at=&wu-&s-&i-&\rt[²]{tin}&-μμL}
					{\xx{1sg.o}&\xx{pfv}&\xx{xtn}&\xx{stv}&\rt[²]{see}&\·\xx{var}}
		\end{itemize}
	\item	indefinite experiencer subject;
		essentially the same as the indefinite human subject but frozen in certain sets of
		verbs describing impersonal experiences (e.g.\ feel of wind, flavour of food);
		cannot be replaced by some other referent
		\begin{itemize}
		\item	\vbform{xóon wuduwanúk}{pfv}[obj intr, \fm{∅}, ach]{north wind was felt}
				\vbmorph{xóon&wu-&\gm{du-}&wa-&\rt[²]{nuk}&-μH}
					{n·wind& \xx{pfv}&\xx{ind.s}&\xx{stv}&\rt[²]{feel}&\·\xx{var}}
		\end{itemize}
	\item	expletive/filler subject;
		occurs in some verbs to fill the subject position without referring to anything;
		cannot be replaced by some other referent
		\begin{itemize}
		\item	\vbform{x̱at kawduwasáy}{pfv}[obj intr, \fm{∅}, ach]{I got hot/sweaty}
				\vbmorph{x̱at=&ka-&w-&\gm{du-}&wa-&\rt[¹]{saÿ}&-μH}
					{\xx{1sg.o}&\xx{qual}&\xx{pfv}&\xx{xpl}&\xx{stv}&\rt[¹]{radiate}&\·\xx{var}}
		\item	\vbform{haa kawduwakʼéin}{pfv}[obj intr, \fm{g}, mot]{we jumped}
				\vbmorph{haa=&ka-&w-&\gm{du-}&wa-&\rt[¹]{kʼeᴴn}&-μμH}
					{\xx{1pl.o}&\xx{qual}&\xx{pfv}&\xx{xpl}&\xx{stv}&\rt[¹]{jump}&\·\xx{var}}
			\versus \vbform{x̱wajikʼéin}{pfv}[subj intr, \fm{g}, mot]{I jumped}
				\vbmorph{ʷ-&x̱a-&d-&sh-&i-&\rt[¹]{kʼeᴴn}&-μμH}
					{\xx{pfv}&\xx{1sg.s}&\xx{mid}&\xx{pej}&\xx{stv}&\rt[¹]{jump}&\·\xx{var}}
		\end{itemize}
	\end{enumerate}

\item[duk-]
	inalienable incorporated noun \fm{dook} ‘skin’

\item[dzi]\label{m:dzi}
	≡ \fm{d-s-i-}
	combination of voice \X{d-},
		valency \X{s-},
		and stative \X[i-stv]{i-}
	\begin{itemize}
	\item	\vbform{sh wutudzi.ée}{pfv}[tr, \fm{∅}, \fm{-μμH} act]{we cooked ourselves}
			\vbmorph{sh=&wu-&tu-&\gm{d-}&\gm{s-}&\gm{i-}&\rt[¹]{.i}&-μμH}
				{\xx{rflx.o}&\xx{pfv}&\xx{1pl.s}&\xx{mid}&\xx{csv}&\xx{stv}&\rt[¹]{cooked}&\·\xx{var}}
		\versus \vbform{wutusi.ée}{pfv}{we cooked him/her/it}
			\vbmorph{wu-&tu-&s-&i-&\rt[¹]{.i}&-μμH}
				{\xx{pfv}&\xx{1pl.s}&\xx{csv}&\xx{stv}&\rt[¹]{cooked}&\·\xx{var}}
	\end{itemize}
\end{morphdesc}

