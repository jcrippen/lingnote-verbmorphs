%!TEX root = ../lingnote-verbmorphs.tex

\subsection{X̱}\label{sec:alphalist-xh}
\begin{morphdesc}[resume*=alphalist]
\item[x̱-]\label{m:x̱-1sg}
	first person singular subject
	\newline
	allomorphs:
	\begin{allolist}
	\item[x̱-]	basic form
	\item[\X{x̱a-}]	form with epenthetic (filler) vowel \fm{á}
	\end{allolist}
	\begin{itemize}
	\item	\vbform{laaḵʼásk kax̱satʼaak}{impfv}[tr, \fm{∅}, \fm{-μμL} act]{I am pressing black seaweed}
			\vbmorph{laaḵʼásk&ka-&\gm{x̱-}&sa-&\rt[¹]{tʼak}&-μμL}
				{blk.seaweed&\xx{hsfc}&\xx{1sg.s}&\xx{csv}&\rt[¹]{pressed}&\·\xx{var}}
		\versus \vbform{laaḵʼásk x̱ax̱á}{impfv}[tr, \fm{∅}, \fm{-μH} act]{I am eating black seaweed}
			\vbmorph{laaḵʼásk&x̱a-&\rt[²]{x̱a}&-μH}
				{blk.seaweed&\xx{1sg.s}&\rt[²]{eat}&\·\xx{var}}
	\end{itemize}

\item[x̱-]\label{m:x̱-g̱cnj}
	allomorph of conjugation \fm{g̱-} when in a syllable coda
	\begin{itemize}
	\item	\fm{kax̱lax̱óotʼ} (imp; tr, \fm{g̱}, \fm{-μμH} act) ‘(you sg.)\ chop/adze it!’ in syllable \fm{kax̱}\newline
		(not \fm[*]{kag̱alax̱óotʼ} or \fm[*]{kaḵlax̱óotʼ})\newline
		versus \fm{kag̱aylax̱óotʼ} ‘you pl.\ chop/adze it!’ in syllable \fm{g̱ay}
	\end{itemize}

\item[x̱-]\label{m:x̱-mod}
	allomorph of modality \fm{g̱-} when in a syllable coda
	\begin{itemize}
	\item	\fm{at gax̱toox̱áa} (prosp; tr, \fm{∅}, \fm{-μH} act) ‘we will eat something’ with \fm{x̱-}\newline
		(not \fm[*]{at gag̱atoox̱áa} or \fm[*]{at gaḵtoox̱áa})\newline
		versus \fm{at gug̱ax̱áa} (prosp) ‘s/he/it will eat something’ with \fm{g̱a-}
	\end{itemize}

\item[-x̱]\label{m:-x̱}
	repetitive suffix indicating iteration of an eventuality;
	occurs in several different contexts:
		(items \ref{item:-x̱-rep-conjclass}, 
			\ref{item:-x̱-rep-motderiv},
			\ref{item:-x̱-rep-oddrep})
			in repetitive imperfective forms,
		(\ref{item:-x̱-rep-derivasp})
			in secondary aspect derivation paradigms
			based on \fm{-x̱} repetitive imperfective forms,
		(\ref{item:-x̱-rep-combo})
			in combination with other repetitive
			or derivational suffixes,
		(\ref{item:-x̱-rep-roots})
			and perhaps frozen in some stems and roots;
		the association between \fm{-x̱} and the semantically unspecified
			\fm{∅} conjugation class could imply that
			\fm{-x̱} is a relatively abstract or generic
			kind of iterativity (similar to \X{-ch})
			in contrast with more specific kinds of iterativity 
			that are expressed by other repetitive suffixes;
	phonologically identical with the pertingent postposition \fm{-x̱} ‘of, contacting’
		and possibly from the same origin especially
		given the use of the postposition \fm{-x̱}
		with repetitive imperfective aspect forms of the motion derivation
		\motderiv{NP-t/x̱/dé}{∅, \fm{-μμL} rep}{arriving at NP}
		and the motion derivation
		\motderiv{NP-x̱}{∅, \fm{-x̱} rep}{stuck in place at NP}
		which provides a \fm{-x̱} repetitive imperfective
		(see item \ref{item:-x̱-rep-motderiv-x̱} below) 
	\newline
	allomorphs:
	\begin{allolist}
	\item[-x̱]	basic form
	\item[\X{-x̱w}]	form with explicit labialization
	\end{allolist}
	\begin{enumerate}
	\item	\label{item:-x̱-rep-conjclass}
		repetitive suffix in repetitive imperfective forms predicted
			by lexically specified \fm{∅} conjugation class;
		this includes both the forms that \citeauthor{leer:1991}
			considers to be repetitive imperfectives
			as well as forms that he considers to be “primary imperfectives”
			(thus “A-x̱” in \cite[iii]{leer:1978b}
			and “x̣-Processive” in \cite[245]{leer:1991}),
			see \cite[119]{crippen:2019} for discussion;
		\citeauthor{leer:1991} identifies \fm{-x̱} repetitive imperfectives as
			“the default Imperfective for causatives” \parencite[245]{leer:1991}
			but does not clarify if this is due to a causative
			form being assigned \fm{∅} conjugation class membership
			so this area needs further investigation
	\item	\label{item:-x̱-rep-motderiv}
		repetitive suffix in repetitive imperfective forms
			arising from motion derivations that assign
			\fm{∅} conjugation class;
		these are identical to the \fm{-x̱} repetitive imperfectives
			from lexically specified \fm{∅} conjugation class
			(see \ref{item:-x̱-rep-conjclass} above),
			with the only difference being that they arise from
			motion derivations rather than lexically specified conjugation class;
		only the motion derivations listed here provide
			\fm{-x̱} repetitive imperfective forms
			and other \fm{∅} conjugation class motion derivations
			instead provide repetitive imperfective forms with
				\X{-μμL},
				\X{-ch},
				or \X{-k}
		\begin{itemize}
		\item[◦] \X{g̱unayéi=} \~\ \X{g̱unéi=}…\fm{-x̱}
			(\ref{item:-x̱-rep-motderiv-g̱unayéi})
		\item[◦] \fm{NP-xʼ} …\fm{-x̱}
			(\ref{item:-x̱-rep-motderiv-loc})
		\item[◦]	\fm{NP-xʼ} \X[ÿax̱=facing]{ÿax̱=}…\fm{-x̱}
			(\ref{item:-x̱-rep-motderiv-loc+yax̱})
		\item[◦]	\fm{\xx{pvb}-i=}…\fm{-x̱}
			(\ref{item:-x̱-rep-motderiv-i})
		\item[◦]	\fm{NP-x̱} …\fm{-x̱}
			(\ref{item:-x̱-rep-motderiv-x̱})
		\item[◦]	\X{ÿetx̱=} \~\ \X{ÿedax̱=}…\fm{-x̱}
			(\ref{item:-x̱-rep-motderiv-yetx̱})
		\item[◦]	\X{a-}\X[ÿa-face]{ÿa-}\X[u-irr]{u-}\X{d-}…\fm{-x̱}
			(\ref{item:-x̱-rep-motderiv-ayaud})
		\end{itemize}
		\begin{enumerate}
		\item	\label{item:-x̱-rep-motderiv-g̱unayéi}
			\fm{∅} conjugation class repetitive imperfectives with
			\X{g̱unayéi=} \~\ \X{g̱unéi=} ‘beginning, starting, initiating’
			\begin{itemize}
			\item	\fm{–ḵúx̱x̱}
				from \fm{\rt[¹]{ḵux̱}} ‘go by boat, other vehicle’
				\newline
				using \motderiv{g̱unayéi=}{\fm{∅}, \fm{-x̱} rep}{start, begin} in
				\newline
				\vbform{g̱unayéi ḵúx̱x̱}{rep impfv}[subj intr, \fm{∅}, mot]{she/he/it repeatedly starts going by boat}
					\parencite[57]{story:1966}
					\vbmorph{g̱unayéi=&\rt[¹]{ḵux̱}&-μH&\gm{-x̱}}
						{begin&\rt[¹]{go.boat}&\·\xx{var}&\·\xx{rep}}
				\versus \vbform{g̱unayéi uwaḵúx̱}{pfv}{she/he/it started going by boat}
					\vbmorph{g̱unayéi=&u-&wa-&\rt[¹]{ḵux̱}&-μH}
						{begin&\xx{zpfv}&\xx{stv}&\rt[¹]{go.boat}&\·\xx{var}}
			\end{itemize}
		\item	\label{item:-x̱-rep-motderiv-loc}
			\fm{∅} conjugation class repetitive imperfectives with
			a postposition phrase headed by locative \fm{-xʼ}
			or its allomorph \fm{-μ}
			\begin{itemize}
			\item	
			\end{itemize}
		\item	\label{item:-x̱-rep-motderiv-loc+yax̱}
			\fm{∅} conjugation class repetitive imperfectives with
			a postposition phrase headed by locative \fm{-xʼ}
			or its allomorph \fm{-μ}
			and the preverb \X[ÿax̱=facing]{ÿax̱=} ‘facing’
			\begin{itemize}
			\item	\fm{–gwátlx̱}
				from \fm{\rt[¹]{gwaᴴtl}} ‘roll, rock’
				\newline
				using \motderiv{NP-xʼ ÿax̱=}{\fm{∅}, \fm{-x̱} rep}{turning over at/by NP} in
				\newline
				\vbform{áa yax̱ gwátlx̱}{rep impfv}[obj intr, \fm{∅}, mot]{she/he/it repeatedly capsizes there}
					\vbmorph{á&-μ&ÿax̱=&\rt[¹]{gwatl}&-μH&-\gm{x̱}}
						{\xx{3n}&\·\xx{loc}&facing&\rt[¹]{roll}&\·\xx{var}&\·\xx{rep}}
				\versus \vbform{áa yax̱ uwagwátl}{pfv}{she/he/it capsized there}
					\vbmorph{á&-μ&ÿax̱=&u-&wa-&\rt[¹]{gwatl}&-μH&-x̱}
						{\xx{3n}&\·\xx{loc}&facing&\xx{zpfv}&\xx{stv}&\rt[¹]{roll}&\·\xx{var}&\·\xx{rep}}
			\end{itemize}
		\item	\label{item:-x̱-rep-motderiv-i}
			\fm{∅} conjugation class repetitive imperfectives with
			a preverb ending with the unique postposition \fm{-i}
			\begin{itemize}
			\item	\fm{–gútx̱}
				from \fm{\rt[¹]{gut}} ‘singular go’
				\newline
				using \motderiv{gági=}{\fm{∅}, \fm{-x̱} rep}{emerging, out into open} in
				\newline
				\vbform{gági gútx̱}{rep impfv}[subj intr, \fm{∅}, mot]{she/he/it repeatedly goes out into the open}
					\vbmorph{gág&-i=&\rt[¹]{gut}&-μH&-x̱}
						{protrude&\·\xx{loc}&\rt[¹]{go.\xx{sg}}&\·\xx{var}&\·\xx{rep}}
				\versus \vbform{gági uwagút}{pfv}{she/he/it went out into the open}
					\parencite[20.78]{story-naish:1973}
					\vbmorph{gág&-i=&u-&wa-&\rt[¹]{gut}&-μH}
						{protrude&\·\xx{loc}&\xx{zpfv}&\xx{stv}&\rt[¹]{go.\xx{sg}}&\·\xx{var}}
			\end{itemize}
		\item	\label{item:-x̱-rep-motderiv-x̱}
			\fm{∅} conjugation class repetitive imperfectives with
			a postposition phrase headed by pertingent \fm{-x̱}
			\begin{itemize}
			\item	
			\end{itemize}
		\item	\label{item:-x̱-rep-motderiv-yetx̱}
			\fm{∅} conjugation class repetitive imperfectives with
			the preverb \X{ÿetx̱=} \~\ \X{ÿedax̱=}
			\begin{itemize}
			\item	
			\end{itemize}
		\item	\label{item:-x̱-rep-motderiv-ayaud}
			\fm{∅} conjugation class repetitive imperfectives with
			the perambulative revertive motion derivation
			containing \X{a-} + \X[ÿa-face]{ÿa-} + \X[u-irr]{u-} + \X{d-}
			\begin{itemize}
			\item	
			\end{itemize}
		\end{enumerate}
	\item	\label{item:-x̱-rep-oddrep}
		repetitive suffix in unexpected repetitive imperfective forms;
		these are cases where a verb has a repetitive imperfective form with \fm{-x̱}
			that is not predicted from other grammatical properties like
			\fm{∅} conjugation class membership
			or the application of a \fm{∅} conjugation class motion derivation;
		excluded here are what \citeauthor{leer:1991} considers “primary imperfectives”
			with \fm{-x̱} that belong to the \fm{∅} conjugation class
			(represented as “A-x̱” in \cite[iii]{leer:1978b}, or as
			“x̣-Processive” in \cite[245]{leer:1991})
			because these are predicted by their conjugation class membership,
			so the following are only those from \fm{n}, \fm{g̱}, or \fm{g̱}
			conjugation classes
		\begin{enumerate}
		\item	non-\fm{∅} conjugation class achievement verbs
			that have a repetitive imperfective with \fm{-x̱}
			\begin{itemize}
			\item	\fm{–.éix̱} ‘end move, grow’
				from \fm{\rt{.a}} ‘end move, grow’
			\item	\fm{–.eix̱} ‘answer call’
				from \fm{\rt{.e}} ‘answer call’
					(invariable \X{-μμL} stem \fm{–.ei})
			\item	\fm{–héix̱} ‘dig, move dirt’
				from \fm{\rt{ha}} ‘dig, move dirt’
			\item	\fm{–sʼéḵx̱} ‘tan by smoking’
				from \fm{\rt{sʼeḵ}} ‘smoke’
			\end{itemize}
		\item	non-\fm{∅} conjugation class activity verbs
			that have a repetitive imperfective with \fm{-x̱}
			\begin{itemize}
			\item	\fm{–kʼwátʼx̱} ‘lay egg’
				from \fm{\rt{kʼwatʼ}} ‘egg’
			\item	\fm{–teix̱} ‘sg.\ sleep’
				from \fm{\rt{taᴸ}} ‘sg.\ sleep’
			\item	\fm{–tóowx̱} \~\ \fm{téewx̱} ‘read’
				from \fm{\rt{tuᴴw}} \~\ \fm{\rt{tiᴴw}} ‘count, read’
			\end{itemize}
		\item	non-\fm{∅} conjugation class state verbs
			that have a repetitive imperfective with \fm{-x̱}
			\begin{itemize}
			\item	\fm{–geix̱} ‘become big’
				from \fm{\rt{ge}} ‘big, much’
			\item	\fm{–kaakx̱} ‘thick’
				from \fm{\rt{kak}} ‘thick’
			\item	\fm{–kákx̱} ‘thick’
				from \fm{\rt{kak}} ‘thick’
			\item	\fm{–teex̱} ‘be’
				from \fm{\rt{tiᴸ}} ‘be’
			\item	\fm{–ÿátʼx̱} ‘become long’
				from \fm{\rt[¹]{ÿatʼ}} ‘long’
			\end{itemize}
		\item	state verbs that have an unexpected stative repetitive imperfective
			with \X[i-stv]{i-} \~\ \X[ÿa-stv]{ÿa-} and \fm{-x̱}
			\begin{itemize}
			\item	\fm{–chʼáchʼx̱} ‘spotted’
				from \fm{\rt[¹]{chʼachʼ}} ‘spotted’ in
				\newline
				\vbform{kadlichʼáchʼx̱}{rep impfv}[obj intr, \fm{∅}, \fm{-x̱} state]{she/he/it is spotted}
				\parencites[206.2878]{story-naish:1973}[10/240]{leer:1973}
					\vbmorph{ka-&d-&lˢ-&i-&\rt[¹]{chʼachʼ}&-μH&-x̱}
						{\xx{hsfc}&\xx{mid}&\xx{xtn}&\xx{stv}&\rt[¹]{spotted}&\·\xx{var}&\·\xx{rep}}
				\versus \vbform{kawdichʼáchʼ}{pfv}{she/he/it has gotten spots}
				\parencites[206.2876]{story-naish:1973}[10/240]{leer:1973}
					\vbmorph{ka-&w-&d-&i-&\rt[¹]{chʼachʼ}&-μH}
						{\xx{hsfc}&\xx{pfv}&\xx{mid}&\xx{stv}&\rt[¹]{spotted}&\·\xx{var}}
				\newline
				compare \fm{\rt{chʼalʼ}} ‘pale’ below
			\item	\fm{–chʼálʼx̱} ‘pale spotted’
				from \fm{\rt[¹]{chʼalʼ}} ‘pale’
				\newline
				\vbform{kakaachʼálʼx̱}{rep impfv}[obj intr, conj?, \fm{-x̱} state]{she/he/it has (pale) spots}
				\parencite[10/238]{leer:1973}
					\vbmorph{ka-&ka-&μ-&\rt[¹]{chʼalʼ}&-μH&-x̱}
						{\xx{hsfc}&\xx{qual}&\xx{stv}&\rt[¹]{pale}&\·\xx{var}&\·\xx{rep}}
				\newline
				no other forms of this verb are attested;
				compare the nouns
					\fm{chʼáalʼ} ‘pale (skin)’
						\parencites[760]{kelly-willard:1905}[10/238]{leer:1973}
					and \fm{chʼáalʼ} ‘willow’,
					also \fm{tlʼáatlʼ} ‘something yellow’
						\parencite[08/220]{leer:1973},
					\fm{sʼáasʼ} ‘pine siskin, goldfinch’
						\parencite[09/196]{leer:1973}
			\end{itemize}
		\end{enumerate}
	\item	\label{item:-x̱-rep-derivasp}
		repetitive suffix in secondary aspect derivation paradigms
		based on \fm{-x̱} repetitive imperfective forms
		\begin{itemize}
		\item	\fm{−teex̱}
			from \fm{\rt[¹]{tiᴸ}} ‘be, exist’ in
			\newline
			\vbform{áxʼ yéi haa wooteex̱}{rep pfv}[obj intr, \fm{n}, state]{we were living there}
			\parencite[74.33]{dauenhauer-dauenhauer:1987}
				\vbmorph{á&-xʼ&yéi=&haa=&wu-&μ-&\rt[¹]{tiᴸ}&-μμL&\gm{-x̱}}
					{\xx{3n}&\·\xx{loc}&thus&\xx{1pl.o}&\xx{pfv}&\xx{stv}&\rt[¹]{be}&\·\xx{var}&\·\xx{rep}}
		\end{itemize}
	\item	\label{item:-x̱-rep-combo}
		in combination with other repetitive or derivational suffixes
		\begin{enumerate}
		\item	with \X{-aa} \~\ \X{-áa} as \X{-x̱aa} \~\ \X{-x̱áa};
			see \X{-x̱aa} for more detail
		\item	with \X{-xʼ} as \X{-x̱xʼ} \~\ \X{-x̱wxʼ} \~\ \X{-x̱xʼw};
			see \X{-x̱xʼ} for more detail
		\end{enumerate}
	\item	\label{item:-x̱-rep-roots}
		potentially identifiable as a frozen suffix in some stems and roots
		\begin{enumerate}
		\item	CVCC stems with possible frozen \fm{-x̱};
			this excludes forms ending with \fm{…tx̱} [\ipa{tχ}]
				which are contractions of the postposition \fm{-dáx̱} ‘from’
				(such as \fm{tuwáatx̱} ‘by virtue of’
					from \fm{tú} ‘inside’
					+ \fm{ÿá} ‘face’
					+ \fm{-dáx̱} ‘from’)
			as well as forms that may instead have the pertingent postposition
				\fm{-x̱} ‘of, contacting’
			\begin{itemize}
			\item	\fm{húnx̱w} ‘older brother of man’
				from unknown \fm{\rt{hun}};
				compare independent third human pronoun \fm{hú} ‘him/her’
			\item	\fm{ḵúlx̱w} ‘mud pushed up in wrinkles’
				from unknown \fm{\rt{ḵul}},
				probably related to \fm{ḵútlʼkw} ‘mud’
				from unknown \fm{\rt{ḵutlʼ}};
				compare \fm{ḵúlkw} ‘rotten wood’
				and \fm{\rt{hutlʼ}} ‘wrinkled’
			\item	\fm{kʼwálx̱} ‘fiddlehead fern’
				from unknown \fm{\rt{kʼwal}}
			\item	\fm{náalx̱} ‘wealth, riches; large halibut’
				from unknown \fm{\rt{nal}};
				compare \fm{\rt{na}} ‘drink, dampen, oil’,
				\fm{\rt{nal}} ‘steam, blow nose’,
				\fm{\rt{natlʼ}} \~\ \fm{\rt{natsʼ}} ‘pruned, wrinkled from water; clumsy’,
				\fm{\rt{naᴴsh}} ‘shake’
			\item	\fm{shakeedatáxʼx̱} ‘white-crowned sparrow’
				probably from \fm{\rt{taxʼ}} ‘bite’
				and so literally ‘repeatedly bitten on top’
			\item	\fm{tlʼikʼwx̱} \~\ \fm{tlʼúkʼx̱} ‘worm, larva, caterpillar, snake’
				from unknown \fm{\rt{tlʼikʼw}} \~\ \fm{\rt{tlʼukʼ}};
				compare \fm{\rt{tlʼakʼ}} ‘wet’,
				\fm{\rt{tlʼeᴴl}} ‘fish guts, milt’ (noun \fm{tlʼéil}),
				\fm{tlʼíl} ’penis’
				\fm{\rt{tlʼix}} ‘trash, dirt’,
				\fm{\rt{tlʼiḵ}} \~\ \fm{\rt{tlʼeḵ}} ‘finger’,
				\fm{\rt{tlʼuḵ}} ‘rotten’
					(noun \fm{tlʼooḵ} ‘scab, sore’)
			\item	\fm{shátx̱} ‘older sibling of woman’
				perhaps from \fm{shaawát} ‘girl, woman’
				itself from \fm{sháa} ‘woman’ (root \fm{\rt{shaʷ}} ‘woman; marry’)
				+ \fm{ÿát} ‘child’
			\end{itemize}
		\item	CVC roots and stems with possible frozen \fm{-x̱};
			this excludes forms that may instead have
				the postposition \fm{-x̱} ‘of, contacting’
			\begin{itemize}
			\item	\fm{eex̱} \~\ \fm{eix̱} ‘grease, oil’;
				compare \fm{\rt{.i}} ‘cooked’
			\item	\fm{\rt{g̱ax̱}} ‘cry’
				(noun \fm{g̱aax̱} ‘crying’);
				compare \fm{\rt{g̱a}} ‘bashful’,
				\fm{\rt{g̱as}} ‘taboo, proscribed’
			\item	\fm{\rt{tax̱}} ‘dewy, wet with with droplets’
				(noun \fm{ḵukatáx̱} ‘dew, water drops’);
				compare \fm{\rt{ta}} ‘boil’,
				\fm{\rt{ta}} ‘fat’ (noun \fm{taaÿ})
			\item	\fm{\rt{lax̱}} ‘die (tree, bush)’
				(noun \fm{láax̱} ‘standing dead tree’);
				compare \fm{\rt{laxw}} ‘starve’
				(noun \fm{laaxw} ‘starvation’)
			\item	\fm{\rt{lʼex̱}} ‘dance’
				(noun \fm{alʼeix̱} ‘dance’);
				compare \fm{\rt{lʼe}} ‘promiscuous’,
				\fm{lʼée} ‘blanket used as wealth’,
				\fm{guklʼéinxw} ‘wool ear ornaments’
			\item	\fm{\rt{nix̱}} \~\ \fm{\rt{nex̱}} ‘safe, rescued’
				(noun \fm{g̱aneex̱} \~\ \fm{g̱aneix̱} ‘recovery, salvation’);
				compare \fm{\rt{niᴸ}} \~\ \fm{\rt{neᴸ}} ‘happen, occur’
			\item	\fm{sʼaax̱} ‘marmot, groundhog, whistler’;
				compare \fm{kanalsʼáak} ‘squirrel’ with unknown \fm{\rt{sʼak}}
			\item	\fm{sóox̱} ‘early, ahead, too soon’
				(also \fm{woosh jisóox̱} ‘trying to beat each other to it’);
				compare \fm{súg̱aa} ‘for the future’
					(probably with postposition \fm{-g̱áa}),
				\fm{yeisú} ‘still, yet’ (with \fm[*]{ÿee} ‘time’),
				\fm{\rt{suᴸ}} ‘supernatural help’
			\item	\fm{téix̱} ‘boiled food’
				from \fm{\rt{ta}} ‘boil’
			\item	\fm{\rt{tʼex̱}} ‘fish with hook’
				(nouns \fm{tʼeix̱}, \fm{tʼéx̱aa} ‘fishhook’);
				compare \fm{\rt{tʼaᴸ}} ‘bent, dented, pressed’
					(also \fm{\rt{tʼak}}),
				\fm{\rt{tʼiᴸ}} \~\ \fm{\rt{tʼeᴸ}} ‘find’
			\item	\fm{tleix̱} ‘forever, always (future)’;
				compare \fm{tle} ‘just then, well, so’,
				\fm{tlákw} ‘always (past)’,
				\fm{tláakw} ‘quickly, angrily’
			\item	\fm{\rt{tsuᴴx̱}} \~\ \fm{\rt{tsiᴴx̱w}} ‘dam, block’
				(noun \fm{tsóox̱} ‘barricade’);
				compare \fm{\rt{tsuᴴw}} ‘push pl.\ sticks; plant with dibble stick; move household’
			\item	\fm{xʼéix̱} ‘king or spider crab’;
				compare \fm{\rt{xʼa}} ‘twist to separate fibres’
			\item	\fm{xwéix̱} ‘steambath’
				and \fm{kaxwéix̱} ‘highbush cranberry’
				from \fm{\rt{xu}} ‘steam cook’
			\item	\fm{x̱éex̱} ‘boreal or saw-whet owl’;
				compare \fm{\rt{x̱i}} \~\ \fm{\rt{x̱e}} ‘overnight’
			\item	\fm{\rt{x̱ux̱}} ‘summon; ask for; compose song’,
				perhaps also \fm{x̱úx̱} ‘husband’;
				compare \fm{x̱oo} ‘among, amidst’
				\fm{x̱ooní} ‘relative, friend’,
				\fm{x̱ooní} \~\ \fm{x̱einí} ‘in addition’
			\item	\fm{yéix̱} ‘deadfall trap’;
				compare \fm{\rt{yex̱}} ‘make, build; whittle’,
				\fm{\rt{ya}} ‘backpack’,
				\fm{\rt{ÿa}} ‘move uncertainly; happen’,
				\fm{\rt{ÿa}} ‘lower, anchor; comb, whet’
			\end{itemize}
		\end{enumerate}
	\end{enumerate}

\item[x̱a-]\label{m:x̱a-}
	allomorph of first person singular subject \X[x̱-1sg]{x̱-} with epenthetic (filler) vowel

\item[x̱ʼa-]\label{m:x̱ʼa-}
	allomorph of \X{x̱ʼe-} ‘mouth’

\item[-x̱aa]\label{m:-x̱aa}
	amissive suffix, indicates failure of an attempt to hit a target;
	part of the ‘miss target’ derivation made up of:
		qualifier \X[ÿa-qual]{ÿa-}
		+ extensional \X{s-}/\X{lˢ-}
		+ amissive \fm{-x̱aa} \~\ \fm{-x̱áa}
		with \fm{∅} conjugation class;
	probably derived from repetitive \X{-x̱} and unknown \X{-áa} \~\ \X{-aa};
	if treated as a single suffix \fm{-x̱aa} \~\ \fm{-x̱áa}
		then it is glossed as \xx{miss}
		otherwise \xx{rep} + \xx{unkn};
	applicable to any verb that denotes striking a target in some manner,
		meaning ‘attempt to strike target but miss’;
	attested with the roots
		\begin{inlinelist}
		\item	\fm{\rt{dzu}} ‘throw at’
		\item	\fm{\rt{gwal}} ‘punch, strike’
		\item	\fm{\rt{ḵʼish}} ‘slap with stick’
		\item	\fm{\rt{shaᴴt}} ‘grab’
		\item	\fm{\rt{tʼach}} ‘slap’
		\item	\fm{\rt{tʼuᴴk}} ‘shoot (arrow)’
		\item	\fm{\rt{.uᴴn}} ‘shoot (gun)’
		\item	\fm{\rt{x̱ich}} ‘club, spank’
		\end{inlinelist}
	\newline
	allomorphs:
	\begin{allolist}
	\item[-x̱aa]	L tone form occurs after an H tone stem
	\item[\X{-x̱áa}]	H tone form occurs after an L tone stem (unattested)
	\end{allolist}
	\begin{itemize}
	\item	\vbform{ayawsi.únx̱aa}{pfv}[tr, \fm{∅}, ach]{she/he/it shot at him/her/it and missed}
			\vbmorph{a-&ÿa-&w-&s-&i-&\rt[²]{.uᴴn}&-μH&\gm{-x̱aa}}
				{\xx{3>3}&\xx{qual}&\xx{pfv}&\xx{xtn}&\xx{stv}&\rt[²]{shoot}&\·\xx{var}&\·\xx{miss}}
		\versus \vbform{aawa.ún}{pfv}{she/he/it shot him/her/it}
			\vbmorph{a-&μʷ-&wa-&\rt[²]{.uᴴn}&-μH}
				{\xx{3>3}&\xx{pfv}&\xx{stv}&\rt[²]{shoot}&\·\xx{var}}
	\item	\vbform{ayawlishátx̱aa}{pfv}[tr, \fm{∅}, ach]{she/he/it grabbed at it and missed}
			\vbmorph{a-&ÿa-&w-&lˢ-&i-&\rt[²]{shaᴴt}&-μH&\gm{-x̱aa}}
				{\xx{3>3}&\xx{qual}&\xx{pfv}&\xx{xtn}&\xx{stv}&\rt[²]{grab}&\·\xx{var}&\·\xx{miss}}
		\versus \vbform{aawasháat}{pfv}[tr, \fm{g}, ach]{she/he/it grabbed him/her/it}
			\vbmorph{a-&μʷ-&wa-&\rt[²]{shaᴴt}&-μμH}
				{\xx{3>3}&\xx{pfv}&\xx{stv}&\rt[²]{grab}&\·\xx{var}}
	\item	\vbform{ayawlidzéix̱aa}{pfv}[tr, \fm{∅}, ach]{she/he/it threw at it and missed}
			\vbmorph{a-&ÿa-&w-&\gm{lˢ-}&i-&\rt[²]{dzu}&-μᵉμH&\gm{-x̱aa}}
				{\xx{3>3}&\xx{qual}&\xx{pfv}&\xx{xtn}&\xx{stv}&\rt[²]{throw}&\·\xx{var}&\·\xx{miss}}
		\versus \vbform{aawadzóo}{pfv}[tr, \fm{∅}, ach]{she/he/it threw at him/her/it}
			\vbmorph{a-&μʷ-&wa-&\rt[²]{dzu}&-μμH}
				{\xx{3>3}&\xx{pfv}&\xx{stv}&\rt[²]{throw}&\·\xx{var}}
	\end{itemize}

\item[-x̱áa]\label{m:-x̱áa}
	allomorph of \X{-x̱aa} with H tone, used after L tone syllable (polar tone);
	unattested but predicted from patterns of \X{-aa} \~\ \X{-áa}

\item[x̱áni=]\label{m:x̱áni=}
	Locational preverb ‘near’ indicating location near some entity.
	Derived from the relational noun \fm{x̱án} ‘near’
		with the special locative postposition allomorph
		\X[-i-loc]{-i} \~\ \X[-í-loc]{-í} ‘at’
		(instead of the regular allomorphs \fm{-xʼ} and \fm{-μ}).
	The locative postposition allomorph is unique in that it only occurs
		with a small handful of preverbs such as \fm{x̱áni=},
		for which see the detailed entry of \X[-i-loc]{-i}.
	Since \fm{x̱áni=} is composed of \fm{x̱án} ‘near’ and \fm{-i} ‘at’
		it can be represented as a sequence of segments \fm{x̱án-i=}
		just as well as a single unit \fm{x̱áni=}.
	Unlike all of the other preverbs that occur with locative \X[-i-loc]{-i},
		\fm{x̱áni=} is based on an inalienable noun.
	Because of this it may occur with an overt possessor unlike other preverbs.
	\begin{enumerate}
	\item	Forms with an overt possessor immediately preceding the preverb.
		Unlike other locational and directional preverbs which usually have an implicit origo
			or an origo indicated elsewhere in the sentence,
			these forms explicitly give the origo of the preverb as the possessor.
		\begin{itemize}
		\item	\vbform{déix̱ haa x̱áni uwax̱ée}{pfv}[obj intr, \fm{∅}, ach]{he stayed with us two nights}
			\parencite[144.1949]{story-naish:1973}
				\vbmorph{déix̱&haa&\gm{x̱án}&\gm{-i=}&u-&wa-&\rt[¹]{x̱i}&-μμH}
					{two&\xx{1pl.psr}&near&\·\xx{loc}&\xx{zpfv}&\xx{stv}&\rt[¹]{overnight}&\·\xx{var}}
			\exand \vbform{ḵaa x̱áni uwax̱ée}{pfv}{he stayed overnight by them}
			\parencite[16.138]{nyman-leer:1993}
				\vbmorph{ḵaa&\gm{x̱án}&\gm{-i=}&u-&wa-&\rt[¹]{x̱i}&-μμH}
					{\xx{ind.h.psr}&near&\·\xx{loc}&\xx{zpfv}&\xx{stv}&\rt[¹]{overnight}&\·\xx{var}}
		\item	\vbform{ax̱ x̱áni gug̱atʼei}{prosp}[subj intr, \fm{n?}, ach, inv]{he will stay (at home) with me}
			\parencite[210.2934]{story-naish:1973}
				\vbmorph{ax̱&x̱án&\gm{-i=}&g-&u-&g̱a-&\rt[¹]{tʼeᴸ}&-μμL}
					{\xx{1sg.psr}&near&\·\xx{loc}&\xx{gcnj}&\xx{irr}&\xx{mod}&\rt[¹]{stay.home}&\·\xx{var}}
		\end{itemize}
	\item	Forms without an overt possessor, thus acting like a typical locational preverb
			with an implicit origo or an origo indicated elsewhere in the sentence.
		These forms are much more uncommon than the forms with an overt possessor.
		\begin{itemize}
		\item	\vbform{x̱áni yux̱ woogoot}{pfv}[subj intr, \fm{n}, mot]{she went out near (him)}
			\parencite[259.8]{swanton:1909}
				\vbmorph{\gm{x̱án}&\gm{-i=}&yux̱=&wu-&μ-&\rt[¹]{gut}&-μμL}
					{near&\·\xx{loc}&out&\xx{pfv}&\xx{stv}&\rt[¹]{go.\xx{sg}}&\·\xx{var}}
		\end{itemize}
	\end{enumerate}

\item[x̱at=]
	first person singular object;
	similar to independent pronoun \fm{x̱át} ‘me’ but with L tone instead of H tone
	\newline
	allomorphs:
	\begin{allolist}
	\item[x̱at=]	typical form
	\item[ax̱=]	possessor of incorporated noun
	\end{allolist}
	\begin{itemize}
	\item	\vbform{x̱at yisiteen}{pfv}[tr, \fm{g̱}, ach]{you saw me}
			\vbmorph{\gm{x̱at=}&ÿ-&i-&s-&i-&\rt[²]{tin}&-μμL}
				{\xx{1sg.o}&\xx{pfv}&\xx{2sg.s}&\xx{xtn}&\xx{stv}&\rt[²]{see}&\·\xx{var}}
		\versus \vbform{ix̱wsiteen}{pfv}{I saw you}
			\vbmorph{i-&ʷ-&x̱-&s-&i-&\rt[²]{tin}&-μμL}
				{\xx{1sg.o}&\xx{pfv}&\xx{1sg.s}&\xx{xtn}&\xx{stv}&\rt[²]{see}&\·\xx{var}}
	\end{itemize}

\item[x̱ʼe-]\label{m:x̱ʼe-}
	incorporated noun ‘mouth’

\item[x̱w]
	≡ \fm{ʷ-x̱-} combination of
		perfective \fm{ʷ-}
		and first person singular subject \fm{x̱-}

\item[-x̱w]\label{m:-x̱w}
	labialized form of \X{-x̱}

\item[x̱wa]
	≡ \fm{ʷ-x̱a-} combination of
		perfective \fm{ʷ-}
		and first person singular subject \fm{x̱a-}

\item[-x̱wxʼ]\label{m:-x̱wxʼ}
	≡ \fm{-x̱w-xʼ}
	combination of repetitive \X{-x̱w}
		and plural/repetitive \X{-xʼ};
	this is an orthographic variant of \X{-x̱xʼ},
		which see

\item[-x̱xʼ]\label{m:-x̱xʼ}
	≡ \fm{-x̱-xʼ}
	combination of repetitive \X{-x̱}
		and plural/repetitive \X{-xʼ},
		see those entries for further details;
	this combination is attested only with the two roots
		\fm{\rt[¹]{tiᴸ}} ‘be, exist’
		and \fm{\rt[²]{.u}} ‘put, place, leave’
		\parencite[153]{leer:1991};
	compare with \X{-kwt} ≡ \X{-kw} + \X{-t}
	\newline
	allomorphs:
	\begin{allolist}
	\item[\X{-x̱xʼ}]	basic form
	\item[\X{-x̱wxʼ}]alternate orthographic form
	\item[\X{-x̱xʼw}]alternate orthographic form
	\end{allolist}
	\begin{itemize}
	\item	\fm{–tíx̱xʼw}
		from \fm{\rt[¹]{tiᴸ}} ‘be, exist’ in
		\newline
		\vbform{áa yéi tíx̱xʼw}{rep impfv}[obj intr, \fm{n}, \fm{-μμL} state]{they are regularly there}
		\parencites[06/159]{leer:1973}[190.487, 292.17]{dauenhauer-dauenhauer:1987}[200.699]{nyman-leer:1993}
			\vbmorph{á&-μ&yéi=&\rt[¹]{tiᴸ}&-μH&-x̱&\gm{-xʼw}}
				{\xx{3n}&\·\xx{loc}&thus&\rt[¹]{be}&\·\xx{var}&\·\xx{rep}&\·\xx{rep}}
		\versus \vbform{áa yéi yatee}{impfv}{she/he/it is there}
			\vbmorph{á&-μ&yéi=&ÿa-&\rt[¹]{tiᴸ}&-μμL}
				{\xx{3n}&\·\xx{loc}&thus&\xx{stv}&\rt[¹]{be}&\·\xx{var}}
		\newline
		the labialization with \fm{-xʼw} is unexpected and the orthography
			for \fm{–tíx̱xʼw} is unclear if this is
			only on the one coda consonant [\ipa{tʰíχxʼʷ}]
			or on both coda consonants [\ipa{tʰíχʷxʼʷ}]
	\item	\fm{–.úx̱xʼ}
		from \fm{\rt[²]{.u}} ‘put, place, leave’ in
		\newline
		\vbform{áa yéi du.úx̱xʼ}{rep impfv}[tr, \fm{n}, \fm{-μμL} act]{people put it there}
		\parencites[59]{story:1966}[02/295]{leer:1973}
		\vbmorph{á&-μ&yéi=&du-&\rt[²]{.u}&-μH&-x̱&\gm{-xʼ}}
			{\xx{3n}&\·\xx{loc}&thus&\xx{ind.h.s}&\rt[²]{put}&\·\xx{var}&\·\xx{rep}&\·\xx{rep}}
		\versus \vbform{áa yéi ana.wéich}{hab}{she/he always puts it there}
		\parencite[02/296]{leer:1973}
		\vbmorph{á&-μ&yéi=&a-&na-&\rt[²]{.u}&-μᵉμH&-ch}
			{\xx{3n}&\·\xx{loc}&thus&\xx{3>3}&\xx{ncnj}&\rt[²]{put}&\·\xx{var}&\·\xx{rep}}
		\exalso \vbform{áa yéi na.oo}{imp}{put it there}
		\parencite[02/295]{leer:1973}
		\vbmorph{á&-μ&yéi=&na-&\rt[²]{.u}&-μμL}
			{\xx{3n}&\·\xx{loc}&thus&\xx{ncnj}&\rt[²]{put}&\·\xx{var}}
		\newline
		both suffixes \fm{-x̱} and \fm{-xʼ} are predictably labialized 
			in \fm{–.úx̱xʼ} [\ipa{ʔúχʷxʼʷ}]
			although this is not shown by the orthography
	\end{itemize}

\item[-x̱xʼw]\label{m:-x̱xʼw}
	≡ \fm{-x̱-xʼw}
	combination of repetitive \X{-x̱}
		and plural/repetitive \X{-xʼw};
	this is an orthographic variant of \X{-x̱xʼ},
		which see

\end{morphdesc}
