%!TEX root = ../lingnote-verbmorphs.tex

\subsection{I}\label{sec:alphalist-i}
\begin{morphdesc}[resume*=alphalist]
\item[i-]\label{m:i-2sg}
	second person singular subject or object; long vowel allomorphs are \fm{ee-} and \fm{ee=};
	subject versus object is typically distinguished by position in the verb word but can sometimes be ambiguous
	\begin{enumerate}
	\item	second person singular subject
		\begin{itemize}
		\item	\vbform{x̱at iyatéen}{impfv}[tr, \fm{g}, \fm{-μμH} state, only impfv]{you sg.\ can see me}
				\vbmorph{x̱at=&i-&ya-&\rt[²]{tin}&-μμH}
					{\xx{1sg.o}&\xx{2sg.s}&\xx{stv}&\rt[²]{see}&\·\xx{var}}
			\versus \vbform{ayatéen}{impfv}{she/he/it can see him/her/it}
				\vbmorph{a-&ya-&\rt[²]{tin}&-μμH}
					{\xx{3>3}&\xx{stv}&\rt[²]{see}&\·\xx{var}}
		\end{itemize}
	\item	second person singular оbject
		\begin{itemize}
		\item	\vbform{ix̱aatéen}{impfv}[tr, \fm{g}, \fm{-μμH} state, only impfv]{I can see you sg.}
				\vbmorph{i-&x̱a-&μ-&\rt[²]{tin}&-μμH}
					{\xx{2sg.o}&\xx{1sg.s}&\xx{stv}&\rt[²]{see}&\·\xx{var}}
		\end{itemize}
	\item	ambiguous: either second person subject or object in certain imperfective verb forms;
		when one of two arguments is third person and thus not indicated by the verb,
		the \fm{i-} argument prefix is ambiguous between subject and object and must
		be distinguished by other means (third person subject or object phrase,
		discourse context, etc.);
		this ambiguity does not arise if qualifiers, incorporated nouns, or aspect prefixes
		are present because they occur between the subject and object prefixes and so
		distinguish subject and object
		\begin{itemize}
		\item	\vbform{iyatéen}{impfv}[tr, \fm{g}, \fm{-μμH} state, only impfv]{you sg.\ can see him/her/it}
				\vbmorph{i-&ya-&\rt[²]{tin}&-μμH}
					{\gm{\xx{2sg.s}}&\xx{stv}&\rt[²]{see}&\·\xx{var}}
			\versus \vbform{iyatéen}{impfv}{she/he/it can see you sg.}
				\vbmorph{i-&ya-&\rt[²]{tin}&-μμH}
					{\gm{\xx{2sg.o}}&\xx{stv}&\rt[²]{see}&\·\xx{var}}
		\end{itemize}
	\end{enumerate}

\item[i-]\label{m:i-stv}
	stative prefix of classifier;
	\newline
	allomorphs:
	\begin{allolist}
	\item[ÿa-]	full syllable
	\item[wa-]	labialized form of \fm{ÿa-}
	\item[μ-]	lengthening of preceding vowel
	\end{allolist}

\item[-i]\label{m:-i-rel}
	relative clause suffix

\item[-í]\label{m:-í-sub}
	subordinate clause suffix

\item[-iḵ]\label{m:-iḵ}
	allomorph of \X[-ḵ-dprv]{-ḵ} with labialization and epenthetic (filler) \fm{i}

\item[-íḵ]\label{m:-íḵ}
	allomorph of \X[-ḵ-dprv]{-ḵ} with labialization and epenthetic (filler) \fm{í}

\item[-ín]\label{m:-ín-ctng}
	contingent mood suffix

\item[-ín]\label{m:-ín-past}
	past tense suffix
\end{morphdesc}
