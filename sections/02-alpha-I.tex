%!TEX root = ../lingnote-verbmorphs.tex

\subsection{I}\label{sec:alphalist-i}
\begin{morphdesc}[resume*=alphalist]
\item[i-]\label{m:i-2sg}
	second person singular subject or object; long vowel allomorphs are \fm{ee-} and \fm{ee=};
	subject versus object is typically distinguished by position in the verb word but can sometimes be ambiguous
	\begin{enumerate}
	\item	second person singular subject
		\begin{itemize}
		\item	\vbform{x̱at iyatéen}{impfv}[tr, \fm{g}, \fm{-μμH} state, only impfv]{you sg.\ can see me}
				\vbmorph{x̱at=&i-&ya-&\rt[²]{tin}&-μμH}
					{\xx{1sg.o}&\xx{2sg.s}&\xx{stv}&\rt[²]{see}&\·\xx{var}}
			\versus \vbform{ayatéen}{impfv}{she/he/it can see him/her/it}
				\vbmorph{a-&ya-&\rt[²]{tin}&-μμH}
					{\xx{3>3}&\xx{stv}&\rt[²]{see}&\·\xx{var}}
		\end{itemize}
	\item	second person singular оbject
		\begin{itemize}
		\item	\vbform{ix̱aatéen}{impfv}[tr, \fm{g}, \fm{-μμH} state, only impfv]{I can see you sg.}
				\vbmorph{i-&x̱a-&μ-&\rt[²]{tin}&-μμH}
					{\xx{2sg.o}&\xx{1sg.s}&\xx{stv}&\rt[²]{see}&\·\xx{var}}
		\end{itemize}
	\item	ambiguous: either second person subject or object in certain imperfective verb forms;
		when one of two arguments is third person and thus not indicated by the verb,
		the \fm{i-} argument prefix is ambiguous between subject and object and must
		be distinguished by other means (third person subject or object phrase,
		discourse context, etc.);
		this ambiguity does not arise if qualifiers, incorporated nouns, or aspect prefixes
		are present because they occur between the subject and object prefixes and so
		distinguish subject and object
		\begin{itemize}
		\item	\vbform{iyatéen}{impfv}[tr, \fm{g}, \fm{-μμH} state, only impfv]{you sg.\ can see him/her/it}
				\vbmorph{i-&ya-&\rt[²]{tin}&-μμH}
					{\gm{\xx{2sg.s}}&\xx{stv}&\rt[²]{see}&\·\xx{var}}
			\versus \vbform{iyatéen}{impfv}{she/he/it can see you sg.}
				\vbmorph{i-&ya-&\rt[²]{tin}&-μμH}
					{\gm{\xx{2sg.o}}&\xx{stv}&\rt[²]{see}&\·\xx{var}}
		\end{itemize}
	\end{enumerate}

\item[i-]\label{m:i-stv}
	stative prefix of classifier;
	\newline
	allomorphs:
	\begin{allolist}
	\item[ÿa-]	full syllable
	\item[wa-]	labialized form of \fm{ÿa-}
	\item[μ-]	lengthening of preceding vowel
	\end{allolist}

\item[-i]\label{m:-i-rel}
	relative clause suffix

\item[-i]\label{m:-i-sub}
	Allomorph of the subordinate clause suffix \X[-í-sub]{-í} with L tone.

\item[-i]\label{m:-i-loc}
	Locative suffix \fm{-í} \~\ \fm{-i} which occurs only as part of a few preverbal postposition
		phrases, indicating the position (or direction?)\ of a motion event.
	Conventionally identified as an allomorph of the more general locative postposition
		\fm{-xʼ} \~\ \fm{-μ} \parencites[33, 134, 138, 301]{leer:1991}
		so it can be glossed as \xx{loc}.
	But unlike the ordinary locative postposition this \fm{-í} \~\ \fm{-i} form is restricted to a
		preverbal position and is attested only with the seven nouns that are listed below.
	Semantically it seems to have the same meanings as the locative postposition but we lack
		elicited data and counterexamples to confirm or deny this equivalence.
	The vast majority of forms are L tone, but it has an H tone allomorph after one noun so
		therefore it has polar tone
		like the subordinate clause suffix \X[-í-sub]{-í} \~\ \X[-i-sub]{-i}
		and the possessed noun suffix \fm{-í} \~\ \fm{-i},
		and unlike the relative clause suffix \X[-i-rel]{-i} which occurs only with L tone.
	In theory, given that this suffix has the same phonological behaviour as the subordinate clause
		and possessed noun suffixes, there are possible phonologically conditioned allomorphs
		\fm{-yí} and \fm{-yi}, \fm{-ú} and \fm{-u}, and \fm{-wú} and \fm{-wu}
		given the right nouns, but none of these have been identified.
	Separate entries are given elsewhere with greater detail for each the individual preverbs
		containing this suffix, for which see the listed forms below.
	Most instances seem to reflect a motion derivation
		\motderiv{NP-i}{∅, \fm{-x̱} rep}{at/on/to NP}
		making them similar to \X{g̱unayéi=} ‘starting, beginning’.
	\newline
	Allomorphs:
	\begin{allolist}
	\item[-i]	form with L tone occurring after an H tone syllable
	\item[{\X[-í-loc]{-í}}]
			form with H tone occurring after an L tone syllable
	\end{allolist}
	\begin{enumerate}
	\item\label{item:-i-loc-dáag̱i}
		In \X{dáag̱i=} ‘inland’ from \fm{dáaḵ} ‘inland’.
	\item\label{item:-i-loc-éeg̱i}
		In \X{éeg̱i=} \~\ \X{éig̱i=} ‘on beach’ from \fm{éeḵ} \~\ \fm{éiḵ} ‘beach’.
	\item\label{item:-i-loc-gáani}
		In \X{gáani=} ‘outside’ from \fm{gáan} ‘outside, outdoors’.
	\item\label{item:-i-loc-gági}
		In \X{gági=} ‘out into open’ from \fm{gáak} ‘protrusion’.
	\item\label{item:-i-loc-héeni}
		In \X{héeni=} ‘into water’ from \fm{héen} ‘water, river’.
	\item\label{item:-i-loc-neilí}
		In \X{neilí=} ‘inside’ from \fm{neil} ‘home, inside, indoors’.
	\item\label{item:-i-loc-x̱áni}
		In \X{x̱áni=} ‘nearing’ from \fm{x̱án} ‘near’
	\end{enumerate}

\item[-í]\label{m:-í-sub}
	Subordinate clause suffix indicating syntactic embedding of the clause containing the
		verb with this suffix.

\item[-í]\label{m:-í-loc}
	Allomorph of the locative preverb suffix \X[-i-loc]{-í} with H tone.

\item[-iḵ]\label{m:-iḵ}
	allomorph of \X[-ḵ-dprv]{-ḵ} with labialization and epenthetic (filler) \fm{i}

\item[-íḵ]\label{m:-íḵ}
	allomorph of \X[-ḵ-dprv]{-ḵ} with labialization and epenthetic (filler) \fm{í}

\item[-ín]\label{m:-ín-ctng}
	contingent mood suffix

\item[-ín]\label{m:-ín-past}
	past tense suffix
\end{morphdesc}
