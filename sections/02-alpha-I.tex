%!TEX root = ../lingnote-verbmorphs.tex

\subsection{I}\label{sec:alphalist-i}
\begin{morphdesc}[resume*=alphalist]
\item[i-]\label{m:i-2sg}
	second person singular subject or object; long vowel allomorphs are \fm{ee-} and \fm{ee=};
	subject versus object is typically distinguished by position in the verb word but can sometimes be ambiguous
	\begin{enumerate}
	\item	second person singular subject
		\begin{itemize}
		\item	\vbform{x̱at iyatéen}{impfv}[tr, \fm{g}, \fm{-μμH} state, only impfv]{you sg.\ can see me}
				\vbmorph{x̱at=&i-&ya-&\rt[²]{tin}&-μμH}
					{\xx{1sg.o}&\xx{2sg.s}&\xx{stv}&\rt[²]{see}&\·\xx{var}}
			\versus \vbform{ayatéen}{impfv}{she/he/it can see him/her/it}
				\vbmorph{a-&ya-&\rt[²]{tin}&-μμH}
					{\xx{3>3}&\xx{stv}&\rt[²]{see}&\·\xx{var}}
		\end{itemize}
	\item	second person singular оbject
		\begin{itemize}
		\item	\vbform{ix̱aatéen}{impfv}[tr, \fm{g}, \fm{-μμH} state, only impfv]{I can see you sg.}
				\vbmorph{i-&x̱a-&μ-&\rt[²]{tin}&-μμH}
					{\xx{2sg.o}&\xx{1sg.s}&\xx{stv}&\rt[²]{see}&\·\xx{var}}
		\end{itemize}
	\item	ambiguous: either second person subject or object in certain imperfective verb forms;
		when one of two arguments is third person and thus not indicated by the verb,
		the \fm{i-} argument prefix is ambiguous between subject and object and must
		be distinguished by other means (third person subject or object phrase,
		discourse context, etc.);
		this ambiguity does not arise if qualifiers, incorporated nouns, or aspect prefixes
		are present because they occur between the subject and object prefixes and so
		distinguish subject and object
		\begin{itemize}
		\item	\vbform{iyatéen}{impfv}[tr, \fm{g}, \fm{-μμH} state, only impfv]{you sg.\ can see him/her/it}
				\vbmorph{i-&ya-&\rt[²]{tin}&-μμH}
					{\gm{\xx{2sg.s}}&\xx{stv}&\rt[²]{see}&\·\xx{var}}
			\versus \vbform{iyatéen}{impfv}{she/he/it can see you sg.}
				\vbmorph{i-&ya-&\rt[²]{tin}&-μμH}
					{\gm{\xx{2sg.o}}&\xx{stv}&\rt[²]{see}&\·\xx{var}}
		\end{itemize}
	\end{enumerate}

\item[i-]\label{m:i-stv}
	stative prefix of classifier;
	\newline
	allomorphs:
	\begin{allolist}
	\item[ÿa-]	full syllable
	\item[wa-]	labialized form of \fm{ÿa-}
	\item[μ-]	lengthening of preceding vowel
	\end{allolist}

\item[-i]\label{m:-i-rel}
	relative clause suffix

\item[-i]\label{m:-i-sub}
	Allomorph of the subordinate clause suffix \X[-í-sub]{-í} with L tone.

\item[-i]\label{m:-i-loc}
	Locative suffix \fm{-í} \~\ \fm{-i} which occurs only as part of a few preverbal postposition
		phrases, indicating the position (or direction?)\ of a motion event.
	Conventionally identified as an allomorph of the more general locative postposition
		\fm{-xʼ} \~\ \fm{-μ} \parencites[33, 134, 138, 301]{leer:1991}
		so it can be glossed as \xx{loc}.
	But unlike the ordinary locative postposition this \fm{-í} \~\ \fm{-i} form is restricted to a
		preverbal position and is attested only with the seven nouns that are listed below.
	Semantically it seems to have the same meanings as the locative postposition but we lack
		elicited data and counterexamples to confirm or deny this equivalence.
	The vast majority of forms are L tone, but it has an H tone allomorph after one noun so
		therefore it has polar tone
		like the subordinate clause suffix \X[-í-sub]{-í} \~\ \X[-i-sub]{-i}
		and the possessed noun suffix \fm{-í} \~\ \fm{-i},
		and unlike the relative clause suffix \X[-i-rel]{-i} which occurs only with L tone.
	In theory, given that this suffix has the same phonological behaviour as the subordinate clause
		and possessed noun suffixes, there are possible phonologically conditioned allomorphs
		\fm{-yí} and \fm{-yi}, \fm{-ú} and \fm{-u}, and \fm{-wú} and \fm{-wu}
		given the right nouns, but none of these have been identified.
	Most instances of this suffix seem to reflect a motion derivation
		\motderiv{NP-i}{∅, \fm{-x̱} rep}{at/on/to NP}
		suggesting membership in the class of motion derivations like
		\motderiv{\X{g̱unayéi=}}{∅, \fm{-x̱} rep}{starting off, setting out, beginning}.
	Each of the preverbs with locative \fm{-í} \~\ \fm{-i} are given below accompanied by a single
		example; for greater detail and discussion see the separate entries for each preverb.
	\newline
	Allomorphs:
	\begin{allolist}
	\item[-i]	form with L tone occurring after an H tone syllable
	\item[{\X[-í-loc]{-í}}]
			form with H tone occurring after an L tone syllable
	\end{allolist}
	\begin{enumerate}
	\item\label{item:-i-loc-dáag̱i}
		In \X{dáag̱i=} ‘inland’ from \fm{dáaḵ} ‘inland’.
		\begin{itemize}
		\item	\vbform{dáag̱i daaḵ uwalʼíxʼ}{pfv}[obj intr, \fm{∅}, ach]{it was broken all the way up}
			\parencite[18.172]{nyman-leer:1993}
				\vbmorph{\gm{dáaḵ}&\gm{-i=}&daaḵ=&u-&wa-&\rt{lʼixʼ}&-μH}
					{inland&\·\xx{loc}&inland&\xx{zpfv}&\xx{stv}&\rt{break}&\·\xx{var}}
		\end{itemize}
	\item\label{item:-i-loc-éeg̱i}
		In \X{éeg̱i=} \~\ \X{éig̱i=} ‘on beach’ from \fm{éeḵ} \~\ \fm{éiḵ} ‘beach’.
		\begin{itemize}
		\item	\vbform{du tláa éeg̱i daxáash}{impfv}[subj intr, \fm{n}, \fm{-μμH} act]{his mother is cutting (fish) on the beach}
			\parencite[315.14]{swanton:1909}
				\vbmorph{du&tláa&\gm{éeg̱i=}&da-&\rt[²]{xash}&-μμH}
					{\xx{3h.psr}&mother&on.beach&\xx{apsv}&\rt[²]{cut}&\·\xx{var}}
		\end{itemize}
	\item\label{item:-i-loc-gáani}
		In \X{gáani=} ‘outside’ from \fm{gáan} ‘outside, outdoors’.
		\begin{itemize}
		\item	\vbform{gáani yux̱ woogoot}{pfv}[subj intr, \fm{n}, mot]{it went outside}
			\parencite[220.54]{dauenhauer-dauenhauer:1987}
				\vbmorph{\gm{gáan}&\gm{-i=}&\gm{yux̱=}&wu-&μ-&\rt[¹]{gut}&-μμL}
					{outside&\·\xx{loc}&out&\xx{pfv}&\xx{stv}&\rt[¹]{go.\xx{sg}}&\·\xx{var}}
		\end{itemize}
	\item\label{item:-i-loc-gági}
		In \X{gági=} ‘out into open’ from \fm{gáak} ‘protrusion’.
		\begin{itemize}
		\item	\vbform{gági uwagút}{pfv}[subj intr, \fm{∅}, mot]{he appeared}
			\parencite[20.78]{story-naish:1973}
				\vbmorph{\gm{gág}&\gm{-i=}&u-&wa-&\rt[¹]{gut}&-μH}
					{open&\·\xx{loc}&\xx{zpfv}&\xx{stv}&\rt[¹]{go.\xx{sg}}&\·\xx{var}}
		\end{itemize}
	\item\label{item:-i-loc-héeni}
		In \X{héeni=} ‘into water’ from \fm{héen} ‘water, river’.
		\begin{itemize}
		\item	\vbform{héeni kawdax̱dudliyaa}{pfv}[tr, \fm{g̱}, ach]{they lowered each}
			\parencite[96.300]{dauenhauer-dauenhauer:1987}
				\vbmorph{\gm{héen}&\gm{-i=}&ka-&w-&dax̱-&du-&d-&l-&i-&\rt[²]{ÿa}&-μμL}
					{water&\·\xx{loc}&\xx{qual}&\xx{pfv}&\xx{distb}&\xx{ind.h.s}&\xx{mid}&\xx{xtn}&\xx{stv}&\rt[²]{lower}&\·\xx{var}}
		\end{itemize}
	\item\label{item:-i-loc-neilí}
		In \X{neilí=} ‘inside’ from \fm{neil} ‘home, inside, indoors’.
		\begin{itemize}
		\item	\vbform{neilí wuḵeiyi shaawát}{rel pfv}[subj intr, \fm{g̱}, ach]{women who sat inside}
			\parencite[266.132]{dauenhauer-dauenhauer:1987}
				\vbmorph{\gm{neil}&\gm{-í=}&wu-&\rt[¹]{ḵe}&-μμL&-yi&shaawát}
					{inside&\·\xx{loc}&\xx{pfv}&\rt[¹]{sit.\xx{pl}}&\·\xx{var}&\·\xx{rel}&woman}
		\end{itemize}
	\item\label{item:-i-loc-x̱áni}
		In \X{x̱áni=} ‘nearing’ from \fm{x̱án} ‘near’.
		This case is notable for also occuring with possessors.
		\begin{itemize}
		\item	\vbform{x̱áni yux̱ woogoot}{pfv}[subj intr, \fm{n}, mot]{she went out near (him)}
			\parencite[259.8]{swanton:1909}
				\vbmorph{\gm{x̱án}&\gm{-i=}&yux̱=&wu-&μ-&\rt[¹]{gut}&-μμL}
					{near&\·\xx{loc}&out&\xx{pfv}&\xx{stv}&\rt[¹]{go.\xx{sg}}&\·\xx{var}}
		\end{itemize}
	\item\label{item:-i-loc-others}
		Given \fm{dáag̱i=} (\#\ref{item:-i-loc-dáag̱i}) ‘inland’ we might expect
			\fm[?]{dáagi=} from \fm{dáak} ‘out to sea’ (compare \X{daak=}),
			but there are no attested examples of this.
		Likewise, given \fm{neilí=} (\#\ref{item:-i-loc-neilí}) we might expect
			\fm[?]{ÿáni=} from \fm{ÿán} ‘shore’ (compare \X{ÿan=}),
			but again there are no attested examples.
	\item\label{item:-i-loc-discussion}
		If \fm{-í} \~\ \fm{-i} is indeed the same underlying postposition as the locative
			\fm{-xʼ} \~\ \fm{-μ} postposition then we want an explanation for why this
			allomorph is used instead of \fm{-xʼ}.
		There is as yet no predictive explanation for the contexts where \fm{-í} \~\ \fm{-i}
			occurs other than simply listing the nouns with which it is found in preverbs
			as presented above.
		Furthermore, there is no obvious historical reason for \fm{-í} \~\ \fm{-i} to have
			developed from \fm{-xʼ} and the similarity with the possessive suffix
			\fm{-í} \~\ \fm{-i} and the subordinate clause suffix \fm{-í} \~\ \fm{-i}
			is totally unexplained so this suffix is very much an outstanding puzzle
			in the intersection of postpositon and preverb phenomena.
	\end{enumerate}

\item[-í]\label{m:-í-sub}
	Subordinate clause suffix indicating syntactic embedding of the clause containing the
		verb with this suffix.

\item[-í]\label{m:-í-loc}
	Allomorph of the locative preverb suffix \X[-i-loc]{-í} with H tone.

\item[-iḵ]\label{m:-iḵ}
	allomorph of \X[-ḵ-dprv]{-ḵ} with labialization and epenthetic (filler) \fm{i}

\item[-íḵ]\label{m:-íḵ}
	allomorph of \X[-ḵ-dprv]{-ḵ} with labialization and epenthetic (filler) \fm{í}

\item[-ín]\label{m:-ín-ctng}
	contingent mood suffix

\item[-ín]\label{m:-ín-past}
	past tense suffix
\end{morphdesc}
