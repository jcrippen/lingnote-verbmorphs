%!TEX root = ../lingnote-verbmorphs.tex

\section{Alphabetic listing of verb morphemes}\label{sec:alphalist}

\subsection{A}\label{sec:alphalist-a}
\raggedright
\begin{morphdesc}[series=alphalist]
\item[a-]\label{m:a-}
	argument marking prefix in the same position as object prefixes/proclitics;
	\newline
	combinations:
	\begin{allolist}
	\item[\X{aawa}]	≡ \fm{a-μʷ-wa-} with perfective \X{μʷ-} and stative \X{wa-}
	\item[\X{am}]	≡ \fm{a-m-} with perfective \X{m-}
	\item[\X{aw}]	≡ \fm{a-w-} with perfective \X[w-pfv]{w-}
	\item[\X{awu}]	≡ \fm{a-wu-} with perfective \X{wu-}
	\item[\X{oo}]	≡ \fm{a-u-} with irrealis \X[u-irr]{u-}
			or \fm{∅} conj perfective \X[u-pfv]{u-}
	\item[\X{oowa}]	≡ \fm{a-u-} with irrealis \X[u-irr]{u-}
			or \fm{∅} conj perfective \X[u-pfv]{u-}
			and stative \X{wa-}
	\item[\X{ee}]	≡ \fm{a-i-} with second person singular subject \X[i-2sg]{i-}
	\item[\X{eeÿa}]	≡ \fm{a-i-} with second person singular subject \X[i-2sg]{i-}
			and stative \X[ÿa-stv]{ÿa-}
	\end{allolist}
	\begin{enumerate}
	\item\label{m:a-3>3}
		3>3 agreement of transitive verb: indicates existence of third person subject
		and third person object (regardless of whether these are actually spoken)
		\begin{itemize}
		\item	\vbform{aawax̱áa}{pfv}[tr, \fm{∅}, \fm{-μH} act]{she/he/it ate him/her/it}
			\vbmorph{\gm{a-}&μʷ-&wa-&\rt[²]{x̱a}&-μμH}
				{\xx{3>3}&\xx{pfv}&\xx{stv}&\rt[²]{eat}&\·\xx{var}}
			\versus \vbform{wutuwax̱áa}{pfv}{we ate it}
			\vbmorph{wu-&tu-&wa-&\rt[²]{x̱a}&-μμH}
				{\xx{pfv}&\xx{1pl.s}&\xx{stv}&\rt[²]{eat}&\·\xx{var}}
		\end{itemize}
	\item\label{m:a-ind.h.o}
		indefinite nonhuman object of some transitive verbs
			instead of \X{at=};
		verbs that use \fm{a-} in this way usually also use \fm{a-} 3>3
		and the reason for using \fm{a-} instead of \fm{at=} is still unclear
		\begin{itemize}
		\item	\vbform{alʼóon}{impfv}[tr, \fm{n}, \fm{-μμH} act]{s/he/it is hunting something}
			\vbmorph{\gm{a-}&\rt[²]{lʼuᴴn}&-μμH}
				{\xx{ind.n.o}&\rt[²]{hunt}&\·\xx{var}}
			\andnot{	\vbform[*]{at lʼóon}{impfv}{s/he/it is hunting something}
				using \fm{at=}}
			\versus \vbform{alʼóon}{impfv}{s/he/it is hunting it}
			\vbmorph{a-&\rt[²]{lʼuᴴn}&-μμH}
				{\xx{3>3}&\rt[²]{hunt}&\·\xx{var}}
			\notealso{compare \vbform{g̱áx̱ alʼóon}{impfv}{s/he/it is hunting rabbits}}
			\notealso{(\fm{g̱áx̱ alʼóon} cannot mean
				‘s/he/it is hunting something rabbits’)}
		\end{itemize}
	\item\label{m:a-ind.h.s}
		indefinite human subject (not object!)\ of subject intransitive verbs
		instead of \X{du-};
		possibly all subject intransitive verbs use \fm{a-} rather than \fm{du-}?
		\begin{itemize}
		\item	\vbform{ax̱éxʼw}{impfv}[subj intr, \fm{n}, \fm{-μH} act]{people are sleeping}
				\vbmorph{\gm{a-}&\rt[¹]{x̱exʼw}&-μH}
					{\xx{ind.h.s}&\rt[¹]{sleep·\xx{pl}}&\·\xx{var}}
				\andnot{\fm[*]{dux̱éxʼw} ‘people are sleeping’
					using \fm{du-} \xx{ind.n.s} ‘people’}
			\versus \vbform{toox̱éxʼw}{impfv}{we are sleeping}
				\vbmorph{too- \rt[¹]{x̱exʼw} -μH}
					{\xx{1pl·s} \rt[¹]{sleep·\xx{pl}} \·\xx{var}}
		\item	\vbform{aawa.aat}{pfv}[subj intr, \fm{n}, mot]{people went}
			\vbmorph{\gm{a-}&μʷ-&wa-&\rt[¹]{.at}&-μμL}
				{\xx{ind.h.s}&\xx{pfv}&\xx{stv}&\rt[¹]{go·\xx{pl}}&\·\xx{var}}
			\andnot{\fm[*]{wuduwa.aat} ‘people went’ 
				using \fm{du-} \xx{ind.h.s} ‘someone, people’}
			\versus \vbform{wutuwa.aat}{pfv}{we went}
				\vbmorph{wu- &tu-&wa- &\rt[¹]{.at}&-μμL}
					{\xx{pfv}&\xx{1pl.s}&\xx{stv}&\rt[¹]{go·\xx{pl}}&\·\xx{var}}
		\end{itemize}
	\item\label{m:a-xpl}
		nonreferential expletive (filler) object, does not refer to anything;
		it is unclear why this is used instead of suppressing the object with
		antipassive voice \X{d-}
		\begin{itemize}
		\item	\vbform{awdigaan}{pfv}[impers, \fm{g̱}, ach]{it sunshined}
			\vbmorph{\gm{a-}&w-&d-&i-&\rt[¹]{gan}&-μμL}
				{\xx{xpl}&\xx{pfv}&\xx{mid}&\xx{stv}&\rt[²]{burn}&\·\xx{var}}
				\andnot{\fm[*]{g̱agaan awdigaan} ‘sun sunshined’: no object allowed}
			\versus \vbform{sʼeenáa kawdigán}{pfv}[obj intr, \fm{∅}, ach]{the lamp shone, lit up}
				\vbmorph{sʼeenáa&ka-&w-&d-&i-&\rt[¹]{gan}&-μH}
					{lamp&\xx{hsfc}&\xx{pfv}&\xx{mid}&\xx{stv}&\rt[¹]{burn}&\·\xx{var}}
		\end{itemize}
	\end{enumerate}

\item[aa=]\label{m:aa=}
	partitive proclitic ‘one of, some of’;
	can apply to either singular or plural quantity of the referent depending on context;
	derived from the independent partitive pronoun \fm{aa} ‘one, some’
		but distinguished from it by position because unlike independent pronouns
		the proclitic can occur between preverbs and verb
	\begin{enumerate}
	\item	partitive third person object of transitive verb
		\begin{itemize}
		\item	\vbform{aa wutusi.ée}{pfv}[tr, \fm{∅}, \fm{-μμH} act]{we cooked one/some of them}
				\vbmorph{\gm{aa=}&wu-&tu-&s-&i-&\rt[¹]{.i}&-μμH}
					{\xx{part.o}&\xx{pfv}&\xx{1pl.s}&\xx{csv}&\xx{stv}&\rt[¹]{cooked}&\·\xx{var}}
			\versus \vbform{wutusi.ée}{pfv}{we cooked him/her/it/them}
				\vbmorph{wu-&tu-&s-&i-&\rt[¹]{.i}&-μμH}
					{\xx{pfv}&\xx{1pl.s}&\xx{csv}&\xx{stv}&\rt[¹]{cooked}&\·\xx{var}}
		\end{itemize}
	\item	partitive third person object of object intransitive verb
		\begin{itemize}
		\item	\vbform{yéi aa yatee}{impfv}[obj intr, \fm{n}, \fm{-μμL} state]{one is/some are that way}
				\vbmorph{yéi=&\gm{aa=}&ÿa-&\rt[¹]{tiᴸ}&-μμL}
					{thus&\xx{part.o}&\xx{stv}&\rt[¹]{be}&\·\xx{var}}
			\versus \vbform{yéi yatee}{impfv}{she/he/it is that way}
				\vbmorph{yéi=&ÿa-&\rt[¹]{tiᴸ}&-μμL}
					{thus&\xx{stv}&\rt[¹]{be}&\·\xx{var}}
		\end{itemize}
	\item	partitive third person subject of subject intransitive verb
		\begin{itemize}
		\item	\vbform{aa woo.aat}{pfv}[subj intr, \fm{n}, mot]{some of them went}
				\vbmorph{\gm{aa=}&wu-&μ-&\rt[ˢ]{.at}&-μμL}
					{\xx{part.s}&\xx{pfv}&\xx{stv}&\rt[ˢ]{go·\xx{pl}}&\·\xx{var}}
			\versus \vbform{woo.aat}{pfv}[subj intr, \fm{n}, mot]{they went}
				\vbmorph{wu-&μ-&\rt[ˢ]{.at}&-μμL}
					{\xx{pfv}&\xx{stv}&\rt[ˢ]{go·\xx{pl}}&\·\xx{var}}
		\item	\vbform{aa woonook}{pfv}[subj intr, \fm{g̱}, mot]{one of them sat down}
				\vbmorph{\gm{aa=}&wu-&μ-&\rt[ˢ]{nuk}&-μμL}
					{\xx{part.s}&\xx{pfv}&\xx{stv}&\rt[ˢ]{sit·\xx{sg}}&\·\xx{var}}
			\versus \vbform{woonook}{pfv}{she/he/it sat down}
				\vbmorph{wu-&μ-&\rt[ˢ]{nuk}&-μμL}
					{\xx{pfv}&\xx{stv}&\rt[ˢ]{sit·\xx{sg}}&\·\xx{var}}
		\end{itemize}
	\end{enumerate}

\item[aawa]\label{m:aawa}
	≡ \fm{a-μʷ-wa-}
	combination of argument marking \X{a-},
		perfective \X{μʷ-},
		and stative \X{wa-};
	compare \X{awu} ≡ \fm{a-wu-} and \X{aw} ≡ \fm{a-w-}
	\begin{itemize}
	\item	\vbform{aawajáḵ}{pfv}[tr, \fm{∅}, ach]{she/he/it killed him/her/it}
			\vbmorph{\gm{a-}&\gm{μʷ-}&\gm{wa-}&\rt[²]{jaḵ}&-μH}
				{\xx{3>3}&\xx{pfv}&\xx{stv}&\rt[²]{kill}&\·\xx{var}}
	\end{itemize}

\item[ach=]\label{m:ach=}
	variant form of third person proximate object \fm{ash=};
	probably derived from third person pronoun \fm{á} + ergative \fm{-ch}
		or reflexive pronoun \fm{sh}
	\begin{enumerate}
	\item	
	\item	
	\end{enumerate}

\item[ách]\label{m:ách}
	postposition phrase immediately preceding some verbs, not actually a verb morpheme;
	third person nonhuman pronoun \fm{á} and ergative/instrumental \fm{-ch}

\item[-áchʼ]\label{m:-áchʼ}
	allomorph of unknown suffix \fm{-chʼ} with epenthetic (filler) vowel \fm{á}

\item[-álʼ]\label{m:-alʼ}
	allomorph of repetitive \fm{-lʼ} with epenthetic (filler) vowel \fm{á}

\item[-áḵw]\label{m:-áḵw}
	allomorph of deprivative \fm{-ḵ} \~\ \fm{-ḵw} with epenthetic (filler) vowel \fm{á}

\item[am]\label{m:am}
	≡ \fm{a-m-}
	combination of argument marking \X{a-}
		and perfective \X{m-};
	compare \X{aw} ≡ \fm{a-w-} and \X{awu} ≡ \fm{a-wu-} and \X{aawa} ≡ \fm{a-μʷ-wa-};
	because \fm{m-} only arises in a syllable coda,
		forms like \fm[*]{amu} and \fm[*]{aama} do not occur
	\begin{itemize}
	\item	\vbform{awsi.ée}{pfv}[tr, \fm{∅}, \fm{-μμH} act]{s/he/it cooked him/her/it}
			\vbmorph{\gm{a-}&\gm{w-}&s-&i-&\rt[¹]{.i}&-μμH}
				{\xx{3>3}&\xx{pfv}&\xx{csv}&\xx{stv}&\rt[¹]{cooked}&\·\xx{var}}
	\end{itemize}

\item[-án]\label{m:-án}
	restorative suffix, indicates restoration of a previous situation;
	possibly occurs as part of \fm{-chʼán} \~\ \fm{-shán} but meaning in this is unclear

\item[as=]\label{m:as=}
	allomorph of human pluralizer \fm{has=} for third person subject or object;
	mostly occurs in Southern \&\ Tongass varieties
	\begin{itemize}
	\item	\vbform{as dustaaÿch}{hab}[tr, \fm{∅}, ach]{they would boil it}
		(Tongass dialect) \parencite[24.80]{leer:1978}
		\vbmorph{\gm{as=}&du-&d-&s-&\rt[¹]{taᴸ}&-μμ&-ÿ&-ch}
			{\xx{plh}&\xx{ind.h.s}&\xx{mid}&\xx{csv}&\rt[¹]{boil}&\·\xx{var}&\·\xx{ÿsfx}&\·\xx{rep}}
		\versus \vbform{has dustáaych}{hab}{they would boil it} (Northern dialect)
		\vbmorph{has=&du-&d-&s-&\rt[¹]{taᴸ}&-μμH&-ÿ&-ch}
			{\xx{plh}&\xx{ind.h.s}&\xx{mid}&\xx{csv}&\rt[¹]{boil}&\·\xx{var}&\·\xx{ÿsfx}&\·\xx{rep}}
	\end{itemize}

\item[-ásʼ]\label{m:-ásʼ}
	allomorph of repetitive \fm{-sʼ} with epenthetic (filler) vowel \fm{á}

\item[ash=]\label{m:ash=}
	object proclitic;
	probably derived from third person pronoun \fm{á} + reflexive pronoun \fm{sh}
		or ergative \fm{-ch};
		see also \fm{ach=} used instead of \fm{ash=} in some verbs
	\begin{enumerate}
	\item	third person proximate human object
	\item	special reflexive object
	\end{enumerate}

\item[at=]\label{m:at=}
	indefinite nonhuman object ‘something, stuff’;
	derived from the noun \fm{át} ‘thing’ (as in \fm{wé át} ‘that thing’);
	see also \fm{a-} used instead of \fm{at=} in some verbs
	\begin{itemize}
	\item	\vbform{at wutusiteen}{pfv}[tr, \fm{g̱}, ach]{we saw something}
			\vbmorph{\gm{at=}&wu-&tu-&s-&i-&\rt[²]{tin}&-μμL}
				{\xx{ind.n.o}&\xx{pfv}&\xx{1pl.s}&\xx{xtn}&\xx{stv}&\rt[²]{see}&\·\xx{var}}
		\versus \vbform{wutusiteen}{pfv}{we saw him/her/it}
			\vbmorph{&wu-&tu-&s-&i-&\rt[²]{tin}&-μμL}
				{&\xx{pfv}&\xx{1pl.s}&\xx{xtn}&\xx{stv}&\rt[²]{see}&\·\xx{var}}
	\item	\vbform{at wusiteen}{pfv}[tr, \fm{g̱}, ach]{s/he/it saw something}
			\vbmorph{\gm{at=}&wu-&s-&i-&\rt[²]{tin}&-μμL}
				{\xx{ind.n.o}&\xx{pfv}&\xx{xtn}&\xx{stv}&\rt[²]{see}&\·\xx{var}}
		\versus \vbform{awsiteen}{pfv}{s/he/it saw him/her/it}
			\vbmorph{a-&w-&s-&i-&\rt[²]{tin}&-μμL}
				{\xx{3>3}&\xx{pfv}&\xx{xtn}&\xx{stv}&\rt[²]{see}&\·\xx{var}}
	\end{itemize}

\item[-átʼ]\label{m:-átʼ}
	allomorph of repetitive \fm{-tʼ} with epenthetic (filler) vowel \fm{á}

\item[aw]\label{m:aw}
	≡ \fm{a-w-}
	combination of argument marking \X{a-}
		and perfective \X[w-pfv]{w-};
	compare \X{awu} ≡ \fm{a-wu-} and \X{aawa} ≡ \fm{a-μʷ-wa-}
	\begin{itemize}
	\item	\vbform{awsi.ée}{pfv}[tr, \fm{∅}, \fm{-μμH} act]{s/he/it cooked him/her/it}
			\vbmorph{\gm{a-}&\gm{w-}&s-&i-&\rt[¹]{.i}&-μμH}
				{\xx{3>3}&\xx{pfv}&\xx{csv}&\xx{stv}&\rt[¹]{cooked}&\·\xx{var}}
	\end{itemize}

\item[awu]\label{m:awu}
	≡ \fm{a-wu-}
	combination of argument marking \X{a-}
		and perfective \X{wu-};
	occurs where stative \fm{ÿa-} \~\ \fm{i-} is suppressed
		such as in negative, past tense, and subordinate clause forms;
	compare \X{aawa} ≡ \fm{a-μʷ-wa-} and \X{aw} ≡ \fm{a-w-}
	\begin{itemize}
	\item	\vbform{tléil awux̱á}{neg pfv}[tr, \fm{∅}, \fm{-μH} act]{she/he/it didn’t eat him/her/it}
			\vbmorph{tléil&\gm{a-}&\gm{wu-}&\rt[²]{x̱a}&-μH}
				{\xx{neg}&\xx{3>3}&\xx{pfv}&\rt[²]{eat}&\·\xx{var}}
		\versus \vbform{aawax̱áa}{pfv}{she/he/it ate him/her/it}
			\vbmorph{a-&μʷ-&wa-&\rt[²]{x̱a}&-μμH}
				{\xx{3>3}&\xx{pfv}&\xx{stv}&\rt[²]{eat}&\·\xx{var}}
	\item	\vbform{awux̱áayin}{past pfv}{she/he/it had eaten him/her/it}
			\vbmorph{\gm{a-}&\gm{wu-}&\rt[²]{x̱a}&-μμH&-yin}
				{\xx{3>3}&\xx{pfv}&\rt[²]{eat}&\·\xx{var}&\·\xx{past}}
	\item	\vbform{awux̱áayi}{sub pfv}{while/when she/he/it had eaten him/her/it}
			\vbmorph{\gm{a-}&\gm{wu-}&\rt[²]{x̱a}&-μμH&-yi}
				{\xx{3>3}&\xx{pfv}&\rt[²]{eat}&\·\xx{var}&\·\xx{sub}}
	\end{itemize}

\item[ax̱=]\label{m:ax̱=}
	allomorph ‘my’ of \fm{x̱at=} ‘me’ first person singular object,
		only used as possessor of incorporated nouns;
	derived from possessive pronoun \fm{ax̱} ‘my’ (compare \fm{ax̱ keidlí áwé} ‘it is my dog’);
	some speakers disprefer \fm{ax̱=} in verbs and only use \fm{x̱at=}
	\begin{itemize}
	\item	\vbform{ax̱ shalxáash}{impfv}[tr, \fm{n}, \fm{-μμH} act]{she/he/it is cutting my hair}
			\vbmorph{\gm{ax̱=}&sha-&l-&\rt[²]{xash}&-μμH}
				{\xx{1sg.o}&head&\xx{xtn}&\rt[²]{cut}&\·\xx{var}}
		\versus 	\vbform{x̱at shalxáash}{impfv}{she/he/it is cutting my hair}
			\vbmorph{x̱at=&sha-&l-&\rt[²]{xash}&-μμH}
				{\xx{1sg.o}&head&\xx{xtn}&\rt[²]{cut}&\·\xx{var}}
	\end{itemize}
\end{morphdesc}

\subsection{C}\label{sec:alphalist-c}

\begin{morphdesc}[resume*=alphalist]
\item[-ch]
	repetitive suffix

\item[-chʼ]
	unknown suffix;
	possibly a part of intensifier \fm{-chʼán} (with restorative \fm{-án})
		and unknown \fm{-chʼálʼ} (with repetitive \fm{-álʼ}),
	otherwise documented only in two denominal verbs
	\begin{enumerate}
	\item	lexicalized element in \fm{g̱eeg̱áchʼ} \~\ \fm{g̱eig̱áchʼ} (noun) ‘swing, hammock’
		(also \fm{g̱eeg̱áchʼaa} with instrument \fm{-aa})
		derived from \fm{\rt[¹]{g̱iḵ}} \~\ \fm{\rt[¹]{g̱eḵ}} ‘swing’
		\begin{itemize}
		\item	\vbform{awlig̱eiḵ}{pfv}[tr, \fm{n}, mot]{she/he/it swung him/her/it}
				\vbmorph{a-&w-&l-&i-&\rt[¹]{g̱eḵ}&-μμL}
					{\xx{3>3}&\xx{pfv}&\xx{csv}&\xx{stv}&\rt[¹]{swing}&\·\xx{var}}
			\versus \vbform{ash koolg̱eig̱áchʼ}{impfv}[subj intr, conj?, act]{she/he/it is playing on a swing/hammock}
				\vbmorph{ash=&ka-&u-&d-&l-&\rt[¹]{g̱eḵ}&-μμL&\gm{-áchʼ}}
					{\xx{rflx.o}&\xx{qual}&\xx{irr}&\xx{mid}&\xx{csv}&\rt[¹]{swing}&\·\xx{var}&\·\xx{unkn}}
		\end{itemize}
	\item	lexicalized element together with deprivative \fm{-áḵw} in verb
		derived from \fm{séew} (noun) ‘rain’
		\begin{itemize}
		\item	\vbform{kawdudliséewchʼáḵw}{pfv}[obj intr, conj?, ach?]{it (berry) is full of rain (thus tasteless)}
			\parencite[56]{story:1966}
				\vbmorph{ka-&w-&du-&d-&l-&i-&\rt{siw}&-μμH&\gm{-chʼ}&-áḵw}
					{\xx{sro}&\xx{pfv}&\xx{xpl}&\xx{mid}&\xx{intr}&\xx{stv}&\rt{rain}&\·\xx{var}&\·\xx{unkn}&\·\xx{dprv}}
		\end{itemize}
	\end{enumerate}

\item[-chʼán]
	intensifier suffix; allomorph \fm{-shán} used after an ejective consonant

\item[chush=]
	allomorph of reflexive object \fm{sh=}

\end{morphdesc}

\subsection{D}\label{sec:alphalist-d}
\begin{morphdesc}[resume*=alphalist]
\item[d-]\label{m:d-}
	voice prefix, traditionally analyzed as part of the classifier;
	suppresses an argument (passive, antipassive)
	or reduces the scope of its reference (middle);
	\newline
	allomorphs:
	\begin{allolist}
	\item[d-]	basic form
	\item[\X{da-}]	with epenthetic (filler) vowel \fm{a}
			(no \fm{s-}/\fm{l-}/\fm{lˢ-}/\fm{sh-} and no \fm{i-})
	\end{allolist}
	combinations:
	\begin{allolist}
	\item[\X{di}]	≡ \fm{d-i-} with stative \X[i-stv]{i-}
	\item[\X{dli}]	≡ \fm{d-l-i-} with valency \X{l-}/\X{lˢ-} and stative \X[i-stv]{i-}
	\item[\X{dzi}]	≡ \fm{d-s-i-} with valency \X{s-} and stative \X[i-stv]{i-}
	\item[\X{ji}]	≡ \fm{d-sh-i-} with valency \X{sh-} and stative \X[i-stv]{i-}
	\item[\X{…l}]	≡ \fm{d-s-} with valency \X{l-}/\X{lˢ-} (no \fm{i-})
	\item[\X{…s}]	≡ \fm{d-s-} with valency \X{s-} (no \fm{i-})
	\item[\X{…sh}]	≡ \fm{d-sh-} with valency \X{sh-} (no \fm{i-})
	\end{allolist}
	the \fm{…s} / \fm{…l} / \fm{…sh} forms must be preceded by a vowel,
		with epenthetic (filler) \fm{i} inserted if no preceding prefixes provide a vowel,
		see those entries for examples
%		(the hypothetical \fm[*]{dza} / \fm[*]{dla} / \fm[*]{ja}
%			and \fm[*]{…dz} / \fm[*]{…dl} / \fm[*]{…j} do not occur
%			and are instead \fm{…s} / \fm{…l} / \fm{…sh})
	\begin{enumerate}
	\item	middle voice
	\item	passive voice
	\item	antipassive voice
	\end{enumerate}

\item[da-]\label{m:da-}
	allomorph of voice prefix \X{d-} with epenthetic (filler) vowel;
	occurs only whenever there is no stative \X[i-stv]{i-}
		and no valency \X{s-}, \X{l-}/\X{lˢ-}, or \X{sh-}
	\begin{itemize}
	\item	\vbform{tléil yan sh wudax̱eech}{neg pfv}[tr, \fm{∅}, mot]{he did not throw himself down}
			\vbmorph{tléil&yan=&sh=&wu-&\gm{da-}&\rt[²]{x̱ich}&-μμL}
				{\xx{neg}&ground&\xx{rflx.o}&\xx{pfv}&\xx{mid}&\rt[²]{throw}&\·\xx{var}}
		\versus \vbform{yan sh wudix̱ích}{pfv}{he threw himself down}
		\parencite[227.3217]{story-naish:1973}
			\vbmorph{yan=&sh=&wu-&d-&i-&\rt[²]{x̱ich}&-μH}
				{ground&\xx{rflx.o}&\xx{pfv}&\xx{mid}&\xx{stv}&\rt[²]{throw}&\·\xx{var}}
	\end{itemize}

\item[daa-]
	inalienable incorporated noun \fm{daa} ‘around, about, surrounding’

\item[daak=]
	directional preverb ‘out to sea (away from land)’;
	derived from directional noun \fm{dáak} ‘out at sea’
		(compare \fm{dákde=})

\item[dáag̱i=]
	directional noun ‘inland (away from water body)’
		with special locative postposition \fm{-í} \~\ \fm{-i} ‘at’;
	derived from directional noun \fm{dáaḵ} ‘inland’
		(compare \fm{daaḵ=}, \fm{dáḵde=}; noun \fm{daḵká} ‘on inland’)

\item[daaḵ=]
	directional preverb ‘inland (away from water body)’;
	derived from directional noun \fm{dáaḵ} ‘inland’
		(compare \fm{dáag̱i=}, \fm{dáḵde=}; noun \fm{daḵká} ‘on inland’)

\item[daa.it-]
	inalienable incorporated noun \fm{daa.ít} ‘joint’;
	possibly developed from a combination of \fm{daa} ‘around’ and \fm{ít} ‘following’
		although compositional meaning is unclear

\item[dag̱a-]
	allomorph of distributive or non-human pluralizer \fm{dax̱=};
	position of this allomorph is uncertain as it is only attested in forms without
	argument or aspectual prefixes

\item[dákde=]
	directional preverb ‘out to sea (away from land)’
		with allative postposition \fm{-dé} \~\ \fm{-de} ‘toward’
	derived from directional noun \fm{dáak} ‘out at sea’
		(compare \fm{daak=})
\item[dáḵde=]
	directional noun \fm{dáaḵ} ‘inland (away from water body)’
		with allative postposition \fm{-dé} \~\ \fm{-de} ‘toward’
	derived from directional noun \fm{dáaḵ} ‘inland’
		(compare \fm{dáag̱i=}, \fm{daaḵ=}; noun \fm{daḵká} ‘on inland’)

\item[dax̱=]
	distributive pluralizer or non-human pluralizer;
	can occur before human pluralizer \fm{has=} but not after

\item[deik=]
	variant form of preverb \fm{daak=} ‘out to sea’
		used in Southern and Transitional Northern communities;
	the reason for using \fm{deik=} versus \fm{daak=} is still unclear;
	compare similar \fm{deiḵ=} versus \fm{daaḵ=}
	\begin{itemize}
	\item	\vbform{deik ḵoowatín}{pfv}[subj intr, \fm{∅}, ach+mot]{he has gotten vision}
		(Southern dialect) \parencite[06/212]{leer:1973}
			\vbmorph{\gm{deik=}&ḵu-&μʷ-&wa-&\rt[²]{tin}&-μH}
				{out&\xx{areal}&\xx{pfv}&\xx{stv}&\rt[²]{see}&\·\xx{var}}
		\versus \vbform{daak ḵoowatín}{pfv}{he has gotten vision} (Northern dialect)
			\vbmorph{daak=&ḵu-&μʷ-&wa-&\rt[²]{tin}&-μH}
				{out&\xx{areal}&\xx{pfv}&\xx{stv}&\rt[²]{see}&\·\xx{var}}
	\end{itemize}

\item[deiḵ=]
	variant form of preverb \fm{daaḵ=} ‘inland’ 
		used in Southern and Transitional Northern communities;
	the reason for using \fm{deiḵ=} versus \fm{daaḵ=} is still unclear;
	compare similar \fm{deik=} versus \fm{daak=}
	\begin{itemize}
	\item	\vbform{i chkáx̱ deiḵ tí}{imp}[tr, \fm{∅}, mot]{put it (glove) on your hand}
		(Southern dialect) \parencite[05/79]{leer:1973}
			\vbmorph{i&ji-&ká&-x̱&\gm{deiḵ=}&\rt[²]{ti}&-μH}
				{\xx{2sg.psr}&hand&\xx{hsfc}&\·\xx{pert}&on&\rt[²]{handle}&\·\xx{var}}
		\versus \vbform{i jikáx̱ daaḵ tí}{imp}{put it (glove) on your hand} (Northern dialect)
			\vbmorph{i&ji-&ká&-x̱&daaḵ=&\rt[²]{ti}&-μH}
				{\xx{2sg.psr}&hand&\xx{hsfc}&\·\xx{pert}&on&\rt[²]{handle}&\·\xx{var}}
	\end{itemize}

\item[di]\label{m:di}
	≡ \fm{d-i-}
	combination of voice \X{d-}
		and stative \X[i-stv]{i-}
	\begin{itemize}
	\item	\vbform{sh tuditéen}{impfv}[tr, \fm{∅}, \fm{-μμH} state]{we can see ourselves}
			\vbmorph{sh=&tu-&\gm{d-}&\gm{i-}&\rt[²]{tin}&-μμH}
				{\xx{rflx.o}&\xx{1pl.s}&\xx{mid}&\xx{stv}&\rt[²]{see}&\·\xx{var}}
		\versus \vbform{tuwatéen}{impfv}{we can see him/her/it}
			\vbmorph{tu-&wa-&\rt[²]{tin}&-μμH}
				{\xx{1pl.s}&\xx{stv}&\rt[²]{see}&\·\xx{var}}
	\end{itemize}

\item[dli]\label{m:dli}
	≡ \fm{d-l-i-}
	combination of voice  \X{d-},
		valency \X{l-}/\X{lˢ-},
		and stative \X[i-stv]{i-}
	\begin{itemize}
	\item	\vbform{sh wutudlitlʼíx}{pfv}[tr, \fm{∅}, ach]{we made ourselves dirty}
			\vbmorph{sh=&wu-&tu-&\gm{d-}&\gm{l-}&\gm{i-}&\rt[¹]{tlʼix}&-μH}
				{\xx{rflx.o}&\xx{pfv}&\xx{1pl.s}&\xx{mid}&\xx{csv}&\xx{stv}&\rt[¹]{dirt}&\·\xx{var}}
		\versus \vbform{wutulitlʼíx}{pfv}{we made him/her/it dirty}
			\vbmorph{wu-&tu-&l-&i-&\rt[¹]{tlʼix}&-μH}
				{\xx{pfv}&\xx{1pl.s}&\xx{csv}&\xx{stv}&\rt[¹]{dirt}&\·\xx{var}}
	\end{itemize}

\item[du-]\label{m:du-}
	\begin{enumerate}
	\item	indefinite human subject of transitive verbs;
		see \fm{ḵaa=} and \fm{ḵu-} for indefinite human object,
		and see \fm{a-} for indefinite human subject of subject intransitive verbs
		\begin{itemize}
		\item	\vbform{x̱at wuduwax̱oox̱}{pfv}[tr, \fm{g̱}, ach]{someone/people summoned me}
				\vbmorph{x̱at=&wu-&\gm{du-}&wa-&\rt[²]{x̱ux̱}&-μμL}
					{\xx{1sg.o}&\xx{pfv}&\xx{ind.h.s}&\xx{stv}&\rt[²]{summon}&\·\xx{var}}
			\versus \vbform{x̱at woox̱oox̱}{pfv}{she/he/it summoned me}
				\vbmorph{x̱at=&wu-&μ-&\rt[²]{x̱ux̱}&-μμL}
					{\xx{1sg.o}&\xx{pfv}&\xx{stv}&\rt[²]{summon}&\·\xx{var}}
		\item	\vbform{x̱at wududziteen}{pfv}[tr, \fm{g̱}, ach]{someone/people saw me}
				\vbmorph{x̱at=&wu-&\gm{du-}&d-&s-&i-&\rt[²]{tin}&-μμL}
					{\xx{1sg.o}&\xx{pfv}&\xx{ind.h.s}&\xx{mid}&\xx{xtn}&\xx{stv}&\rt[²]{see}&\·\xx{var}}
			\versus \vbform{x̱at wusiteen}{pfv}{she/he/it saw me}
				\vbmorph{x̱at=&wu-&s-&i-&\rt[²]{tin}&-μμL}
					{\xx{1sg.o}&\xx{pfv}&\xx{xtn}&\xx{stv}&\rt[²]{see}&\·\xx{var}}
		\end{itemize}
	\item	indefinite experiencer subject;
		essentially the same as the indefinite human subject but frozen in certain sets of
		verbs describing impersonal experiences (e.g.\ feel of wind, flavour of food);
		cannot be replaced by some other referent
		\begin{itemize}
		\item	\vbform{xóon wuduwanúk}{pfv}[obj intr, \fm{∅}, ach]{north wind was felt}
				\vbmorph{xóon&wu-&\gm{du-}&wa-&\rt[²]{nuk}&-μH}
					{n·wind& \xx{pfv}&\xx{ind.s}&\xx{stv}&\rt[²]{feel}&\·\xx{var}}
		\end{itemize}
	\item	expletive/filler subject;
		occurs in some verbs to fill the subject position without referring to anything;
		cannot be replaced by some other referent
		\begin{itemize}
		\item	\vbform{x̱at kawduwasáy}{pfv}[obj intr, \fm{∅}, ach]{I got hot/sweaty}
				\vbmorph{x̱at=&ka-&w-&\gm{du-}&wa-&\rt[¹]{saÿ}&-μH}
					{\xx{1sg.o}&\xx{qual}&\xx{pfv}&\xx{xpl}&\xx{stv}&\rt[¹]{radiate}&\·\xx{var}}
		\item	\vbform{haa kawduwakʼéin}{pfv}[obj intr, \fm{g}, mot]{we jumped}
				\vbmorph{haa=&ka-&w-&\gm{du-}&wa-&\rt[¹]{kʼeᴴn}&-μμH}
					{\xx{1pl.o}&\xx{qual}&\xx{pfv}&\xx{xpl}&\xx{stv}&\rt[¹]{jump}&\·\xx{var}}
			\versus \vbform{x̱wajikʼéin}{pfv}[subj intr, \fm{g}, mot]{I jumped}
				\vbmorph{ʷ-&x̱a-&d-&sh-&i-&\rt[¹]{kʼeᴴn}&-μμH}
					{\xx{pfv}&\xx{1sg.s}&\xx{mid}&\xx{pej}&\xx{stv}&\rt[¹]{jump}&\·\xx{var}}
		\end{itemize}
	\end{enumerate}

\item[duk-]
	inalienable incorporated noun \fm{dook} ‘skin’

\item[dzi]\label{m:dzi}
	≡ \fm{d-s-i-}
	combination of voice \X{d-},
		valency \X{s-},
		and stative \X[i-stv]{i-}
	\begin{itemize}
	\item	\vbform{sh wutudzi.ée}{pfv}[tr, \fm{∅}, \fm{-μμH} act]{we cooked ourselves}
			\vbmorph{sh=&wu-&tu-&\gm{d-}&\gm{s-}&\gm{i-}&\rt[¹]{.i}&-μμH}
				{\xx{rflx.o}&\xx{pfv}&\xx{1pl.s}&\xx{mid}&\xx{csv}&\xx{stv}&\rt[¹]{cooked}&\·\xx{var}}
		\versus \vbform{wutusi.ée}{pfv}{we cooked him/her/it}
			\vbmorph{wu-&tu-&s-&i-&\rt[¹]{.i}&-μμH}
				{\xx{pfv}&\xx{1pl.s}&\xx{csv}&\xx{stv}&\rt[¹]{cooked}&\·\xx{var}}
	\end{itemize}
\end{morphdesc}

\subsection{E}\label{sec:alphalist-e}
\begin{morphdesc}[resume*=alphalist]
\item[ee]\label{m:ee}
	≡ \fm{a-i-}
	combination of argument marking \X{a-}
		and second person singular subject \X[i-2sg]{i-}

\item[ee-]
	allomorph of second person singular subject \fm{i-}

\item[ee=]
	allomorph of second person singular object \fm{i-}

\item[éeg̱i=]
	directional noun ‘beach’ with special locative postposition \fm{-í} \~\ \fm{-i} ‘at’,
	variant form \fm{éig̱i=};
	derived from noun \fm{éeḵ} \~\ \fm{éiḵ} ‘beach’;
	compare \fm{ÿeeḵ=} \~\ \fm{ÿeiḵ=} \~\ \fm{eèḵ=}

\item[eèḵ=]
	variant form of directional preverb \fm{ÿeeḵ=} \~\ \fm{ÿeiḵ=} used in Tongass Tlingit

\item[eeÿa]\label{m:eeÿa}
	≡ \fm{a-i-ÿa-}
	combination of argument marking \X{a-}
		and second person singular subject \X[i-2sg]{i-}
		and stative \X[ÿa-stv]{ÿa-}

\item[éig̱i=]
	variant form of directional noun \fm{éeg̱i=} ‘beach’ used in some Northern varieties;
	arises from uvular lowering of \fm{ée} to \fm{éi}
\end{morphdesc}

\subsection{G}\label{sec:alphalist-g}
\begin{morphdesc}[resume*=alphalist]
\item[g-, ga-]
	\fm{g} conjugation class prefix, upward spatial orientation;
	prospective aspect prefix with irrealis \fm{w-} and modal \fm{g̱-};
	can occur together with comparative \fm{g-}

\item[g-, ga-]
	irregular allomorph of comparative \fm{ka-};
	identifiable by occurence together with 
	\begin{itemize}
	\item	\vbform{chʼa yéi googéikʼ}{impfv}[obj intr, \fm{n}, \fm{-μμH} cmpv state]{just a little}
			\vbmorph{chʼa&yéi=&g-&u-&μ-&\rt[¹]{ge}&-μμH&-kʼ}
				{just&thus&\xx{cmpv}&\xx{irr}&\xx{stv}&\rt[¹]{big}&\·\xx{var}&\·\xx{dim}}
		\versus \vbform{yagéi}{impfv}[obj intr, \fm{g}, \fm{-μμH} state]{it is big}
			\vbmorph{ÿa-&\rt[¹]{ge}&-μμH}
				{\xx{stv}&\rt[¹]{big}&\·\xx{var}}
	\end{itemize}

\item[ga-]
	self-benefactive prefix, occurs with transitive verbs and requires \fm{d-};
	unclear if a \fm{g-} allomorph is possible;
	predicted to cooccur with \fm{g-} conjugation prefix but not attested;
	unclear if cooccurrence with \fm{ga-} comparative is possible
	\begin{itemize}
	\item	\vbform{at gawtudzi.ée}{pfv}[tr, \fm{∅}, ach]{we cooked something for ourselves}
			\vbmorph{at=&\gm{ga-}&w-&tu-&\gm{d-}&s-&i-&\rt[¹]{.i}&-μμH}
				{\xx{ind.n.o}&\xx{sben}&\xx{pfv}&\xx{1pl.s}&\xx{mid}&\xx{csv}&\xx{stv}&\rt[¹]{cooked}&\·\xx{var}}
		\versus \vbform{at wutusi.ée}{pfv}{we cooked something}
			\vbmorph{at=&wu-&tu-&s-&i-&\rt[¹]{.i}&-μμH}
				{\xx{ind.n.o}&\xx{pfv}&\xx{1pl.s}&\xx{csv}&\xx{stv}&\rt[¹]{cooked}&\·\xx{var}}
	\end{itemize}

\item[gági=]
	directional preverb ‘emerging, out into the open’;
	derived from noun \fm{gáak} ‘protrusion’
		with special locative postposition \fm{-í} \~\ \fm{-i};
	occurs in motion derivation
		\motderiv{gági}{\fm{∅}, \fm{-x̱} rep}{emerging, out into the open}
	\begin{itemize}
	\item	\fm{gági uwaháa du waḵshayeexʼ chʼáakʼ ḵuyéik}
		‘it emerged before his eyes, the eagle spirit’
		\parencite[01/6]{leer:1973}
			\vbmorph{gági=&u-&wa-&\rt[¹]{haᴸ}&-μμH}
				{emerge=&\xx{zpfv}&\xx{stv}&\rt[¹]{appear}&\·\xx{var}}
	\end{itemize}

\item[gug̱a]
	≡ \fm{g-u-g̱a-}
	combination of conjugation \fm{g-},
		irrealis \fm{u-},
		and  modal \fm{g̱a-},
		together indicating prospective (‘future’) aspect;
	this form occurs when there is no subject prefix and no
		immediately preceding vowel from an incorporated noun, object prefix, preverb, etc.;
	\fm{kg̱wa} occurs instead if there is a preceding vowel;
	compare \fm{kuḵa} with first person singular subject \fm{x̱a-}
	\begin{itemize}
	\item	\vbform{at gug̱ax̱áa}{prosp}[tr, \fm{∅}, \fm{-μμH} act]{she/he/it will eat something}
			\vbmorph{at=&g-&u-&g̱a-&\rt[²]{x̱a}&-μμH}
				{\xx{ind.n.o}&\xx{gcnj}&\xx{irr}&\xx{mod}&\rt[²]{eat}&\·\xx{var}}
		\versus \vbform{akg̱wax̱áa}{prosp}{she/he/it will eat him/her/it}
			\vbmorph{a-&g-&u-&g̱a-&\rt[²]{x̱a}&-μμH}
				{\xx{3>3}&\xx{gcnj}&\xx{irr}&\xx{mod}&\rt[²]{eat}&\·\xx{var}}
	\end{itemize}
\end{morphdesc}

\subsection{G̱}\label{sec:alphalist-gh}
\begin{morphdesc}[resume*=alphalist]
\item[g̱-, g̱a-]
	\fm{g̱} conjugation class prefix, downward spatial orientation;
	can occur together with modal \fm{g̱-}

\item[g̱-, g̱a-]
	modal prefix in prospective aspect,
		hortative mood,
		potential mood,
		and contingent mood;
	can occur together with \fm{g̱-} conjugation class prefix in hortative, potential, contingent;
	\newline
	allomorphs:
	\begin{allolist}
	\item[-x̱]	consonant in coda of a syllable
	\end{allolist}
	combinations:
	\begin{allolist}
	\item[ḵ, ḵa]	≡ \fm{g̱-x̱a-} with first person singular subject \fm{x̱a-}
	\end{allolist}
	\begin{enumerate}
	\item	prospective aspect:
			conjugation \fm{g-}
			+ irrealis \fm{u-}
			+ modal \fm{g̱-}
		\begin{itemize}
		\item	\fm{gug̱atáa} (prosp; \fm{n}, \fm{-μH} act) ‘s/he/it will sleep’
				with \fm{g-u-g̱a-\rt[¹]{taᴸ}-μμH}\newline
			versus \fm{wootaa} (pfv) ‘s/he/it slept’
				with \fm{wu-μ-\rt[¹]{taᴸ}-μμL}
		\end{itemize}
	\item	hortative mood:
			conjugation \fm{n-}/\fm{g̱-}/\fm{g-}
			+  modal \fm{g̱-}
	\item	potential mood:
			irrealis \fm{u-}
			+ conjugation \fm{n-}/\fm{g̱-}/\fm{g-}
			+ modal \fm{g̱-}
	\item	contingent mood:
			conjugation \fm{n-}/\fm{g̱-}/\fm{g-}
			+ modal \fm{g̱-}
			(+ \fm{-n} + \fm{-ín})
	\end{enumerate}

\item[g̱unéi=]
	variant form of inceptive \fm{g̱unayéi} ‘starting, beginning’, arising from contraction;
	some speakers use only \fm{g̱unéi} with verbs and have \fm{g̱unayéi} only as a noun

\item[g̱ax̱=]
	incorporated noun ‘crying’, saturates object;
	derived from \fm{\rt[¹]{g̱ax̱}} ‘cry’
	\begin{itemize}
	\item	plural \fm{g̱ax̱satí} (impfv; subj intr?, \fm{g}, \fm{-μH} act) ‘they cry’ (with \fm{s-\rt[¹]{tiᴸ}} ‘be’)\newline
		versus singular \fm{g̱áax̱} (impfv; subj intr, \fm{g}, \fm{-μμH} act) ‘s/he/it cries’
	\item	\fm{kei gax̱ gax̱yisatée} (prosp) ‘you pl.\ will cry’
		\parencite[60.683]{story-naish:1973}
	\end{itemize}

\item[g̱unayéi=]
	inceptive preverb indicating initiation of motion or other eventuality;
	variant form \fm{g̱unéi} arising from contraction;
	derived from the noun \fm{g̱unayéi} ‘elsewhere, different place’
		from \fm{g̱una} ‘different, other’
		and \fm{yé} \~\ \fm{yéi} ‘place, way’
		probably with \fm{-μ} allomorph of locative postposition \fm{-xʼ}
	\begin{enumerate}
	\item	initiation of motion;
		motion derivation
			\fm{g̱unayéi} \~\ \fm{g̱unéi} (\fm{∅}, \fm{-x̱} rep) ‘starting off, setting out’
		\begin{itemize}
		\item	\fm{g̱unayéi wutuwa.át} (pfv; subj intr, \fm{∅}, \fm{-x̱} rep) ‘we started off’\newline
			versus \fm{wutuwa.aat} (pfv; subj intr, \fm{n}, \fm{yoo=i-…-k} rep) ‘we went’
		\end{itemize}
	\item	initiation of other eventuality;
		eventuality derivation
			\fm{g̱unayéi} \~\ \fm{g̱unéi} (\fm{∅}, ach, \fm{-x̱} rep)
				‘beginning, starting, initiating’
		\begin{itemize}
		\item	\fm{g̱unayéi aawax̱áa} (pfv; tr, \fm{∅}, ach) ‘s/he/it started eating him/her/it’\newline
			versus \fm{aawax̱áa} (pfv; tr, \fm{∅}, \fm{-μH} act) ‘s/he/it ate him/her/it’
		\end{itemize}
	\end{enumerate}
\end{morphdesc}

\subsection{H}\label{sec:alphalist-h}
\begin{morphdesc}[resume*=alphalist]
\item[haa=]
	first person plural object
	(compare \fm{haa keidlí áwé} ‘it is our dog’);
	although this is homophonous with the possessive pronoun,
		\fm{haa=} as an object is not necessarily possessive
	\begin{itemize}
	\item	\fm{haa yisiteen} (pfv; tr, \fm{g̱}, ach) ‘you sg.\ saw us’\newline
		versus \fm{x̱at yisiteen} (pfv) ‘you sg.\ saw me’
	\end{itemize}

\item[has=]
	human pluralizer for third person;
	\newline
	allomorphs:
	\begin{allolist}
	\item[as=]	onset glottal stop instead of \fm{h} used in Southern and Tongass varieties
	\item[s=]	lone consonant, usually coda of a preceding syllable
	\end{allolist}
	note that since Tlingit is number neutral (nouns are not singular by default),
		a form without \fm{has=} may still refer to plural humans,
		i.e.\ \fm{has=} is not required for third person human plural arguments
	\begin{enumerate}
	\item	human pluralizer for third person subject
		\begin{itemize}
		\item	\fm{tʼá aawax̱áa} (pfv; tr, \fm{∅}, \fm{-μH} act) ‘s/he/it ate king salmon’\newline
			versus \fm{tʼá has aawax̱áa} (pfv) ‘they (humans) ate king salmon’
		\end{itemize}
	\item	human pluralizer for third person object
		\begin{itemize}
		\item	\fm{has tushikʼáan} (impfv; tr, \fm{g}, \fm{-μμH} state) ‘we hate them’\newline
			versus \fm{yee tushikʼáan} (impfv) ‘we hate you guys’
		\end{itemize}
	\item	human pluralizer for both third person subject and third person object;
		some speakers do not accept this use of \fm{has=}
			for both subject and object at the same time
		\begin{itemize}
		\item	\fm{has awsiteen} (pfv; tr, \fm{g}, ach) ‘they saw them’
			or ‘s/he/it saw them’ or ‘they saw him/her/it’
		\end{itemize}
	\end{enumerate}
\end{morphdesc}

\subsection{I}\label{sec:alphalist-i}
\begin{morphdesc}[resume*=alphalist]
\item[i-]\label{m:i-2sg}
	second person singular subject or object; long vowel allomorphs are \fm{ee-} and \fm{ee=};
	subject versus object is typically distinguished by position in the verb word but can sometimes be ambiguous
	\begin{enumerate}
	\item	second person singular subject
		\begin{itemize}
		\item	\vbform{x̱at iyatéen}{impfv}[tr, \fm{g}, \fm{-μμH} state, only impfv]{you sg.\ can see me}
				\vbmorph{x̱at=&i-&ya-&\rt[²]{tin}&-μμH}
					{\xx{1sg.o}&\xx{2sg.s}&\xx{stv}&\rt[²]{see}&\·\xx{var}}
			\versus \vbform{ayatéen}{impfv}{she/he/it can see him/her/it}
				\vbmorph{a-&ya-&\rt[²]{tin}&-μμH}
					{\xx{3>3}&\xx{stv}&\rt[²]{see}&\·\xx{var}}
		\end{itemize}
	\item	second person singular оbject
		\begin{itemize}
		\item	\vbform{ix̱aatéen}{impfv}[tr, \fm{g}, \fm{-μμH} state, only impfv]{I can see you sg.}
				\vbmorph{i-&x̱a-&μ-&\rt[²]{tin}&-μμH}
					{\xx{2sg.o}&\xx{1sg.s}&\xx{stv}&\rt[²]{see}&\·\xx{var}}
		\end{itemize}
	\item	ambiguous: either second person subject or object in certain imperfective verb forms;
		when one of two arguments is third person and thus not indicated by the verb,
		the \fm{i-} argument prefix is ambiguous between subject and object and must
		be distinguished by other means (third person subject or object phrase,
		discourse context, etc.);
		this ambiguity does not arise if qualifiers, incorporated nouns, or aspect prefixes
		are present because they occur between the subject and object prefixes and so
		distinguish subject and object
		\begin{itemize}
		\item	\vbform{iyatéen}{impfv}[tr, \fm{g}, \fm{-μμH} state, only impfv]{you sg.\ can see him/her/it}
				\vbmorph{i-&ya-&\rt[²]{tin}&-μμH}
					{\gm{\xx{2sg.s}}&\xx{stv}&\rt[²]{see}&\·\xx{var}}
			\versus \vbform{iyatéen}{impfv}{she/he/it can see you sg.}
				\vbmorph{i-&ya-&\rt[²]{tin}&-μμH}
					{\gm{\xx{2sg.o}}&\xx{stv}&\rt[²]{see}&\·\xx{var}}
		\end{itemize}
	\end{enumerate}

\item[i-]\label{m:i-stv}
	stative prefix of classifier;
	\newline
	allomorphs:
	\begin{allolist}
	\item[ÿa-]	full syllable
	\item[wa-]	labialized form of \fm{ÿa-}
	\item[μ-]	lengthening of preceding vowel
	\end{allolist}
\end{morphdesc}

\subsection{J}\label{sec:alphalist-j}
\begin{morphdesc}[resume*=alphalist]
\item[ji-]\label{m:ji-}
	incorporated noun indicating hand or possession;
	qualifier indicating object with extended projections (fingers);
	derived from relational nouns \fm{jín} ‘hand’ and \fm{jee} ‘possession’

\item[ji]\label{m:ji}
	≡ \fm{d-sh-i-}
	combination of voice \X{d-},
		valency \X{sh-},
		and stative \X[i-stv]{i-}
\end{morphdesc}

\subsection{K}\label{sec:alphalist-k}
\begin{morphdesc}[resume*=alphalist]
\item[k-, ka-]
	incorporated noun ‘horizontal surface’,
	derived from relational noun \fm{ká} ‘horizontal surface, flat top of’;
	can occur together with one of
		qualifier \fm{ka-} ‘small and round’
		or qualifier \fm{ka-} of unknown meaning
		or comparative \fm{ka-}

\item[k-, ka-]
	qualifier ‘small and round’;
	can occur together with one of
		incorporated noun \fm{ka-} ‘horizontal surface’
		or qualifier \fm{ka-} of unknown meaning
		or comparative \fm{ka-}

\item[k-, ka-]
	qualifier of unknown meaning;
	can occur together with one of
		qualifier \fm{ka-} ‘small and round’ 
		or incorporated noun \fm{ka-} ‘horizontal surface’
		or comparative \fm{ka-}

\item[k-, ka-]
	comparative prefix, used along with irrealis \fm{u-} \~\ \fm{oo-} \~\ \fm{w-};
	required in comparative forms of state verbs that denote dimensions (e.g.\ short, heavy);
	can occur together with one of
		qualifier \fm{ka-} ‘small and round’
		or incorporated noun \fm{ka-} ‘horizontal surface’
		or qualifier \fm{ka-} of unknown meaning
	\begin{itemize}
	\item	\vbform{ḵúdáx̱ koodáal}{impfv}[obj intr, \fm{n}, \fm{-μμH} cmpv state]{she/he/it is too heavy}
			\vbmorph{ḵúdáx̱&k-&u-&μ-&\rt[¹]{dal}&-μμH}
				{too.much&\xx{cmpv}&\xx{irr}&\xx{stv}&\rt[¹]{heavy}&\·\xx{var}}
		\versus \vbform{yadál}{impfv}[obj intr, \fm{n}, \fm{-μH} state]{she/he/it is heavy}
			\vbmorph{ÿa-&\rt[¹]{dal}&-μH}
				{\xx{stv}&\rt[¹]{heavy}&\·\xx{var}}
	\end{itemize}

\item[-k]
	repetitive suffix;
	allomorph \fm{-kw} with labialization

\item[-kw]
	allomorph of repetitive \fm{-k} with labialization

\item[-kʼ]
	diminutive suffix;
	allomorph \fm{-kʼw} with labialization

\item[-kʼw]
	allomorph of diminutive \fm{-kʼ} with labialization

\item[kaawa]
	≡ \fm{ka-μʷ-wa-}
	combination of any prefix of the form \fm{ka-},
		perfective \fm{μʷ-},
		and stative \fm{wa-}
	\begin{itemize}
	\item	\vbform{kaawagaan}{pfv}[obj intr, \fm{g̱}, ach]{she/he/it burned}
			\vbmorph{\gm{ka-}&\gm{μʷ-}&\gm{wa-}&\rt[¹]{gan}&-μμL}
				{\xx{hsfc}&\xx{pfv}&\xx{stv}&\rt[¹]{burn}&\·\xx{var}}
		\versus \vbform{woogaan}{pfv}[obj intr, \fm{g̱}, ach]{she/he/it burned}
			\vbmorph{wu-&μ-&\rt[¹]{gan}&-μμL}
				{\xx{pfv}&\xx{stv}&\rt[¹]{burn}&\·\xx{var}}
	\end{itemize}

\item[keey-]
	inalienable incorporated noun \fm{keey} ‘knee’
	\begin{itemize}
	\item	\vbform{yan x̱at keeyshakawdligásʼ}{pfv}[obj intr, \fm{∅}, mot]{I fell down and skidded on my knees}
		\parencite[193.2689]{story-naish:1973}
			\vbmorph{yan=&x̱at=&\gm{keey-}&sha-&ka-&w-&d-&l-&i-&\rt[¹]{gasʼ}&-μH}
				{ground&\xx{1sg.o}&knee&head&\xx{hsfc}&\xx{pfv}&\xx{mid}&\xx{xtn}&\xx{stv}&\rt[¹]{end·fall}&\·\xx{var}}
	\end{itemize}

\item[keeya]
	≡ \fm{ka-μʷ-i-ya-}
	combination of any prefix of the form \fm{ka-},
		perfective \fm{μʷ-},
		second person singular subject \fm{i-},
		and stative \fm{ÿa-}
	\begin{itemize}
	\item	\vbform{keeyayúk}{pfv}[tr, \fm{∅}, ach]{you sg.\ shook him/her/it}
			\vbmorph{\gm{ka-}&\gm{μʷ-}&\gm{i-}&\gm{ya-}&\rt[²]{yuᴴk}&-μH}
				{\xx{qual}&\xx{pfv}&\xx{2sg.s}&\xx{stv}&\rt[²]{shake}&\·\xx{var}}
		\versus \vbform{akaawayúk}{pfv}[tr, \fm{∅}, ach]{she/he/it shook him/her/it}
			\vbmorph{a-&ka-&μʷ-&wa-&\rt[²]{yuᴴk}&-μH}
				{\xx{3>3}&\xx{qual}&\xx{pfv}&\xx{stv}&\rt[²]{shake}&\·\xx{var}}
	\end{itemize}

\item[kḵwa]
	≡ \fm{g-u-g̱-x̱a-}
	combination of conjugation \fm{g-},
		irrealis \fm{u-},
		and modal \fm{g̱a-},
			together indicating prospective (‘future’) aspect,
		along with first person singular subject \fm{x̱-} / \fm{x̱a-};
	this form occurs when there is an
		immediately preceding vowel (incorporated noun, object prefix, preverb, etc.);
	\fm{kuḵa} occurs instead if there is no preceding vowel
	\begin{itemize}
	\item	\fm{yee kḵwax̱áa} (prosp) ‘I will eat you (pl.)’
			with \fm{ÿee=g-u-g̱-x̱a-}\newline
		versus \fm{at kuḵax̱áa} (prosp; \fm{∅}, \fm{-μμH} act) ‘I will eat something’
			with \fm{at=g-u-g̱-x̱a-}
	\end{itemize}

\item[kuḵa]
	≡ \fm{g-u-g̱-x̱a-}
	combination of conjugation \fm{g-},
		irrealis \fm{u-} prefix,
		and modal \fm{g̱a-},
			together indicating prospective (‘future’) aspect,
		along with first person singular subject \fm{x̱-} / \fm{x̱a-};
	this form occurs when there is no
		immediately preceding vowel (incorporated noun, object prefix, preverb, etc.);
	\fm{kḵwa} occurs instead if there is a preceding vowel
	\begin{itemize}
	\item	\fm{at kuḵax̱áa} (prosp; \fm{∅}, \fm{-μμH} act) ‘I will eat something’
			with \fm{at=g-u-g̱-x̱a-}\newline
		versus \fm{yee kḵwax̱áa} (prosp) ‘I will eat you (pl.)’
			with \fm{ÿee=g-u-g̱-x̱a-}
	\end{itemize}

\item[kuḵwa]
	variant of \fm{kuḵa}
	
\item[kwḵa]
	variant of \fm{kḵwa} used primarily in \cite{story-naish:1973}
\end{morphdesc}

\subsection{Ḵ}\label{sec:alphalist-kh}
\begin{morphdesc}[resume*=alphalist]
\item[ḵ, ḵa]
	≡ \fm{g̱-x̱-}
	combination of first person singular subject \fm{x̱-} / \fm{x̱a-} with either one of
		conjugation \fm{g̱-}
		or modal \fm{g̱-}
	\begin{enumerate}
	\item	with conjugation \fm{g̱-}: \fm{g̱-x̱a-} → \fm{ḵa}
	\item	with modal\fm{g̱-}: \fm{g̱-x̱a-} → \fm{ḵa}
	\end{enumerate}

\item[-ḵ]
	optative/prohibitive modality suffix;
	\newline
	allomorphs:
	\begin{allolist}
	\item[-ḵw] 	with labialization,
	\item[-íḵ \~\ -iḵ] with epenthetic (filler) vowel \fm{í} \~\ \fm{i}
	\item[-úḵ \~\ -uḵ] with epenthetic (filler) vowel \fm{í} \~\ \fm{i} and labialization
	\end{allolist}
	always occurs with a preceding particle (see below) and usually with irrealis \fm{u-},
	in contrast with deprivative \fm{-ḵ} which does not require a particle and irrealis;
	both opt./prohib.\ \fm{-ḵ} and depriv.\ \fm{-ḵ} are related to Eyak negative \fm{-ɢ}
	\begin{enumerate}
	\item	with optative particle \fm{gu.aal} ‘hopefully’
			and usually also \fm{gushí} \~\ \fm{kwshé} ‘maybe’
	\item	with prohibitive particle \fm{líl} ‘don’t’
			or alternatively with negative particle \fm{tléil} ‘not’
	\end{enumerate}

\item[-ḵ]
	deprivative suffix \fm{-ḵ} \~\ \fm{-ḵw} ‘lacking, removed’;
	\newline
	allomorphs:
	\begin{allolist}
	\item[-ḵw]	with labialization
	\item[-áḵw]	with epenthetic (filler) vowel \fm{á}
	\end{allolist}
	contrast optative/prohibitive \fm{-ḵ} which is accompanied by a particle;
	both depriv.\ \fm{-ḵ} and opt./prohib.\ \fm{-ḵ} are related to Eyak negative \fm{-ɢ}
	\begin{enumerate}
	\item	lacking, without, removed:
		\fm{\rt{shisʼ}} ‘raw’, \fm{\rt{til}} ‘shoe’, \fm{\rt{tsaxʼ}} ‘mitten’
	\item	unclear meaning, followed by plural \fm{-xʼ}:
		\fm{\rt{tsin}} ‘strong; expensive’, \fm{\rt{ÿatʼ}} ‘long’
	\item	unclear meaning, lexically specified with certain roots (see also \fm{-áḵw})ː
		\fm{\rt{xʼwásʼ}} ‘numb’, \fm{\rt{ÿash}} ‘scarce’
	\end{enumerate}

\item[ḵaa=]
	allomorph of indefinite human object \fm{ḵu-} ‘someone, people, one, them’;
	possibly like \fm{ax̱=} used only as possessor of incorporated noun
		(compare \fm{ḵaa keidlí áwé} ‘it’s someone’s dog’)
	\begin{itemize}
	\item	\fm{ḵaa seiwa.áx̱} (pfv; tr, \fm{∅}, ach) ‘s/he/it heard someone’s voice’
	\end{itemize}

\item[ḵu-]
	indefinite human object ‘someone, people, one, them’;
	allomorph \fm{ḵaa=} as possessor of incorporated noun;
	compare PP pronoun \fm{ḵú-} ‘someone, people, one, them’
	\begin{itemize}
	\item	\fm{ḵuwsiteen} (pfv; tr, \fm{g̱}, ach) ‘s/he/it saw someone/people’
	\end{itemize}

\item[ḵu-]
	areal prefix indicating space, area, extent, or weather;
	compare \fm{ḵúx̱(-de)=} ‘back, returning’, \fm{ḵut=} ‘lost’
	\begin{itemize}
	\item	\fm{ḵuwakʼéi} (impfv; impers, \fm{g}, \fm{-μμH} state) ‘it is good weather’\newline
		versus \fm{yakʼéi} (impfv; obj intr) ‘it is good’
	\end{itemize}
\end{morphdesc}

\subsection{L}\label{sec:alphalist-l}
\begin{morphdesc}[resume*=alphalist]
\item[l-]\label{m:l-}
	valency prefix of classifier
	\begin{enumerate}
	\item	argument addition
		\begin{enumerate}
		\item	lone argument of intransitive
		\item	causative
		\item	applicative
		\end{enumerate}
	\item	spatial extension
		\begin{enumerate}
		\item	extended entity
		\item	extended eventuality
		\end{enumerate}
	\end{enumerate}

\item[lˢ-]\label{m:lˢ-}
	allomorph of valency \fm{s-} \~\ \fm{sa-};
	occurs when any fricative
		\fm{s}, \fm{sʼ}, \fm{l}, \fm{lʼ}, \fm{sh}
	or any affricate
		\fm{dz}, \fm{ts}, \fm{tsʼ},
		\fm{dl}, \fm{tl}, \fm{tlʼ},
		\fm{j}, \fm{ch}, \fm{chʼ}
	occurs in the onset or coda of the stem syllable;
	phonetically indistinguishable from \fm{l-} \~\ \fm{la-} and may be represented as such
	if the distinction is not important
	\begin{itemize}
	\item	\fm{lichán} (impfv; obj intr, \fm{g}, \fm{-μH} invar.\ state) ‘it stinks’
		(not *\fm{sichán})
	\item	\fm{wutuliyíḵsʼ} (rep pfv; tr, \fm{n}, mot) ‘we repeatedly pulled it (long obj.)’\newline
		versus \fm{wutusiyeeḵ} (pfv) ‘we pulled it (long obj.)’
	\end{itemize}

\item[…l]\label{m:…l}
	≡ \fm{d-l-}
	combination of voice \X{d-}
		and valency \X{l-} or \X{lˢ-},
	appears only as a coda consonant and so requires a preceding vowel
	\begin{itemize}
	\item	\fm{sh ilg̱ásʼx̱} (rep impfv; tr, \fm{∅}, ach) ‘s/he/it repeatedly scratches self’
			with \fm{d-l-}\newline
		versus \fm{sh wudlig̱ásʼ} (pfv) ‘s/he/it scratched self’
			with \fm{d-l-i-}
	\item	\fm{tléil sh kawulháachʼ} (neg pfv; tr, \fm{n}, \fm{-μμH} state) ‘s/he/it didn’t shame self’
			with \fm{d-l-}
		versus \fm{sh kawdliháachʼ} (pfv) ‘s/he/it shamed self’
			with \fm{d-l-i-}
	\end{itemize}

\item[-lʼ]
	repetitive suffix, limited to a few verbs;
	allomorph \fm{-álʼ} with epenthetic (filler) vowel \fm{á};\
	probably historically related to \fm{-sʼ}

\item[la-]\label{m:la-val}
	allomorph of valency \X{l-}

\item[la-]\label{m:la-throat}
	allomorph of incorporated noun \X{le-} ‘throat’

\item[lˢa-]\label{m:lˢa-}
	allomorph of valency \X{lˢ-}

\item[le-]\label{m:le-}
	incorporated noun indicating throat or inside of mouth

\item[li]
	≡ \fm{l-i-}
	combination of valency \X{l-} or \X{lˢ-}
		and stative \X[i-stv]{i-}
\end{morphdesc}

\subsection{M}\label{sec:alphalist-m}
\begin{morphdesc}[resume*=alphalist]
\item[m-]\label{m:m-}
	allomorph of perfective \fm{wu-} in coda of a syllable;
	currently used instead of perfective \fm{w-} only in Inland Tlingit varieties,
	but may also occur elsewhere in older Tlingit (e.g.\ song lyrics, set phrases)
	\begin{itemize}
	\item	\fm{amsiteen} (pfv; tr, \fm{g̱}, ach) ‘s/he/it caught sight of (saw) him/her/it’\newline
		(instead of \fm{awsiteen})\newline
		versus \fm{x̱at wusiteen} (pfv) ‘s/he/it caught sight of (saw) me’
	\end{itemize}
\end{morphdesc}

\subsection{N}\label{sec:alphalist-n}
\begin{morphdesc}[resume*=alphalist]
\item[n-, na-]
	\begin{enumerate}
	\item	\fm{n} conjugation class prefix, horizontal spatial orientation
		\begin{itemize}
		\item	\fm{nayx̱éixʼw!} (imp; subj intr, \fm{n}, \fm{-μμH} act) ‘you guys (go to) sleep!’
		\item	\fm{naḵahoon} (hort; tr, \fm{n}, \fm{-μμH} act) ‘let me sell it’
		\end{itemize}
	\item	progressive aspect prefix
		\begin{itemize}
		\item	\fm{yaa nx̱ax̱éin} (prog; tr, \fm{∅}, act) ‘I am going along eating it’\newline
			versus \fm{x̱ax̱á} (impfv) ‘I am eating it’
		\end{itemize}
	\end{enumerate}

\item[-n]
	stem suffix of uncertain meaning;
	causes ablaut /\ipa{a}, \ipa{u}/ → [\ipa{eː}] of \fm{\rt{Ca}} and \fm{\rt{Cu}} roots
	except for \fm{\rt[¹]{naᴸ}} ‘die’ and \fm{\rt[²]{ya}} ‘pack’
	\begin{enumerate}
	\item	with progressive aspect
		\begin{itemize}
		\item	\fm{yaa anax̱éin} (prog; tr, \fm{∅}, \fm{-μH} act) ‘s/he/it is going along eating him/her/it’ with root \fm{\rt[²]{x̱a}} ‘eat’\newline
			(not \fm[*]{yaa anax̱áan})\newline
			versus \fm{aawax̱áa} (pfv) ‘s/he/it ate him/her/it’
		\item	\fm{yaa anaskwéin} (prog; subj intr, \fm{∅}, ach) ‘s/he/it is coming to know him/her/it’ with root \fm{\rt[²]{kuᴸ}} ‘know’\newline
			(not \fm[*]{yaa anaskóon})\newline
			versus \fm{awsikóo} (pfv) ‘s/he/it came to know him/her/it’
		\end{itemize}
	\item	with conditional mood
	\item	with contingent mood
	\item	irregularly in a few imperfective state verbs
	\end{enumerate}
\end{morphdesc}

\subsection{O}\label{sec:alphalist-o}
\begin{morphdesc}[resume*=alphalist]
\item[oo-]\label{m:oo-}
	allomorph of irrealis \fm{u-}
	
\item[oo]\label{m:oo}
	≡ \fm{a-u-}
	combination of argument marking \X{a-}
		and either irrealis \X[u-irr]{u-}
			or \fm{∅} conjugation perfective \X[u-pfv]{u-}

\item[oowa]\label{m:oowa}
	≡ \fm{a-u-wa-}
	combination of argument marking \X{a-}
		and either irrealis \X[u-irr]{u-}
			or \fm{∅} conj perfective \X[u-pfv]{u-}
		and stative \X{wa-}
\end{morphdesc}

\subsection{S}\label{sec:alphalist-s}
\begin{morphdesc}[resume*=alphalist]
\item[s-]\label{m:s-}
	valency prefix of classifier
	\begin{enumerate}
	\item	argument addition
		\begin{enumerate}
		\item	lone argument of intransitive
		\item	causative
		\item	applicative
		\end{enumerate}
	\item	spatial extension
		\begin{enumerate}
		\item	extended entity
		\item	extended eventuality
		\end{enumerate}
	\end{enumerate}

\item[…s]\label{m:…s}
	≡ \fm{d-s-}
	combination of voice \X{d-}
		and valency \X{s-},
	appears only as a coda consonant and so requires a preceding vowel
	\begin{itemize}
	\item	\fm{yax̱ sh x̱asnei} (rep impfv; tr, \fm{∅}, ach) ‘I repeatedly dress myself’
			with \fm{d-s-}\newline
		versus \fm{yan sh x̱wadzinéi} (pfv) ‘I dressed myself’
			with \fm{d-s-i-}
	\end{itemize}

\item[s=]
	allomorph of human pluralizer \fm{has=} for third person subject or object

\item[sa-]\label{m:sa-val}
	allomorph of valency \X{s-}

\item[sa-]\label{m:sa-voice}
	allomorph of incorporated noun \X{se-} ‘voice’

\item[se-]\label{m:se-}
	incorporated noun indicating voice or vocalization

\item[sh=]
	reflexive object;
	requires middle voice \fm{d-}

\item[sh-]\label{m:sh-}
	valency prefix of classifier
	\begin{enumerate}
	\item	pejorative
		\begin{enumerate}
		\item	pejorative entity
		\item	pejorative eventuality
		\end{enumerate}
	\item	negative
	\item	unclear meaning
	\end{enumerate}

\item[…sh]\label{m:…sh}
	≡ \fm{d-sh-}
	combination of voice \X{d-}
		and valency \X{sh-},
	appears only as a coda consonant and so requires a preceding vowel
	\begin{itemize}
	\item	\fm{yaa sh kanx̱ashxʼáḵw} (prog; tr, \fm{n}, ach) ‘I am making myself comfortable’
			with \fm{d-sh-}\newline
		versus \fm{sh kax̱wjixʼaaḵw} (pfv) ‘I made myself comfortable’
			with \fm{d-sh-i-}
	\end{itemize}

\item[sha-]\label{m:sha-val}
	allomorph of valency \X{sh-}

\item[sha-]
	incorporated noun indicating head or hair of the head;
	derived from relational noun \fm{shá} ‘head’

\item[shakux=]
	incorporated noun ‘thirst’,
	saturates object argument;
	derived from \fm{shá} ‘head’ and \fm{\rt[¹]{kux}} ‘dry’
		in verb \fm{x̱at shaawakúx} (pfv; obj intr, \fm{∅}, ach) ‘I got thirsty’
		suggesting a nominalization \fm{shakoox} ‘thirsting’
	\begin{itemize}
	\item	\fm{ax̱ éet shakux uwaháa} (pfv; obj intr, \fm{∅}, mot) ‘thirst appeared to me’,
		i.e.\ ‘I got thirsty’
		\parencite[01/11]{leer:1973}
	\end{itemize}

\item[-shán]
	allomorph of intensifier \fm{-chʼán},
	used when immediately following an ejective consonant

\item[shi]
	≡ \fm{sh-i-}
	combination of valency \X{sh-}
		and stative \X[i-stv]{i-}

\item[si]
	≡ \fm{s-i-}
	combination of valency \X{s-}
		and stative \X[i-stv]{i-}
\end{morphdesc}

\subsection{T}\label{sec:alphalist-t}
\begin{morphdesc}[resume*=alphalist]
\item[tu-]
	first person plural subject;
	note that \cite{story-naish:1973} write all cases of \fm{tu-} as \fm{too-}
		so they do not distinguish the two allomorphs
	\begin{itemize}
	\item	\fm{wutuwax̱áa} (pfv; tr, \fm{∅}, \fm{-μH} act) ‘we ate it’\newline
		versus \fm{toox̱á} (impfv) ‘we eat it; we are eating it’
	\end{itemize}

\item[tu-]
	incorporated noun ‘inside; mind, emotion, bodily spirit’;
	derived from relational noun \fm{tú} ‘inside of (hollow object)’
	used metaphorically as ‘mind, emotion, bodily spirit’ as in \fm{ax̱ toowú yanéekw} ‘my mind hurts’

\item[too-]
	allomorph of first person plural subject \fm{tu-}
	\begin{itemize}
	\item	\fm{toox̱á} (pfv; tr, \fm{∅}, \fm{-μH} act) ‘we eat it; we are eating it’\newline
		versus \fm{wutuwax̱áa} (pfv) ‘we ate it’
	\end{itemize}
\end{morphdesc}

\subsection{U}\label{sec:alphalist-u}
\begin{morphdesc}[resume*=alphalist]
\item[u-]\label{m:u-irr}
	irrealis prefix

\item[u-]\label{m:u-pfv}
	\fm{∅} conjugation class perfective,
	occurring in some perfective aspect forms and some habitual aspect forms of
	\fm{∅} conjugation class verbs
\end{morphdesc}

\subsection{W}\label{sec:alphalist-w}
\begin{morphdesc}[resume*=alphalist]
\item[ʷ-]\label{m:ʷ-pfv}
	allomorph of perfective \fm{wu-} when followed by
	first person singular subject \fm{x̱-} \~\ \fm{x̱a-},
	labializing its fricative to form \fm{x̱w} or \fm{x̱wa}
	(phonetically [\ipa{χʷ}] or [\ipa{χʷa}])

\item[ʷ-]\label{m:ʷ-irr}
	allomorph of irrealis \fm{u-}

\item[w-]\label{m:w-pfv}
	allomorph of perfective \fm{wu-} in coda of a syllable;
	in Inland Tlingit \fm{m-} is used instead, may also occur elsewhere in older Tlingit
		(e.g.\ song lyrics);
	19th century Tlingit occasionally has full \fm{wu-} rather than \fm{w-},
		e.g.\ \fm{awusikóo du éesh hídi} ‘she knew her father’s house’
		\parencite[255.7]{swanton:1909}
	\begin{itemize}
	\item	\fm{awsiteen} (pfv; tr, \fm{g̱}, ach) ‘s/he/it caught sight of (saw) him/her/it’\newline
		versus \fm{x̱at wusiteen} (pfv) ‘s/he/it caught sight of (saw) me’
	\end{itemize}

\item[w-]\label{m:w-irr}
	allomorph of irrealis \fm{u-}

\item[w̃-]\label{m:w̃-}
	variant form of perfective \fm{w-} allomorph in coda of a syllable;
	occurs in some varieties where \fm{m-} formerly occurred, compare \fm{m-} still used
	in Teslin and Carcross/Tagish Tlingit varieties

\item[wa-]\label{m:wa-}
	allomorph of stative \fm{ÿa-} when preceded by labialized (round) sound

\item[wu-]\label{m:wu-}
	perfective prefix used in most perfective aspect forms;
	see also \fm{∅} conjugation class perfective \fm{u-};
	the perfective prefix has a very complex phonology with many different patterns that depend
	on both	preceding and following prefixes;
	see individual entries for specific details and verb prefix charts for comprehensive patterns;
	reconstructed as \fm[*]{ŋu-} \~\ \fm[*]{ŋʷ-} cognate with Proto-Dene \fm[*]{ŋi-} perfective
	\newline
	allomorphs:
	\begin{allolist}
	\item[\X{m-}]	coda consonant following a vowel, variant form of \fm{w-} used in Teslin
			and Carcross/Tagish varieties
	\item[{\X[u-pfv]{u-}}]
			special allomorph only used with \fm{∅} conjugation class verbs
	\item[{\X[ʷ-pfv]{ʷ-}}]
			labialization of a consonant, with first person singular subject \fm{x̱-} \~\ \fm{x̱a-}
	\item[{\X[w-pfv]{w-}}]
			coda consonant following a vowel (i.e.\ \fm{u} of \fm{wu-} deleted)
	\item[\X{w̃-}]	coda consonant following a vowel, retaining the nasalization of former \fm{m-}
	\item[{\X[ÿ-pfv]{ÿ-}}]
			delabialized \fm{y} or \fm{ÿ} preceding or merged with a front vowel
	\item[\X{ÿu-}]	abstract representation reflecting labialized and delabialized forms
	\item[\X{μʷ-}]	lengthening of preceding vowel with labialization spread to other prefixes
	\end{allolist}
	combinations:
	\begin{allolist}
	\item[x̱w]	≡ \fm{ʷ-x̱-} with first person singular subject \fm{x̱-}
	\item[x̱wa]	≡ \fm{ʷ-x̱a-} with first person singular subject \fm{x̱a-}
	\item[ÿ]	≡ \fm{ÿ-i-} with second singular subject \X[i-2sg]{i-}
	\item[ÿee]	≡ \fm{ÿ-i-μ-} with second singular subject \X[i-2sg]{i-}
			and stative \X{μ-}
	\item[ÿeeÿ]	≡ \fm{ÿu-ÿi-} with second plural subject \X{ÿi-}
	\item[ÿeeÿ]	≡ \fm{ÿu-ÿi-μ-} with second plural subject \X{ÿi-}
			and stative \X{μ-}
	\item[ÿeeÿCi]	≡ \fm{ÿu-ÿi-C-i-} with second plural subject \X{ÿi-}
			and valency \X{s-} or \X{l-}/\X{lˢ-} or \X{sh-}
			and stative \X[i-stv]{i-}
	\item[ÿi]	≡ \fm{ÿ-i-} with second singular subject \X[i-2sg]{i-}
	\end{allolist}
	\begin{itemize}
	\item	\vbform{at wuduwax̱áa}{pfv}[tr, \fm{∅}, \fm{-μH} act]{someone/people ate something}
			\vbmorph{at=&wu-&du-&wa-&\rt[²]{x̱a}&-μμH}
				{\xx{ind.n.o}&\xx{pfv}&\xx{ind.h.s}&\xx{stv}&\rt[²]{eat}&\·\xx{var}}
		\versus \vbform{at dux̱á}{impfv}{someone/people is/are eating something}
			\vbmorph{at=&du-&\rt[²]{x̱a}&-μH}
				{\xx{ind.n.o}&\xx{ind.h.s}&\rt[²]{eat}&\·\xx{var}}
	\end{itemize}

\item[wush=]
	variant form of reciprocal object \fm{woosh=}

\item[wooch=]
	variant form of reciprocal object \fm{woosh=}

\item[woosh=]
	reciprocal object
\end{morphdesc}

\subsection{X̱}\label{sec:alphalist-xh}
\begin{morphdesc}[resume*=alphalist]
\item[x̱-, x̱a-]
	first person singular subject
	\begin{itemize}
	\item	\fm{laaḵʼásk kax̱satʼaak} (impfv; tr, \fm{∅}, \fm{-μμL} act) ‘I am pressing black seaweed’\newline
		versus \fm{laaḵʼásk x̱ax̱á} (impfv; tr, \fm{∅}, \fm{-μH} act) ‘I am eating black seaweed’
	\end{itemize}

\item[x̱-]
	allomorph of conjugation \fm{g̱-} when in a syllable coda
	\begin{itemize}
	\item	\fm{kax̱lax̱óotʼ} (imp; tr, \fm{g̱}, \fm{-μμH} act) ‘(you sg.)\ chop/adze it!’ in syllable \fm{kax̱}\newline
		(not \fm[*]{kag̱alax̱óotʼ} or \fm[*]{kaḵlax̱óotʼ})\newline
		versus \fm{kag̱aylax̱óotʼ} ‘you pl.\ chop/adze it!’ in syllable \fm{g̱ay}
	\end{itemize}

\item[x̱-]
	allomorph of modality \fm{g̱-} when in a syllable coda
	\begin{itemize}
	\item	\fm{at gax̱toox̱áa} (prosp; tr, \fm{∅}, \fm{-μH} act) ‘we will eat something’ with \fm{x̱-}\newline
		(not \fm[*]{at gag̱atoox̱áa} or \fm[*]{at gaḵtoox̱áa})\newline
		versus \fm{at gug̱ax̱áa} (prosp) ‘s/he/it will eat something’ with \fm{g̱a-}
	\end{itemize}

\item[-x̱]
	repetitive suffix;
	\newline
	allomorphs:
	\begin{allolist}
	\item[-x̱w]	with labialization
	\end{allolist}
	\begin{enumerate}
	\item	repetitive suffix predicted for \fm{∅} conjugation class verbs
	\item	repetitive suffix in derived aspect paradigms
	\item	repetitive suffix lexically required for some roots
	\item	possibly part of amissive \fm{-x̱aa} ‘miss target’
	\end{enumerate}

\item[-x̱aa \~\ -x̱áa]
	amissive suffix ‘miss target’;
	part of derivation \motderiv{ÿa-s/lˢ-…-x̱áa}{∅}{miss target}
		with qualifier \fm{ÿa-}
		and extensional valency \fm{s-} (or \fm{lˢ-})
		and \fm{∅} conjugation class;
	apparently derived from repetitive \fm{-x̱} and unknown \fm{-áa};
	occurs with the roots
		\fm{\rt{.uᴴn}} ‘shoot (gun)’,
		\fm{\rt{tʼuᴴk}} ‘shoot (arrow)’,
		\fm{\rt{tʼach}} ‘slap’,
		\fm{\rt{dzu}} ‘throw at’,
		\fm{\rt{shaᴴt}} ‘grab’,
		\fm{\rt{gwal}} ‘punch, strike’,
		\fm{\rt{ḵʼish}} ‘slap with stick’,
		\fm{\rt{x̱ich}} ‘club, spank’;
	not related to stems \fm{–x̱aa} or \fm{–x̱áa} formed from \fm{\rt{x̱a}} roots
	nor to the discourse particle \fm{x̱áa} ‘you see; indeed’
	\newline
	allomorphs:
	\begin{allolist}
	\item[-x̱aa]	L tone form occurs after an H tone stem
	\item[-x̱áa]	H tone form occurs after an L tone stem
	\end{allolist}
	\begin{itemize}
	\item	\vbform{ayawsi.únx̱aa}{pfv}[tr, \fm{∅}, ach]{she/he/it shot at him/her/it and missed}
			\vbmorph{a-&\gm{ÿa-}&w-&\gm{s-}&i-&\rt[²]{.uᴴn}&-μH&\gm{-x̱aa}}
				{\xx{3>3}&\xx{qual}&\xx{pfv}&\xx{xtn}&\xx{stv}&\rt[²]{shoot}&\·\xx{var}&\·\xx{miss}}
		\versus \vbform{aawa.ún}{pfv}{she/he/it shot him/her/it}
			\vbmorph{a-&μʷ-&wa-&\rt[²]{.uᴴn}&-μH}
				{\xx{3>3}&\xx{pfv}&\xx{stv}&\rt[²]{shoot}&\·\xx{var}}
	\item	\vbform{ayawlishátx̱aa}{pfv}[tr, \fm{∅}, ach]{she/he/it grabbed at it and missed}
			\vbmorph{a-&\gm{ÿa-}&w-&\gm{lˢ-}&i-&\rt[²]{shaᴴt}&-μH&\gm{-x̱aa}}
				{\xx{3>3}&\xx{qual}&\xx{pfv}&\xx{xtn}&\xx{stv}&\rt[²]{grab}&\·\xx{var}&\·\xx{miss}}
		\versus \vbform{aawasháat}{pfv}[tr, \fm{g}, ach]{she/he/it grabbed him/her/it}
			\vbmorph{a-&μʷ-&wa-&\rt[²]{shaᴴt}&-μμH}
				{\xx{3>3}&\xx{pfv}&\xx{stv}&\rt[²]{grab}&\·\xx{var}}
	\item	\vbform{ayawlidzéix̱aa}{pfv}[tr, \fm{∅}, ach]{she/he/it threw at it and missed}
			\vbmorph{a-&\gm{ÿa-}&w-&\gm{lˢ-}&i-&\rt[²]{dzu}&-μₑμH&\gm{-x̱aa}}
				{\xx{3>3}&\xx{qual}&\xx{pfv}&\xx{xtn}&\xx{stv}&\rt[²]{throw}&\·\xx{var}&\·\xx{miss}}
		\versus \vbform{aawadzóo}{pfv}[tr, \fm{∅}, ach]{she/he/it threw at him/her/it}
			\vbmorph{a-&μʷ-&wa-&\rt[²]{dzu}&-μμH}
				{\xx{3>3}&\xx{pfv}&\xx{stv}&\rt[²]{throw}&\·\xx{var}}
	\end{itemize}

\item[x̱at=]
	first person singular object;
	similar to independent pronoun \fm{x̱át} ‘me’ but with L tone instead of H tone
	\newline
	allomorphs:
	\begin{allolist}
	\item[x̱at=]	typical form
	\item[ax̱=]	possessor of incorporated noun
	\end{allolist}
	\begin{itemize}
	\item	\vbform{x̱at yisiteen}{pfv}[tr, \fm{g̱}, ach]{you saw me}
			\vbmorph{\gm{x̱at=}&ÿ-&i-&s-&i-&\rt[²]{tin}&-μμL}
				{\xx{1sg.o}&\xx{pfv}&\xx{2sg.s}&\xx{xtn}&\xx{stv}&\rt[²]{see}&\·\xx{var}}
		\versus \vbform{ix̱wsiteen}{pfv}{I saw you}
			\vbmorph{i-&ʷ-&x̱-&s-&i-&\rt[²]{tin}&-μμL}
				{\xx{1sg.o}&\xx{pfv}&\xx{1sg.s}&\xx{xtn}&\xx{stv}&\rt[²]{see}&\·\xx{var}}
	\end{itemize}

\item[-x̱w]
	allomorph with labialization of repetitive \fm{-x̱}

\item[x̱w]
	≡ \fm{ʷ-x̱-} combination of
		perfective \fm{ʷ-}
		and first person singular subject \fm{x̱-}

\item[x̱wa]
	≡ \fm{ʷ-x̱a-} combination of
		perfective \fm{ʷ-}
		and first person singular subject \fm{x̱a-}

\end{morphdesc}

\subsection{Y}\label{sec:alphalist-y}
\begin{morphdesc}[resume*=alphalist]
\item[ÿ-]\label{m:ÿ-2sg}
	allomorph of second person singular subject \X[i-2sg]{i-}

\item[ÿ-]\label{m:ÿ-2pl}
	allomorph of second person plural subject \X{ÿi-}
	
\item[ÿ-]\label{m:ÿ-pfv}
	allomorph of perfective \X{wu-}

\item[ÿ-]\label{m:ÿ-face}
	allomorph of incorporated noun \X[ÿa-face]{ÿa-} ‘face’

\item[ÿ-]\label{m:ÿ-qual}
	allomorph of qualifier \X[ÿa-qual]{ÿa-} of unknown meaning

\item[ÿa-]\label{m:ÿa-stv}
	allomorph of stative \X[i-stv]{i-}
	\begin{itemize}
	\item	\vbform{yadál}{impfv}[obj intr, \fm{g}, \fm{-μH} state]{she/he/it is heavy}
			\vbmorph{\gm{ÿa-}&\rt[¹]{dal}&-μH}
				{\xx{stv}&\rt[¹]{heavy}&\·\xx{var}}
		\versus \vbform{si.áatʼ}{impfv}[obj intr, \fm{g}, \fm{-μμH} state]{she/he/it is cold}
			\vbmorph{s-&i-&\rt[⁰]{.atʼ}&-μμH}
				{\xx{intr}&\xx{stv}&\rt[⁰]{cold}&\·\xx{var}}
	\end{itemize}

\item[ÿa-]\label{m:ÿa-face}
	incorporated noun indicating vertical surface or face,
	derived from the relational noun \fm{ÿá} ‘face’;
	can occur together with qualifier \X[ÿa-qual]{ÿa-} of unknown meaning

\item[ÿa-]\label{m:ÿa-qual}
	qualifier of unknown meaning;
	can occur together with incorporated noun \X[ÿa-face]{ÿa-} ‘face’

\item[ÿaa=]
	directional preverb indicating progression or movement along a space
	(compare \fm{\rt[²]{ÿa}} ‘move’,
		directional noun \fm{diÿáa} ‘across, other side’,
		\fm{niÿaa} ‘direction’);
	\begin{enumerate}
	\item	progression, used in progressive aspect for \fm{∅} and \fm{n} conjugation class verbs
		\begin{itemize}
		\item	\fm{yaa x̱at nalnítl} (prog; obj intr, \fm{∅}, ach) ‘I am getting fat’\newline
			versus \fm{x̱at wudlinítl} (pfv) ‘I got fat’
		\end{itemize}
	\item	movement along a space
		\begin{enumerate}
		\item	motion derivation
				\fm{ÿaa} (\fm{g̱}, \fm{yei=…-ch} rep) ‘down along’
				(\fm{yei} in repetitive blocks \fm{ÿaa})
		\item	motion derivation
				\fm{ÿaa} \~\ \fm{ÿa-u-} (\fm{∅}, \fm{-ch} rep) ‘obliquely, circuitously’
		\end{enumerate}
	\end{enumerate}

\item[ÿaa=]
	preverb indicating mental phenomenon, limited to a couple of verbs;
	uncertain if it can occur together with \fm{ÿaa} ‘along’;
	possibly related to Proto-Dene \fm[*]{yən-} \~\ \fm[*]{yiːn-} ‘mind’ and Eyak \fm{ʔiːlih} ‘mind’
	\begin{itemize}
	\item	\fm{yaa ḵux̱dzigéi} (impfv; subj intr, \fm{g}, \fm{-μμH} state) ‘I am smart, wise’
	\item	\fm{yaa aḵoowlig̱át} (pfv; tr, \fm{∅}, ach) ‘s/he/it forgot him/her/it’
	\end{itemize}

\item[ÿaan=]
	incorporated noun indicating hunger,
	saturates object argument;
	derived from noun \fm{ÿaan} ‘hunger’ (now rare)
	\begin{itemize}
	\item	\fm{ax̱ éet yaan uwaháa} (pfv; obj intr, \fm{∅}, mot) ‘hunger appeared to me’ (i.e.\ ‘I got hungry’)\newline
		(not \fm[*]{yaan ax̱ éet uwaháa})
	\end{itemize}

\item[ÿan=]
	directional preverb indicating motion to shore, motion to ground, or termination;
	allomorphs are \fm{ÿax̱} and \fm{ÿánde}:
		\fm{ÿax̱} is used with repetitive,
		\fm{ÿánde} with progressive and prospective,
		and \fm{ÿan} elsewhere (e.g.\ pfv, imp);
	morphologically a specialization of the
		\fm{NP-\{t,x̱,dé\}} (\fm{∅}, \fm{-μμL} rep) ‘arriving at NP’
		motion derivation,
	so the \fm{ÿan} probably used to end with \fm{-t} ‘to a point’ punctual postposition
		as \fm[*]{ÿant};
	derived from noun \fm{ÿán} ‘shore’
		(< Pre-Tlingit \fm[*]{ŋanʰ} < Proto-Na-Dene \fm[*]{ŋənˀ} ‘ground, earth’)
	\begin{enumerate}
	\item	motion on water to shore,
		can be translated ‘ashore’;
		motion derivation
			\fm{ÿan} / \fm{yax̱} / \fm{ÿánde} (\fm{∅}, \fm{-μμL} rep) ‘ashore’
	\item	motion to ground or other horizontal surface,
		can be translated ‘down’ or ‘on ground’;
		motion derivation
			\fm{ÿan} / \fm{yax̱} / \fm{ÿánde} (\fm{∅}, \fm{-μμL} rep) ‘on ground’
			optionally with incorporates (\fm{kʼi-} ‘base’ for ‘setting up, erecting’,
			\fm{sha-} ‘head’ for ‘leaning against’)
	\item	termination of eventuality,
		can be translated ‘ending, terminating, finishing’;
		eventuality/motion derivation
			\fm{ÿan} \~\ \fm{yax̱} \~\ \fm{ÿánde} (\fm{∅}, \fm{-μμL} rep) ‘ending, finishing’
			optionally with \fm{NP-xʼ} ‘coming to rest at NP’;
		derives from metaphor of ‘shore’ as ‘end of journey’ and thus ‘end of event’
	\end{enumerate}

\item[ÿánde=]
	allomorph of directional preverb \fm{ÿan} ‘ashore’ or ‘ending’
	with allative postposition \fm{-dé} \~\ \fm{-de} ‘toward’
	\begin{itemize}
	\item	\fm{yánde gax̱tooḵóox̱} (prosp; subj intr, \fm{∅}, mot) ‘we are going to boat ashore’\newline
		versus \fm{yan wutuwaḵúx̱} (pfv) ‘we boated ashore’
	\end{itemize}

\item[ÿata=]
	incorporated noun ‘sleep’,
	saturates object argument;
	apparently derived from \fm{ÿá} ‘face’ and \fm{\rt[¹]{taᴸ}} ‘sg.\ sleep’
	\begin{itemize}
	\item	\fm{ax̱ éet yataawaháa} (pfv; obj intr, \fm{∅}, mot) ‘sleep appeared to me’, i.e. ‘I got sleepy’
	\item	\fm{ax̱ yaadáx̱ yataawahaa} (pfv; obj intr, \fm{g̱}, mot) ‘sleep disappeared from my face’,
		i.e.\ ‘I became wakeful’
	\end{itemize}

\item[ÿax̱=]
	allomorph of directional preverb \fm{ÿan} ‘ashore’ or ‘ending’
	with perlative postposition \fm{-x̱} ‘contacting’;
	used only with repetitive versus \fm{ÿánde} (prog, prosp) or \fm{ÿan} (pfv, imp, etc.)
	\begin{itemize}
	\item	\fm{yax̱ tooḵoox̱} (rep impfv; subj intr, \fm{∅}, mot) ‘we repeatedly boat ashore’\newline
		versus \fm{yan wutuwaḵúx̱} (pfv) ‘we boated ashore’
	\end{itemize}

\item[ÿee-]
	allomorph of second person singular subject \fm{ÿi-}

\item[ÿee=]
	allomorph of second person singular object \fm{ÿi-}

\item[ÿee=]
	incorporated noun indicating time

\item[ÿee]
	≡ \fm{wu-i-μ}
	combination of perfective \fm{wu-}
		and second person singular subject \fm{i-}
		and stative \fm{μ-}

\item[ÿeeÿ]
	second person plural subject \fm{ÿi-} combined with either one or both of
		perfective \fm{wu-}
		and stative \fm{ÿa-} \~\ \fm{i-}
	\begin{enumerate}
	\item	\fm{ÿeeÿ} ≡ \fm{ÿi-ÿa-}
		with stative \fm{ÿa-}
	\item	\fm{ÿeeÿ} ≡ \fm{wu-ÿi-}
		with perfective \fm{wu-}
	\item	\fm{ÿeeÿ} ≡ \fm{wu-ÿi-ÿa-}
		with perfective \fm{wu-}
		and stative \fm{ÿa-}
	\item	\fm{ÿeeÿsi} ≡ \fm{ÿi-s-i-}
		with valency \fm{s-}
			(or \fm{ÿeeÿli} \fm{l-} or \fm{ÿeeÿshi} \fm{sh-})
		and stative \fm{i-}
	\item	\fm{ÿeeÿsi} ≡ \fm{wu-ÿi-s-i-}
		with perfective \fm{wu-}
		and valency \fm{s-}
			(or \fm{ÿeeÿli} \fm{l-} or \fm{ÿeeÿshi} \fm{sh-})
		and stative \fm{i-}
	\end{enumerate}

\item[ÿeeḵ=]
	directional preverb ‘beach’, variant forms \fm{ÿeiḵ=} and \fm{eèḵ=};
	derived from noun \fm{éeḵ} \~\ \fm{éiḵ} ‘beach’;
	compare \fm{éeg̱i=} \~\ \fm{éig̱i=}

\item[yei=]
	direction preverb ‘down’;
	may reflect \fm{g̱} conjugation class or a \fm{∅} conjugation class motion derivation;
	part of directional element paradigm of \fm{\rt{ÿiⁿ}} ‘down’:
		\fm{(di)ÿée} ‘below’, \fm{(di)ÿín-de} ‘to below’, \fm{(di)ÿee-naa} ‘downward’;
	related to \fm{ÿee} ‘beneath, below’
	\begin{enumerate}
	\item	reflects \fm{g̱} conjugation class in prospective, progressive, and repetitive imperfective
	\item	\fm{g̱} conjugation class motion derivation
	\item	\fm{∅} conjugation class motion derivation
	\end{enumerate}

\item[yéi=]
	manner preverb ‘thus, so’;
	derived from noun \fm{yéi} \~\ \fm{yé} ‘place, way, manner’

\item[ÿeiḵ=]
	variant form of directional preverb \fm{ÿeeḵ=} ‘beach’ used in some Northern varieties
	arises from uvular lowering of \fm{ée} to \fm{éi};
	derived from noun \fm{éeḵ} \~\ \fm{éiḵ} ‘beach’

\item[ÿi-]\label{m:ÿi-}
	second person plural subject or object; long vowel allomorphs are \fm{ÿee-} and \fm{ÿee=}
	\begin{enumerate}
	\item	second person plural subject
	\item	second person plural object
	\end{enumerate}

\item[ÿi]
	≡ \fm{wu-i-}
	combination of perfective \fm{wu-} and
		second person singular subject \fm{i-}

\item[ÿu-]\label{m:ÿu-}
	abstract representation of perfective \fm{wu-};
	this form does not actually occur in speech, instead see
		\fm{wu-}, \fm{w-}, \fm{m-}, \fm{μʷ-} \fm{ÿi}, \fm{ÿee}, \fm{ÿeeÿ}

\item[yoo=]
	alternating eventuality preverb

\item[yóo=]
	quotative preverb

\end{morphdesc}

\subsection{Symbols}\label{sec:alphalist-sym}
\begin{morphdesc}[resume*=alphalist]
\item[μ-]\label{m:μ-}
	allomorph of stative \fm{ÿa-} \~\ \fm{i-}
	\begin{itemize}
	\item	\fm{x̱at yatéen} (impfv; tr, \fm{∅}, \fm{-μμH} state) ‘s/he/it can see me’
			with \fm{ÿa-}\newline
		versus \fm{x̱aatéen} (impfv) ‘I can see him/her/it’
			with \fm{μ-}\newline
		(not normally \fm[*]{x̱ayatéen}, although some speakers also permit this form)
	\end{itemize}

\item[μʷ-]\label{m:μʷ-}
	allomorph of perfective \fm{wu-} when preceded by CV and followed by stative \fm{ÿa-}
	\begin{itemize}
	\item	\fm{aawajáḵ} (pfv; \fm{∅}, ach) ‘s/he/it killed him/her/it’
			with \fm{a-μʷ-wa-\rt[¹]{jaḵ}-μH}\newline
		versus \fm{wutuwajáḵ} (pfv) ‘we killed him/her/it’
			with \fm{wu-tu-wa-\rt[¹]{jaḵ}-μH}
	\end{itemize}

\item[-μL]
	stem variation: short vowel with low tone [\ipa{V̀}]
	\begin{itemize}
	\item	\fm{neil uwagudi ḵáa} (pfv rel; subj intr, \fm{∅}, mot) ‘man who went home’
		with \fm{\rt[¹]{gut}} ‘sg.\ go’ and \fm{-μL}
		in \fm{∅} conjugation class perfective aspect relative clause
	\end{itemize}

\item[-μH]
	stem variation: short vowel with high tone [\ipa{V́}]
	\begin{itemize}
	\item	\fm{neil uwagút} (pfv; subj intr, \fm{∅}, mot) ‘s/he/it went home’
		with \fm{\rt[¹]{gut}} ‘sg.\ go’ and \fm{-μH}
		in \fm{∅} conjugation class perfective aspect main clause
	\end{itemize}

\item[-μμL]
	stem variation: long vowel with low tone [\ipa{V̀ː}]
	\begin{itemize}
	\item	\fm{neildé woogoot} (pfv; subj intr, \fm{n}, mot) ‘s/he/it went homeward’
		with \fm{\rt[¹]{gut}} ‘sg.\ go’ and \fm{-μμL}
		in \fm{n} conjugation class perfective aspect main clause
	\end{itemize}

\item[-μμH]
	stem variation: long vowel with high tone [\ipa{V́ː}]
	\begin{itemize}
	\item	\fm{neildé gug̱agóot} (prosp; subj intr, \fm{∅}/\fm{n}, mot) ‘s/he/it will go home’
		with \fm{\rt[¹]{gut}} ‘sg.\ go’ and \fm{-μμH}
		in prospective aspect main clause
	\end{itemize}

\item[-μₑμL]
	stem variation: ablaut (/\ipa{a, u}/ > [\ipa{e}]) long vowel with low tone [\ipa{èː}];
	normally occurs only with \fm{\rt{CVᴸ}} (Tongass \fm{\rt{CVʰ}}) roots
	\begin{itemize}
	\item	\fm{x̱ateix̱} (rep impfv; subj intr, \fm{n}, \fm{-μH} act) ‘I repeatedly sleep’
			with \fm{\rt[¹]{taᴸ}} ‘sg.\ sleep’ and \fm{-μₑμL-x̱}\newline
		versus
		\fm{x̱atá} (impfv) ‘I am sleeping’
			with \fm{-μH}\newline
		but \fm{yaa nx̱atéin} (prog) ‘I am falling asleep’
			with \fm{-μₑμH-n}
	\end{itemize}

\item[-μₑμH]
	stem variation: ablaut (/\ipa{a, u}/ > [\ipa{e}]) long vowel with high tone [\ipa{éː}];
	normally occurs only with \fm{\rt{CV}} roots
	\begin{itemize}
	\item	\fm{x̱ax̱éix̱} (rep impfv; tr, \fm{∅}, \fm{-μH} act) ‘I repeatedly eat it’
			with \fm{\rt[²]{x̱a}} ‘eat’ and \fm*{-μₑμH-x̱}\newline
		versus
		\fm{x̱ax̱á} (impfv) ‘I am eating it’
			with \fm{-μH}
	\end{itemize}

\item[-⊗]
	stem variation: irregular deletion of final consonant and short vowel with high tone;
	only occurs in imperatives with \fm{\rt[¹]{gut}} ‘sg go’,
			\fm{\rt[¹]{.at}} ‘pl go’,
			\fm{\rt[¹]{nuk}} ‘sg sit’
	\begin{itemize}
	\item	\vbform{neildé nagú!}{imp}[subj intr, \fm{n}, mot]{(you sg.)\ go home!}
			\vbmorph{neil&-dé&na-&\rt[ˢ]{gu\gm{t}}&\gm{-⊗}}
				{home&\·\xx{all}&\xx{ncnj}&\rt[ˢ]{go·\xx{sg}}&\·\xx{var}}
		\versus \vbform{neildé yeegoot}{pfv}{you sg.\ went home}
			\vbmorph{neil&-dé&ÿ-&i-&μ-&\rt[ˢ]{gut}&-μμL}
				{home&\·\xx{all}&\xx{pfv}&\xx{2sg.s}&\xx{stv}&\rt[ˢ]{go·\xx{sg}}&\·\xx{var}}
	\item	\vbform{neildé nay.á!}{imp}[subj intr, \fm{n}, mot]{you guys go home!}
			\vbmorph{neil&-dé&na-&ÿ-&\rt[ˢ]{.a\gm{t}}&\gm{-⊗}}
				{home&\·\xx{all}&\xx{ncnj}&\xx{2pl.s}&\rt[ˢ]{go·\xx{pl}}&\·\xx{var}}
		\versus \vbform{neildé yeey.aat}{pfv}{you guys went home}
			\vbmorph{neil&-dé&ÿ-&ÿ-&μ-&\rt[ˢ]{.at}&-μμL}
				{home&\·\xx{all}&\xx{pfv}&\xx{2pl.s}&\xx{stv}&\rt[ˢ]{go·\xx{pl}}&\·\xx{var}}
	\item	\vbform{g̱anú!}{imp}[subj intr, \fm{g̱}, mot]{(you sg.)\ sit down!}
			\vbmorph{g̱a-&\rt[ˢ]{nu\gm{k}}&\gm{-⊗}}
				{\xx{g̱cnj}&\rt[ˢ]{sit·\xx{sg}}&\·\xx{var}}
		\versus \vbform{yeenook}{pfv}{you sg.\ sat down}
			\vbmorph{ÿ-&i-&μ-&\rt[ˢ]{nuk}&-μμL}
				{\xx{pfv}&\xx{2sg.s}&\xx{stv}&\rt[ˢ]{sit·\xx{sg}}&\·\xx{var}}
	\end{itemize}
\end{morphdesc}
