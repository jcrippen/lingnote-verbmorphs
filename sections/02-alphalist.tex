%!TEX root = ../lingnote-verbmorphs.tex

\section{Alphabetic listing of verb morphemes}\label{sec:alphalist}

\subsection{A}\label{sec:alphalist-a}
\raggedright
\begin{morphdesc}[series=alphalist]
\item[a-]\label{m:a-}
	argument marking prefix in the same position as object prefixes/proclitics;
	\newline
	allomorphs:
	\begin{allolist}
	\item[\X{ⱥ-}]	symbol used in glosses indicating absence of expected \fm{a-}
			due to preceding ergative \fm{-ch}
	\end{allolist}
	combinations:
	\begin{allolist}
	\item[\X{aawa}]	≡ \fm{a-μʷ-wa-} with perfective \X{μʷ-} and stative \X{wa-}
	\item[\X{am}]	≡ \fm{a-m-} with perfective \X{m-}
	\item[\X{aw}]	≡ \fm{a-w-} with perfective \X[w-pfv]{w-}
	\item[\X{awu}]	≡ \fm{a-wu-} with perfective \X{wu-}
	\item[\X{ax̱}]	≡ \fm{a-x̱-} with first person singular subject \X[x̱-1sg]{x̱-}
				or \fm{g̱} conjugation \X[x̱-g̱cnj]{x̱-}
				or modal \X[x̱-mod]{x̱-}
	\item[{\X[aÿ-a-ÿ]{aÿ}}]
			≡ \fm{a-ÿ-} with second person plural subject \X[ÿ-2pl]{ÿ-}
	\item[{\X[aÿ-a-ʷ-ÿ]{aÿ}}]
			≡ \fm{a-ʷ-ÿ-} with perfective \X[ʷ-pfv]{ʷ-}
				and second person singular subject \X[ÿ-2sg]{ÿ-}
	\item[\X{ee}]	≡ \fm{a-i-} with second person singular subject \X[i-2sg]{i-}
	\item[{\X[eeÿa-a-i-ÿa]{eeÿa}}]
			≡ \fm{a-i-ÿa-} with second person singular subject \X[i-2sg]{i-}
				and stative \X[ÿa-stv]{ÿa-}
	\item[{\X[eeÿa-a-ʷ-i-ÿa]{eeÿa}}]
			≡ \fm{a-ʷ-i-ÿa-} with perfective \X[ʷ-pfv]{ʷ-}
				and second person singular subject \X[i-2sg]{i-}
				and stative \X[ÿa-stv]{ÿa-}
	\item[\X{oo}]	≡ \fm{a-u-} with irrealis \X[u-irr]{u-}
				or \fm{∅} conj perfective \X[u-pfv]{u-}
	\item[\X{oowa}]	≡ \fm{a-u-} with irrealis \X[u-irr]{u-}
				and stative \X{wa-}
	\item[\X{oox̱}]	≡ \fm{a-u-x̱-} with irrealis \X[u-irr]{u-}
				and first person singular subject \X[x̱-1sg]{x̱-}
				or \fm{g̱} conjugation \X[x̱-g̱cnj]{x̱-}
				or modal \X[x̱-mod]{x̱-}
	\end{allolist}
	\begin{enumerate}
	\item\label{m:a-3>3}
		3>3 agreement of transitive verb: indicates existence of third person subject
		and third person object (regardless of whether these are actually spoken)
		\begin{itemize}
		\item	\vbform{aawax̱áa}{pfv}[tr, \fm{∅}, \fm{-μH} act]{she/he/it ate him/her/it}
			\vbmorph{\gm{a-}&μʷ-&wa-&\rt[²]{x̱a}&-μμH}
				{\xx{3>3}&\xx{pfv}&\xx{stv}&\rt[²]{eat}&\·\xx{var}}
			\versus \vbform{wutuwax̱áa}{pfv}{we ate it}
			\vbmorph{wu-&tu-&wa-&\rt[²]{x̱a}&-μμH}
				{\xx{pfv}&\xx{1pl.s}&\xx{stv}&\rt[²]{eat}&\·\xx{var}}
		\end{itemize}
	\item\label{m:a-ind.h.o}
		indefinite nonhuman object of some transitive verbs
			instead of \X{at=};
		verbs that use \fm{a-} in this way usually also use \fm{a-} 3>3
		and the reason for using \fm{a-} instead of \fm{at=} is still unclear
		\begin{itemize}
		\item	\vbform{alʼóon}{impfv}[tr, \fm{n}, \fm{-μμH} act]{s/he/it is hunting something}
			\vbmorph{\gm{a-}&\rt[²]{lʼuᴴn}&-μμH}
				{\xx{ind.n.o}&\rt[²]{hunt}&\·\xx{var}}
			\andnot{	\vbform[*]{at lʼóon}{impfv}{s/he/it is hunting something}
				using \fm{at=}}
			\versus \vbform{alʼóon}{impfv}{s/he/it is hunting it}
			\vbmorph{a-&\rt[²]{lʼuᴴn}&-μμH}
				{\xx{3>3}&\rt[²]{hunt}&\·\xx{var}}
			\notealso{compare \vbform{g̱áx̱ alʼóon}{impfv}{s/he/it is hunting rabbits}}
			\notealso{(\fm{g̱áx̱ alʼóon} cannot mean
				‘s/he/it is hunting something rabbits’)}
		\end{itemize}
	\item\label{m:a-ind.h.s}
		indefinite human subject (not object!)\ of subject intransitive verbs
		instead of \X{du-};
		possibly all subject intransitive verbs use \fm{a-} rather than \fm{du-}?
		\begin{itemize}
		\item	\vbform{ax̱éxʼw}{impfv}[subj intr, \fm{n}, \fm{-μH} act]{people are sleeping}
				\vbmorph{\gm{a-}&\rt[¹]{x̱exʼw}&-μH}
					{\xx{ind.h.s}&\rt[¹]{sleep·\xx{pl}}&\·\xx{var}}
				\andnot{\fm[*]{dux̱éxʼw} ‘people are sleeping’
					using \fm{du-} \xx{ind.n.s} ‘people’}
			\versus \vbform{toox̱éxʼw}{impfv}{we are sleeping}
				\vbmorph{too- \rt[¹]{x̱exʼw} -μH}
					{\xx{1pl·s} \rt[¹]{sleep·\xx{pl}} \·\xx{var}}
		\item	\vbform{aawa.aat}{pfv}[subj intr, \fm{n}, mot]{people went}
			\vbmorph{\gm{a-}&μʷ-&wa-&\rt[¹]{.at}&-μμL}
				{\xx{ind.h.s}&\xx{pfv}&\xx{stv}&\rt[¹]{go·\xx{pl}}&\·\xx{var}}
			\andnot{\fm[*]{wuduwa.aat} ‘people went’ 
				using \fm{du-} \xx{ind.h.s} ‘someone, people’}
			\versus \vbform{wutuwa.aat}{pfv}{we went}
				\vbmorph{wu- &tu-&wa- &\rt[¹]{.at}&-μμL}
					{\xx{pfv}&\xx{1pl.s}&\xx{stv}&\rt[¹]{go·\xx{pl}}&\·\xx{var}}
		\end{itemize}
	\item\label{m:a-xpl}
		nonreferential expletive (filler) object, does not refer to anything;
		it is unclear why this is used instead of suppressing the object with
		antipassive voice \X{d-}
		\begin{itemize}
		\item	\vbform{awdigaan}{pfv}[impers, \fm{g̱}, ach]{it sunshined}
			\vbmorph{\gm{a-}&w-&d-&i-&\rt[¹]{gan}&-μμL}
				{\xx{xpl}&\xx{pfv}&\xx{mid}&\xx{stv}&\rt[²]{burn}&\·\xx{var}}
				\andnot{\fm[*]{g̱agaan awdigaan} ‘sun sunshined’: no object allowed}
			\versus \vbform{sʼeenáa kawdigán}{pfv}[obj intr, \fm{∅}, ach]{the lamp shone, lit up}
				\vbmorph{sʼeenáa&ka-&w-&d-&i-&\rt[¹]{gan}&-μH}
					{lamp&\xx{hsfc}&\xx{pfv}&\xx{mid}&\xx{stv}&\rt[¹]{burn}&\·\xx{var}}
				\notealso{(object \fm{sʼeenáa} ‘lamp’ allowed)}
		\end{itemize}
	\end{enumerate}

\item[ⱥ-]\label{m:ⱥ-}
	allomorph of 3>3 agreement \X{a-} specifically when preceded by ergative \fm{-ch};
	this is not actually a real morpheme but is used in glosses to indicate that
		an otherwise expected \fm{a-} is absent
		due to the presence of ergative \fm{-ch} immediately preceding the verb;
	certain elements (some preverbs, focus particles) are transparent to \fm{-ch … ⱥ-}
		but other things (NPs, PPs, adverbs) block the effect
	\begin{itemize}
	\item	\vbform{ax̱ x̱úx̱ch uwa.ún}{pfv}[tr, \fm{∅}, ach]{my husband shot him/her/it}
		\parencite[183.303]{nyman-leer:1993}
			\vbmorph{ax̱&x̱úx̱&-ch&\gm{ⱥ-}&u-&wa-&\rt[²]{.uᴴn}&-μH}
				{\xx{1sg.psr}&husband&\·\xx{erg}&\xx{3>3}&\xx{zpfv}&\xx{stv}&\rt[²]{shoot}&\·\xx{var}}
		\versus \vbform{ax̱ x̱úx̱ aawa.ún}{pfv}{she/he/it shot my husband}
			\vbmorph{ax̱&x̱úx̱&a-&μʷ-&wa-&\rt[²]{.uᴴn}&-μH}
				{\xx{1sg.psr}&husband&\xx{3>3}&\xx{pfv}&\xx{stv}&\rt[²]{shoot}&\·\xx{var}}
	\item	\vbform{Lingítch áwé yéi uwasáa}{pfv}[tr, \fm{∅}, ach]{it was Tlingits that named it so}
		\parencite[144.128]{dauenhauer-dauenhauer:1987}
			\vbmorph{Lingít&-ch&áwé&yéi=&\gm{ⱥ-}&u-&wa-&\rt[²]{sa}&-μμH}
				{Tlingit&\·\xx{erg}&\xx{foc}&thus&\xx{3>3}&\xx{zpfv}&\xx{stv}&\rt[²]{name}&\·\xx{var}}
		\versus \vbform{aawasáa}{pfv}{she/he/it named him/her/it so}
			\vbmorph{yéi=&ⱥ-&u-&wa-&\rt[²]{sa}&-μμH}
				{thus&\xx{3>3}&\xx{zpfv}&\xx{stv}&\rt[²]{name}&\·\xx{var}}		\item	\vbform{x̱áat ḵwáanich ásgíyú wusineex̱}{pfv}[tr, \fm{g̱}, ach]{salmon people apparently rescued him}
		\parencite[312.38]{swanton:1909}
			\vbmorph{x̱áat&ḵwáan&-i&-ch&ásgíyú&\gm{ⱥ-}&wu-&s-&i-&\rt[¹]{nix̱}&-μμL}
				{salmon&people&\xx{poss}&\xx{erg}&\xx{foc}&\xx{3>3}&\xx{pfv}&\xx{csv}&\xx{stv}&\rt[¹]{safe}&\·\xx{var}}
		\versus \vbform{awsineex̱}{pfv}{she/he/it rescued him/her/it}
			\vbmorph{a-&w-&s-&i-&\rt[¹]{nix̱}&-μμL}
				{\xx{3>3}&\xx{pfv}&\xx{csv}&\xx{stv}&\rt[¹]{safe}&\·\xx{var}}
	\end{itemize}

\item[aa=]\label{m:aa=}
	partitive proclitic ‘one of, some of’;
	can apply to either singular or plural quantity of the referent depending on context;
	derived from the independent partitive pronoun \fm{aa} ‘one, some’
		but distinguished from it by position because unlike independent pronouns
		the proclitic can occur between preverbs and verb
	\begin{enumerate}
	\item	partitive third person object of transitive verb
		\begin{itemize}
		\item	\vbform{aa wutusi.ée}{pfv}[tr, \fm{∅}, \fm{-μμH} act]{we cooked one/some of them}
				\vbmorph{\gm{aa=}&wu-&tu-&s-&i-&\rt[¹]{.i}&-μμH}
					{\xx{part.o}&\xx{pfv}&\xx{1pl.s}&\xx{csv}&\xx{stv}&\rt[¹]{cooked}&\·\xx{var}}
			\versus \vbform{wutusi.ée}{pfv}{we cooked him/her/it/them}
				\vbmorph{wu-&tu-&s-&i-&\rt[¹]{.i}&-μμH}
					{\xx{pfv}&\xx{1pl.s}&\xx{csv}&\xx{stv}&\rt[¹]{cooked}&\·\xx{var}}
		\end{itemize}
	\item	partitive third person object of object intransitive verb
		\begin{itemize}
		\item	\vbform{yéi aa yatee}{impfv}[obj intr, \fm{n}, \fm{-μμL} state]{one is/some are that way}
				\vbmorph{yéi=&\gm{aa=}&ÿa-&\rt[¹]{tiᴸ}&-μμL}
					{thus&\xx{part.o}&\xx{stv}&\rt[¹]{be}&\·\xx{var}}
			\versus \vbform{yéi yatee}{impfv}{she/he/it is that way}
				\vbmorph{yéi=&ÿa-&\rt[¹]{tiᴸ}&-μμL}
					{thus&\xx{stv}&\rt[¹]{be}&\·\xx{var}}
		\end{itemize}
	\item	partitive third person subject of subject intransitive verb
		\begin{itemize}
		\item	\vbform{aa woo.aat}{pfv}[subj intr, \fm{n}, mot]{some of them went}
				\vbmorph{\gm{aa=}&wu-&μ-&\rt[ˢ]{.at}&-μμL}
					{\xx{part.s}&\xx{pfv}&\xx{stv}&\rt[ˢ]{go·\xx{pl}}&\·\xx{var}}
			\versus \vbform{woo.aat}{pfv}[subj intr, \fm{n}, mot]{they went}
				\vbmorph{wu-&μ-&\rt[ˢ]{.at}&-μμL}
					{\xx{pfv}&\xx{stv}&\rt[ˢ]{go·\xx{pl}}&\·\xx{var}}
		\item	\vbform{aa woonook}{pfv}[subj intr, \fm{g̱}, mot]{one of them sat down}
				\vbmorph{\gm{aa=}&wu-&μ-&\rt[ˢ]{nuk}&-μμL}
					{\xx{part.s}&\xx{pfv}&\xx{stv}&\rt[ˢ]{sit·\xx{sg}}&\·\xx{var}}
			\versus \vbform{woonook}{pfv}{she/he/it sat down}
				\vbmorph{wu-&μ-&\rt[ˢ]{nuk}&-μμL}
					{\xx{pfv}&\xx{stv}&\rt[ˢ]{sit·\xx{sg}}&\·\xx{var}}
		\end{itemize}
	\end{enumerate}

\item[-a]\label{m:-a}
	allomorph of instrument suffix \X{-aa} \~\ \X{-áa};
	occurs between an H tone syllable and \fm{-ÿi}
		so that \fm{-aa} becomes a short vowel
	\begin{itemize}
	\item	\vbform{koolitsʼígwayi át}{rel impfv}[obj intr, \fm{g}?, inv state]{thing which is a delicate, touchy issue}
			\vbmorph{ka-&u-&lˢ-&i-&\rt{tsʼikw}&-μH&\gm{-a}&-ÿi&át}
				{\xx{qual}&\xx{irr}&\xx{xtn}&\xx{stv}&\rt{delicate}&\·\xx{var}&\·\xx{inst}&\·\xx{rel}&thing}
		\versus \vbform{koolitsʼígwaa}{impfv}{she/he/it is a delicate, touchy issue}	
			\vbmorph{ka-&u-&lˢ-&i-&\rt{tsʼikw}&-μH&-aa}
				{\xx{qual}&\xx{irr}&\xx{xtn}&\xx{stv}&\rt{delicate}&\·\xx{var}&\·\xx{inst}}
	\item	predicted to occur with subordinate \X[-ÿi-sub]{-ÿi}
		but no examples
	\item	\fm{ax̱ gúxʼayi} ‘my cup, can, dipper’
			\vbmorph{ax̱&\rt{guxʼ}&-μH&\gm{-a}&-ÿi}
				{\xx{1sg.psr}&\rt{dip}&\·\xx{var}&\·\xx{inst}&\·\xx{poss}}
		\versus \fm{gúxʼaa} ‘cup, can, dipper’
			\vbmorph{\rt{guxʼ}&-μL&-aa}
				{\rt{dip}&\·\xx{var}&\·\xx{inst}}
	\end{itemize}

\item[-á]\label{m:-á}
	allomorph of instrument suffix \X{-aa} \~\ \X{-áa};
	occurs between an L tone syllable
		so that \fm{-aa} becomes a short vowel
	\begin{itemize}
	\item	predicted to occur with relative \X[-ÿi-rel]{-ÿi} and subordinate \X[-ÿi-sub]{-ÿi}
		but no examples
	\item	\fm{ax̱ sʼeenáyi} ‘my lamp’
			\vbmorph{ax̱&\rt{sʼin}&-μμL&\gm{-á}&-ÿi}
				{\xx{1sg.psr}&\rt{lamp}&\·\xx{var}&\·\xx{inst}&\·\xx{poss}}
		\versus \fm{sʼeenáa} ‘lamp’
			\vbmorph{\rt{sʼin}&-μμL&-áa}
				{\rt{lamp}&\·\xx{var}&\·\xx{inst}}
	\end{itemize}

\item[-aa]\label{m:-aa}
	instrument suffix, indicates an instrument used for an event;
	has polar tone opposite the preceding syllable
		so \fm{-aa} after an H tone syllable
		and \X{-áa} after an L tone syllable;
	becomes a short vowel when followed by \fm{-ÿi}
		so \X{-a} + \fm{-ÿi}
			(not \fm[*]{-a-ÿí})
		and \X{-á} + \fm{-ÿi};
	this suffix usually occurs in nouns derived from verbs (nominalizations)
		but it also occurs in some verbs where the the verb is probably
		derived from the noun that includes \fm{-aa} \~\ \fm{-áa},
		supported by the fact that when this suffix occurs in verbs
		the verb stem is always invariable;
	the meaning of \fm{-aa} \~\ \fm{-áa} is similar to English \fm{-er} as in \fm{dipper}
		but it only applies to an instrument and never an agent,
		so \fm{gúxʼaa} is only a tool and never a person;
	glossed \xx{inst} versus the instrumental postposition \fm{-n} \~\ \fm{een} \~\ \fm{teen}
		which is glossed \xx{instr};
	probably also part of \X{-jaa} \~\ \X{-jáa} with repetitive \X{-ch}
		and part of \X{-x̱aa} \~\ \X{-x̱áa} with repetitive \X{-x̱}
		but the composition of meaning in these is unclear
	\newline
	allomorphs:
	\begin{allolist}
	\item[-aa]	L tone form after used after H tone syllable
	\item[\X{-áa}]	H tone form after used after L tone syllable
	\item[\X{-a}]	short vowel form of \fm{-aa} when followed by \fm{-ÿi}
	\item[\X{-á}]	short vowel form of \fm{-áa} when followed by \fm{-ÿi}
	\end{allolist}
	part of:
	\begin{allolist}
	\item[\X{-jaa} \~\ \X{-jáa}]
			with repetitive \X{-ch} as a suffix in some nouns
	\item[\X{-x̱aa} \~\ \X{-x̱áa}]
			with repetitive \X{-x̱} in ‘miss target’ derivation
	\end{allolist}
	\begin{enumerate}
	\item	instrument suffix forming nouns from verb roots,
		denoting an instrument used for the event described by the root
		\begin{itemize}
		\item	\fm{óonaa} ‘rifle, gun’ (literally ‘instrument for shooting’)
				\vbmorph{\rt{.uᴴn}&-μμH&\gm{-aa}}
					{\rt{shoot}&\·\xx{var}&\·\xx{inst}}
			\versus \vbform{aawa.ún}{pfv}[tr, \fm{∅}, ach]{she/he/it shot him/her/it}
				\vbmorph{a-&μʷ-&wa-&\rt[²]{.uᴴn}&-μH}
					{\xx{3>3}&\xx{pfv}&\xx{stv}&\rt[²]{shoot}&\·\xx{var}}
		\item	\fm{gúxʼaa} ‘cup, can, dipper’ (literally ‘instrument for dipping’)
				\vbmorph{\rt{guxʼ}&-μH&\gm{-aa}}
					{\rt{dip}&\·\xx{var}&\·\xx{inst}}
			\versus \vbform{agóoxʼ}{impfv}[tr, \fm{∅}, \fm{-μμH} act]{she/he/it dips up/out him/her/it}
				\vbmorph{a-&\rt[²]{guxʼ}&-μμH}
					{\xx{3>3}&\rt[²]{dip}&\·\xx{var}}
		\end{itemize}
	\item	instrument suffix retained in verbs derived from nouns with this suffix;
		in some cases only the verb based on the noun with \fm{-aa} is attested,
			but in other cases both the verb without \fm{-aa}
			and the verb from the noun with \fm{-aa} are attested,
			though the difference in meaning between these, if any, is unclear
		\begin{itemize}
		\item	\vbform{dusdeegáa}{impfv}[tr, \fm{∅}?, inv act]{people dipnet for it}
			\parencite[91.1143]{story-naish:1973}
				\vbmorph{du-&d-&s-&\rt{dik}&-μμL&\gm{-áa}}
					{\xx{ind.h.s}&\xx{mid}&\xx{tr}&\rt{dipnet}&\·\xx{var}&\xx{inst}}
			\versus \fm{deegáa} ‘dipnet’ (literally ‘instrument for dipnetting’)
				\vbmorph{\rt{dik}&-μμL&\gm{-áa}}
					{\rt{dipnet}&\·\xx{var}&\·\xx{inst}}
			\versus \vbform{awdzidéek}{pfv}[tr, \fm{∅}?, ach?]{she/he/it dipnetted for things}
			\parencite[91.1142]{story-naish:1973}
				\vbmorph{a-&w-&d-&s-&i-&\rt{dik}&-μμH}
					{\xx{ind.n.o}&\xx{pfv}&\xx{mid}&\xx{tr}&\xx{stv}&\rt{dipnet}&\·\xx{var}}
		\end{itemize}
	\item	unknown suffix in verbs with the ‘pretend activity’ derivation
			made up of:
			reflexive \X{ash=} \~\ \X{ach=}
			+ qualifier \X[ka-qual]{ka-}
			+ irrealis \X[u-irr]{u-}
			± extensional \X{s-}/\X{lˢ-} or \X{l-} or pejorative \X{sh-}
			± unknown \fm{-aa}
			\parencites[55]{story:1966}[654]{crippen:2019};
			may be glossed as \xx{inst} or \xx{unkn};
			attested with the roots
			\begin{inlinelist}
			\item	\fm{\rt{chʼitʼ}} \~\ \fm{\rt{chʼetʼ}} ‘ball’
			\item	\fm{\rt{dlen}} ‘tempt’
			\item	\fm{\rt{gulʼ}} ‘one-eye’
			\item	\fm{\rt{g̱iḵ}} \~\ \fm{\rt{g̱eḵ}} ‘swing’
			\item	\fm{\rt{g̱ixʼ}} ‘creak, squeak’
			\item	\fm{\rt{kitsʼ}} ‘rock’
			\item	\fm{\rt{kʼeᴴn}} ‘jump’
			\item	\fm{\rt{ḵux̱}} ‘go by boat’
			\item	\fm{\rt{ḵʼish}} ‘swat, hit with stick’
			\item	\fm{\rt{taḵ}} ‘poke’
			\item	\fm{\rt{tʼach}} ‘slap; swim’
			\item	\fm{\rt{tʼaxʼ}} ‘flick’
			\item	\fm{\rt{x̱ʼilʼ}} ‘slide’
			\end{inlinelist}
			see also \fm{\rt{tsin}} ‘alive, strong’ under \X{-áa} allomorph
		\begin{itemize}
		\item	\vbform{has ash koosḵúx̱aa}{impfv}[subj intr, \fm{∅}?, inv act]{they are playing toy boat}
			\parencite[152.2071]{story-naish:1973}
				\vbmorph{has=&ash=&ka-&u-&d-&s-&\rt[¹]{ḵux̱}&-μμH&\gm{-aa}}
					{\xx{plh}&\xx{rflx.o}&\xx{qual}&\xx{irr}&\xx{mid}&\xx{csv}&\rt[¹]{go.boat}&\·\xx{var}&\·\xx{inst}}
			\versus \vbform{wooḵoox̱}{pfv}[subj intr, \fm{n}, mot]{she/he/it went by boat}
				\vbmorph{wu-&μ-&\rt[¹]{ḵux̱}&-μμL}
					{\xx{pfv}&\xx{stv}&\rt[¹]{go.boat}&\·\xx{var}}
		\item	\vbform{has ash koolḵʼíshaa}{impfv}[subj intr, \fm{∅}?, inv act]{they are playing baseball, softball, etc.}
			\parencite[152.2070]{story-naish:1973}
				\vbmorph{has=&ash=&ka-&u-&d-&lˢ-&\rt[²]{ḵʼish}&-μμH&\gm{-aa}}
					{\xx{plh}&\xx{rflx.o}&\xx{qual}&\xx{irr}&\xx{mid}&\xx{csv}&\rt[²]{swat}&\·\xx{var}&\·\xx{inst}}
			\versus \vbform{aawaḵʼísh}{pfv}[tr, \fm{∅}, ach]{she/he/it swatted, batted him/her/it}
				\vbmorph{a-&μʷ-&wa-&\rt[²]{ḵʼish}&-μH}
					{\xx{3>3}&\xx{pfv}&\xx{stv}&\rt[²]{swat}&\·\xx{var}}
		\item	\vbform{has ash koolchʼéitʼaa}{impfv}[subj intr, \fm{∅}?, inv act]{they are playing ball}
			\parencite[152.2069]{story-naish:1973}
				\vbmorph{has=&ash=&ka-&u-&d-&lˢ-&\rt{chʼetʼ}&-μμH&\gm{-aa}}
					{\xx{plh}&\xx{rflx.o}&\xx{qual}&\xx{irr}&\xx{mid}&\xx{csv}&\rt{ball}&\·\xx{var}&\·\xx{inst}}
			\versus \fm{koochʼéitʼaa} ‘ball’
				\vbmorph{ka-&u-&\rt{chʼetʼ}&-μμH&\gm{-aa}}
					{\xx{qual}&\xx{irr}&\rt{ball}&\·\xx{var}&\·\xx{inst}}
		\item	\vbform{has ash koolkʼéinaa}{impfv}[subj intr, \fm{∅}?, inv act]{they are playing at jumping}
			\parencite[152.2073]{story-naish:1973}
				\vbmorph{has=&ash=&ka-&u-&d-&l-&\rt[¹]{kʼeᴴn}&-μμH&\gm{-aa}}
					{\xx{plh}&\xx{rflx.o}&\xx{qual}&\xx{irr}&\xx{mid}&\xx{csv}&\rt{ball}&\·\xx{var}&\·\xx{inst}}
			\versus \vbform{héent wujikʼén}{pfv}[subj intr, \fm{∅}, mot]{she/he/it jumped into the water}
			\parencite[71.850]{story-naish:1973}
				\vbmorph{héen&-t&wu-&d-&sh-&i-&\rt[¹]{kʼeᴴn}&-μH}
					{water&\·\xx{pnct}&\xx{pfv}&\xx{mid}&\xx{pej}&\xx{stv}&\rt[¹]{jump}&\·\xx{var}}
		\item	\vbform{a kát ash kux̱ashx̱ʼílʼaa}{impfv}[subj intr, \fm{∅}?, inv act]{I am sledding on it}
			\parencite[98]{leer:1963}
				\vbmorph{a&ká&-t&ash=&ka-&u-&x̱a-&d-&sh-&\rt[¹]{kʼeᴴn}&-μμH&\gm{-aa}}
					{\xx{3n.psr}&\xx{hsfc}&\xx{pnct}&\xx{rflx.o}&\xx{qual}&\xx{irr}&\xx{1sg.s}&\xx{mid}&\xx{pej}&\rt[¹]{slide}&\·\xx{var}&\·\xx{inst}}
			\versus \fm{koox̱ʼílʼaa yeit} ‘recreational sled’
				\vbmorph{ka-&u-&\rt[¹]{x̱ʼilʼ}&-μH&\gm{-aa}&yee-&át}
					{\xx{qual}&\xx{irr}&\rt[¹]{slide}&\·\xx{var}&\·\xx{instr}&below&thing}
			\versus \vbform{wushix̱ʼéelʼ}{pfv}[subj intr, \fm{n}, mot]{she/he/it slid}
			\parencite[98]{leer:1963}
				\vbmorph{wu-&sh-&i-&\rt[¹]{x̱ʼilʼ}&-μμH}
					{\xx{pfv}&\xx{pej}&\xx{stv}&\rt[¹]{slide}&\·\xx{var}}
		\end{itemize}
	\item	unknown suffix in a few verbs, possibly onomatopoeia
		\begin{itemize}
		\item	\vbform{aatlein ax̱altsʼíxaa}{impfv}[subj intr, \fm{∅}?, inv act]{I sneeze a lot}
			\parencite[200.2788]{story-naish:1973}
				\vbmorph{aatlein&a-&x̱a-&d-&lˢ-&\rt{tsʼix}&-μH&\gm{-aa}}
					{lots&\xx{xpl}&\xx{1sg.s}&\xx{mid}&\xx{xtn}&\rt{sneeze}&\·\xx{var}&\·\xx{unkn}}
			\versus	\vbform{akḵwaltsʼíxaa}{prosp}{I’m going to sneeze}
			\parencite[200.2789]{story-naish:1973}
				\vbmorph{a-&k-&w-&g̱-&x̱a-&d-&lˢ-&\rt{tsʼix}&-μH&\gm{-aa}}
					{lots&\xx{xpl}&\xx{gcnj}&\xx{irr}&\xx{mod}&\xx{1sg.s}&\xx{mid}&\xx{xtn}&\rt{sneeze}&\·\xx{var}&\·\xx{unkn}}
		\item	\vbform{kalitsʼígwaa}{impfv}[obj intr, \fm{g}?, inv state]{she/he/it is a delicate matter}
				\vbmorph{ka-&lˢ-&i-&\rt{tsʼikw}&-μH&\gm{-aa}}
					{\xx{qual}&\xx{xtn}&\xx{stv}&\rt{delicate}&\·\xx{var}&\·\xx{unkn}}
		\end{itemize}
	\item	meaningless suffix in a few borrowed nouns;
		although this is not originally a suffix it acts like the instrument suffix
			in the same phonological way despite not contributing any meaning;
		for convenience this can be glossed as \xx{inst} like the real instrument suffix
		\begin{itemize}
		\item	\fm{dáanaa} ‘dollar, money, silver’
			from Chinook Jargon \fm{dála} (same meaning)
			ultimately from English \fm{dollar}
				\vbmorph{\rt{dan}&-μμH&\gm{-aa}}
					{\rt{money}&\·\xx{var}&\·\xx{instr}}
			\versus \fm{ax̱ dáanayi} ‘my dollar, money, silver’
				\vbmorph{ax̱&\rt{dan}&-μμH&\gm{-a}&-ÿi}
					{\xx{1sg.psr}&\rt{money}&\·\xx{var}&\·\xx{inst}&\·\xx{poss}}
		\item	\fm{shgóonaa} ‘sailboat, schooner’
			from English \fm{schooner}
				\vbmorph{sh=&\rt{gun}&-μμH&\gm{-aa}}
					{\xx{unkn}&\rt{sailboat}&\·\xx{var}&\·\xx{instr}}
			\versus \fm{ax̱ shgóonayi} ‘my sailboat, my schooner’
				\vbmorph{ax̱&sh=&\rt{gun}&-μμH&\gm{-a}&-ÿi}
					{\xx{1sg.psr}&\xx{unkn}&\rt{sailboat}&\·\xx{var}&\·\xx{instr}&\·\xx{poss}}
		\end{itemize}
	\item	meaningless suffix retained in verbs derived from borrowed nouns with this suffix
		\begin{itemize}
		\item	\vbform{jididáanaa}{impfv}[obj intr, \fm{g}?, inv state]{she/he/it has (lots of) money, is rich}
			\parencite[05/41]{leer:1973}
				\vbmorph{ji-&d-&i-&\rt[¹]{dan}&-μμH&\gm{-aa}}
					{hand&\xx{mid}&\xx{stv}&\rt[¹]{money}&\·\xx{var}&\·\xx{inst}}
			\versus \vbform{x̱at jidadáanayi kát}{sub impfv}{if I were rich}
			\parencite[15]{leer:1963}
				\vbmorph{x̱at=&ji-&da-&\rt[¹]{dan}&-μμH&\gm{-a}&-ÿi&ká&-t}
					{\xx{1sg.o}&hand&\xx{mid}&\rt[¹]{money}&\·\xx{var}&\·\xx{inst}&\·\xx{sub}&\xx{hsfc}&\·\xx{pnct}}
		\end{itemize}
	\item	part of unknown suffix \X{-jaa} \~\ \X{-jáa} in nouns derived from verb roots,
			see that entry for more detail
	\item	part of amissive \X{-x̱aa} \~\ \X{-x̱áa} in verbs with the ‘miss target’ derivation made up of:
		qualifier \X[ÿa-qual]{ÿa-}
		+ extensional \X{s-}/\X{lˢ-}
		+ repetitive \X{-x̱}
		+ unknown \fm{-aa} \~\ \X{-áa}
		with \fm{∅} conjugation class;
		see amissive \X{-x̱aa} \~\ \X{-x̱áa} for more detail;
		if treated as a single suffix \fm{-x̱aa} \~\ \fm{-x̱áa}
			this is glossed as \xx{miss},
			otherwise \xx{rep} + \xx{unkn}
		\begin{itemize}
		\item	\vbform{ayawsi.únx̱aa}{pfv}[tr, \fm{∅}, ach]{she/he/it shot at him/her/it and missed}
				\vbmorph{a-&ÿa-&w-&s-&i-&\rt[²]{.uᴴn}&-μH&\gm{-x̱}&\gm{-aa}}
					{\xx{3>3}&\xx{qual}&\xx{pfv}&\xx{xtn}&\xx{stv}&\rt[²]{shoot}&\·\xx{var}&\·\xx{rep}&\·\xx{unkn}} or
				\vbmorph{a-&ÿa-&w-&s-&i-&\rt[²]{.uᴴn}&-μH&\gm{-x̱aa}}
					{\xx{3>3}&\xx{qual}&\xx{pfv}&\xx{xtn}&\xx{stv}&\rt[²]{shoot}&\·\xx{var}&\·\xx{miss}}				
			\versus \vbform{aawa.ún}{pfv}{she/he/it shot him/her/it}
				\vbmorph{a-&μʷ-&wa-&\rt[²]{.uᴴn}&-μH}
					{\xx{3>3}&\xx{pfv}&\xx{stv}&\rt[²]{shoot}&\·\xx{var}}
		\end{itemize}
	\end{enumerate}

\item[-áa]\label{m:-áa}
	allomorph of \X{-aa} with H tone, used after L tone syllable (polar tone);
	for phonological reasons this allomorph is less common than \fm{-aa};
	attested only with nouns except for \vbform{sh yáx̱ awooltseenáa}{impfv}{she/he/it is exercising}
		and \vbform{dusdeegáa}{impfv}{people dipnet for it} shown below
	\begin{itemize}
	\item	\vbform{sh yáx̱ awooltseenáa}{impfv}[subj intr, conj?, inv act]{she/he/it is exercising}
		\parencite[09/162]{leer:1973} from \fm{\rt{tsin}} ‘alive, strong’
		\vbmorph{sh&yá&-x̱&a-&ÿa-&u-&d-&lˢ-&\rt{tsin}&-μμL&-áa}
			{\xx{rflx}&face&\·\xx{pert}&\xx{xpl}&face&\xx{irr}&\xx{mid}&\xx{csv}&\rt[¹]{alive}&\·\xx{var}&\·\xx{inst}}
		\newline
		includes motion derivation \motderiv{ÿaa= \~\ ÿa-u-}{∅, \fm{-ch} rep}{obliquely, circuitously, aside};
		similar in meaning to ‘pretend activity’ derivation (see \X{-aa})
			but with different structure
	\item	\fm{deegáa} ‘dipnet’
			\vbmorph{\rt{dik}&-μμL&\gm{-áa}}
				{\rt{dipnet}&\·\xx{var}&\·\xx{inst}}
		\versus \vbform{asdéek}{impfv}[tr, \fm{∅}, \fm{-μμH} act]{she/he/it dipnets for him/her/it}
			\vbmorph{a-&d-&s-&\rt{dik}&-μμH}
				{\xx{3>3}&\xx{mid}&\xx{csv}&\rt{dipnet}&\·\xx{var}}
		\versus \vbform{dusdeegáa}{impfv}[tr, \fm{∅}?, inv act]{people dipnet for it}
			\parencite[91.1143]{story-naish:1973}
				\vbmorph{du-&d-&s-&\rt{dik}&-μμL&\gm{-áa}}
					{\xx{ind.h.s}&\xx{mid}&\xx{tr}&\rt{dipnet}&\·\xx{var}&\xx{inst}}
	\end{itemize}
	because there are not many nouns with \fm{-áa} they are listed here for reference:
	\begin{itemize}
	\item	\fm{deegáa} ‘dipnet’
		from \fm{\rt{dik}} ‘fish by dipnet’
			\vbmorph{\rt{dik}&-μμL&\gm{-áa}}
				{\rt{dipnet}&\·\xx{var}&\·\xx{inst}}
	\item	\fm{dzeenáa} \~\ \fm{dzeináa} ‘small animal leg snare’
		from unknown \fm{\rt{dzin}} \~\ \fm{\rt{dzen}}
			\vbmorph{\rt{dzin}&-μμL&\gm{-áa}}
				{\rt{\xx{unkn}}&\·\xx{var}&\·\xx{inst}}
	\item	\fm{dzoonáa} \~\ \fm{dzeenáa} \~\ \fm{dzanáa} ‘dart, missile’
		from \fm{\rt{dzuᴸ}} ‘throw at’
			\vbmorph{\rt{dzuᴸ}&-μμL&-n&\gm{-áa}}
				{\rt{throw}&\·\xx{var}&\·\xx{nsfx}&\·\xx{inst}}
		\newline
		the form \fm{dzeináa} with ablaut \fm{-μᵉμL} from \fm{-n} is expected
			but apparently does not occur, perhaps because
			the stem has been reanalyzed
	\item	\fm{sʼeenáa} ‘lamp, light’
		from unknown \fm{\rt{sʼin}}
			\vbmorph{\rt{sʼin}&-μμL&\gm{-áa}}
				{\rt{\xx{unkn}}&\·\xx{var}&\·\xx{inst}}
	\item	\fm{g̱aatáa} ‘trap’
		from \fm{\rt{g̱at}} ‘split, fall apart’
			\vbmorph{\rt{g̱at}&-μμL&\gm{-áa}}
				{\rt{fall.apart}&\·\xx{var}&\·\xx{inst}}
	\item	\fm{woosheenáa} ‘staff, cane, walking stick’
		probably from \fm{\rt{shi}} ‘reach out hand, touch’ (but \X{-μμL})
			\vbmorph{ÿa-&u-&\rt{shi}&-μμL&-n&\gm{-áa}}
				{\xx{qual}&\xx{irr}&\rt{\xx{unkn}}&\·\xx{var}&\·\xx{nsfx}&\·\xx{inst}}
		\newline
		includes motion derivation \motderiv{ÿaa= \~\ ÿa-u-}{∅, \fm{-ch} rep}{obliquely, circuitously, aside};
		compare \fm{sheeÿ} ‘right (hand)’ and \fm{sheeÿ} ‘limb, knot’
	\item	\fm{wootsaag̱áa} ‘staff, cane, walking stick’
		from \fm{\rt{tsaḵ}} ‘poke, prod’
			\vbmorph{ÿa-&u-&\rt{tsaḵ}&-μμL&\gm{-áa}}
				{\xx{qual}&\xx{irr}&\rt{\xx{unkn}}&\·\xx{var}&\·\xx{inst}}
		\newline
		includes motion derivation \motderiv{ÿaa= \~\ ÿa-u-}{∅, \fm{-ch} rep}{obliquely, circuitously, aside}
	\item	\fm{tináa} \~\ \fm{teenáa} ‘copper shield’
		perhaps from \fm{\rt{tin}} ‘see’
			\vbmorph{\rt{tin}&-μμL&\gm{-áa}}
				{\rt{see}&\·\xx{var}&\·\xx{inst}}
	\item	\fm{xeisáa} ‘small animal or bird trap’
		from \fm{\rt{xes}} ‘catch with container’
			\vbmorph{\rt{xes}&-μμL&\gm{-áa}}
				{\rt{capture}&\·\xx{var}&\·\xx{inst}}
	\item	\fm{xaadáa} ‘veil, fine netting’
		from \fm{\rt{xat}} ‘pull, tighten; fasten’
			\vbmorph{\rt{xat}&-μμL&\gm{-áa}}
				{\rt{pull}&\·\xx{var}&\·\xx{inst}}
	\item	\fm{xeejáa} ‘springpole (of snare)’
		from \fm{\rt{xich}} ‘bend over’
			\vbmorph{\rt{x̱ich}&-μμL&\gm{-áa}}
				{\rt{bend.over}&\·\xx{var}&\·\xx{inst}}
	\item	\fm{xʼeesháa} ‘bucket’
		from unknown \fm{\rt{xʼish}}
			\vbmorph{\rt{xʼish}&-μμL&\gm{-áa}}
				{\rt{\xx{unkn}}&\·\xx{var}&\·\xx{inst}}
		\newline
		compare \fm{\rt{xʼish}} ‘skin’ and \fm{xʼíshaa} ‘skinning knife’
	\item	\fm{x̱oonáa} ‘tree debarker’
		from \fm{\rt{x̱uᴴw}} ‘peel bark’ (noun \fm{x̱óow} ‘slab of bark’)
			\vbmorph{\rt{x̱uᴴw}&-μμL&-n&\gm{-áa}}
				{\rt{debark}&\·\xx{var}&\·\xx{nsfx}&\·\xx{inst}}
	\end{itemize}

\item[aawa]\label{m:aawa}
	≡ \fm{a-μʷ-wa-}
	combination of argument marking \X{a-},
		perfective \X{μʷ-},
		and stative \X{wa-};
	compare \X{awu} ≡ \fm{a-wu-} and \X{aw} ≡ \fm{a-w-}
		as well as \X[eeÿa-a-ʷ-i-ÿa]{eeÿa} ≡ \fm{a-ʷ-i-ÿa-}
		and \X{oowa} ≡ \fm{a-u-wa-}
	\begin{itemize}
	\item	\vbform{aawajáḵ}{pfv}[tr, \fm{∅}, ach]{she/he/it killed him/her/it}
			\vbmorph{\gm{a-}&\gm{μʷ-}&\gm{wa-}&\rt[²]{jaḵ}&-μH}
				{\xx{3>3}&\xx{pfv}&\xx{stv}&\rt[²]{kill}&\·\xx{var}}
		\versus \vbform{awsi.átʼ}{pfv}[tr, \fm{∅}, ach]{she/he/it cooled him/her/it}
			\vbmorph{a-&w-&s-&i-&\rt[¹]{.átʼ}&-μH}
				{\xx{3>3}&\xx{pfv}&\xx{csv}&\xx{stv}&\rt[⁰]{cold}&\·\xx{var}}
	\end{itemize}

\item[ach=]\label{m:ach=}
	allomorph of third person proximate human object \X{ash=};
	may reflect older forms of reflexive with affricate \fm{ch} (compare \X{chush=})
		that otherwise became \fm{sh} (compare \X{sh=}),
	or perhaps derived from third person pronoun \fm{á} + postposition \fm{-ch}
		(compare \X{ách});
	apparently no reliable difference in meaning or context between \fm{ash=} and \fm{ach=},
		nor any phonological contexts that distinguish them;
	dialect distribution remains to be studied, but mostly attested in Inland Northern
	\begin{itemize}
	\item	\vbform{has ach x̱ʼawóosʼ}{impfv}[tr, \fm{n}, \fm{-μμH} act]{they question him/her}
		\parencite[60.396]{nyman-leer:1993}
			\vbmorph{has=&\gm{ach=}&x̱ʼe-&\rt[²]{wusʼ}&-μμH}
				{\xx{plh}&\xx{3prx.o}&mouth&\rt[²]{question}&\·\xx{var}}
		\versus \vbform{has ax̱ʼawóosʼ}{impfv}{they question him/her/it}
			\vbmorph{has=&a-&x̱ʼe-&\rt[²]{wusʼ}&-μμH}
				{\xx{plh}&\xx{3>3}&mouth&\rt[²]{question}&\·\xx{var}}
	\item	\vbform{du een ach katoolyádi}{sub impfv}[subj intr, \fm{n}, inv \fm{-μH} act]{that we play with him/her}
		\parencite[188.434]{dauenhauer-dauenhauer:1987}
			\vbmorph{du&ee&-n&\gm{ach=}&ka-&too-&d-&l-&\rt[¹]{ÿat}&-μH&-i}
				{\xx{3h}&\xx{base}&\·\xx{instr}&\xx{rflx.o}&\xx{qual}&\xx{1pl.s}&\xx{mid}&\xx{csv}&\rt[¹]{child}&\·\xx{var}&\·\xx{sub}}
		\versus \vbform{du een ash kanax̱toolyát}{hort}{let us play with him/her}
			\parencite[189.452]{dauenhauer-dauenhauer:1987}
			\vbmorph{du&ee&-n&ash=&ka-&na-&x̱-&too-&d-&l-&\rt[¹]{ÿat}&-μH}
				{\xx{3h}&\xx{base}&\·\xx{instr}&\xx{rflx.o}&\xx{qual}&\xx{ncnj}&\xx{mod}&\xx{1pl.s}&\xx{mid}&\xx{csv}&\rt[¹]{child}&\·\xx{var}&}
		\newline
		(note that these two examples are from the same speaker)
	\end{itemize}

\item[ách]\label{m:ách}
	≡ \fm{á-ch}
	combination of third person nonhuman pronoun \fm{á}
		and applicative instrumental postposition \fm{-ch};
	this is a postposition phrase required immediately before a few verbs and is
	not actually a verb morpheme although it resembles (and may be becoming) a preverb
	\begin{itemize}
	\item	\vbform{du jeedé ách ax̱wsiwóo}{pfv}[subj intr, \fm{∅}, ach]{I sent him/her/it to him/her/it}
		\parencite[03/296]{leer:1973}
			\vbmorph{du&jee&-dé&\gm{á}&\gm{-ch}&a-&ʷ-&x̱-&s-&i-&\rt[²]{wu}&-μμH}
				{\xx{3h.psr}&pos’n&\·\xx{all}&\xx{3n}&\·\xx{instr}&\xx{xpl}&\xx{pfv}&\xx{1sg.s}&\xx{appl}&\xx{stv}&\rt[²]{send}&\·\xx{var}}
	\end{itemize}

\item[-áchʼ]\label{m:-áchʼ}
	allomorph of unknown suffix \X{-chʼ} with epenthetic (filler) vowel \fm{á};
	this \fm{-áchʼ} is attested only in the noun
		\fm{g̱eeg̱áchʼ} \~\ \fm{g̱eig̱áchʼ} ‘hammock, swing for baby’
		as well as in verbs and nouns derived from this noun;
	since the meaning of \fm{-chʼ} \~\ \fm{-áchʼ} is unknown it is glossed \xx{unkn}
	\begin{itemize}
	\item	\fm{g̱eeg̱áchʼ} \~\ \fm{g̱eig̱áchʼ} ‘hammock, swing for baby’
		\vbmorph{\rt{g̱eḵ}&-μμL&\gm{-áchʼ}}
			{\rt{swing}&\·\xx{var}&\·\xx{unkn}}
		\begin{itemize}
		\item	\fm{g̱eeg̱áchʼaa} \~\ \fm{g̱eig̱áchʼaa} ‘swing’
			\parencite[f02/193]{leer:1973} with suffix \X{-aa} ‘instrument for’;
			implies verb → noun → verb → noun
		\end{itemize}
	\item	\vbform{ash koolg̱eig̱áchʼ}{impfv}[subj intr, conj?, act]{she/he/it is playing on a swing/hammock}
			\vbmorph{ash=&ka-&u-&d-&l-&\rt[¹]{g̱eḵ}&-μμL&\gm{-áchʼ}}
				{\xx{rflx.o}&\xx{qual}&\xx{irr}&\xx{mid}&\xx{csv}&\rt[¹]{swing}&\·\xx{var}&\·\xx{unkn}}
		\versus \vbform{awlig̱eiḵ}{pfv}[tr, \fm{n}, mot]{she/he/it swung him/her/it}
			\vbmorph{a-&w-&l-&i-&\rt[¹]{g̱eḵ}&-μμL}
				{\xx{3>3}&\xx{pfv}&\xx{csv}&\xx{stv}&\rt[¹]{swing}&\·\xx{var}}
		\notealso{(\fm{ash koolg̱eig̱áchʼ} is presumably verb → noun → verb)}
	\item	possibly in \fm{tlʼaaḵʼwáchʼ} ‘sourdock, wild rhubarb (\textit{Rumex} spp.)’
		\parencite[f01/251]{leer:1973} implying a stem \fm{tlʼaaḵʼw},
		although \textcite[79]{leer:1978b} analyzes this with a stem \fm{ḵʼwáchʼ}
		perhaps related to obscure \fm{\rt{ḵʼwash}} ‘peel’
		and \fm{x̱ʼwaash} ‘large sea urchin’ \parencite[f01/215]{leer:1973}
	\end{itemize}

\item[-álʼ]\label{m:-álʼ}
	allomorph of repetitive \X{-lʼ} with epenthetic (filler) vowel \fm{á};
	attested only with two verbs but may be possible with others given attested nouns
	\begin{enumerate}
	\item	repetitive suffix attested with two verb roots;
		with \fm{\rt[²]{chuxʼ}} ‘touch lightly, graze’ the \fm{-álʼ} occurs only in the
			repetitive imperfective form
		but with \fm{…néegwálʼ} ‘paint’ the \fm{-álʼ} occurs in all forms
			and so forms a frozen disyllabic stem;
		the order of \fm{-álʼ}\X{-k} in \fm{yoo ayanéegwálʼk} is interesting
		\begin{itemize}
		\item	\vbform{achóoxʼálʼ}{rep impfv}[tr, \fm{∅}, \fm{-μμH} act]{she/he/it repeatedly touches him/her/it}
			\parencites[10/227]{leer:1973}[598]{leer:1976}
				\vbmorph{a-&\rt[²]{chuxʼ}&-μμH&\gm{-álʼ}}
					{\xx{3>3}&\rt[²]{graze}&\·\xx{var}&\·\xx{rep}}
			\versus \vbform{achóoxʼ}{impfv}{she/he/it touches him/her/it}
				\vbmorph{a-&\rt[²]{chuxʼ}&-μμH}
					{\xx{3>3}&\rt[²]{graze}&\·\xx{var}}
		\item	\vbform{anéegwálʼ}{impfv}[tr, \fm{n}, inv act]{she/he/it paints him/her/it}
				\vbmorph{a-&\rt[²]{nikw}&-μμH&\gm{-álʼ}}
					{\xx{3>3}&\rt[²]{paint}&\·\xx{var}&\·\xx{rep}}
			\versus \vbform{yoo ayanéegwálʼk}{rep impfv}{she/he/it repeatedly paints him/her/it}
				\vbmorph{yoo=&a-&ÿa-&\rt[²]{nikw}&-μμH&\gm{-álʼ}&-k}
					{\xx{alt}&\xx{3>3}&\xx{stv}&\rt[²]{paint}&\·\xx{var}&\·\xx{rep}&\·\xx{rep}}
		\end{itemize}
	\item	unclear meaning in a handful of nouns
		\begin{itemize}
		\item	\fm{ḵéichʼálʼ} \~\ \fm{ḵéechʼálʼ} ‘seam’ from \fm{\rt{ḵa}} ‘stitch, sew’
			\vbmorph{\rt{ḵa}&-μᵉμH&-chʼ&\gm{-álʼ}}
				{\rt{stitch}&\·\xx{var}&\·\xx{unkn}&\·\xx{rep}}
			\newline
			see \X{-chʼálʼ} for more detail on this noun
		\item	\fm{néegwálʼ} ‘paint’ from unknown \fm{\rt{nikw}}
			possibly originally from \fm{\rt{ni}} \~\ \fm{\rt{ne}} ‘occur, do’ 
			+ repetitive \X{-kw}
				\vbmorph{\rt{nikw}&-μμH&\gm{-álʼ}}
					{\rt{paint?}&\·\xx{var}&\·\xx{rep}}
			\newline
			also occurs in:
			\begin{itemize}
			\item	\fm{kanéegwálʼ} ‘berries and salmon eggs’
				\vbmorph{ka-&\rt{nikw}&-μμH&\gm{-álʼ}}
					{\xx{sro}&\rt{paint?}&\·\xx{var}&\·\xx{rep}}
			\item	\fm{x̱ʼanéegwálʼ} ‘lipstick’
				\vbmorph{x̱ʼe-&\rt{nikw}&-μμH&\gm{-álʼ}}
					{mouth&\rt{paint?}&\·\xx{var}&\·\xx{rep}}
			\end{itemize}
		\item	\fm{táaxʼálʼ} ‘needle’ from \fm{\rt{taxʼ}} ‘bite’
			\vbmorph{\rt{taxʼ}&-μμH&\gm{-álʼ}}
				{\rt{bite}&\·\xx{var}&\·\xx{rep}}
		\item	\fm{tʼaag̱álʼ} ‘fastening peg’ perhaps from
			\fm{\rt{tʼaḵ}} ‘shift position, aside’ (noun \fm{tʼaaḵ} ‘beside’)
			\vbmorph{\rt{tʼaḵ}&-μμL&\gm{-álʼ}}
				{\rt{aside}&\·\xx{var}&\·\xx{rep}}
			\newline
			compare 
			\begin{inlinelist}
			\item	\fm{tʼáḵlʼ} ‘projecting bone’ (see \X{-lʼ}\!)
			\item	\fm{tʼáaḵw} ‘joist’
			\item	\fm{tʼáax̱ʼw} ‘wart’
			\item	\fm{\rt{tʼak}} ‘dent’
			\item	\fm{\rt{tʼakw}} ‘slap tail, smack’
			\item	\fm{\rt{tʼax̱}} ‘gape’
			\end{inlinelist}
		\item	\fm{tsaag̱álʼ} ‘spear’ from \fm{\rt{tsaḵ}} ‘poke, prod’
			\vbmorph{\rt{tsaḵ}&-μμL&\gm{-álʼ}}
				{\rt{poke}&\·\xx{var}&\·\xx{rep}}
		\item	\fm{x̱eeygwálʼ} ‘pack strap, lashing’ from \fm{\rt{x̱iÿ}} ‘pack’
			\vbmorph{\rt{x̱iÿ}&-μμL&-kw&\gm{-álʼ}}
				{\rt{pack}&\·\xx{var}&\·\xx{rep}&\·\xx{rep}}
		\item	\fm{x̱ʼéexʼwálʼ} \~\ \fm{x̱ʼéixʼwálʼ} ‘safety pin’ from \fm{\rt{x̱ʼixʼ}} ‘wedge’
			\vbmorph{\rt{x̱ʼixʼ}&-μμH&\gm{-álʼ}}
				{\rt{wedge}&\·\xx{var}&\·\xx{rep}}
			\newline
			also occurs in:
			\begin{itemize}
			\item	\fm{shax̱ʼéexʼwálʼ} ‘hair clip, barette’
				\vbmorph{sha-&\rt{x̱ʼixʼ}&-μμH&\gm{-álʼ}}
					{head&\rt{wedge}&\·\xx{var}&\·\xx{rep}}
			\end{itemize}
		\item	\fm{x̱ʼéigwálʼ} ‘safety pin’ from \fm{\rt{x̱ʼe}} ‘mouth’
			\vbmorph{\rt{x̱ʼe}&-μμH&-kw&\gm{-álʼ}}
				{\rt{mouth}&\·\xx{var}&\·\xx{rep}&\·\xx{rep}}
			\newline
			probably a reanalysis of \fm{x̱ʼéexʼwálʼ} \~\ \fm{x̱ʼéixʼwálʼ} above;
			also occurs in:
			\begin{itemize}
			\item	\fm{shax̱ʼéegwálʼ} ‘hair clip, barette’
				\vbmorph{sha-&\rt{x̱ʼe}&-μμH&-kw&\gm{-álʼ}}
					{head&\rt{mouth}&\·\xx{var}&\·\xx{rep}&\·\xx{rep}}
			\end{itemize}
		\end{itemize}
	\end{enumerate}

\item[-áḵw]\label{m:-áḵw}
	allomorph of deprivative \X[-ḵ-dprv]{-ḵ} \~\ \X[-ḵw-dprv]{-ḵw}
		with epenthetic (filler) vowel \fm{á};
	generally describes a situation where something is deprived, lacking, or removed;
	this \fm{-áḵw} form is attested specifically in the following verb stems
		where it has a clear compositional meaning of ‘remove, deprive’:
	\begin{enumerate}
	\item	derivational suffix meaning ‘remove, deprive’ in six verb stems
		\begin{itemize}
		\item	\fm{–geiÿáḵw} ‘claim as payment’
			from \fm{\rt{geᴴÿ}} ‘repay’ (but \X{-μμL}) in
			\newline
			\vbform{algeiÿáḵw}{impfv}[tr, \fm{∅}?, inv act]{she/he/it claims him/her/it as/for payment}
			\parencite[f05/74]{leer:1973}
				\vbmorph{a-&l-&\rt[²]{geᴴÿ}&-μμL&\gm{-áḵw}}
					{\xx{3>3}&\xx{xtn}&\rt[²]{repay}&\·\xx{var}&\·\xx{dprv}}
			\versus \vbform{awsigéÿ}{pfv}[tr, \fm{∅}, ach]{she/he/it has repaid for him/her/it}
				\vbmorph{a-&w-&s-&i-&\rt[²]{geᴴÿ}&-μH}
					{\xx{3>3}&\xx{pfv}&\xx{xtn}&\xx{stv}&\rt[²]{repay}&\·\xx{var}}
		\item	\fm{–g̱eiÿáḵw} ‘scoop out (of shell)’
			from \fm{\rt{g̱e}} ‘between’ (noun \fm{g̱ei}) in
			\newline
			\vbform{alg̱eiÿáḵw}{impfv}[tr, \fm{n}, inv act]{she/he/it scoops it out (of shell)}
			\parencite[f02/147]{leer:1973}
				\vbmorph{a-&l-&\rt{g̱e}&-μμL&\gm{-ÿáḵw}}
					{\xx{3>3}&\xx{tr}&\rt{between}&\·\xx{var}&\·\xx{dprv}}
		\item	\fm{–nóoxʼáḵw} ‘remove shell of’
			from \fm{\rt{nuxʼ}} ‘shell’ (noun \fm{nóoxʼ}) in
			\newline
			\vbform{kadulnóoxʼáḵw}{impfv}[tr, \fm{∅}?, inv act]{they remove the shell from it}
			\parencite[171.2350]{story-naish:1973}
				\vbmorph{ka-&du-&d-&l-&\rt{nuxʼ}&-μμH&\gm{-áḵw}}
					{\xx{qual}&\xx{ind.h.s}&\xx{mid}&\xx{csv}&\rt{shell}&\·\xx{var}&\·\xx{dprv}}
		\item	\fm{–tlʼéiláḵw} ‘remove milt of’
			from \fm{\rt{tlʼeᴴl}} ‘milt’ (noun \fm{tlʼéil}) in
			\newline
			\vbform{altlʼéiláḵw}{impfv}[tr, \fm{n}, inv act]{she/he/it removes the milt from him/her/it}
				\vbmorph{a-&l-&\rt{tlʼeᴴl}&-μμH&\gm{-áḵw}}
					{\xx{3>3}&\xx{tr}&\rt{milt}&\·\xx{var}&\·\xx{dprv}}
		\item	\fm{–xʼwánjáḵw} ‘remove boots’
			from \fm{\rt{xʼwan}} ‘boot’ in
			\newline
			\vbform{kawdlixʼwánjáḵw}{pfv}[subj intr?, \fm{∅}?, ach?]{she/he/it removed boots}
			\parencite[f04/77]{leer:1973}
				\vbmorph{ka-&w-&d-&lˢ-&i-&\rt{xʼwan}&-μH&-ch&\gm{-áḵw}}
					{\xx{qual}&\xx{pfv}&\xx{mid}&\xx{tr}&\xx{stv}&\rt{boot}&\·\xx{var}&\·\xx{rep}&\·\xx{dprv}}
		\item	\fm{–x̱aaÿáḵw} ‘shed, lose hair, go bald’
			from \fm{\rt{x̱aw}} ‘fur, hair’ (noun \fm{x̱aaw}) in
			\newline
			\vbform{wudlix̱aaÿáḵw}{pfv}[obj intr, \fm{n}, ach]{she/he/it lost hair, shed, went bald}
			\parencite[786]{leer:1976}
				\vbmorph{wu-&d-&l-&i-&\rt{x̱aw}&-μμL&\gm{-áḵw}}
					{\xx{pfv}&\xx{pasv}&\xx{xtn}&\xx{stv}&\rt{fur}&\·\xx{var}&\·\xx{dprv}}
		\end{itemize}
	\item	derivational suffix ‘remove, deprive’ in verb stems
		where the phonological composition is irregular
		or the underlying root has an unknown meaning:
		\begin{itemize}
		\item	\fm{–.éiÿáḵw} ‘have injured limb’
			from unknown \fm{\rt{.e}} or \fm{\rt{.eÿ}} or \fm{\rt{.a}} in
			\newline
			\vbform{kawdi.éiÿáḵw}{pfv}[obj intr, \fm{g̱}?, ach?]{she/he/it is incapacitated; it (body part) is dislocated, useless}
			\parencite[02/21]{leer:1973}
				\vbmorph{ka-&w-&d-&i-&\rt{.eÿ}&-μμH&\gm{-áḵw}}
					{\xx{qual}&\xx{pfv}&\xx{mid}&\xx{stv}&\rt{\xx{unkn}}&\·\xx{var}&\·\xx{dprv}}
			\newline
			\citeauthor{leer:1976} points to \fm{\rt{.a}} ‘end move, extend’ \parencite[78]{leer:1976}
				but the compositional meaning is unclear;
			possibly related to \fm{\rt{.ek}} ‘weak, paralyzed’
		\item	\fm{–ÿaax̱áḵw} ‘plan, intend’
			from unknown \fm{\rt{ÿax̱}} in
			\newline
			\vbform{yéi awliÿaax̱áḵw}{pfv}[tr, conj?, ach?]{she/he/it wanted, planned it thus}
				\vbmorph{yéi=&a-&w-&l-&i-&\rt{ÿax̱}&-μμL&\gm{-áḵw}}
					{thus&\xx{3>3}&\xx{pfv}&\xx{xtn}&\xx{stv}&\rt{\xx{unkn}}&\·\xx{var}&\·\xx{dprv}}
			\newline
			probably related to \fm{\rt{.aḵw}} ‘direct, command, plan, try’
				and \fm{\rt{ÿaḵw}} ‘compare, liken’;
			\citeauthor{leer:1976} points to \fm{\rt{ÿaḵw}} ‘bequeath, pass on’ \parencite[12]{leer:1978b}
		\item	\fm{–x̱oonáḵw} ‘drowned’
			likely from \fm{\rt{x̱un}} ‘relative, friend’ (noun \fm{x̱oon-í}) in
			\newline
			\vbform{wudix̱oonáḵw}{pfv}[obj intr?, \fm{g̱}?, ach?]{she/he/it drowned}
			\parencite[f02/77]{leer:1973}
				\vbmorph{wu-&d-&i-&\rt{x̱un}&-μμL&\gm{-áḵw}}
					{\xx{pfv}&\xx{pasv}&\xx{stv}&\rt{relative}&\·\xx{var}&\·\xx{dprv}}
		\item	\fm{–chʼéeÿáḵw} ‘slow, late’
			from unknown \fm{\rt{chʼi}} or \fm{\rt{chʼiÿ}} in
			\newline
			\vbform{lichʼéeÿáḵw}{impfv}{obj intr, \fm{g}, inv state}{she/he/it is slow, late}
				\vbmorph{lˢ-&i-&\rt{chʼiÿ}&-μμH&\gm{-áḵw}}
					{\xx{intr}&\xx{stv}&\rt{early?}&\·\xx{var}&\·\xx{dprv}}
		\end{itemize}
	\item	derivational suffix in the stem
		\fm{–séewchʼáḵw} ‘watery, tasteless’
		together with the noun \fm{séew} ‘rain’
		and the suffix \X{-chʼáḵw}
		which is probably a combination of
		the unknown suffix \X{-chʼ}
		and \fm{-áḵw},
		see \X{-chʼáḵw} for details;
	\item	unclear meaning but probably deprivative in several nouns and adjectives
		\begin{itemize}
		\item	\fm{gaawáḵ} ‘saskatoonberry’
			perhaps from \fm{\rt{gaw}} ‘noise’
				\vbmorph{\rt{gaw}&-μμL&\gm{-áḵ}}
					{\rt{noise}&\·\xx{var}&\·\xx{dprv}}
		\item	\fm{gantáḵw} \~\ \fm{kantáḵw} ‘lupine’
			perhaps from \fm{\rt{gan}} ‘burn’ (but \X{-μL})
				\vbmorph{\rt{gan}&-μL&-t&\gm{-áḵw}}
					{\rt{burn}&\·\xx{var}&\·\xx{rep}&\·\xx{dprv}}
			\newline
			but \fm{kantáḵw} instead suggests unknown \fm{\rt{kan}}
			for which compare
			\begin{inlinelist}
			\item	\fm{\rt{kan}} ‘dance in return’
			\item	\fm{kánaa} ‘flag, dance staff’ (with \X{-aa})
			\item	\fm{chookán} ‘grass’
			\item	\fm{keekán} ‘going to see’
			\item	\fm{g̱uwakaan} ‘deer’
			\item	\fm{káani} ‘sibling-in-law’
			\end{inlinelist}
		\item	\fm{goosháḵw} \~\ \fm{gooshúḵ} ‘nine’
			from \fm{\rt{gush}} ‘dorsal fin, thumb’
				\vbmorph{\rt{gush}&-μμL&\gm{-áḵw}}
					{\rt{thumb}&\·\xx{var}&\·\xx{dprv}}
		\item	\fm{katʼéitʼáḵw} ‘berry on stem after freeze’
			from \fm{\rt{tʼaᴸ}} ‘hot; ripe’ (but \X{-μᵉμH})
				\vbmorph{ka-&\rt{tʼaᴸ}&-μᵉμH&-tʼ&\gm{-áḵw}}
					{\xx{sro}&\rt{ripe}&\·\xx{var}&\·\xx{rep}&\·\xx{dprv}}
		\item	\fm{kax̱duḵéisʼáḵw} \~\ \fm{katḵéisʼáḵw} ‘quilt’
			from \fm{\rt{ḵa}} ‘stitch, sew’
				\vbmorph{ká&-x̱&du-&\rt[²]{ḵa}&-μμᵉH&-sʼ&\gm{-áḵw}}
					{\xx{hsfc}&\·\xx{pert}&\xx{ind.h.s}&\rt[²]{stitch}&\·\xx{var}&\·\xx{rep}&\·\xx{dprv}}
				 or \vbmorph{ká&-t&\rt[²]{ḵa}&-μμᵉH&-sʼ&\gm{-áḵw}}
				 	{\xx{hsfc}&\·\xx{pnct}&\rt[²]{stitch}&\·\xx{var}&\·\xx{rep}&\·\xx{dprv}}
		\item	\fm{tlʼeitáḵw} (prenominal adjective) ‘pure, honest’
			from unknown \fm{\rt{tlʼet}}
			probably from \fm{\rt{tlʼeᴴn}} ‘impure, unclean’ (but \X{-μμH})
				\vbmorph{\rt{tlʼet}&-μμL&\gm{-áḵw}}
					{\rt{impure}&\·\xx{var}&\·\xx{dprv}}
		\item	\fm{ḵutx̱ naaxʼáḵw} ‘genocide or extinction of a nation’
			from \fm{\rt{na}} ‘clan, nation’
				\vbmorph{ḵutx̱&\rt{na}&-μμL&-xʼ&-áḵw}
					{too.much&\rt{clan}&\·\xx{var}&\·\xx{pl}&\·\xx{dprv}}
		\item	\fm{laanáḵw} ‘barren female ruminant’
			from unknown \fm{\rt{lan}}
			possibly related to \fm{laaw} ‘penis’ or \fm{lʼaa} ‘breast’
				\vbmorph{\rt{lan}&-μμL&\gm{-áḵw}}
					{\rt{\xx{unkn}}&\·\xx{var}&\·\xx{dprv}}
		\item	\fm{shax̱dáḵw} \~ \fm{shuḵdáḵ} ‘legendary man-eating shark’
			from unknown \fm{\rt{shax̱}} or \fm{\rt{shuḵ}}
			possibly related to \fm{sháḵʼw} \~\ \fm{shúḵʼ} ‘legendary wolf-like animal’
				\vbmorph{\rt{shax̱}&-μL&-t&\gm{-áḵw}}
					{\rt{\xx{unkn}}&\·\xx{var}&\·\xx{rep}&\·\xx{dprv}}
			\newline
		\item	\fm{tʼaawáḵ} ‘Canada goose’
			from \fm{tʼaaw} ‘feather’
				\vbmorph{\rt{tʼaw}&-μμL&\gm{-áḵ}}
					{\rt{feather}&\·\xx{var}&\·\xx{dprv}}
		\item	\fm{Tlʼanaxéedáḵw} ‘Property/Wealth Woman’
			from earlier \fm{Tlʼeinax̱xéedáḵw} \parencite[“ʟ!ê′nᴀxx̣ī′dᴀq” in][]{swanton:1909}
				\vbmorph{tlʼeiḵ&-náx̱&\rt{xit}&-μμH&\gm{-áḵw}}
					{finger&\·\xx{perl}&\rt{scratch}&\·\xx{var}&\·\xx{dprv}}
		\item	\fm{x̱aatlʼáḵw} ‘mouth ulcer’ from unknown \fm{\rt{x̱atlʼ}}
			possibly related to \fm{\rt[²]{x̱a}} ‘eat’ and/or \fm{x̱aatlʼ} ‘freshwater grass’
				\vbmorph{\rt{x̱atlʼ}&-μμH&\gm{-áḵw}}
					{\rt{\xx{unkn}}&\·\xx{var}&\·\xx{dprv}}
			\newline
			and \fm{washtux̱aatlʼáḵw} ‘cheek ulcer, cold sore’
				with \fm{wásh} ’cheek’ + \fm{tú} ‘inside’
		\end{itemize}
	\end{enumerate}

\item[am]\label{m:am}
	≡ \fm{a-m-}
	combination of argument marking \X{a-}
		and perfective \X{m-};
	compare \X{aw} ≡ \fm{a-w-}, \X{awu} ≡ \fm{a-wu-}, and \X{aawa} ≡ \fm{a-μʷ-wa-};
	because \fm{m-} only occurs in a syllable coda,
		forms like \fm[*]{amu} cannot occur,
		see \X{m-} for more details
	\begin{itemize}
	\item	\vbform{amsi.ée}{pfv}[tr, \fm{∅}, \fm{-μμH} act]{s/he/it cooked him/her/it}
			\vbmorph{\gm{a-}&\gm{m-}&s-&i-&\rt[¹]{.i}&-μμH}
				{\xx{3>3}&\xx{pfv}&\xx{csv}&\xx{stv}&\rt[¹]{cooked}&\·\xx{var}}
		\versus \vbform{awsi.ée}{pfv}{s/he/it cooked him/her/it}
			\vbmorph{a-&w-&s-&i-&\rt[¹]{.i}&-μμH}
				{\xx{3>3}&\xx{pfv}&\xx{csv}&\xx{stv}&\rt[¹]{cooked}&\·\xx{var}}
	\end{itemize}

\item[-án]\label{m:-án}
	restorative suffix, indicates restoration of a previous situation;
	possibly also occurs as part of intensifier \X{-chʼán} \~\ \X{-shán}
		but its contribution to the meaning in this is unclear

\item[-ani]\label{m:-ani}
	suffix with unknown meaning

\item[as=]\label{m:as=}
	allomorph of human pluralizer \fm{has=} for third person subject or object;
	mostly occurs in Southern \&\ Tongass varieties
	\begin{itemize}
	\item	\vbform{as dustaaÿch}{hab}[tr, \fm{∅}, ach]{they would boil it}
		(Tongass dialect) \parencite[24.80]{leer:1978}
		\vbmorph{\gm{as=}&du-&d-&s-&\rt[¹]{taᴸ}&-μμ&-ÿ&-ch}
			{\xx{plh}&\xx{ind.h.s}&\xx{mid}&\xx{csv}&\rt[¹]{boil}&\·\xx{var}&\·\xx{ÿsfx}&\·\xx{rep}}
		\versus \vbform{has dustáaych}{hab}{they would boil it} (Northern dialect)
		\vbmorph{has=&du-&d-&s-&\rt[¹]{taᴸ}&-μμH&-ÿ&-ch}
			{\xx{plh}&\xx{ind.h.s}&\xx{mid}&\xx{csv}&\rt[¹]{boil}&\·\xx{var}&\·\xx{ÿsfx}&\·\xx{rep}}
	\end{itemize}

\item[-ás]\label{m:-ás}
	allomorph of unknown suffix \X{-s} with epenthetic (filler) vowel \fm{á};
	only attested with the root \fm{\rt[²]{ḵe}} \~\ \fm{\rt[²]{ḵi}} ‘pay’
		so its meaning is unknown;
	this allomorph is found in most varieties of Tlingit together with \X{-n} as \X{nás},
		but in Tongass Tlingit the \X{-s} allomorph is used without \fm{-n}
	\begin{itemize}
	\item	\vbform{a káxʼ aawaḵéinás}{pfv}[tr, \fm{n}, ach]{she/he/it asked him/her/it for it in exchange}
		\parencites[82.1003]{story-naish:1973}[868]{leer:1973}
			\vbmorph{a-&μʷ-&wa-&\rt[²]{ḵe}&-μμH&-n&\gm{-ás}}
				{\xx{3>3}&\xx{pfv}&\xx{stv}&\rt[²]{pay}&\·\xx{var}&\·\xx{nsfx}&\·\xx{unkn}}
		\versus \vbform{aawaḵéi}{pfv}[tr, \fm{n}, ach]{she/he/it paid him/her/it}
			\vbmorph{a-&μʷ-&wa-&\rt[²]{ḵe}&-μμH}
				{\xx{3>3}&\xx{pfv}&\xx{stv}&\rt[²]{pay}&\·\xx{var}}
	\end{itemize}

\item[-ásʼ]\label{m:-ásʼ}
	allomorph of repetitive \X{-sʼ} with epenthetic (filler) vowel \fm{á};
	attested only in one verb derived from the noun \fm{x̱aanásʼ} ‘raft’
		but may be identified in some other nouns;
	in all cases its meaning is unknown but presumably originally repetitive
	\begin{enumerate}
	\item	in the noun \fm{x̱aanásʼ} ‘raft’ and in verbs derived from this noun
		\begin{itemize}
		\item	\fm{x̱aanásʼ} ‘raft’ from \fm{\rt{x̱a}} ‘paddle (canoe)’
			\vbmorph{\rt{x̱a}&-μμL&-n&\gm{-ásʼ}}
				{\rt{paddle}&\·\xx{var}&\·\xx{nsfx}&\·\xx{unkn}}
		\item	\vbform{awdlix̱aanásʼ}{pfv}[tr, conj?, ach?]{she/he/it made him/her/it into a raft}
			\parencite[787]{leer:1976}
			\vbmorph{a-&w-&d-&lˢ-&i-&\rt[²]{x̱a}&-μμL&-n&\gm{-ásʼ}}
				{\xx{3>3}&\xx{pfv}&\xx{mid}&\xx{xtn}&\xx{stv}&\rt[²]{paddle}&\·\xx{var}&\·\xx{nsfx}&\·\xx{unkn}}
		\item	\vbform{yan awtudlix̱aanásʼ}{pfv}{we went by raft}
			\parencite[165.2271]{story-naish:1973}
			\vbmorph{ÿan=&a-&w-&tu-&d-&lˢ-&i-&\rt[²]{x̱a}&-μμL&-n&\gm{-ásʼ}}
				{\xx{term}&\xx{xpl}&\xx{pfv}&\xx{1pl.s}&\xx{mid}&\xx{xtn}&\xx{stv}&\rt[²]{paddle}&\·\xx{var}&\·\xx{nsfx}&\·\xx{unkn}}
		\end{itemize}
	\item	possibly identifiable in two nouns with unclear etymology
		\begin{itemize}
		\item	\fm{kʼoodásʼ} ‘shirt, tunic’ from unknown \fm{\rt{kʼut}}
			\vbmorph{\rt{kʼut}&-μμL&\gm{-ásʼ}}
				{\rt{\xx{unkn}}&\·\xx{var}&\·\xx{unkn}}
			\newline
			perhaps related to \fm{\rt{kʼut}} ‘bounce back, rebound’
				but the composition of meaning is unclear
		\item	\fm{taagwásʼ} ‘sunray skate’ perhaps from \fm{\rt{takw}} ‘winter’
			\vbmorph{\rt{takw}&-μμL&\gm{-ásʼ}}
				{\rt{winter}&\·\xx{var}&\·\xx{unkn}}
		\end{itemize}
	\end{enumerate}

\item[ash=]\label{m:ash=}
	object proclitic;
	probably derived from third person pronoun \fm{á} + reflexive pronoun \fm{sh}
		or ergative \fm{-ch};
		see also \X{ach=} used instead of \fm{ash=} by some speakers;
	\newline
	allomorphs:
	\begin{allolist}
	\item[ash=]	basic form
	\item[\X{ach=}]	affricate \fm{ch} instead of fricative \fm{sh}
	\end{allolist}
	\begin{enumerate}
	\item	third person proximate human object
	\item	special reflexive object
	\end{enumerate}

\item[at=]\label{m:at=}
	indefinite nonhuman object ‘something, stuff’;
	derived from the noun \fm{át} ‘thing’ (as in \fm{wé át} ‘that thing’);
	see also \fm{a-} used instead of \fm{at=} in some verbs
	\begin{itemize}
	\item	\vbform{at wutusiteen}{pfv}[tr, \fm{g̱}, ach]{we saw something}
			\vbmorph{\gm{at=}&wu-&tu-&s-&i-&\rt[²]{tin}&-μμL}
				{\xx{ind.n.o}&\xx{pfv}&\xx{1pl.s}&\xx{xtn}&\xx{stv}&\rt[²]{see}&\·\xx{var}}
		\versus \vbform{wutusiteen}{pfv}{we saw him/her/it}
			\vbmorph{&wu-&tu-&s-&i-&\rt[²]{tin}&-μμL}
				{&\xx{pfv}&\xx{1pl.s}&\xx{xtn}&\xx{stv}&\rt[²]{see}&\·\xx{var}}
	\item	\vbform{at wusiteen}{pfv}[tr, \fm{g̱}, ach]{s/he/it saw something}
			\vbmorph{\gm{at=}&wu-&s-&i-&\rt[²]{tin}&-μμL}
				{\xx{ind.n.o}&\xx{pfv}&\xx{xtn}&\xx{stv}&\rt[²]{see}&\·\xx{var}}
		\versus \vbform{awsiteen}{pfv}{s/he/it saw him/her/it}
			\vbmorph{a-&w-&s-&i-&\rt[²]{tin}&-μμL}
				{\xx{3>3}&\xx{pfv}&\xx{xtn}&\xx{stv}&\rt[²]{see}&\·\xx{var}}
	\end{itemize}

\item[-át]\label{m:-át}
	suffix with unknown meaning,
		attested only in one verb derived from a noun,
		ultimately from an unknown root;
	possibly related to \X{-t} but this is only speculation;
	could instead be a suffix \fm{-kʼát} but that is also unattested elsewhere,
		though there is a poorly documented particle \fm{kʼát}
		in the phrase \fm{chʼa kʼát} ‘at least’
		which unfortunately does not shed much light on this case;
	another alternative parse is diminutive \fm{-kʼ} + \fm{-át}
		but this also has an unclear meaning;
	sometimes instead attested as \X{-átʼ}
		which may be remodeled by analogy with \X{-tʼ} and 
		and the alternation between \X{-sʼ} \~\ \X{-ásʼ}
	\begin{itemize}
	\item	verb stem \fm{–tlʼéekʼát} ‘thread stick through to stiffen’
		from noun \fm{tlʼéekʼát} ‘barbecue crosspiece’
		from unknown \fm{\rt{tlʼikʼ}}
		possibly related to \fm{\rt{tlʼiᴴn}} ‘tie hair, cloth’
		\newline
		\vbform{aawatlʼéekʼát}{pfv}[tr, conj?, ach?]{she/he/it threaded sticks through him/her/it to stiffen}
		\parencite[08/253]{leer:1973}
			\vbmorph{a-&μʷ-&wa-&\rt{tlʼikʼ}&-μμH&\gm{-át}}
				{\xx{3>3}&\xx{pfv}&\xx{stv}&\rt{thread.thru?}&\·\xx{var}&\·\xx{unkn}}
	\end{itemize}

\item[-átʼ]\label{m:-átʼ}
	possibly an allomorph of repetitive \X{-tʼ} with epenthetic (filler) vowel \fm{á};
	attested only in one verb derived from a noun,
		ultimately from an unknown root;
	sometimes instead attested as \X{-át};
	given the rarity of this form and its variation, ejective \fm{-átʼ}
		may be remodeled from \fm{-át} by analogy with \fm{-tʼ} 
		and the alternation between \X{-sʼ} \~\ \X{-ásʼ},
		suggesting that the \fm{-át} form is more conservative;
	see \X{-át} for more discussion
	\begin{itemize}
	\item	verb stem \fm{–tlʼéekátʼ} ‘thread stick through to stiffen’
		from noun \fm{tlʼéekátʼ} ‘barbecue crosspiece’
		from unknown \fm{\rt{tlʼik}}
		possibly related to \fm{\rt{tlʼiᴴn}} ‘tie hair, cloth’
		\newline
		\vbform{wutuwatlʼéekátʼ}{pfv}[tr, conj?, ach?]{we threaded sticks through it}
		\parencite[227.3206]{story-naish:1973}
			\vbmorph{wu-&tu-&wa-&\rt{tlʼik}&-μμH&\gm{-átʼ}}
				{\xx{pfv}&\xx{1pl.s}&\xx{stv}&\rt{thread.thru?}&\·\xx{var}&\·\xx{unkn}}
	\end{itemize}

\item[aw]\label{m:aw}
	≡ \fm{a-w-}
	combination of argument marking \X{a-}
		and perfective \X[w-pfv]{w-};
	compare \X{awu} ≡ \fm{a-wu-} and \X{aawa} ≡ \fm{a-μʷ-wa-}
	\begin{itemize}
	\item	\vbform{awsi.ée}{pfv}[tr, \fm{∅}, \fm{-μμH} act]{s/he/it cooked him/her/it}
			\vbmorph{\gm{a-}&\gm{w-}&s-&i-&\rt[¹]{.i}&-μμH}
				{\xx{3>3}&\xx{pfv}&\xx{csv}&\xx{stv}&\rt[¹]{cooked}&\·\xx{var}}
	\end{itemize}

\item[awu]\label{m:awu}
	≡ \fm{a-wu-}
	combination of argument marking \X{a-}
		and perfective \X{wu-};
	occurs where stative \fm{ÿa-} \~\ \fm{i-} is suppressed
		such as in negative, past tense, and subordinate clause forms;
	compare \X{aawa} ≡ \fm{a-μʷ-wa-} and \X{aw} ≡ \fm{a-w-}
	\begin{itemize}
	\item	\vbform{tléil awux̱á}{neg pfv}[tr, \fm{∅}, \fm{-μH} act]{she/he/it didn’t eat him/her/it}
			\vbmorph{tléil&\gm{a-}&\gm{wu-}&\rt[²]{x̱a}&-μH}
				{\xx{neg}&\xx{3>3}&\xx{pfv}&\rt[²]{eat}&\·\xx{var}}
		\versus \vbform{aawax̱áa}{pfv}{she/he/it ate him/her/it}
			\vbmorph{a-&μʷ-&wa-&\rt[²]{x̱a}&-μμH}
				{\xx{3>3}&\xx{pfv}&\xx{stv}&\rt[²]{eat}&\·\xx{var}}
	\item	\vbform{awux̱áayin}{past pfv}{she/he/it had eaten him/her/it}
			\vbmorph{\gm{a-}&\gm{wu-}&\rt[²]{x̱a}&-μμH&-yin}
				{\xx{3>3}&\xx{pfv}&\rt[²]{eat}&\·\xx{var}&\·\xx{past}}
	\item	\vbform{awux̱áayi}{sub pfv}{while/when she/he/it had eaten him/her/it}
			\vbmorph{\gm{a-}&\gm{wu-}&\rt[²]{x̱a}&-μμH&-yi}
				{\xx{3>3}&\xx{pfv}&\rt[²]{eat}&\·\xx{var}&\·\xx{sub}}
	\end{itemize}

\item[ax̱]\label{m:ax̱}
	≡ \fm{a-x̱-}
	combination of argument marking \X{a-}
		and either first person singular subject \X[x̱-1sg]{x̱-}
			or \fm{g̱} conjugation \X[x̱-g̱cnj]{x̱-}
			or modal \X[x̱-mod]{x̱-}

\item[ax̱=]\label{m:ax̱=}
	allomorph ‘my’ of \fm{x̱at=} ‘me’ first person singular object,
		only used as possessor of incorporated nouns;
	derived from possessive pronoun \fm{ax̱} ‘my’ (compare \fm{ax̱ keidlí áwé} ‘it is my dog’);
	some speakers disprefer \fm{ax̱=} in verbs and only use \fm{x̱at=}
	\begin{itemize}
	\item	\vbform{ax̱ shalxáash}{impfv}[tr, \fm{n}, \fm{-μμH} act]{she/he/it is cutting my hair}
			\vbmorph{\gm{ax̱=}&sha-&l-&\rt[²]{xash}&-μμH}
				{\xx{1sg.o}&head&\xx{xtn}&\rt[²]{cut}&\·\xx{var}}
		\versus 	\vbform{x̱at shalxáash}{impfv}{she/he/it is cutting my hair}
			\vbmorph{x̱at=&sha-&l-&\rt[²]{xash}&-μμH}
				{\xx{1sg.o}&head&\xx{xtn}&\rt[²]{cut}&\·\xx{var}}
	\end{itemize}

\item[aÿ]\label{m:aÿ-a-ÿ}
	≡ \fm{a-ÿ-}
	combination of argument marking \X{a-}
		and second person plural subject \X[ÿ-2pl]{ÿ-};
	distinct from \X[aÿ-a-ʷ-ÿ]{aÿ} ≡ \fm{a-ʷ-ÿ-}
		which has perfective \X[ʷ-pfv]{ʷ-}
		and second person \emph{singular} subject \X[ÿ-2sg]{ÿ-}

\item[aÿ]\label{m:aÿ-a-ʷ-ÿ}
	≡ \fm{a-ʷ-ÿ-}
	combination of argument marking \X{a-}
		and perfective \X[ʷ-pfv]{ʷ-}
		and second person singular subject \X[ÿ-2sg]{ÿ-};
	distinct from \X[aÿ-a-ÿ]{aÿ} ≡ \fm{a-ÿ-}
		which has second person \emph{plural} subject \X[ÿ-2pl]{ÿ-};
\end{morphdesc}

\subsection{C}\label{sec:alphalist-c}

\begin{morphdesc}[resume*=alphalist]
\item[-ch]\label{m:-ch}
	repetitive suffix

\item[-chʼ]\label{m:-chʼ}
	suffix with unknown meaning;
	discussed by \textcite[56]{story:1966},
		appears to be derivational (present in all forms)
		and not inflectional (present only in some forms);
	may be part of unknown \X{-chʼáḵw} with deprivative \X{-áḵw} ‘lacking’,
		part of unknown \X{-chʼálʼ} with repetitive \X{-álʼ},
		and part of intensifier \X{-chʼán} with restorative \X{-án};
	otherwise documented only in one verb as \X{-áchʼ} (which see);
	because the meaning of \fm{-chʼ} \~\ \fm{-áchʼ} is unknown it is glossed as \xx{unkn}
	\newline
	allomorphs:
	\begin{allolist}
	\item[-chʼ]	single consonant form
	\item[\X{-áchʼ}]	form with epenthetic (filler) vowel \fm{á}
	\item[\X{-sh}]	form after ejective consonant in \X{-chʼán} \~\ \X{-shán}
	\end{allolist}
	\begin{enumerate}
	\item	in \fm{g̱eeg̱áchʼ} \~\ \fm{g̱eig̱áchʼ} ‘hammock, swing for baby’
		and related forms
		see \X{-áchʼ}
	\item	in \fm{kawdudliséewchʼáḵw} ‘it (berry) is full of rain (and so tasteless)’
		see \X{-chʼáḵw}
	\item	in \fm{ḵéichʼálʼ} \~\ \fm{ḵéechʼálʼ} ‘seam’
		see \X{-chʼálʼ}
	\item	in \fm{kuli.áax̱chʼán} ‘she/he/it is fascinating, interesting to hear’
		and related forms
		see \X{-chʼán}
	\item	possibly identifiable as the final consonant in some CVC verb roots:
		\begin{inlinelist}
		\item	\fm{\rt{.achʼ}} ‘insufficient’
		\item	\fm{\rt{chʼachʼ}} ‘spotted, polka-dotted’
			(compare \fm{\rt{chʼalʼ}} ‘pale, spotted’, \fm{chʼáalʼ} ‘willow’,
			\fm{tlʼáatlʼ} ‘yellow salmonberry’,
			also perhaps \fm{cháasʼ} ‘humpy or pink salmon’)
		\item	\fm{\rt{duchʼ}} ‘tweak, pinch and twist’
			(compare \fm{\rt{dutlʼ}} ‘roll up’ and \fm{\rt{tuchʼ}} ‘rub with hand’)
		\item	\fm{\rt{duchʼ}} ‘cut into chunks’
			(also noun \fm{g̱áaxʼw kadóochʼi} ‘herring eggs crumbled into chunks’)
		\item	\fm{\rt{g̱wachʼ}} \~\ \fm{\rt{g̱uchʼ}} ‘wrap in blanket’
		\item	\fm{\rt{hachʼ}} ‘shameful’
			(compare \fm{\rt{hasʼ}} ‘vomit’ and \fm{\rt{hatlʼ}} ‘crap’)
		\item	\fm{\rt{kuchʼ}} ‘curly’
			(also noun \fm{kakóochʼi} ‘fiddlehead (of fern)’)
		\item	\fm{\rt{kuchʼ}} ‘fart noiselessly’
			(also noun \fm{kóochʼ} ‘noiseless fart’)
		\item	\fm{\rt{ḵichʼ}} ‘watch covertly, spy’
		\item	\fm{\rt{ḵʼichʼ}} ‘scabby’
			(also noun \fm{ḵʼéechʼ} ‘scar with scab’)
		\item	\fm{\rt{tuchʼ}} ‘rub with hand’
			(compare \fm{\rt{duchʼ}} ‘tweak’ and \fm{\rt{dutlʼ}} ‘roll up’)
		\item	\fm{\rt{tʼuchʼ}} ‘char’
			(also noun \fm{tʼoochʼ} ‘charcoal’,
			\fm{waḵlitaaktʼoochʼí} ‘iris of eye’,
			compare \fm{Duktʼootlʼ} ‘Black Skin’)
		\item	\fm{\rt{wuchʼ}} ‘murky’
			(compare \fm{\rt{wusʼ}} ‘murky’)
		\item	\fm{\rt{wuchʼ}} ‘stubborn’
			(compare \fm{\rt{wutlʼ}} ‘stubborn’ and \fm{\rt{wusʼ}} ‘tough’)
		\item	\fm{\rt{ÿachʼ}} ‘too short’
			(compare \fm{\rt{ÿatʼ}} ‘long’ and \fm{\rt{ÿatlʼ}} ‘short’,
			also in noun \fm{sʼuḵkulayáachʼi} ‘lowest, shortest rib’)
		\end{inlinelist}
	\item	possibly identifiable as the final consonant in some CVCC nouns:
		\begin{inlinelist}
		\item	\fm{ḵʼánchʼ} ‘unidentified species of seaweed’
			(compare \fm{ḵʼaan} ‘dolphin’ and \fm{néiḵʼán} ‘stone weir’)
		\item	\fm{tláxchʼ} ‘dead branches (for tinder)’
			(compare \fm{\rt{tlax̱}} ‘mouldy’)
		\item	\fm{xíxchʼ} ‘frog’
			(compare \fm{\rt{xix}} ‘fall, move through space’)
		\item	\fm{yáxwchʼ} \~\ \fm{yúxchʼ} ‘sea otter’
			(compare \fm{\rt{yux̱ʼ}} ‘waterlogged’)
		\end{inlinelist}
	\item	possibly identifiable as the final consonant in some CVVC nouns:
		\begin{inlinelist}
		\item	\fm{éechʼ} ‘boulder’
		\item	\fm{iḵnáachʼ} \~\ \fm{eḵnáachʼ} ‘brass’
			(\fm{eeḵ} \~\ \fm{eiḵ} ‘copper’ with unidentified \fm{náachʼ},
			compare \fm{\rt{natlʼ}} \~\ \fm{\rt{natsʼ}}
				‘pruned, wrinkled from water; clumsy’)
		\item	\fm{ḵáachʼ} ‘red seaweed’
		\item	\fm{kʼóochʼ} ‘rear projection (snowshoe, spear)’
			(compare \fm{kʼóolʼ} ‘tailbone’)
		\item	\fm{ḵʼéechʼ} ‘scar with scab’
			(also verb \fm{\rt{ḵʼichʼ}} ‘scabby’)
		\item	\fm{lakʼéechʼ} ‘nape of neck’
			(\fm{la-} \~\ \fm{le-} ‘neck’ with unidentified \fm{kʼéechʼ})
		\item	\fm{lugóochʼ} ‘lobe of nostril, nasal alae’
			(\fm{lú} ‘nose’ with unidentified \fm{góochʼ},
			compare \fm{gangóosh} ‘two-eared headdress’)
		\item	\fm{lugwáachʼ} ‘rhinocerous auklet’
			(\fm{lú} ‘nose’ with unidentified \fm{gwáachʼ},
			compare \fm{\rt{gwatlʼ}} ‘fold, bend’,
			\fm{\rt{gwalʼ}} ‘curl’,
			and \fm{\rt{gwaᴴsh}} ‘hop on one leg’)
		\item	\fm{nóochʼi} ‘fish heart; gill bone’
		\item	\fm{sháachʼ} ‘smelt, sardine, young herring’
		\item	\fm{sháachʼ} ‘unidentified lichen; licorice’
			(also in \fm{ashayakiksháachʼi} ‘unidentified lichen’,
			\fm{asxʼaansháachʼi} ‘unidentified bird species’)
		\item	\fm{tʼoochʼ} ‘charcoal’
			(also in \fm{waḵlitaaktʼoochʼí} ‘iris of eye’,
			verb \fm{\rt{tʼuchʼ}} ‘char’,
			compare \fm{Duktʼootlʼ} ‘Black Skin’)
		\item	\fm{yéichʼ} ‘canoe carving depth peg’
		\end{inlinelist}
	\end{enumerate}

\item[-chʼáḵw]\label{m:-chʼáḵw}
	suffix with unknown meaning, but possibly a combination of unknown \X{-chʼ}
		and deprivative \X{-áḵw} ‘lacking’;
	attested only in one verb derived from the noun \fm{séew} ‘rain’
	\begin{itemize}
	\item	\vbform{kawdudliséewchʼáḵw}{pfv}[obj intr, conj?, ach?]{it (berry) is full of rain (and so tasteless)}
		\parencite[56]{story:1966}
			\vbmorph{ka-&w-&du-&d-&l-&i-&\rt{siw}&-μμH&\gm{-chʼ}&\gm{-áḵw}}
				{\xx{sro}&\xx{pfv}&\xx{xpl}&\xx{mid}&\xx{intr}&\xx{stv}&\rt{rain}&\·\xx{var}&\·\xx{unkn}&\·\xx{dprv}}
	\end{itemize}

\item[-chʼálʼ]\label{m:-chʼálʼ}
	suffix with unknown meaning, but possibly a combination of unknown \X{-chʼ}
		and repetitive \X{-álʼ};
	attested only in one noun derived from the root \fm{\rt[²]{ḵa}} ‘stitch, sew’;
	compare the stem \fm{–ḵéilʼútʼ} ‘lick seam’ with \X{-lʼútʼ}
	\begin{itemize}
	\item	\fm{ḵéichʼálʼ} \~\ \fm{ḵéechʼálʼ} ‘seam’
		\vbmorph*{\rt{ḵa}&-μᵉμH&\gm{-chʼ}&\gm{-álʼ}}
			{\rt{stitch}&\·\xx{var}&\·\xx{unkn}&\·\xx{rep}}
		\newline
		the form \fm{ḵéichʼálʼ} has \X{-μᵉμH} stem variation
			with ablaut of /\ipa{a}/ → [\ipa{e}] as is normal with a consonant suffix
			after a /\ipa{Ca}/ or /\ipa{Cu}/ root,
		but the form \fm{ḵéechʼálʼ} is unexpected and suggests reanalysis
			with interdialectal reversal of uvular lowering
	\end{itemize}

\item[-chʼán]\label{m:-chʼán}
	intensifier suffix in verbs with the ‘extraordinary state’ derivation
		made up of:
		qualifier \X[ka-qual]{ka-}
		+ irrealis \X[u-irr]{u-}
		+ extensional \X{s-}/\X{lˢ-}
		+ state \X[i-stv]{i-}
		+ intensifier \X{-chʼán} \~\ \fm{-shán}
		with \fm{g} conjugation class
		\parencite[655]{crippen:2019};
	discussed by \textcite[56]{story:1966}
		who shows alternation between basic \fm{-chʼán}
		versus \X{-shán} after an ejective consonant as a kind of dissimilation
		\parencite[878]{crippen:2019};
	the exact meaning is unclear but this suffix is associated with experience of
		an intense “pleasurable or fearful reaction” \parencite[56]{story:1966}
		to a situation, thus the label ‘intensifier’;
	\fm{-chʼán} is possibly formed by combination of
		unknown \X{-chʼ} and restorative \X{-án}
		and \fm{-shán} by unknown \X{-sh} and restorative \X{-án},
		but the composition of meaning is unclear;
	if treated as a single suffix \fm{-chʼán} \~\ \fm{-shán}
		it is glossed as \xx{intns},
		otherwise \fm{-chʼ} \xx{unkn} or \fm{-sh} \xx{unkn}
		+ \fm{-án} \xx{rest};
	\fm{-chʼán} is specifically attested with the roots
		\begin{inlinelist}
		\item	\fm{\rt{.ax̱}} ‘hear’
		\item	\fm{\rt{jaḵw}} ‘beat up’
		\item	\fm{–néisʼ} ‘dampen, oil’ (\fm{\rt{na}} + \X{-μᵉμH} + \X{-sʼ})
		\item	\fm{\rt{nik}} ‘tell’
		\item	\fm{\rt{nitl}} \~\ \fm{\rt{netl}} ‘fat (human)’
		\item	\fm{–núkts} ‘sweet, delicious’ (\fm{\rt{nuk}} + \X{-μH} + \X{-ts} with loss of \fm{-ts})
		\item	\fm{\rt{tul}} ‘spin, drill’
		\item	\fm{–tʼáaÿ} ‘hot’ (\fm{\rt{tʼaᴸ}} + \X{-μμH} + \X{-ÿ})
		\item	\fm{\rt{.ush}} ‘pout’
		\item	\fm{\rt{.uᴴw}} ‘buy’
		\item	\fm{\rt{was}} ‘roast’
		\item	\fm{\rt{wash}} ‘yawn’
		\end{inlinelist}
	but see \X{-shán} for more examples
	\newline
	allomorphs:
	\begin{allolist}
	\item[-chʼán]	basic form
	\item[\X{-shán}]	form used after an ejective consonant
	\end{allolist}
	\begin{itemize}
	\item	\vbform{kuli.áax̱chʼán}{impfv}[obj intr, \fm{g}, inv state]{she/he/it is fascinating, interesting to hear}
		\parencites[02/172]{leer:1973}[120]{leer:1976}
			\vbmorph{ka-&ʷ-&l-&i-&\rt{.ax̱}&-μμH&\gm{-chʼán}}
				{\xx{qual}&\xx{irr}&\xx{xtn}&\xx{stv}&\rt{hear}&\·\xx{var}&\·\xx{intns}}
		\versus \vbform{aawa.áx̱}{pfv}[tr, \fm{∅}, ach]{she/he/it heard him/her/it}
			\vbmorph{a-&μʷ-&wa-&\rt[²]{.ax̱}&-μH}
				{\xx{3>3}&\xx{pfv}&\xx{stv}&\rt[²]{hear}&\·\xx{var}}
	\item	\vbform{alijáaḵwchʼán}{impfv}[tr, \fm{g}?, inv state]{she/he/it likes to beat up, fight with him/her/it}
		\parencite[132.223]{dauenhauer-dauenhauer:1987}
			\vbmorph{a-&lˢ-&i-&\rt[²]{jaḵw}&-μμH&\gm{-chʼán}}
				{\xx{3>3}&\xx{xtn}&\xx{stv}&\rt[²]{beat.up}&\·\xx{var}&\xx{intns}}
		\versus \vbform{ajáaḵw}{impfv}[tr, \fm{n}, \fm{-μμH} act]{she/he/it beats up him/her/it}
			\vbmorph{a-&\rt[²]{jaḵw}&-μμH}
				{\xx{3>3}&\rt[²]{beat.up}&\·\xx{var}}
	\item	\vbform{x̱ʼalinóokchʼán}{impfv}[obj intr, \fm{g}?, inv state]{she/he/it looks delicious}
		\parencite[04/219]{leer:1973}
			\vbmorph{x̱ʼe-&l-&i-&\rt[⁰]{nuk}&-μμH&\gm{-chʼán}}
				{mouth&\xx{intr}&\xx{stv}&\rt{sweet}&\·\xx{var}&\·\xx{intns}}
		\versus \vbform{linúkts}{impfv}[obj intr, \fm{g}, inv state]{she/he/it is sweet (tasting)}
			\vbmorph{lˢ-&i-&\rt[⁰]{nuk}&-μH&-ts}
				{\xx{intr}&\xx{stv}&\rt[⁰]{sweet}&\·\xx{var}&\·\xx{unkn}}
	\end{itemize}
	the \fm{-chʼán} suffix is also attested in one noun:
	\begin{itemize}
	\item	\fm{toolchʼán} ‘top, spinning toy’
		from \fm{\rt[²]{tul}} ‘spin, drill’
		\vbmorph{\rt[²]{tul}&-μμL&-chʼ&-án}
			{\rt[²]{drill}&\·\xx{var}&\·\xx{unkn}&\·\xx{rest}}
	\end{itemize}

\item[chush=]\label{m:chush=}
	allomorph of reflexive object \X{sh=}

\end{morphdesc}

\subsection{D}\label{sec:alphalist-d}
\begin{morphdesc}[resume*=alphalist]
\item[d-]\label{m:d-}
	voice prefix, traditionally analyzed as part of the classifier;
	suppresses an argument (passive, antipassive)
	or reduces the scope of its reference (middle);
	\newline
	allomorphs:
	\begin{allolist}
	\item[d-]	basic form
	\item[\X{da-}]	with epenthetic (filler) vowel \fm{a}
			(no \fm{s-}/\fm{l-}/\fm{lˢ-}/\fm{sh-} and no \fm{i-})
	\end{allolist}
	combinations:
	\begin{allolist}
	\item[\X{di}]	≡ \fm{d-i-} with stative \X[i-stv]{i-}
	\item[\X{dli}]	≡ \fm{d-l-i-} with valency \X{l-}/\X{lˢ-} and stative \X[i-stv]{i-}
	\item[\X{dzi}]	≡ \fm{d-s-i-} with valency \X{s-} and stative \X[i-stv]{i-}
	\item[\X{ji}]	≡ \fm{d-sh-i-} with valency \X{sh-} and stative \X[i-stv]{i-}
	\item[\X{…l}]	≡ \fm{d-s-} with valency \X{l-}/\X{lˢ-} (no \fm{i-})
	\item[\X{…s}]	≡ \fm{d-s-} with valency \X{s-} (no \fm{i-})
	\item[\X{…sh}]	≡ \fm{d-sh-} with valency \X{sh-} (no \fm{i-})
	\end{allolist}
	the \fm{…s} / \fm{…l} / \fm{…sh} forms must be preceded by a vowel,
		with epenthetic (filler) \fm{i} inserted if no preceding prefixes provide a vowel,
		see those entries for examples
%		(the hypothetical \fm[*]{dza} / \fm[*]{dla} / \fm[*]{ja}
%			and \fm[*]{…dz} / \fm[*]{…dl} / \fm[*]{…j} do not occur
%			and are instead \fm{…s} / \fm{…l} / \fm{…sh})
	\begin{enumerate}
	\item	middle voice
	\item	passive voice
	\item	antipassive voice
	\end{enumerate}

\item[da-]\label{m:da-}
	allomorph of voice prefix \X{d-} with epenthetic (filler) vowel;
	occurs only whenever there is no stative \X[i-stv]{i-}
		and no valency \X{s-}, \X{l-}/\X{lˢ-}, or \X{sh-}
	\begin{itemize}
	\item	\vbform{tléil yan sh wudax̱eech}{neg pfv}[tr, \fm{∅}, mot]{he did not throw himself down}
			\vbmorph{tléil&yan=&sh=&wu-&\gm{da-}&\rt[²]{x̱ich}&-μμL}
				{\xx{neg}&ground&\xx{rflx.o}&\xx{pfv}&\xx{mid}&\rt[²]{throw}&\·\xx{var}}
		\versus \vbform{yan sh wudix̱ích}{pfv}{he threw himself down}
		\parencite[227.3217]{story-naish:1973}
			\vbmorph{yan=&sh=&wu-&d-&i-&\rt[²]{x̱ich}&-μH}
				{ground&\xx{rflx.o}&\xx{pfv}&\xx{mid}&\xx{stv}&\rt[²]{throw}&\·\xx{var}}
	\end{itemize}

\item[daa-]
	inalienable incorporated noun \fm{daa} ‘around, about, surrounding’

\item[daak=]
	directional preverb ‘out to sea (away from land)’;
	derived from directional noun \fm{dáak} ‘out at sea’
		(compare \fm{dákde=})

\item[dáag̱i=]
	directional noun ‘inland (away from water body)’
		with special locative postposition \fm{-í} \~\ \fm{-i} ‘at’;
	derived from directional noun \fm{dáaḵ} ‘inland’
		(compare \fm{daaḵ=}, \fm{dáḵde=}; noun \fm{daḵká} ‘on inland’)

\item[daaḵ=]
	directional preverb ‘inland (away from water body)’;
	derived from directional noun \fm{dáaḵ} ‘inland’
		(compare \fm{dáag̱i=}, \fm{dáḵde=}; noun \fm{daḵká} ‘on inland’)

\item[daa.it-]
	inalienable incorporated noun \fm{daa.ít} ‘joint’;
	possibly developed from a combination of \fm{daa} ‘around’ and \fm{ít} ‘following’
		although compositional meaning is unclear

\item[dag̱a-]
	allomorph of distributive or non-human pluralizer \fm{dax̱=};
	position of this allomorph is uncertain as it is only attested in forms without
	argument or aspectual prefixes

\item[dákde=]
	directional preverb ‘out to sea (away from land)’
		with allative postposition \fm{-dé} \~\ \fm{-de} ‘toward’
	derived from directional noun \fm{dáak} ‘out at sea’
		(compare \fm{daak=})
\item[dáḵde=]
	directional noun \fm{dáaḵ} ‘inland (away from water body)’
		with allative postposition \fm{-dé} \~\ \fm{-de} ‘toward’
	derived from directional noun \fm{dáaḵ} ‘inland’
		(compare \fm{dáag̱i=}, \fm{daaḵ=}; noun \fm{daḵká} ‘on inland’)

\item[dax̱=]
	distributive pluralizer or non-human pluralizer;
	can occur before human pluralizer \fm{has=} but not after

\item[deik=]
	variant form of preverb \fm{daak=} ‘out to sea’
		used in Southern and Transitional Northern communities;
	the reason for using \fm{deik=} versus \fm{daak=} is still unclear;
	compare similar \fm{deiḵ=} versus \fm{daaḵ=}
	\begin{itemize}
	\item	\vbform{deik ḵoowatín}{pfv}[subj intr, \fm{∅}, ach+mot]{he has gotten vision}
		(Southern dialect) \parencite[06/212]{leer:1973}
			\vbmorph{\gm{deik=}&ḵu-&μʷ-&wa-&\rt[²]{tin}&-μH}
				{out&\xx{areal}&\xx{pfv}&\xx{stv}&\rt[²]{see}&\·\xx{var}}
		\versus \vbform{daak ḵoowatín}{pfv}{he has gotten vision} (Northern dialect)
			\vbmorph{daak=&ḵu-&μʷ-&wa-&\rt[²]{tin}&-μH}
				{out&\xx{areal}&\xx{pfv}&\xx{stv}&\rt[²]{see}&\·\xx{var}}
	\end{itemize}

\item[deiḵ=]
	variant form of preverb \fm{daaḵ=} ‘inland’ 
		used in Southern and Transitional Northern communities;
	the reason for using \fm{deiḵ=} versus \fm{daaḵ=} is still unclear;
	compare similar \fm{deik=} versus \fm{daak=}
	\begin{itemize}
	\item	\vbform{i chkáx̱ deiḵ tí}{imp}[tr, \fm{∅}, mot]{put it (glove) on your hand}
		(Southern dialect) \parencite[05/79]{leer:1973}
			\vbmorph{i&ji-&ká&-x̱&\gm{deiḵ=}&\rt[²]{ti}&-μH}
				{\xx{2sg.psr}&hand&\xx{hsfc}&\·\xx{pert}&on&\rt[²]{handle}&\·\xx{var}}
		\versus \vbform{i jikáx̱ daaḵ tí}{imp}{put it (glove) on your hand} (Northern dialect)
			\vbmorph{i&ji-&ká&-x̱&daaḵ=&\rt[²]{ti}&-μH}
				{\xx{2sg.psr}&hand&\xx{hsfc}&\·\xx{pert}&on&\rt[²]{handle}&\·\xx{var}}
	\end{itemize}

\item[di]\label{m:di}
	≡ \fm{d-i-}
	combination of voice \X{d-}
		and stative \X[i-stv]{i-}
	\begin{itemize}
	\item	\vbform{sh tuditéen}{impfv}[tr, \fm{∅}, \fm{-μμH} state]{we can see ourselves}
			\vbmorph{sh=&tu-&\gm{d-}&\gm{i-}&\rt[²]{tin}&-μμH}
				{\xx{rflx.o}&\xx{1pl.s}&\xx{mid}&\xx{stv}&\rt[²]{see}&\·\xx{var}}
		\versus \vbform{tuwatéen}{impfv}{we can see him/her/it}
			\vbmorph{tu-&wa-&\rt[²]{tin}&-μμH}
				{\xx{1pl.s}&\xx{stv}&\rt[²]{see}&\·\xx{var}}
	\end{itemize}

\item[dli]\label{m:dli}
	≡ \fm{d-l-i-}
	combination of voice  \X{d-},
		valency \X{l-}/\X{lˢ-},
		and stative \X[i-stv]{i-}
	\begin{itemize}
	\item	\vbform{sh wutudlitlʼíx}{pfv}[tr, \fm{∅}, ach]{we made ourselves dirty}
			\vbmorph{sh=&wu-&tu-&\gm{d-}&\gm{l-}&\gm{i-}&\rt[¹]{tlʼix}&-μH}
				{\xx{rflx.o}&\xx{pfv}&\xx{1pl.s}&\xx{mid}&\xx{csv}&\xx{stv}&\rt[¹]{dirt}&\·\xx{var}}
		\versus \vbform{wutulitlʼíx}{pfv}{we made him/her/it dirty}
			\vbmorph{wu-&tu-&l-&i-&\rt[¹]{tlʼix}&-μH}
				{\xx{pfv}&\xx{1pl.s}&\xx{csv}&\xx{stv}&\rt[¹]{dirt}&\·\xx{var}}
	\end{itemize}

\item[du-]\label{m:du-}
	\begin{enumerate}
	\item	indefinite human subject of transitive verbs;
		see \fm{ḵaa=} and \fm{ḵu-} for indefinite human object,
		and see \fm{a-} for indefinite human subject of subject intransitive verbs
		\begin{itemize}
		\item	\vbform{x̱at wuduwax̱oox̱}{pfv}[tr, \fm{g̱}, ach]{someone/people summoned me}
				\vbmorph{x̱at=&wu-&\gm{du-}&wa-&\rt[²]{x̱ux̱}&-μμL}
					{\xx{1sg.o}&\xx{pfv}&\xx{ind.h.s}&\xx{stv}&\rt[²]{summon}&\·\xx{var}}
			\versus \vbform{x̱at woox̱oox̱}{pfv}{she/he/it summoned me}
				\vbmorph{x̱at=&wu-&μ-&\rt[²]{x̱ux̱}&-μμL}
					{\xx{1sg.o}&\xx{pfv}&\xx{stv}&\rt[²]{summon}&\·\xx{var}}
		\item	\vbform{x̱at wududziteen}{pfv}[tr, \fm{g̱}, ach]{someone/people saw me}
				\vbmorph{x̱at=&wu-&\gm{du-}&d-&s-&i-&\rt[²]{tin}&-μμL}
					{\xx{1sg.o}&\xx{pfv}&\xx{ind.h.s}&\xx{mid}&\xx{xtn}&\xx{stv}&\rt[²]{see}&\·\xx{var}}
			\versus \vbform{x̱at wusiteen}{pfv}{she/he/it saw me}
				\vbmorph{x̱at=&wu-&s-&i-&\rt[²]{tin}&-μμL}
					{\xx{1sg.o}&\xx{pfv}&\xx{xtn}&\xx{stv}&\rt[²]{see}&\·\xx{var}}
		\end{itemize}
	\item	indefinite experiencer subject;
		essentially the same as the indefinite human subject but frozen in certain sets of
		verbs describing impersonal experiences (e.g.\ feel of wind, flavour of food);
		cannot be replaced by some other referent
		\begin{itemize}
		\item	\vbform{xóon wuduwanúk}{pfv}[obj intr, \fm{∅}, ach]{north wind was felt}
				\vbmorph{xóon&wu-&\gm{du-}&wa-&\rt[²]{nuk}&-μH}
					{n·wind& \xx{pfv}&\xx{ind.s}&\xx{stv}&\rt[²]{feel}&\·\xx{var}}
		\end{itemize}
	\item	expletive/filler subject;
		occurs in some verbs to fill the subject position without referring to anything;
		cannot be replaced by some other referent
		\begin{itemize}
		\item	\vbform{x̱at kawduwasáy}{pfv}[obj intr, \fm{∅}, ach]{I got hot/sweaty}
				\vbmorph{x̱at=&ka-&w-&\gm{du-}&wa-&\rt[¹]{saÿ}&-μH}
					{\xx{1sg.o}&\xx{qual}&\xx{pfv}&\xx{xpl}&\xx{stv}&\rt[¹]{radiate}&\·\xx{var}}
		\item	\vbform{haa kawduwakʼéin}{pfv}[obj intr, \fm{g}, mot]{we jumped}
				\vbmorph{haa=&ka-&w-&\gm{du-}&wa-&\rt[¹]{kʼeᴴn}&-μμH}
					{\xx{1pl.o}&\xx{qual}&\xx{pfv}&\xx{xpl}&\xx{stv}&\rt[¹]{jump}&\·\xx{var}}
			\versus \vbform{x̱wajikʼéin}{pfv}[subj intr, \fm{g}, mot]{I jumped}
				\vbmorph{ʷ-&x̱a-&d-&sh-&i-&\rt[¹]{kʼeᴴn}&-μμH}
					{\xx{pfv}&\xx{1sg.s}&\xx{mid}&\xx{pej}&\xx{stv}&\rt[¹]{jump}&\·\xx{var}}
		\end{itemize}
	\end{enumerate}

\item[duk-]
	inalienable incorporated noun \fm{dook} ‘skin’

\item[dzi]\label{m:dzi}
	≡ \fm{d-s-i-}
	combination of voice \X{d-},
		valency \X{s-},
		and stative \X[i-stv]{i-}
	\begin{itemize}
	\item	\vbform{sh wutudzi.ée}{pfv}[tr, \fm{∅}, \fm{-μμH} act]{we cooked ourselves}
			\vbmorph{sh=&wu-&tu-&\gm{d-}&\gm{s-}&\gm{i-}&\rt[¹]{.i}&-μμH}
				{\xx{rflx.o}&\xx{pfv}&\xx{1pl.s}&\xx{mid}&\xx{csv}&\xx{stv}&\rt[¹]{cooked}&\·\xx{var}}
		\versus \vbform{wutusi.ée}{pfv}{we cooked him/her/it}
			\vbmorph{wu-&tu-&s-&i-&\rt[¹]{.i}&-μμH}
				{\xx{pfv}&\xx{1pl.s}&\xx{csv}&\xx{stv}&\rt[¹]{cooked}&\·\xx{var}}
	\end{itemize}
\end{morphdesc}

\subsection{E}\label{sec:alphalist-e}
\begin{morphdesc}[resume*=alphalist]
\item[ee]\label{m:ee}
	≡ \fm{a-i-}
	combination of argument marking \X{a-}
		and second person singular subject \X[i-2sg]{i-}

\item[ee-]
	allomorph of second person singular subject \fm{i-}

\item[ee=]
	allomorph of second person singular object \fm{i-}

\item[éeg̱i=]
	directional noun ‘beach’ with special locative postposition \fm{-í} \~\ \fm{-i} ‘at’,
	variant form \fm{éig̱i=};
	derived from noun \fm{éeḵ} \~\ \fm{éiḵ} ‘beach’;
	compare \fm{ÿeeḵ=} \~\ \fm{ÿeiḵ=} \~\ \fm{eèḵ=}

\item[eèḵ=]
	variant form of directional preverb \fm{ÿeeḵ=} \~\ \fm{ÿeiḵ=} used in Tongass Tlingit

\item[eeÿa]\label{m:eeÿa-a-i-ÿa}
	≡ \fm{a-i-ÿa-}
	combination of argument marking \X{a-}
		and second person singular subject \X[i-2sg]{i-}
		and stative \X[ÿa-stv]{ÿa-};
	same form as \X[eeÿa-a-ʷ-i-ÿa]{eeÿa} ≡ \fm{a-ʷ-i-ÿa-}
		which has perfective \X[ʷ-pfv]{ʷ-}

\item[eeÿa]\label{m:eeÿa-a-ʷ-i-ÿa}
	≡ \fm{a-ʷ-i-ÿa-}
	combination of argument marking \X{a-}
		and perfective \X[ʷ-pfv]{ʷ-}
		and second person singular subject \X[i-2sg]{i-}
		and stative \X[ÿa-stv]{ÿa-};
	same form as \X[eeÿa-a-i-ÿa]{eeÿa} ≡ \fm{a-i-ÿa-}
		which does not have perfective \X[ʷ-pfv]{ʷ-}

\item[éig̱i=]
	variant form of directional noun \fm{éeg̱i=} ‘beach’ used in some Northern varieties;
	arises from uvular lowering of \fm{ée} to \fm{éi}
\end{morphdesc}

\subsection{G}\label{sec:alphalist-g}
\begin{morphdesc}[resume*=alphalist]
\item[g-]\label{m:g-conj}
	\fm{g} conjugation prefix, upward spatial orientation;
	prospective aspect prefix with irrealis \fm{w-} and modal \fm{g̱-};
	can occur together with comparative \fm{g-}

\item[g-]\label{m:g-cmpv}
	irregular allomorph of comparative \X[ka-cmpv]{ka-}
	\begin{itemize}
	\item	\vbform{chʼa yéi googéikʼ}{impfv}[obj intr, \fm{n}, \fm{-μμH} cmpv state]{just a little}
			\vbmorph{chʼa&yéi=&g-&u-&μ-&\rt[¹]{ge}&-μμH&-kʼ}
				{just&thus&\xx{cmpv}&\xx{irr}&\xx{stv}&\rt[¹]{big}&\·\xx{var}&\·\xx{dim}}
		\versus \vbform{yagéi}{impfv}[obj intr, \fm{g}, \fm{-μμH} state]{it is big}
			\vbmorph{ÿa-&\rt[¹]{ge}&-μμH}
				{\xx{stv}&\rt[¹]{big}&\·\xx{var}}
	\end{itemize}

\item[ga-]\label{m:ga-conj}
	allomorph of \fm{g} conjugation prefix \X[g-conj]{g-} with epenthetic (filler) vowel \fm{a}

\item[ga-]\label{m:ga-cmpv}
	irregular allomorph of comparative \X[ka-cmpv]{ka-}

\item[ga-]\label{m:ga-sben}
	self-benefactive prefix, occurs with transitive verbs and requires \fm{d-};
	unclear if a \fm{g-} allomorph is possible;
	predicted to cooccur with \fm{g-} conjugation prefix but not attested;
	unclear if cooccurrence with \fm{ga-} comparative is possible
	\begin{itemize}
	\item	\vbform{at gawtudzi.ée}{pfv}[tr, \fm{∅}, ach]{we cooked something for ourselves}
			\vbmorph{at=&\gm{ga-}&w-&tu-&\gm{d-}&s-&i-&\rt[¹]{.i}&-μμH}
				{\xx{ind.n.o}&\xx{sben}&\xx{pfv}&\xx{1pl.s}&\xx{mid}&\xx{csv}&\xx{stv}&\rt[¹]{cooked}&\·\xx{var}}
		\versus \vbform{at wutusi.ée}{pfv}{we cooked something}
			\vbmorph{at=&wu-&tu-&s-&i-&\rt[¹]{.i}&-μμH}
				{\xx{ind.n.o}&\xx{pfv}&\xx{1pl.s}&\xx{csv}&\xx{stv}&\rt[¹]{cooked}&\·\xx{var}}
	\end{itemize}

\item[gági=]
	directional preverb ‘emerging, out into the open’;
	derived from noun \fm{gáak} ‘protrusion’
		with special locative postposition \fm{-í} \~\ \fm{-i};
	occurs in motion derivation
		\motderiv{gági}{\fm{∅}, \fm{-x̱} rep}{emerging, out into the open}
	\begin{itemize}
	\item	\fm{gági uwaháa du waḵshayeexʼ chʼáakʼ ḵuyéik}
		‘it emerged before his eyes, the eagle spirit’
		\parencite[01/6]{leer:1973}
			\vbmorph{gági=&u-&wa-&\rt[¹]{haᴸ}&-μμH}
				{emerge=&\xx{zpfv}&\xx{stv}&\rt[¹]{appear}&\·\xx{var}}
	\end{itemize}

\item[gug̱a]
	≡ \fm{g-u-g̱a-}
	combination of conjugation \fm{g-},
		irrealis \fm{u-},
		and  modal \fm{g̱a-},
		together indicating prospective (‘future’) aspect;
	this form occurs when there is no subject prefix and no
		immediately preceding vowel from an incorporated noun, object prefix, preverb, etc.;
	\fm{kg̱wa} occurs instead if there is a preceding vowel;
	compare \fm{kuḵa} with first person singular subject \fm{x̱a-}
	\begin{itemize}
	\item	\vbform{at gug̱ax̱áa}{prosp}[tr, \fm{∅}, \fm{-μμH} act]{she/he/it will eat something}
			\vbmorph{at=&g-&u-&g̱a-&\rt[²]{x̱a}&-μμH}
				{\xx{ind.n.o}&\xx{gcnj}&\xx{irr}&\xx{mod}&\rt[²]{eat}&\·\xx{var}}
		\versus \vbform{akg̱wax̱áa}{prosp}{she/he/it will eat him/her/it}
			\vbmorph{a-&g-&u-&g̱a-&\rt[²]{x̱a}&-μμH}
				{\xx{3>3}&\xx{gcnj}&\xx{irr}&\xx{mod}&\rt[²]{eat}&\·\xx{var}}
	\end{itemize}
\end{morphdesc}

\subsection{G̱}\label{sec:alphalist-gh}
\begin{morphdesc}[resume*=alphalist]
\item[g̱-]\label{m:g̱-conj}
	\fm{g̱} conjugation prefix, downward spatial orientation;
	can occur together with modal \fm{g̱-}

\item[g̱a-]\label{m:g̱a-conj}
	allomorph of \fm{g̱} conjugation prefix \X[g̱-conj]{g̱-}

\item[g̱a-]\label{m:g̱a-mod}
	allomorph of modal \X[g̱-mod]{g̱-} with epenthetic (filler) vowel \fm{a}

\item[g̱-]\label{m:g̱-mod}
	modal prefix in prospective aspect,
		hortative mood,
		potential mood,
		and contingent mood;
	can occur together with \fm{g̱-} conjugation class prefix in hortative, potential, contingent;
	\newline
	allomorphs:
	\begin{allolist}
	\item[{\X[g̱a-mod]{g̱a-}}]
			form with epenthetic (filler) vowel \fm{a}
	\item[{\X[x̱-g̱cnj]{x̱-}}]
			form occurring as consonant in coda of a syllable
	\end{allolist}
	combinations:
	\begin{allolist}
	\item[ḵa]	≡ \fm{g̱-x̱a-} with first person singular subject \fm{x̱a-}
	\end{allolist}
	\begin{enumerate}
	\item	prospective aspect:
			conjugation \fm{g-}
			+ irrealis \fm{u-}
			+ modal \fm{g̱-}
		\begin{itemize}
		\item	\vbform{gug̱atáa}{prosp}[subj intr, \fm{n}, \fm{-μH} act]{she/he/it will sleep}
			\vbmorph{g-&u-&\gm{g̱a-}&\rt[¹]{taᴸ}&-μμH}
				{\xx{gcnj}&\xx{irr}&\xx{mod}&\rt[¹]{sleep.\xx{sg}}&\·\xx{var}}
			\versus \vbform{wootaa}{pfv}{she/he/it slept}
			\vbmorph{wu-&μ-&\rt[¹]{taᴸ}&-μμL}
				{\xx{pfv}&\xx{stv}&\rt[¹]{sleep.\xx{sg}}&\·\xx{var}}
		\end{itemize}
	\item	hortative mood:
			conjugation \fm{n-}/\fm{g̱-}/\fm{g-}
			+  modal \fm{g̱-}
		\begin{itemize}
		\item	\vbform{nag̱ataa}{hort}[subj intr, \fm{n}, \fm{-μH} act]{let him/her/it sleep}
			\vbmorph{na-&\gm{g̱a-}&\rt[¹]{taᴸ}&-μμL}
				{\xx{ncnj}&\xx{mod}&\rt[¹]{sleep.\xx{sg}}&\·\xx{var}}
		\end{itemize}
	\item	potential mood:
			irrealis \fm{u-}
			+ conjugation \fm{n-}/\fm{g̱-}/\fm{g-}
			+ modal \fm{g̱-}
	\item	contingent mood:
			conjugation \fm{n-}/\fm{g̱-}/\fm{g-}
			+ modal \fm{g̱-}
			(+ \fm{-n} + \fm{-ín})
	\end{enumerate}

\item[g̱unéi=]
	variant form of inceptive \fm{g̱unayéi} ‘starting, beginning’, arising from contraction;
	some speakers use only \fm{g̱unéi} with verbs and have \fm{g̱unayéi} only as a noun

\item[g̱ax̱=]
	incorporated noun ‘crying’, saturates object;
	derived from \fm{\rt[¹]{g̱ax̱}} ‘cry’
	\begin{itemize}
	\item	\vbform{gax̱satí}{impfv}[subj intr, \fm{g}, \fm{-μH} act]{they cry}
		\vbmorph{\gm{gax̱=}&sa-&\rt[¹]{tiᴸ}&-μH}
			{cry&\xx{appl}&\rt[¹]{be}&\·\xx{var}}
		\versus \vbform{g̱áax̱}{impfv}[subj intr, \fm{g}, \fm{-μμH} act]{she/he/it cries}
		\vbmorph{\rt[¹]{g̱ax̱}&-μμH}
			{\rt[¹]{cry}&\·\xx{var}}
	\item	\vbform{kei gax̱ gax̱yisatée}{prosp}[subj intr, \fm{g}, \fm{-μμH} act]{you pl.\ will cry}
		\parencite[60.683]{story-naish:1973}
		\vbmorph{kei=&\gm{gax̱=}&ga-&w̸-&x̱-&sa-&\rt[¹]{tiᴸ}&-μμH}
			{up=&cry&\xx{gcnj}&\xx{irr}&\xx{mod}&\xx{appl}&\rt[¹]{be}&\·\xx{var}}
	\end{itemize}

\item[g̱unayéi=]
	inceptive preverb indicating initiation of motion or other eventuality;
	variant form \fm{g̱unéi} arising from contraction;
	derived from the noun \fm{g̱unayéi} ‘elsewhere, different place’
		from \fm{g̱una} ‘different, other’
		and \fm{yé} \~\ \fm{yéi} ‘place, way’
		probably with \fm{-μ} allomorph of locative postposition \fm{-xʼ}
	\begin{enumerate}
	\item	initiation of motion;
		motion derivation
			\fm{g̱unayéi} \~\ \fm{g̱unéi} (\fm{∅}, \fm{-x̱} rep) ‘starting off, setting out’
		\begin{itemize}
		\item	\fm{g̱unayéi wutuwa.át} (pfv; subj intr, \fm{∅}, \fm{-x̱} rep) ‘we started off’\newline
			versus \fm{wutuwa.aat} (pfv; subj intr, \fm{n}, \fm{yoo=i-…-k} rep) ‘we went’
		\end{itemize}
	\item	initiation of other eventuality;
		eventuality derivation
			\fm{g̱unayéi} \~\ \fm{g̱unéi} (\fm{∅}, ach, \fm{-x̱} rep)
				‘beginning, starting, initiating’
		\begin{itemize}
		\item	\fm{g̱unayéi aawax̱áa} (pfv; tr, \fm{∅}, ach) ‘s/he/it started eating him/her/it’\newline
			versus \fm{aawax̱áa} (pfv; tr, \fm{∅}, \fm{-μH} act) ‘s/he/it ate him/her/it’
		\end{itemize}
	\end{enumerate}
\end{morphdesc}

\subsection{H}\label{sec:alphalist-h}
\begin{morphdesc}[resume*=alphalist]
\item[haa=]
	first person plural object
	(compare \fm{haa keidlí áwé} ‘it is our dog’);
	although this is homophonous with the possessive pronoun,
		\fm{haa=} as an object is not necessarily possessive
	\begin{itemize}
	\item	\fm{haa yisiteen} (pfv; tr, \fm{g̱}, ach) ‘you sg.\ saw us’\newline
		versus \fm{x̱at yisiteen} (pfv) ‘you sg.\ saw me’
	\end{itemize}

\item[has=]
	human pluralizer for third person;
	\newline
	allomorphs:
	\begin{allolist}
	\item[as=]	onset glottal stop instead of \fm{h} used in Southern and Tongass varieties
	\item[s=]	lone consonant, usually coda of a preceding syllable
	\end{allolist}
	note that since Tlingit is number neutral (nouns are not singular by default),
		a form without \fm{has=} may still refer to plural humans,
		i.e.\ \fm{has=} is not required for third person human plural arguments
	\begin{enumerate}
	\item	human pluralizer for third person subject
		\begin{itemize}
		\item	\fm{tʼá aawax̱áa} (pfv; tr, \fm{∅}, \fm{-μH} act) ‘s/he/it ate king salmon’\newline
			versus \fm{tʼá has aawax̱áa} (pfv) ‘they (humans) ate king salmon’
		\end{itemize}
	\item	human pluralizer for third person object
		\begin{itemize}
		\item	\fm{has tushikʼáan} (impfv; tr, \fm{g}, \fm{-μμH} state) ‘we hate them’\newline
			versus \fm{yee tushikʼáan} (impfv) ‘we hate you guys’
		\end{itemize}
	\item	human pluralizer for both third person subject and third person object;
		some speakers do not accept this use of \fm{has=}
			for both subject and object at the same time
		\begin{itemize}
		\item	\fm{has awsiteen} (pfv; tr, \fm{g}, ach) ‘they saw them’
			or ‘s/he/it saw them’ or ‘they saw him/her/it’
		\end{itemize}
	\end{enumerate}
\end{morphdesc}

\subsection{I}\label{sec:alphalist-i}
\begin{morphdesc}[resume*=alphalist]
\item[i-]\label{m:i-2sg}
	second person singular subject or object; long vowel allomorphs are \fm{ee-} and \fm{ee=};
	subject versus object is typically distinguished by position in the verb word but can sometimes be ambiguous
	\begin{enumerate}
	\item	second person singular subject
		\begin{itemize}
		\item	\vbform{x̱at iyatéen}{impfv}[tr, \fm{g}, \fm{-μμH} state, only impfv]{you sg.\ can see me}
				\vbmorph{x̱at=&i-&ya-&\rt[²]{tin}&-μμH}
					{\xx{1sg.o}&\xx{2sg.s}&\xx{stv}&\rt[²]{see}&\·\xx{var}}
			\versus \vbform{ayatéen}{impfv}{she/he/it can see him/her/it}
				\vbmorph{a-&ya-&\rt[²]{tin}&-μμH}
					{\xx{3>3}&\xx{stv}&\rt[²]{see}&\·\xx{var}}
		\end{itemize}
	\item	second person singular оbject
		\begin{itemize}
		\item	\vbform{ix̱aatéen}{impfv}[tr, \fm{g}, \fm{-μμH} state, only impfv]{I can see you sg.}
				\vbmorph{i-&x̱a-&μ-&\rt[²]{tin}&-μμH}
					{\xx{2sg.o}&\xx{1sg.s}&\xx{stv}&\rt[²]{see}&\·\xx{var}}
		\end{itemize}
	\item	ambiguous: either second person subject or object in certain imperfective verb forms;
		when one of two arguments is third person and thus not indicated by the verb,
		the \fm{i-} argument prefix is ambiguous between subject and object and must
		be distinguished by other means (third person subject or object phrase,
		discourse context, etc.);
		this ambiguity does not arise if qualifiers, incorporated nouns, or aspect prefixes
		are present because they occur between the subject and object prefixes and so
		distinguish subject and object
		\begin{itemize}
		\item	\vbform{iyatéen}{impfv}[tr, \fm{g}, \fm{-μμH} state, only impfv]{you sg.\ can see him/her/it}
				\vbmorph{i-&ya-&\rt[²]{tin}&-μμH}
					{\gm{\xx{2sg.s}}&\xx{stv}&\rt[²]{see}&\·\xx{var}}
			\versus \vbform{iyatéen}{impfv}{she/he/it can see you sg.}
				\vbmorph{i-&ya-&\rt[²]{tin}&-μμH}
					{\gm{\xx{2sg.o}}&\xx{stv}&\rt[²]{see}&\·\xx{var}}
		\end{itemize}
	\end{enumerate}

\item[i-]\label{m:i-stv}
	stative prefix of classifier;
	\newline
	allomorphs:
	\begin{allolist}
	\item[ÿa-]	full syllable
	\item[wa-]	labialized form of \fm{ÿa-}
	\item[μ-]	lengthening of preceding vowel
	\end{allolist}

\item[-iḵ]\label{m:-iḵ}
	allomorph of \X[-ḵ-dprv]{-ḵ} with labialization and epenthetic (filler) \fm{i}

\item[-íḵ]\label{m:-íḵ}
	allomorph of \X[-ḵ-dprv]{-ḵ} with labialization and epenthetic (filler) \fm{í}

\end{morphdesc}

\subsection{J}\label{sec:alphalist-j}
\begin{morphdesc}[resume*=alphalist]
\item[-ja]\label{m:-ja}
	allomorph of unknown suffix \X{-jaa} \~\ \X{-jáa};
	occurs between H tone syllable and \fm{-ÿi} so that \fm{-jaa} has a short vowel

\item[-já]\label{m:-já}
	allomorph of unknown suffix \X{-jaa} \~\ \X{-jáa};
	occurs between L tone syllable and \fm{-ÿi} so that \fm{-jáa} has a short vowel
	\begin{itemize}
	\item	\fm{shaa kʼeeljáyi} ‘windy storm of a mountain’
			\vbmorph*{\rt{kʼil}&-μμL&\gm{-ch}&\gm{-á}&-ÿi}
				{\rt{storm?}&\·\xx{var}&\·\xx{rep}&\·\xx{instr}&\·\xx{poss}}
		\versus \fm{kʼeeljáa} ‘windy storm; strong south wind’
			\vbmorph*{\rt{kʼil}&-μμL&\gm{-ch}&\gm{-áa}}
				{\rt{storm?}&\·\xx{var}&\·\xx{rep}&\·\xx{instr}}
	\end{itemize}

\item[-jaa]\label{m:-jaa}
	suffix with unknown meaning, 
		apparently a combination of repetitive \X{-ch}
		and instrument \X{-aa} \~\ \X{-áa},
		though the composition of meaning is unclear;
	as with \X{-aa} \~\ \X{-áa} this suffix has
		polar tone opposite the preceding syllable
		so \fm{-jaa} after an H tone syllable
		and \X{-jáa} after an L tone syllable,
	and becomes a short vowel when followed by \fm{-ÿi}
		so \X{-ja} + \fm{-ÿi}
			(not \fm[*]{-ja-ÿí})
		and \X{-já} + \fm{-ÿi};
	\newline
	allomorphs:
	\begin{allolist}
	\item[-jaa]	L tone form used after H tone syllable
	\item[\X{-jáa}]	H tone form used after L tone syllable
	\item[\X{-ja}]	short vowel L tone form when followed by \fm{-ÿi}
	\item[\X{-já}]	short vowel H tone form when followed by \fm{-ÿi}
	\end{allolist}

\item[-jáa]\label{m:-jáa}
	allomorph of unknown suffix \X{-jaa} with H tone,
		used after L tone syllable (polar tone);
	only one noun is attested with this suffix as shown below,
		with no corresponding verb root known
	\begin{itemize}
	\item	\fm{kʼeeljáa} ‘windy storm; strong south wind’
		\vbmorph*{\rt{kʼil}&-μμL&\gm{-ch}&\gm{-áa}}
			{\rt{storm?}&\·\xx{var}&\·\xx{rep}&\·\xx{instr}}
	\end{itemize}

\item[ji-]\label{m:ji-}
	incorporated noun indicating hand or possession;
	qualifier indicating object with extended projections (fingers);
	derived from relational nouns \fm{jín} ‘hand’ and \fm{jee} ‘possession’

\item[ji]\label{m:ji}
	≡ \fm{d-sh-i-}
	combination of voice \X{d-},
		valency \X{sh-},
		and stative \X[i-stv]{i-}
\end{morphdesc}

\subsection{K}\label{sec:alphalist-k}
\begin{morphdesc}[resume*=alphalist]
\item[ka-]\label{m:ka-hsfc}
	incorporated noun ‘horizontal surface’,
	derived from relational noun \fm{ká} ‘horizontal surface, flat top of’;
	can occur together with one of
		qualifier \fm{ka-} ‘small and round’
		or qualifier \fm{ka-} of unknown meaning
		or comparative \fm{ka-}

\item[ka-]\label{m:ka-sro}
	qualifier ‘small and round’;
	can occur together with one of
		incorporated noun \fm{ka-} ‘horizontal surface’
		or qualifier \fm{ka-} of unknown meaning
		or comparative \fm{ka-}

\item[ka-]\label{m:ka-qual}
	qualifier of unknown meaning;
	can occur together with one of
		qualifier \fm{ka-} ‘small and round’ 
		or incorporated noun \fm{ka-} ‘horizontal surface’
		or comparative \fm{ka-}

\item[ka-]\label{m:ka-cmpv}
	comparative prefix, used along with irrealis \fm{u-} \~\ \fm{oo-} \~\ \fm{w-};
	required in comparative forms of state verbs that denote dimensions (e.g.\ short, heavy);
	can occur together with one of
		qualifier \fm{ka-} ‘small and round’
		or incorporated noun \fm{ka-} ‘horizontal surface’
		or qualifier \fm{ka-} of unknown meaning
	\begin{itemize}
	\item	\vbform{ḵúdáx̱ koodáal}{impfv}[obj intr, \fm{n}, \fm{-μμH} cmpv state]{she/he/it is too heavy}
			\vbmorph{ḵúdáx̱&k-&u-&μ-&\rt[¹]{dal}&-μμH}
				{too.much&\xx{cmpv}&\xx{irr}&\xx{stv}&\rt[¹]{heavy}&\·\xx{var}}
		\versus \vbform{yadál}{impfv}[obj intr, \fm{n}, \fm{-μH} state]{she/he/it is heavy}
			\vbmorph{ÿa-&\rt[¹]{dal}&-μH}
				{\xx{stv}&\rt[¹]{heavy}&\·\xx{var}}
	\end{itemize}

\item[-k]\label{m:-k}
	repetitive suffix;
	allomorph \fm{-kw} with labialization

\item[-kʼ]\label{m:-kʼ}
	diminutive suffix;
	allomorph \fm{-kʼw} with labialization

\item[kaawa]
	≡ \fm{ka-μʷ-wa-}
	combination of any prefix of the form \fm{ka-},
		perfective \fm{μʷ-},
		and stative \fm{wa-}
	\begin{itemize}
	\item	\vbform{kaawagaan}{pfv}[obj intr, \fm{g̱}, ach]{she/he/it burned}
			\vbmorph{\gm{ka-}&\gm{μʷ-}&\gm{wa-}&\rt[¹]{gan}&-μμL}
				{\xx{hsfc}&\xx{pfv}&\xx{stv}&\rt[¹]{burn}&\·\xx{var}}
		\versus \vbform{woogaan}{pfv}[obj intr, \fm{g̱}, ach]{she/he/it burned}
			\vbmorph{wu-&μ-&\rt[¹]{gan}&-μμL}
				{\xx{pfv}&\xx{stv}&\rt[¹]{burn}&\·\xx{var}}
	\end{itemize}

\item[keey-]
	inalienable incorporated noun \fm{keey} ‘knee’
	\begin{itemize}
	\item	\vbform{yan x̱at keeyshakawdligásʼ}{pfv}[obj intr, \fm{∅}, mot]{I fell down and skidded on my knees}
		\parencite[193.2689]{story-naish:1973}
			\vbmorph{yan=&x̱at=&\gm{keey-}&sha-&ka-&w-&d-&l-&i-&\rt[¹]{gasʼ}&-μH}
				{ground&\xx{1sg.o}&knee&head&\xx{hsfc}&\xx{pfv}&\xx{mid}&\xx{xtn}&\xx{stv}&\rt[¹]{end·fall}&\·\xx{var}}
	\end{itemize}

\item[keeya]
	≡ \fm{ka-μʷ-i-ya-}
	combination of any prefix of the form \fm{ka-},
		perfective \fm{μʷ-},
		second person singular subject \fm{i-},
		and stative \fm{ÿa-}
	\begin{itemize}
	\item	\vbform{keeyayúk}{pfv}[tr, \fm{∅}, ach]{you sg.\ shook him/her/it}
			\vbmorph{\gm{ka-}&\gm{μʷ-}&\gm{i-}&\gm{ya-}&\rt[²]{yuᴴk}&-μH}
				{\xx{qual}&\xx{pfv}&\xx{2sg.s}&\xx{stv}&\rt[²]{shake}&\·\xx{var}}
		\versus \vbform{akaawayúk}{pfv}[tr, \fm{∅}, ach]{she/he/it shook him/her/it}
			\vbmorph{a-&ka-&μʷ-&wa-&\rt[²]{yuᴴk}&-μH}
				{\xx{3>3}&\xx{qual}&\xx{pfv}&\xx{stv}&\rt[²]{shake}&\·\xx{var}}
	\end{itemize}

\item[kḵwa]
	≡ \fm{g-u-g̱-x̱a-}
	combination of conjugation \fm{g-},
		irrealis \fm{u-},
		and modal \fm{g̱a-},
			together indicating prospective (‘future’) aspect,
		along with first person singular subject \fm{x̱-} / \fm{x̱a-};
	this form occurs when there is an
		immediately preceding vowel (incorporated noun, object prefix, preverb, etc.);
	\fm{kuḵa} occurs instead if there is no preceding vowel
	\begin{itemize}
	\item	\fm{yee kḵwax̱áa} (prosp) ‘I will eat you (pl.)’
			with \fm{ÿee=g-u-g̱-x̱a-}\newline
		versus \fm{at kuḵax̱áa} (prosp; \fm{∅}, \fm{-μμH} act) ‘I will eat something’
			with \fm{at=g-u-g̱-x̱a-}
	\end{itemize}

\item[-kt]\label{m:-kt}
	orthographic variant of \X{-kwt} without overt labialization;
	occurs after roots with /\ipa{u}/ where the orthography omits labialization
		because it predictably occurs on consonants after the labial vowel;
	actually a combination of repetitive \X{-kw} and repetitive \X{-t}
		that behaves like a distinct suffix;
	see \X{-kwt} for details
	\begin{itemize}
	\item	\fm{–húkt} ‘characteristically wades’
		from \fm{\rt[¹]{hu}} ‘wade’ in
		\newline
		\vbform{jidihúkt}{impfv}[subj intr?, conj?, state]{she/he/it characteristically wades}
		\parencites[01/173]{leer:1973}[63]{leer:1976}
			\vbmorph{ji-&d-&i-&\rt[¹]{hu}&-μH&-kw&-t}
				{hand&\xx{mid}&\xx{stv}&\rt[¹]{wade}&\·\xx{var}&\·\xx{rep}&\·\xx{ict}}
		\versus \vbform{át jeewdihoo}{pfv}[subj intr?, \fm{n}, mot]{she/he/it waded around there}
		\parencites[240.3409]{story-naish:1973}[01/173]{leer:1973}[63]{leer:1976}
			\vbmorph{á&-t&ji-&μw-&d-&i-&\rt[¹]{hu}&-μμL}
				{\xx{3n}&\·\xx{pnct}&hand&\xx{pfv}&\xx{mid}&\xx{stv}&\rt[¹]{wade}&\·\xx{var}}
	\end{itemize}

\item[kuḵa]
	≡ \fm{g-u-g̱-x̱a-}
	combination of conjugation \fm{g-},
		irrealis \fm{u-} prefix,
		and modal \fm{g̱a-},
			together indicating prospective (‘future’) aspect,
		along with first person singular subject \fm{x̱-} / \fm{x̱a-};
	this form occurs when there is no
		immediately preceding vowel (incorporated noun, object prefix, preverb, etc.);
	\fm{kḵwa} occurs instead if there is a preceding vowel
	\begin{itemize}
	\item	\fm{at kuḵax̱áa} (prosp; \fm{∅}, \fm{-μμH} act) ‘I will eat something’
			with \fm{at=g-u-g̱-x̱a-}\newline
		versus \fm{yee kḵwax̱áa} (prosp) ‘I will eat you (pl.)’
			with \fm{ÿee=g-u-g̱-x̱a-}
	\end{itemize}

\item[kuḵwa]
	variant of \fm{kuḵa}
	
\item[-kw]\label{m:-kw}
	allomorph of repetitive \X{-k} with labialization

\item[-kʼw]\label{m:-kʼw}
	allomorph of diminutive \X{-kʼ} with labialization

\item[-kwt]\label{m:-kwt}
	≡ \fm{-kw-t}
	combination of repetitive \X{-kw}
		and ictive repetitive \X{-t}
		but acts like a distinct suffix;
	occurs only with open syllable CV roots
		in multipositional repetitive state imperfectives
		and in tendency state imperfectives
		(\X{-k} \~\ \X{-kw} is used instead
		with closed syllable CVC roots)
	the compositional meaning of this combination is unclear,
		but both cases are relatively well attested;
	forms with \X{-kt} are based on \X{-kw} and not \X{-k},
		reflecting labialization not indicated in the orthography
		when immediately following a labial vowel /\ipa{u}/,
		so this is orthographic variation and not actual allomorphy
	\newline
	allomorphs:
	\begin{allolist}
	\item[\X{-kt}]	form without overt labialization
	\item[-kwt]	form with labialization
	\end{allolist}
	\begin{enumerate}
	\item	repetitive suffix combination as part of multipositional repetitive
		state imperfectives \parencite[540]{crippen:2019}, indicating
		entities distributed at multiple positions along some path in space
		\begin{itemize}
		\item	\fm{–dákwt} ‘bodies of water lie’
			from \fm{\rt[¹]{daᴸ}} ‘flow’ in
			\newline
			\vbform{áx̱ naadákwt}{impfv}[obj intr, \fm{n}, state]{they (bodies of water) lie here and there along it}
			\parencite[328]{leer:1991}
				\vbmorph{á&-x̱&na-&μ-&\rt[¹]{daᴸ}&-μH&-kw&\gm{-t}}
					{\xx{3n}&\·\xx{pert}&\xx{ncnj}&\xx{stv}&\rt[¹]{flow}&\·\xx{var}&\·\xx{rep}&\·\xx{ict}}
			\versus \vbform{át déin}{pos impfv}{it (body of water) flows, lies there}
			\parencites[05/2]{leer:1973}[313]{leer:1976}
				\vbmorph{á&-t&\rt[¹]{daᴸ}&-μμᵉH&-n}
					{\xx{3n}&\·\xx{pnct}&\rt[¹]{flow}&\·\xx{var}&\·\xx{nsfx}}
			\exand \vbform{héen naadaa}{impfv}[obj impfv, \fm{n}, \fm{-μμL} state]{the river flows}
			\parencites[94.1193]{story-naish:1973}[05/3]{leer:1973}[313]{leer:1976}
				\vbmorph{héen&na-&μ-&\rt[¹]{daᴸ}&-μμL}
					{river&\xx{ncnj}&\xx{stv}&\rt[¹]{flow}&\·\xx{var}}
			\exand \vbform{woodaa}{pfv}{it flowed}
			\parencites[94.1192]{story-naish:1973}[05/2–3]{leer:1973}[313]{leer:1976}
				\vbmorph{wu-&μ-&\rt[¹]{daᴸ}&-μμL}
					{\xx{pfv}&\xx{stv}&\rt[¹]{flow}&\·\xx{var}}
			\newline
			not related to \fm{dákwtasi} ‘fertilizer’ (from \fm{\rt{dakw}} ‘render’)
			or to \fm{nadáakw} ‘table’ (from Chinook Jargon \fm{latáp} from French \fm{la table})
		\end{itemize}
	\item	repetitive suffix combination in tendency state imperfectives,
		indicating a tendency for an event to occur as a characteristic
		property of some entity
		\begin{itemize}
		\item	\fm{–.íkwt} ‘cooks quickly, easily’
			from \fm{\rt[¹]{.i}} ‘cooked’
		\end{itemize}
	\end{enumerate}

\item[kwḵa]
	variant of \fm{kḵwa} used primarily in \cite{story-naish:1973}
\end{morphdesc}

\subsection{Ḵ}\label{sec:alphalist-kh}
\begin{morphdesc}[resume*=alphalist]
\item[ḵ, ḵa]
	≡ \fm{g̱-x̱-}
	combination of first person singular subject \fm{x̱-} / \fm{x̱a-} with either one of
		conjugation \fm{g̱-}
		or modal \fm{g̱-}
	\begin{enumerate}
	\item	with conjugation \fm{g̱-}: \fm{g̱-x̱a-} → \fm{ḵa}
	\item	with modal\fm{g̱-}: \fm{g̱-x̱a-} → \fm{ḵa}
	\end{enumerate}

\item[-ḵ]\label{m:-ḵ-optphib}
	optative/prohibitive modality suffix;
	\newline
	allomorphs:
	\begin{allolist}
	\item[{\X[-ḵw-optphib]{-ḵw}}]
		 	with labialization,
	\item[-íḵ \~\ -iḵ] with epenthetic (filler) vowel \fm{í} \~\ \fm{i}
	\item[-úḵ \~\ -uḵ] with epenthetic (filler) vowel \fm{í} \~\ \fm{i} and labialization
	\end{allolist}
	always occurs with a preceding particle (see below) and usually with irrealis \fm{u-},
	in contrast with deprivative \fm{-ḵ} which does not require a particle and irrealis;
	both the optative/prohibitive and the deprivative are related to 
		the Eyak negative \fm{-ɢ}
		and so presumably descend from a Proto-Na-Dene \fm[*]{-ɢ} 
		that was somehow related to negation
		\parencites{leer:2000b}[872, 876]{crippen:2019}
	\begin{enumerate}
	\item	with optative particle \fm{gu.aal} ‘hopefully’
			and usually also \fm{gushí} \~\ \fm{kwshé} ‘maybe’
	\item	with prohibitive particle \fm{líl} ‘don’t’
			or alternatively with negative particle \fm{tléil} ‘not’
	\end{enumerate}

\item[-ḵ]\label{m:-ḵ-dprv}
	deprivative suffix \fm{-ḵ} \~\ \fm{-ḵw} ‘lacking, removed’;
	generally describes a situation where something is deprived, lacking, or removed,
		but also occurs in some cases with unclear meaning;
	phonologically similar to the optative/prohibitive \X[-ḵ-optphib]{-ḵ}
		but the epenthesized allomorph of the deprivative is \X{-áḵw}
		rather than \X{-iḵ} \~\ \X{-íḵ} or \X{-uḵ} \~\ \X{-úḵ} or the like
		for the optative/prohibitive, showing that they are distinct suffixes;
	also, the optative/prohibitive is always accompanied by a particle
		(\fm{gu.aal}, \fm{líl}, etc.)
		but the deprivative does not require a particle;
	see optative/prohibitive \X[-ḵ-optphib]{-ḵ} for more background and discussion
	\newline
	allomorphs:
	\begin{allolist}
	\item[{\X[-ḵw-dprv]{-ḵw}}]
			with labialization
	\item[\X{-áḵw}]	with epenthetic (filler) vowel \fm{á}
	\end{allolist}
	\begin{enumerate}
	\item	lacking, without, removed;
		some cases are adverbs suggesting otherwise undocumented verbs derived from nouns,
			probably statives with \X[ka-qual]{ka-}
			+ \X{s-}/\X{lˢ-}
			+ \X{d-}
			+ \X[i-stv]{i-}
			+ noun
			+ \fm{-ḵ};
		a couple of cases involve unknown or puzzling roots
		\begin{itemize}
		\item	\fm{kalsʼáaxwḵ} (adverb) ‘hatless’ from \fm{sʼáaxw} ‘hat’ (also \fm{\rt{sʼaxw}} ‘stack’)
			\vbmorph{ka-&d-&lˢ-&\rt{sʼaxw}&-μμH&\gm{-ḵ}}
				{\xx{qual}&\xx{mid}&\xx{intr}&\rt{hat}&\·\xx{var}&\·\xx{dprv}}
		\item	\vbform{lishísʼḵ}{impfv}[obj intr, \fm{g}, inv state]{she/he/it is raw} 
			\vbmorph{lˢ-&i-&\rt{shisʼ}&-μH&\gm{-ḵ}}
				{\xx{intr}&\xx{stv}&\rt{raw?}&\·\xx{var}&\·\xx{dprv}}
			\newline
			also noun \fm{shísʼḵ} ‘raw flesh; sapling wood’;
			some dialects have \fm{shásʼḵ} instead;
			possibly related to \fm{\rt{shisʼ}} ‘squeeze out guts’ 
				as noted by \textcite[59]{story:1966}
			or less likely to \fm{\rt{shi}} ‘reach out’ + repetitive \X{-sʼ};
			\textcite[59]{story:1966} erroneously gives “łɩšɩ́sʼg” = \fm[*]{lishísʼk}
		\item	\fm{kaltéelḵ} (adverb) ‘shoeless’ from \fm{téel} ‘shoe’
			\vbmorph{ka-&d-&lˢ-&\rt{til}&-μμH&\gm{-ḵ}}
				{\xx{qual}&\xx{mid}&\xx{intr}&\rt{shoe}&\·\xx{var}&\·\xx{dprv}}
		\item	\fm{kaltsáaxʼḵ} (adverb) ‘mittenless’ from \fm{tsáaxʼ} ‘mitten, glove’
			\vbmorph{ka-&d-&lˢ-&\rt{tsaxʼ}&-μμH&\gm{-ḵ}}
				{\xx{qual}&\xx{mid}&\xx{intr}&\rt{mitten}&\·\xx{var}&\·\xx{dprv}}
		\end{itemize}
	\item	lexically specified with certain roots, originally deprivative
		but now frozen and noncompositional		
		\begin{itemize}
		\item	\vbform{lixʼwásʼḵ}{impfv}[obj intr, \fm{g}, inv state]{she/he/it is numb}
			\parencites[f04/80]{leer:1973}[762]{leer:1976} from unknown \fm{\rt{xʼwasʼ}}
			\vbmorph{lˢ-&i-&\rt[⁰]{xʼwasʼ}&-μH&\gm{-ḵ}}
				{\xx{intr}&\xx{stv}&\rt[⁰]{feeling?}&\·\xx{var}&\·\xx{dprv}}
			\newline
			also part of \fm{hintakxʼwásʼg̱i} ‘bufflehead’ (\textit{Bucephala albeola} L.)
				\parencite[f04/79]{leer:1973}
				\vbmorph{héen&táak&\rt{xʼwasʼ}&-μH&\gm{-ḵ}&-i}
					{water&below&\rt{\xx{unkn}}&\·\xx{var}&\·\xx{unkn}&\·\xx{poss}}
			\newline
			which might literally mean ‘numb below the surface of the water’
				but the reasoning behind this meaning is unclear
		\item	\vbform{diyáshḵ}{impfv}[obj intr, \fm{n}?, inv state]{she/he/it is scarce, rare}
			\parencites[03/167]{leer:1973}[199]{leer:1976} from unknown \fm{\rt{ÿash}}
			\vbmorph{d-&i-&\rt{ÿash}&-μH&\gm{-ḵ}}
				{\xx{mid}&\xx{stv}&\rt{plenty?}&\·\xx{var}&\xx{dprv}}
			\newline
			perhaps related to \fm{Daasʼadiyáash} ‘Dezadeash Lake’
				and/or \fm{chʼiyaash} ‘flat-bottomed canoe (of Yakutat)’
				both with unknown etymology;
			less likely \fm{\rt{ÿatlʼ}} \~\ \fm{\rt{ÿachʼ}} ‘short’
				and hence \fm{\rt{ÿatʼ}} ‘long’;
			even more speculatively \fm{\rt{ÿat}} ‘child’
				and \fm{\rt{wat}} ‘mature’;
			probably no connection to \fm{\rt{ÿaᴴsh}} ‘platform’ (noun \fm{kaÿáash})
				or \fm{\rt{ÿasʼ}} ‘smooth’
		\end{itemize}
	\item	unclear meaning, followed by plural/repetitive \X{-xʼ}
			and associated with an increase in degree;
		only two cases are attested, both reported by \textcite[59]{story:1966},
			which are notably not listed by \textcites{leer:1973}{leer:1976};
		these could plausibly be mishearings of repetitive \X{-k} instead of deprivative \fm{-ḵ}
		\begin{itemize}
		\item	\vbform{x̱ʼalitsínḵxʼ}{impfv}[obj intr, \fm{g}, inv state]{they are all very expensive}
			\parencite[59]{story:1966} from \fm{\rt{tsin}} ‘alive, strong’
				\vbmorph{x̱ʼe-&lˢ-&i-&\rt[¹]{tsin}&-μH&\gm{-ḵ}&-xʼ}
					{mouth&\xx{xtn}&\xx{stv}&\rt[¹]{strong}&\·\xx{var}&\·\xx{dprv}&\·\xx{pl}}
			\versus \vbform{dax̱ x̱ʼadlitsínxʼ}{impfv}{they are each expensive}
				\parencite[154.1173]{nyman-leer:1993}
				\vbmorph{dax̱=&x̱ʼe-&d-&lˢ-&i-&\rt[¹]{tsin}&-μH&-xʼ}
					{\xx{distb}&mouth&\xx{mid}&\xx{xtn}&\xx{stv}&\rt[¹]{strong}&\·\xx{var}&\·\xx{pl}}
			\versus \vbform{x̱ʼalitseen}{impfv}{she/he/it is expensive}
				\vbmorph{x̱ʼe-&lˢ-&i-&\rt[¹]{tsin}&-μμL}
					{mouth&\xx{xtn}&\xx{stv}&\rt[¹]{strong}&\·\xx{var}}
		\item	\vbform{diyátʼḵxʼ}{impfv}[obj intr, \fm{g}, inv state]{they are very long}
			\parencite[59]{story:1966} from \fm{\rt{ÿatʼ}} ‘long’
				\vbmorph{d-&i-&\rt[¹]{ÿatʼ}&-μH&\gm{-ḵ}&-xʼ}
					{\xx{mid}&\xx{stv}&\rt[¹]{long}&\·\xx{var}&\·\xx{dprv}&\·\xx{pl}}
			\versus \vbform{diyátʼxʼ}{impfv}{they are long}
				\vbmorph{d-&i-&\rt[¹]{ÿatʼ}&-μH&-xʼ}
					{\xx{mid}&\xx{stv}&\rt[¹]{long}&\·\xx{var}&\·\xx{pl}}
			\versus \vbform{yayátʼ}{impfv}{she/he/it is long}
				\vbmorph{ÿa-&\rt[¹]{ÿatʼ}&-μH}
					{\xx{stv}&\rt[¹]{long}&\·\xx{var}&}
		\end{itemize}
	\item	possibly identifiable as the final consonant in a couple of roots,
		originally deprivative but now frozen and noncompositional
		\begin{itemize}
		\item	\fm{\rt{naḵ}} ‘abandoning, leaving behind’ in verbs and as a postposition;
			the hypothetical \fm[*]{\rt{na}} root that would combine with \fm{-ḵ}
				to give \fm{\rt{naḵ}} has not been identified
				and may no longer exist independently
			\begin{enumerate}[label=\alph*.]
			\item	as a verb root \fm{\rt[²]{naḵ}} ‘give up, let go’ in
				\begin{itemize}[label=•]
				\item	\vbform{ajeewanáḵ}{pfv}[tr, \fm{∅}, ach]{she/he/it let him/her/it go}
					\vbmorph{a-&ji-&μʷ-&wa-&\rt[²]{naḵ}&-μH}
						{\xx{3>3}&hand&\xx{pfv}&\xx{stv}&\rt[²]{let.go}&\xx{var}}
				\item	\vbform{ax̱ʼeiwanáḵ}{pfv}[tr, \fm{∅}, ach]{she/he/it gave up it (drinking, smoking)}
					\vbmorph{a-&x̱ʼe-&μʷ-&wa-&\rt[²]{naḵ}&-μH}
						{\xx{3>3}&mouth&\xx{pfv}&\xx{stv}&\rt[²]{let.go}&\xx{var}}
				\item	\vbform{ax̱ʼawsináḵ}{pfv}[tr, \fm{∅}, ach]{she/he/it silenced him/her/it}
					\vbmorph{a-&x̱ʼe-&w-&s-&i-&\rt[²]{naḵ}&-μH}
						{\xx{3>3}&mouth&\xx{pfv}&\xx{csv}&\xx{stv}&\rt[²]{let.go}&\xx{var}}
				\end{itemize}
			\item	as a postposition \fm{náḵ} ‘away from, leaving behind’ in
				\begin{itemize}[label=•]
				\item	\vbform{a náḵ neil oo.aatch}{hab}[subj intr, \fm{∅}, mot]{they would go home from it}
					\parencite[66.64]{dauenhauer-dauenhauer:1987}
					\vbmorph{a&náḵ&neil=&a-&u-&\rt[¹]{.at}&-μμL&-ch}
						{\xx{3n}&away&home&\xx{ind.h.s}&\xx{zpfv}&\rt[¹]{go.\xx{pl}}&\·\xx{var}&\·\xx{rep}}
				\item	\vbform{a waḵnáḵ awlisín}{pfv}[tr, \fm{∅}, ach]{she/he/it hid him/her/it away from his/her/its eyes}
					\parencite[04/77]{leer:1973}
					\vbmorph{a&waaḵ&-náḵ&a-&w-&lˢ-&i-&\rt{sin}&-μH}
						{\xx{3n.psr}&eye&\·away&\xx{3>3}&\xx{pfv}&\xx{csv}&\xx{stv}&\rt{hide}&\·\xx{var}}
				\end{itemize}
			\end{enumerate}
		\item	\fm{sayeiḵ} \~\ \fm{siyeiḵ} ‘next day, day after’
				(also \fm{seig̱ánín} \~\ \fm{seig̱án} ‘tomorrow’
				< \fm[*]{sayeiḵ-án-ín});
			the \fm{sa-} is probably a frozen form of \X{s-}, and the stem \fm{–yeiḵ}
			possibly derives from the noun \fm{ÿee} ‘time’ (see \X[ÿee=time]{ÿee=}) 
			and deprivative \fm{-ḵ}
		\item	\fm{\rt{x̱iḵ}} \~\ \fm{\rt{x̱eḵ}} ‘lack sleep’ in verbs and a noun;
			originally from \fm{\rt{x̱i}} \~\ \fm{\rt{x̱e}} ‘stay overnight’
				and deprivative \fm{-ḵ}
				but reanalyzed as a new root with distinct stem variation,
				compare \fm{\rt{x̱exʼw}} ‘plural sleep’ with plural \X{-xʼw}
			\begin{enumerate}[label=\alph*.]
			\item	as a verb root \fm{\rt{x̱eḵ}} \~\ \fm{\rt{x̱iḵ}} ‘insomnia’ in
				\begin{itemize}[label=•]
				\item	\vbform{wudix̱éḵ}{pfv}[obj intr, \fm{∅}, ach]{she/he/it had insomnia}
					\parencites[f02/23]{leer:1973}[794]{leer:1976}
						\vbmorph{wu-&d-&i-&\rt[¹]{x̱eḵ}&-μH}
							{\xx{pfv}&\xx{mid}&\xx{stv}&\rt[¹]{insomnia}&\·\xx{var}}
				\item	\vbform{awsix̱éḵ}{pfv}[tr, \fm{∅}, ach]{she/he/it woke him/her/it early}
					\vbmorph{a-&w-&s-&i-&\rt{x̱eḵ}&-μH}
						{\xx{3>3}&\xx{pfv}&\xx{csv}&\xx{stv}&\rt[¹]{insomnia}&\·\xx{var}}
				\item	\vbform{ÿax̱éḵsʼ}{impfv}{she/he/it gets up early}
						\vbmorph{ÿa-&\rt[¹]{x̱eḵ}&-μH&-sʼ}
							{\xx{stv}&\rt[¹]{insomnia}&\·\xx{var}&\·\xx{rep}}
				\end{itemize}
			\item	as a noun \fm{x̱eeḵ} \~\ \fm{x̱eiḵ} ‘insomnia; tiredness’
				especially in the phrase
				\vbform{x̱eiḵch uwajáḵ}{pfv}[tr, \fm{∅}, ach]{she/he/it suffered bout of insomnia}
				(literally ‘insomnia killed him/her/it’)
					\vbmorph{x̱eiḵ&-ch&ⱥ-&u-&wa-&\rt[²]{jaḵ}&-μH}
						{insomnia&\·\xx{erg}&\xx{3>3}&\xx{zpfv}&\xx{stv}&\rt[²]{kill}&\·\xx{var}}
			\end{enumerate}
		\end{itemize}
	\end{enumerate}

\item[ḵaa=]
	allomorph of indefinite human object \fm{ḵu-} ‘someone, people, one, them’;
	possibly like \fm{ax̱=} used only as possessor of incorporated noun
		(compare \fm{ḵaa keidlí áwé} ‘it’s someone’s dog’)
	\begin{itemize}
	\item	\fm{ḵaa seiwa.áx̱} (pfv; tr, \fm{∅}, ach) ‘s/he/it heard someone’s voice’
	\end{itemize}

\item[ḵu-]\label{m:ḵu-indef}
	indefinite human object ‘someone, people, one, them’;
	allomorph \fm{ḵaa=} as possessor of incorporated noun;
	compare PP pronoun \fm{ḵú-} ‘someone, people, one, them’
	\begin{itemize}
	\item	\fm{ḵuwsiteen} (pfv; tr, \fm{g̱}, ach) ‘s/he/it saw someone/people’
	\end{itemize}

\item[ḵu-]\label{m:ḵu-areal}
	areal prefix indicating space, area, extent, or weather;
	compare \fm{ḵúx̱(-de)=} ‘back, returning’, \fm{ḵut=} ‘lost’
	\begin{itemize}
	\item	\fm{ḵuwakʼéi} (impfv; impers, \fm{g}, \fm{-μμH} state) ‘it is good weather’\newline
		versus \fm{yakʼéi} (impfv; obj intr) ‘it is good’
	\end{itemize}

\item[-ḵw]\label{m:-ḵw-optphib}
	allomorph of optative/prohibitive \X[-ḵ-optphib]{-ḵ} with labialization

\item[-ḵw]\label{m:-ḵw-dprv}
	allomorph of deprivative \X[-ḵ-dprv]{-ḵ} with labialization

\end{morphdesc}

\subsection{L}\label{sec:alphalist-l}
\begin{morphdesc}[resume*=alphalist]
\item[l-]\label{m:l-}
	valency prefix of classifier
	\begin{enumerate}
	\item	argument addition
		\begin{enumerate}
		\item	lone argument of intransitive
		\item	causative
		\item	applicative
		\end{enumerate}
	\item	spatial extension
		\begin{enumerate}
		\item	extended entity
		\item	extended eventuality
		\end{enumerate}
	\end{enumerate}

\item[lˢ-]\label{m:lˢ-}
	allomorph of valency \fm{s-} \~\ \fm{sa-};
	occurs when any fricative
		\fm{s}, \fm{sʼ}, \fm{l}, \fm{lʼ}, \fm{sh}
	or any affricate
		\fm{dz}, \fm{ts}, \fm{tsʼ},
		\fm{dl}, \fm{tl}, \fm{tlʼ},
		\fm{j}, \fm{ch}, \fm{chʼ}
	occurs in the onset or coda of the stem syllable;
	phonetically indistinguishable from \fm{l-} \~\ \fm{la-} and may be represented as such
	if the distinction is not important
	\begin{itemize}
	\item	\fm{lichán} (impfv; obj intr, \fm{g}, \fm{-μH} invar.\ state) ‘it stinks’
		(not *\fm{sichán})
	\item	\fm{wutuliyíḵsʼ} (rep pfv; tr, \fm{n}, mot) ‘we repeatedly pulled it (long obj.)’\newline
		versus \fm{wutusiyeeḵ} (pfv) ‘we pulled it (long obj.)’
	\end{itemize}

\item[…l]\label{m:…l}
	≡ \fm{d-l-}
	combination of voice \X{d-}
		and valency \X{l-} or \X{lˢ-},
	appears only as a coda consonant and so requires a preceding vowel
	\begin{itemize}
	\item	\fm{sh ilg̱ásʼx̱} (rep impfv; tr, \fm{∅}, ach) ‘s/he/it repeatedly scratches self’
			with \fm{d-l-}\newline
		versus \fm{sh wudlig̱ásʼ} (pfv) ‘s/he/it scratched self’
			with \fm{d-l-i-}
	\item	\fm{tléil sh kawulháachʼ} (neg pfv; tr, \fm{n}, \fm{-μμH} state) ‘s/he/it didn’t shame self’
			with \fm{d-l-}
		versus \fm{sh kawdliháachʼ} (pfv) ‘s/he/it shamed self’
			with \fm{d-l-i-}
	\end{itemize}

\item[-lʼ]\label{m:-lʼ}
	repetitive suffix limited to only a few verbs,
	most often with \fm{\rt[²]{xakw}} ‘whip up, beat (eggs, soapberries, etc.)’;
	probably historically related to \X{-sʼ};
	\newline
	allomorphs:
	\begin{allolist}
	\item[-lʼ]	single consonant form
	\item[\X{-álʼ}]	form with epenthetic (filler) vowel \fm{á}
	\end{allolist}
	\begin{enumerate}
	\item	repetitive suffix attested with three verb roots
		\begin{itemize}
		\item	\vbform{aklaxákwlʼ}{rep impfv}[tr, \fm{∅}, ach]{she/he/it whips up him/her/it}
				\vbmorph{a-&k-&la-&\rt[²]{xakw}&-μH&\gm{-lʼ}}
					{\xx{3>3}&\xx{qual}&\xx{xtn}&\rt[²]{whip}&\·\xx{var}&\·\xx{rep}}
			\versus \vbform{akawlixákw}{pfv}{she/he/it whipped up him/her/it}
				\vbmorph{a-&ka-&w-&l-&i-&\rt[²]{xakw}&-μH}
					{\xx{3>3}&\xx{qual}&\xx{pfv}&\xx{xtn}&\xx{stv}&\rt[²]{whip}&\·\xx{var}}
				
		\item	\vbform{akaagúklʼ}{rep impfv}[tr, \fm{∅}?, state]{she/he/it tries to be skilled at it}
			\parencites[f05/176]{leer:1973}[671]{leer:1976}
				\vbmorph{a-&ka-&μ-&\rt[²]{guᴴk}&-μH&\gm{-lʼ}}
					{\xx{3>3}&\xx{qual}&\xx{stv}&\rt[²]{know.how}&\·\xx{var}&\·\xx{rep}}
			\versus \vbform{ashigóok}{impfv}[tr, \fm{g}/\fm{g̱}, state]{she/he/it knows how to do it}
				\vbmorph{a-&sh-&i-&\rt[²]{guᴴk}&-μμH}
					{\xx{3>3}&\xx{xtn}&\xx{stv}&\rt[²]{know.how}&\·\xx{var}}
		\item	\vbform{kax̱aagúnlʼx̱}{rep impfv}[tr?, conj?, state]{I am trying hard}
			\parencite[f05/126]{leer:1973}
				\vbmorph{ka-&x̱a-&μ-&\rt{gun}&\gm{-lʼ}&-x̱}
					{\xx{qual}&\xx{1sg.s}&\xx{stv}&\rt{try.hard}&\·\xx{rep}&\·\xx{rep}}
		\end{itemize}
	\item	unclear meaning in a handful of nouns
		\begin{itemize}
		\item	\fm{dínlʼ} ‘obstruction’ from unknown \fm{\rt{din}},
			possibly related to \fm{\rt[²]{diᴴn}} ‘trouble’;
			occurs as a part of:
			\begin{itemize}
			\item	\fm{gukyikdínlʼ} ‘hard of hearing’
				\vbmorph{guk-&yik-&\rt{din}&-μH&\gm{-lʼ}}
					{ear&within&\rt{obstruction?}&\·\xx{var}&\·\xx{rep}}
			\item	\fm{shantudínlʼ} ‘idiot, airhead’
				\vbmorph{shaⁿ-&tu-&\rt{din}&-μH&\gm{-lʼ}}
					{head&inside&\rt{obstruction?}&\·\xx{var}&\·\xx{rep}}
			\end{itemize}
		\item	\fm{gúnlʼ} ‘burl, lump’ from unknown \fm{\rt{gun}}
			\vbmorph{\rt{gun}&-μH&\gm{-lʼ}}
				{\rt{lump?}&\·\xx{var}&\·\xx{rep}}
			\newline
			also occurs in
				\begin{inlinelist}
				\item	\fm{aasdaagúnlʼi} ‘burl’
				\item	\fm{jigúnlʼi} ‘wrist knob’
				\item	\fm{lakʼichʼgúnlʼi} ‘occipital knob’
				\item	\fm{leikachóox̱'u gúnlʼi} ‘adam’s apple’
				\item	\fm{leitux̱gúnlʼi} ‘adam’s apple’
				\item	\fm{tlʼeḵkagúnlʼi} ‘knuckle’
				\item	\fm{x̱ʼagúnlʼi} ‘lumpy mouth’
				\item	\fm{x̱ʼusgúnlʼi} ‘ankle knob’
				\end{inlinelist};
			the noun is also the basis of the derived verb
				\vbform{kashigúnlʼ}{impfv}[obj intr, conj?, inv state]{it has a burl};
			probably related to \vbform{kax̱aagúnlʼx̱}{rep impfv}{I am trying hard}
				discussed above	but the meaning relationship is unclear
		\item	\fm{g̱úḵlʼ} ‘swan’ from unknown \fm{\rt{g̱uḵ}}
			\vbmorph{\rt{g̱uḵ}&-μH&\gm{-lʼ}}
				{\rt{\xx{unkn}}&\·\xx{var}&\·\xx{rep}}
			\newline
			also occurs in \fm{áa tug̱úḵlʼi} ‘kind of whitefish’
		\item	\fm{túḵlʼ} ‘young spruce or hemlock; cartilage’ from unknown \fm{\rt{tuḵ}}
			\vbmorph{\rt{tuḵ}&-μH&-lʼ}
				{\rt{flex?}&\·\xx{var}&\·\xx{rep}}
			\newline
			occurs in 
				\begin{inlinelist}
				\item	\fm{lututúḵlʼi} ‘nasal cartilage’
				\item	\fm{sʼaḵshutúḵlʼi} ‘bone end cartilage’
				\item	\fm{sʼaḵx̱ʼaaktúḵlʼi} ‘cartilage between bones’
				\item	\fm{yuwshutúḵlʼi} ‘xiphoid process’
				\end{inlinelist};
			probably related to \fm{dúḵ} ‘cottonwood’;
			possibly related to \fm{tóox̱ʼ} in \fm{shutóox̱ʼ} ‘ankle’
				and \fm{x̱ʼusʼgukshtóox̱ʼ} ‘ankle bone’;
			less likely related to \fm{tooḵ} ‘butt’ or \fm{tóoḵ} ‘needlefish’
		\item	\fm{tʼáḵlʼ} ‘projecting bone’ from unknown \fm{\rt{tʼaḵ}}
			possibly related to \fm{tʼaaḵ} ‘beside’;
			occurs as a part of:
			\begin{itemize}
			\item	\fm{jitʼáḵlʼi} ‘wrist knob’
				\vbmorph{ji-&\rt{tʼaḵ}&-μH&\gm{-lʼ}&-i}
					{hand&\rt{beside?}&\·\xx{var}&\·\xx{rep}&\·\xx{poss}}
			\item	\fm{x̱ʼustʼáḵlʼi} ‘ankle knob’
				\vbmorph{x̱ʼus-&\rt{tʼaḵ}&-μH&\gm{-lʼ}&-i}
					{foot&\rt{beside?}&\·\xx{var}&\·\xx{rep}&\·\xx{poss}}
			\end{itemize}
			also compare \fm{tʼáaḵw} ‘joist, timber; carpentry joint’
		\item	\fm{xákwlʼi} ‘soapberries’ from \fm{\rt{xakw}} ‘whip’
			\vbmorph{\rt{xakw}&-μH&\gm{-lʼ}&-i}
				{\rt{whip}&\·\xx{var}&\·\xx{rep}&\·\xx{nmz}}
			\newline
			nominalization of the verb ‘whip up’ above
		\item	\fm{yadzánlʼ} ‘lumpy face’ from unknown \fm{\rt{dzan}}
			\vbmorph{ÿa-&\rt{dzan}&-μH&\gm{-lʼ}}
				{face&\rt{lump?}&\·\xx{var}&\·\xx{rep}}
			\newline
			possibly related to \fm{dzánti} ‘flounder’;
			may derive from irregular palatalization and affrication of
				\fm{gúnlʼ} ‘burl, lump’ above
		\end{itemize}
	\end{enumerate}

\item[la-]\label{m:la-val}
	allomorph of valency \X{l-}

\item[la-]\label{m:la-throat}
	allomorph of incorporated noun \X{le-} ‘throat’

\item[lˢa-]\label{m:lˢa-}
	allomorph of valency \X{lˢ-}

\item[le-]\label{m:le-}
	incorporated noun indicating throat or inside of mouth

\item[li]
	≡ \fm{l-i-}
	combination of valency \X{l-} or \X{lˢ-}
		and stative \X[i-stv]{i-}

\item[-lʼútʼ]\label{m:-lʼútʼ}
	allomorph of noun \fm{lʼóotʼ} ‘tongue’ used as a suffix in the stem
		\fm{–ḵéilʼútʼ} ‘lick seam’;
	this stem is formed from the root \fm{\rt[²]{ḵa}} ‘stitch, sew’ 
		with stem variation \X{-μᵉμH}
		and \fm{-lʼútʼ};
	compare the noun \fm{ḵéichʼálʼ} ‘seam’ with \fm{-μᵉμH} and \X{-chʼálʼ}
	\begin{itemize}
	\item	\vbform{aḵéilʼútʼ}{impfv}[tr, conj?, inv act]{she/he/it is licking it (seam) to make it hard}
		\parencite[f01/26]{leer:1973}
			\vbmorph{a-&\rt[²]{ḵa}&-μᵉμH&\gm{-lʼútʼ}}
				{\xx{3>3}&\rt[²]{stitch}&\·\xx{var}&\·tongue}
	\end{itemize}
		
		
\end{morphdesc}

\subsection{M}\label{sec:alphalist-m}
\begin{morphdesc}[resume*=alphalist]
\item[m-]\label{m:m-}
	allomorph of perfective \X{wu-} in the coda of a syllable;
	currently used instead of perfective \X[w-pfv]{w-}
		only in some Inland Northern Tlingit varieties (Teslin and Carcross-Tagish),
	but may also occur elsewhere in older Tlingit (e.g.\ song lyrics, set phrases,
	19th century transcriptions)
	\begin{itemize}
	\item	\vbform{amsiteen}{pfv}[tr, \fm{g̱}, ach]{s/he/it caught sight of (saw) him/her/it}
			\vbmorph{a-&\gm{m-}&s-&i-&\rt[²]{tin}&-μμL}
				{\xx{3>3}&\xx{pfv}&\xx{xtn}&\xx{stv}&\rt[²]{see}&\·\xx{var}}
		\versus \vbform{x̱at wusiteen}{pfv}{she/he/it caught sight of (saw) me}
			\vbmorph{x̱at=&wu-&s-&i-&\rt[²]{tin}&-μμL}
				{\xx{1sg.o}&\xx{pfv}&\xx{xtn}&\xx{stv}&\rt[²]{see}&\·\xx{var}}
	\end{itemize}
\end{morphdesc}

\subsection{N}\label{sec:alphalist-n}
\begin{morphdesc}[resume*=alphalist]
\item[n-]\label{m:n-}
	\begin{enumerate}
	\item	\fm{n} conjugation class prefix, horizontal spatial orientation
		\begin{itemize}
		\item	\fm{nayx̱éixʼw!} (imp; subj intr, \fm{n}, \fm{-μμH} act) ‘you guys (go to) sleep!’
		\item	\fm{naḵahoon} (hort; tr, \fm{n}, \fm{-μμH} act) ‘let me sell it’
		\end{itemize}
	\item	progressive aspect prefix
		\begin{itemize}
		\item	\fm{yaa nx̱ax̱éin} (prog; tr, \fm{∅}, act) ‘I am going along eating it’\newline
			versus \fm{x̱ax̱á} (impfv) ‘I am eating it’
		\end{itemize}
	\end{enumerate}

\item[-n]\label{m:-n}
	stem suffix of uncertain meaning;
	causes ablaut /\ipa{a}, \ipa{u}/ → [\ipa{eː}] of \fm{\rt{Ca}} and \fm{\rt{Cu}} roots
	except for \fm{\rt[¹]{naᴸ}} ‘die’ and \fm{\rt[²]{ya}} ‘pack’
	\begin{enumerate}
	\item	with progressive aspect
		\begin{itemize}
		\item	\fm{yaa anax̱éin} (prog; tr, \fm{∅}, \fm{-μH} act) ‘s/he/it is going along eating him/her/it’ with root \fm{\rt[²]{x̱a}} ‘eat’\newline
			(not \fm[*]{yaa anax̱áan})\newline
			versus \fm{aawax̱áa} (pfv) ‘s/he/it ate him/her/it’
		\item	\fm{yaa anaskwéin} (prog; subj intr, \fm{∅}, ach) ‘s/he/it is coming to know him/her/it’ with root \fm{\rt[²]{kuᴸ}} ‘know’\newline
			(not \fm[*]{yaa anaskóon})\newline
			versus \fm{awsikóo} (pfv) ‘s/he/it came to know him/her/it’
		\end{itemize}
	\item	with conditional mood
	\item	with contingent mood
	\item	irregularly in a few imperfective state verbs
	\end{enumerate}

\item[na-]\label{m:na-}
	allomorph of \fm{n} conjugation prefix \X{n-} with epenthetic (filler) vowel \fm{a}

\item[nás]\label{m:nás}
	≡ \fm{-n-ás}
	combination of stem suffix \X{-n}
		and unknown \X{-ás};
	only occurs with the root \fm{\rt[²]{ḵe}} \~\ \fm{\rt[²]{ḵi}} ‘pay’;
	see \X{-ás} for more details
	and see \X{-s} for forms without \fm{-n}
\end{morphdesc}

\subsection{O}\label{sec:alphalist-o}
\begin{morphdesc}[resume*=alphalist]
\item[oo-]\label{m:oo-}
	allomorph of irrealis \fm{u-}
	
\item[oo]\label{m:oo}
	≡ \fm{a-u-}
	combination of argument marking \X{a-}
		and either irrealis \X[u-irr]{u-}
			or \fm{∅} conjugation perfective \X[u-pfv]{u-}

\item[oowa]\label{m:oowa}
	≡ \fm{a-u-wa-}
	combination of argument marking \X{a-}
		and irrealis \X[u-irr]{u-}
		and stative \X{wa-};
	unlike \X[eeÿa-a-i-ÿa]{eeÿa} ≡ \fm{a-i-ÿa} versus \X[eeÿa-a-ʷ-i-ÿa]{eeÿa} ≡ \fm{a-ʷ-i-ÿa},
		the combination \fm{oowa} cannot be perfective
		because \fm{∅} conjugation perfective \X[u-pfv]{u-}
			cannot occur with both argument marking \X{a-}
			and stative \X{wa-} at the same time
		and with perfective \X{wu-} (instead of \X[u-pfv]{u-})
			the form is always \X{aawa} never \fm{oowa}
	\begin{itemize}
	\item	\vbform{yéi oowajée}{impfv}[tr, \fm{n}, \fm{-μμH} state]{she/he/it thinks so about him/her/it}
			\vbmorph{yéi=&\gm{a-}&\gm{u-}&\gm{wa-}&\rt[²]{jiᴸ}&-μμH}
				{thus&\xx{3>3}&\xx{irr}&\xx{stv}&\rt[²]{think}&\·\xx{var}}
		\versus \vbform{yéi aawajee}{pfv}{she/he/it thought so about him/her/it}
			\vbmorph{yéi=&a-&ʷ-&μʷ-&wa-&\rt[²]{jiᴸ}&-μμL}
				{thus&\xx{3>3}&\xx{irr}&\xx{pfv}&\xx{stv}&\rt[²]{think}&\·\xx{var}}
	\end{itemize}

\item[oox̱]\label{m:oox̱}
	≡ \fm{a-u-x̱-}
	combination of argument marking \X{a-}
		and either irrealis \X[u-irr]{u-}
			or \fm{∅} conjugation perfective \X[u-pfv]{u-}
		and either \fm{g̱} conjugation \X[x̱-g̱cnj]{x̱-}
			or modal \X[x̱-mod]{x̱-}
			or first person singular subject \X[x̱-1sg]{x̱-}
\end{morphdesc}

\subsection{S}\label{sec:alphalist-s}
\begin{morphdesc}[resume*=alphalist]
\item[s-]\label{m:s-}
	valency prefix of classifier
	\begin{enumerate}
	\item	argument addition
		\begin{enumerate}
		\item	lone argument of intransitive
		\item	causative
		\item	applicative
		\end{enumerate}
	\item	spatial extension
		\begin{enumerate}
		\item	extended entity
		\item	extended eventuality
		\end{enumerate}
	\end{enumerate}

\item[…s]\label{m:…s}
	≡ \fm{d-s-}
	combination of voice \X{d-}
		and valency \X{s-},
	appears only as a coda consonant and so requires a preceding vowel
	\begin{itemize}
	\item	\fm{yax̱ sh x̱asnei} (rep impfv; tr, \fm{∅}, ach) ‘I repeatedly dress myself’
			with \fm{d-s-}\newline
		versus \fm{yan sh x̱wadzinéi} (pfv) ‘I dressed myself’
			with \fm{d-s-i-}
	\end{itemize}

\item[s=]
	allomorph of human pluralizer \fm{has=} for third person subject or object

\item[-s]\label{m:-s}
	unknown suffix that occurs only with the root \fm{\rt[²]{ḵe}} \~\ \fm{\rt[²]{ḵi}} ‘pay’;
	this form is documented only in Tongass Tlingit, with the allomorph \X{-ás}
		documented elsewhere;
	the meaning of this suffix is unknown because it is only attested with a single root,
		although it could potentially be identified among CVC roots with coda /\ipa{s}/;
	this suffix is distinct from and apparently unrelated to \X{-sʼ}
	\newline
	allomorphs:
	\begin{allolist}
	\item[-s]	basic form
	\item[\X{-ás}]	with epenthetic (filler) vowel \fm{á}
	\end{allolist}
	\begin{itemize}
	\item	\vbform{awli̥ḵeís}{pfv}[tr, \fm{n}?, ach?]{she/he/it gave him/her/it expecting something in return}
		(Tongass dialect) \parencite[f01/68]{leer:1973}
			\vbmorph{a-&w-&lˢ-&i-&\rt[²]{ḵe}&-μμˀ&\gm{-s}}
				{\xx{3>3}&\xx{pfv}&\xx{xtn}&\xx{stv}&\rt[²]{pay}&\·\xx{var}&\·\xx{unkn}}
		\versus \vbform{aawaḵei}{pfv}[tr, \fm{n}, ach]{she/he/it paid him/her/it}
		(Tongass dialect) \parencite[f01/66]{leer:1973}
			\vbmorph{a-&μʷ-&wa-&\rt[²]{ḵe}&-μμ}
				{\xx{3>3}&\xx{pfv}&\xx{stv}&\rt[²]{pay}&\·\xx{var}}
	\end{itemize}

\item[-sʼ]\label{m:-sʼ}
	repetitive suffix only occurring with specific verbs;
	unlike the \X{-ch}, \X{-k}, and \X{-x̱} repetitive suffixes,
		this suffix is never specified by conjugation class
		or as part of a motion derivation;
	for some verbs \fm{-sʼ} occurs instead of
		the conjugation class-specific repetitive suffix (\fm{-ch}, \fm{-k}, \fm{-x̱})
		but other verbs can use either the conjugation class-specific suffix
		or \fm{-sʼ};
	the relatively large number of verbs attested with this suffix
		suggests that it has a productive meaning
		but the details remain unclear;
	the repetitive suffix \X{-lʼ} is historically related,
		similar to the relationship between \X{s-} and \X{l-};
	there may be a broader historical relationship
		between \X{-lʼ}, \X{-sʼ} \X{-tʼ}, and \X{-xʼ}
		since all of these are ejective
	\newline
	allomorphs:
	\begin{allolist}
	\item[-sʼ]	basic form
	\item[\X{-ásʼ}]	form with epenthetic (filler) vowel \fm{á}
	\end{allolist}
	\begin{enumerate}
	\item	repetitive suffix with a wide variety of roots,
		appearing in a repetitive imperfective form
		or in forms derived from the repetitive imperfective
		(secondary aspectual derivations, Leer’s “epiaspect”)
		\begin{itemize}
		\item	\fm{–ḵéisʼ} from \fm{\rt{ḵa}} ‘stitch, sew’ in
			\newline
			\vbform{aḵéisʼ}{rep impfv}[tr, \fm{∅}, ach]{she/he/it is stitching, sewing him/her/it}
				\vbmorph{a-&\rt[²]{ḵa}&-μμᵉH&\gm{-sʼ}}
					{\xx{3>3}&\rt[²]{stitch}&\·\xx{var}&\·\xx{rep}}
			\versus \vbform{aawaḵáa}{pfv}{she/he/it stitched, sewed him/her/it}
				\vbmorph{a-&μʷ-&wa-&\rt[²]{ḵa}&-μμH}
					{\xx{3>3}&\xx{pfv}&\xx{stv}&\rt[²]{stitch}&\·\xx{var}}
			\newline
			compare
			\begin{inlinelist}
			\item	\fm{kax̱duḵéisʼáḵw} \~\ \fm{katḵéisʼáḵw} ‘quilt’ with \X{-áḵw}
			\item	\fm{ḵéichʼálʼ} ‘seam’ with \X{-chʼálʼ}
			\item	\fm{–ḵéilʼútʼ} ‘lick seam’ with \X{-lʼútʼ}
			\item	\fm{ḵéinaa} ‘awl’ with \X{-n} and \X{-aa}
			\end{inlinelist}
		\item	roots attested with \fm{-sʼ} include
			\parencite[533]{crippen:2019}:
			\begin{inlinelist}
			\item	\fm{\rt{cha}} ‘strain, sift’ (\fm{–chéisʼ})
			\item	\fm{\rt{chuk}} ‘rub soft’
			\item	\fm{\rt{chux}} ‘knead’
			\item	\fm{\rt{geᴴÿ}} ‘pay debt’
			\item	\fm{\rt{gish}} ‘soak’
			\item	\fm{\rt{gwal}} ‘beat’
			\item	\fm{\rt{g̱uk}} ‘squeeze’
			\item	\fm{\rt{hin}} ‘water’
			\item	\fm{\rt{jaᴸ}} ‘advise’ (\fm{–jeisʼ})
			\item	\fm{\rt{kik}} ‘shake out’
			\item	\fm{\rt{kel}} ‘soak’
			\item	\fm{\rt{ḵa}} ‘stitch, sew’ (\fm{–ḵéisʼ})
			\item	\fm{\rt{lʼikw}} \~\ \fm{\rt{lʼuk}} ‘blink’ (in \fm{xeitl lʼíkwsʼi} ‘lightning’)
			\item	\fm{\rt{lʼixw}} \~\ \fm{\rt{lʼux}} ‘close eye’
			\item	\fm{\rt{naᴸ}} ‘damp; oil’ (\fm{–neisʼ})
			\item	\fm{\rt{naḵw}} ‘bait, octopus’
			\item	\fm{\rt{nal}} ‘blow nose, steam’
			\item	\fm{\rt{sʼiḵ}} \~\ \fm{\rt{sʼeḵ}} ‘suck; smoke’
			\item	\fm{\rt{taxʼ}} ‘bite’
			\item	\fm{\rt{tiy}} ‘patch’
			\item	\fm{\rt{tuk}} ‘pop’
			\item	\fm{\rt{tʼaᴸ}} ‘hot’ (\fm{–tʼeisʼ})
			\item	\fm{\rt{tʼak}} ‘dent’
			\item	\fm{\rt{tʼixʼ}} ‘hard’
			\item	\fm{\rt{tsik}} ‘roast’
			\item	\fm{\rt{wu}} ‘send for’ (\fm{–wéisʼ})
			\item	\fm{\rt{xa}} ‘pour’ (\fm{–xéisʼ})
			\item	\fm{\rt{xwach}} ‘scrape’
			\item	\fm{\rt{x̱eḵ}} ‘insomnia, wake early’ (see also \X[-ḵ-dprv]{-ḵ})
			\item	\fm{\rt{x̱ik}} ‘flap wings’
			\item	\fm{\rt{x̱ux̱}} ‘summon; compose song’
			\item	\fm{\rt{x̱ʼeᴴÿ}} ‘encourage’
			\item	\fm{\rt{yiḵ}} ‘mark; pull’
			\item	\fm{\rt{yuk}} ‘shake’
			\end{inlinelist}
		\end{itemize}
	\item	unidentified suffix in some nouns with complex codas
		\begin{itemize}
		\item	\fm{g̱áḵsʼi} ‘fish roasted whole by tail’
			from unknown \fm{\rt{g̱aḵ}},
			compare \fm{g̱aaḵ} ‘lynx’,
			\fm{\rt{g̱aḵ}} ‘dog/raven makes noise’,
			\fm{\rt{g̱aḵ}} ‘toss quoits, gambling sticks’
		\item	\fm{ḵáx̱sʼi} ‘coho salmon tied to bush’
			from unknown \fm{\rt{ḵax̱}}
		\item	\fm{náksʼ} ‘cold sore on tongue’
			from unknown \fm{\rt{nak}}
		\item	\fm{x̱ikshakahánsʼi} ‘shoulder fringe on jacket’
			from \fm{\rt{han}} ‘cut into strips’
		\item	\fm{x̱ʼwánsʼ} ‘crumbly substance’,
			\fm{attux̱ʼúnsʼi} ‘buckshot’,
			\fm{attux̱ʼúnsʼi náakw} ‘black pepper’,
			and \fm{tux̱ʼwánsʼi} ‘pellets’
			from \fm{\rt{x̱ʼwan}} \~\ \fm{\rt{x̱ʼun}} ‘(wood) rot to powder’,
			(noun \fm{x̱ʼoon} ‘dry rotten wood for tanning’)
		\end{itemize}
	\item	possibly frozen in some CVC roots with coda /\ipa{sʼ}/
		\begin{itemize}
		\item	\fm{\rt{dasʼ}} in \fm{kadásʼ} ‘hail’
			possibly related to \fm{\rt{tats}} \~\ \fm{\rt{dats}} ‘knock berries off branch’
			or perhaps \fm{\rt{daᴴn}} ‘snow fall heavy; dust’
		\item	\fm{\rt{hisʼ}} ‘borrow’
			perhaps related to \fm{\rt{hiᴸ}} \~\ \fm{\rt{heᴸ}} ‘pay shaman’
		\item	\fm{\rt{ḵisʼ}} ‘flood’
			perhaps related to \fm{\rt{ḵi}} ‘plural sit’
		\item	\fm{\rt{lesʼ}} in \fm{hintaklaleisʼí} ‘first-run king salmon’
			probably related to \fm{\rt{la}} ‘melt, thaw’
		\item	\fm{\rt{sʼusʼ}} ‘thread on stick’
			(noun \fm{sʼóosʼ} ‘stick for drying’, \fm{sʼóosʼani} ‘conifer cone’)
			probably related to \fm{\rt{sʼu}} ‘thin branch; twist to limber’
		\item	\fm{\rt{tesʼ}} ‘limp, flabby’
			(noun \fm{téisʼ} ‘flab’)
			probably related to \fm{\rt{ta}} ‘fat’ (noun \fm{taaÿ} ‘fat’),
			compare \fm{\rt{tetlʼ}} ‘fat (nonhuman)’,
			\fm{téelʼ} ‘dog salmon’
		\item	\fm{\rt{tisʼ}} ‘stare’
			probably related to \fm{\rt{tin}} ‘see’
		\item	\fm{\rt{wesʼ}} in \fm{wéisʼ} ‘louse’
			perhaps related to \fm{\rt{waᴴn}} ‘maggoty’
			and \fm{woon} ‘maggot’
		\item	\fm{\rt{xisʼ}} in \fm{xéesʼ} ‘louse nit’
			compare \fm{xéen} ‘fly (insect)’ and \fm{wéisʼ} ‘louse’
		\item	\fm{\rt{x̱ʼasʼ}} in \fm{x̱ʼásʼ} ‘jaw, mandible’
			probably related to \fm{\rt{x̱ʼe}} ‘mouth’, perhaps via earlier \fm[*]{\rt{x̱ʼa-y}}
			(compare \fm{x̱ʼa-}, \fm{\rt{x̱ʼeᴴÿ}} ‘encourage’)
		\item	\fm{\rt{x̱ʼwasʼ}} ‘shed hair, go bald; cheap’
			(also \fm{shax̱ʼwáasʼ} ‘bald spot’)
			probably related to \fm{\rt{x̱ʼwalʼ}} ‘down, fluff’
		\item	\fm{\rt{ÿesʼ}} ‘dark, discoloured’
			(noun \fm{ÿéisʼ} ‘obsidian’, \fm{yétsʼ} ‘black dye’)
			perhaps related to \fm{\rt{ÿe}} ‘strange’,
			\fm{\rt{ÿa}} ‘resemble’,
			or \fm{\rt{wu}} ‘pale’
		\end{itemize}
	\end{enumerate}


\item[sa-]\label{m:sa-val}
	allomorph of valency \X{s-}

\item[sa-]\label{m:sa-voice}
	allomorph of incorporated noun \X{se-} ‘voice’

\item[se-]\label{m:se-}
	incorporated noun indicating voice or vocalization

\item[sh-]\label{m:sh-}
	valency prefix of classifier
	\begin{enumerate}
	\item	pejorative
		\begin{enumerate}
		\item	pejorative entity
		\item	pejorative eventuality
		\end{enumerate}
	\item	negative
	\item	unclear meaning
	\end{enumerate}

\item[sh=]\label{m:sh=}
	reflexive object;
	requires middle voice \fm{d-}

\item[-sh]\label{m:-sh}
	unknown suffix with very limited distribution
	\begin{enumerate}
	\item	allomorph of \X{-chʼ} in \X{-chʼán} \~\ \X{-shán};
		the form \fm{-shán} occurs instead of intensive \fm{-chʼán}
			when immediately following an ejective consonant;
		since \fm{-chʼán} might be analyzed as \X{-chʼ} + \X{-án},
		the \fm{-shán} form implies \fm{-sh} + \fm{-án};
		see \X{-shán} for more detail
	\item	unknown suffix which occurs only in the stem 
			\fm{–gwéinsh} ‘tinker, fiddle, mess with’
			which is analyzed as \fm{\rt{gwen}-μμH-sh};
		the underlying root is unknown but could plausibly
			arise from either \fm{\rt{gu}} ‘joy, enjoy, fun’
			or \fm{\rt{gu}} ‘poke, stab; sea mammals go in group’
			with unknown \X{-n}
			and predicted stem \X{-μᵉμH};
		this \fm{-sh} could plausibly arise from repetitive \X{-ch}
		\begin{itemize}
		\item	\vbform{algwéinsh}{impfv}[tr?, conj?, act]{she/he/it fiddles with him/her/it}
				\vbmorph{a-&l-&\rt{gwen}&-μμH&\gm{-sh}}
					{\xx{3>3}&\xx{xtn}&\rt{fiddle?}&\·\xx{var}&\·\xx{unkn}}
		\end{itemize}
	\end{enumerate}
	the \fm{-sh} suffix might also be identified in a few nouns:
	\begin{itemize}
	\item	\fm{gúksh} ‘corner’
		(\fm{gúk} ‘ear’?; compare \fm{gukshutú} ‘corner’)
	\item	\fm{gutguníksh} ‘kind of owl’
		(possibly onomatopoiea)
	\item	\fm{sʼíksh} ‘false hellebore, skookum root’
		(unknown root; compare \fm{sʼeek} ‘black bear’,
		\fm{\rt{sʼixw}} ‘sour’,
		\fm{\rt{sʼikw}} ‘crisp’)
	\item	\fm{wéiksh} ‘ulu’
	\end{itemize}

\item[…sh]\label{m:…sh}
	≡ \fm{d-sh-}
	combination of voice \X{d-}
		and valency \X{sh-},
	appears only as a coda consonant and so requires a preceding vowel
	\begin{itemize}
	\item	\fm{yaa sh kanx̱ashxʼáḵw} (prog; tr, \fm{n}, ach) ‘I am making myself comfortable’
			with \fm{d-sh-}\newline
		versus \fm{sh kax̱wjixʼaaḵw} (pfv) ‘I made myself comfortable’
			with \fm{d-sh-i-}
	\end{itemize}

\item[sha-]\label{m:sha-val}
	allomorph of valency \X{sh-}

\item[sha-]
	incorporated noun indicating head or hair of the head;
	derived from relational noun \fm{shá} ‘head’

\item[shakux=]
	incorporated noun ‘thirst’,
	saturates object argument;
	derived from \fm{shá} ‘head’ and \fm{\rt[¹]{kux}} ‘dry’
		in verb \fm{x̱at shaawakúx} (pfv; obj intr, \fm{∅}, ach) ‘I got thirsty’
		suggesting a nominalization \fm{shakoox} ‘thirsting’
	\begin{itemize}
	\item	\fm{ax̱ éet shakux uwaháa} (pfv; obj intr, \fm{∅}, mot) ‘thirst appeared to me’,
		i.e.\ ‘I got thirsty’
		\parencite[01/11]{leer:1973}
	\end{itemize}

\item[-shán]\label{m:-shán}
	allomorph of intensifier \X{-chʼán},
	used when immediately following an ejective consonant;
	occurs as part of the ‘extraordinary state’ derivation
		made up of:
		qualifier \X[ka-qual]{ka-}
		+ irrealis \X[u-irr]{u-}
		+ valency \X{s-}/\X{lˢ-}
		+ state \X[i-stv]{i-}
		+ intensifier \X{-chʼán} \~\ \fm{-shán}
		with \fm{g} conjugation class
		\parencite[655]{crippen:2019};
	possibly formed by combination of unknown \X{-sh}
		and restorative \X{-án};
	the form \fm{-shán} is specifically attested with the roots:
		\begin{inlinelist}
		\item	\fm{\rt{.etsʼ}} ‘handle gingerly’
		\item	\fm{\rt{tisʼ}} ‘stare’
		\item	\fm{\rt{.usʼ}} ‘wash’
		\item	\fm{\rt{walʼ}} ‘break’
		\item	\fm{\rt{wasʼ}} \~\ \fm{\rt{wusʼ}} ‘ask’
		\item	\fm{\rt{x̱itlʼ}} \~\ \fm{\rt{x̱etlʼ}} ‘fear’
		\item	\fm{\rt{x̱ʼwalʼ}} ‘fluff, down’
		\item	\fm{\rt{yeᴴn}} ‘wave’;
		\end{inlinelist}
	see \X{-chʼán} for more discussion and examples
	\begin{itemize}
	\item	\vbform{kulitéesʼshán}{impfv}[obj intr, \fm{g}, inv state]{she/he/it is fascinating to watch}
		\parencite[87.1085]{story-naish:1973}
			\vbmorph{k-&u-&lˢ-&i-&\rt[¹]{tisʼ}&-μμH&\gm{-shán}}
				{\xx{qual}&\xx{irr}&\xx{xtn}&\xx{stv}&\rt[²]{stare}&\·\xx{var}&\·\xx{intns}}
		\versus \vbform{kaawatísʼ}{pfv}[subj intr, \fm{∅}, ach]{she/he/it stared}
			\parencite[06/243]{leer:1973}
			\vbmorph{ka-&μʷ-&wa-&\rt[¹]{tisʼ}&-μH}
				{\xx{qual}&\xx{pfv}&\xx{stv}&\rt[¹]{stare}&\·\xx{var}}
	\item	\vbform{kulix̱éetlʼshán}{impfv}[obj intr, \fm{g}, inv state]{she/he/it is frightening}
		\parencite[63.732]{story-naish:1973}
			\vbmorph{k-&u-&lˢ-&i-&\rt[²]{x̱itlʼ}&-μμH&\gm{-shán}}
				{\xx{qual}&\xx{irr}&\xx{xtn}&\xx{stv}&\rt[²]{fear}&\·\xx{var}&\·\xx{intns}}
		\versus \vbform{áxʼ akoox̱dlix̱éetlʼ}{impfv}[subj intr, \fm{g}, \fm{-μμH} state]{she/he/it is afraid of it}
			\vbmorph{á&-xʼ&a-&k-&oo-&x̱-&d-&lˢ-&i-&\rt[²]{x̱itlʼ}&-μμH}
				{\xx{3n}&\·\xx{loc}&\xx{xpl}&\xx{qual}&\xx{irr}&\xx{1sg.s}&\xx{mid}&\xx{xtn}&\xx{stv}&\rt[²]{fear}&\·\xx{var}}
	\end{itemize}

\item[shi]
	≡ \fm{sh-i-}
	combination of valency \X{sh-}
		and stative \X[i-stv]{i-}

\item[si]
	≡ \fm{s-i-}
	combination of valency \X{s-}
		and stative \X[i-stv]{i-}
\end{morphdesc}

\subsection{T}\label{sec:alphalist-t}
\begin{morphdesc}[resume*=alphalist]
\item[-t]\label{m:-t}
	ictive repetitive suffix describing repeated contact with a target;
	unlike the \X{-ch}, \X{-k}, and \X{-x̱} repetitive suffixes,
		this suffix is never specified by conjugation class
		or as part of a motion derivation;
	called ‘ictive’ from Latin \fm{ictus} ‘a blow, a strike, a beat’
		and so glossed \xx{ict},
		but may be instead be glossed \xx{rep} like other repetitive suffixes;
	apparently productive given its occurrence with a fairly large number of verbs,
		but the grammatical and semantic details have yet to be analyzed;
	historically related to the punctual postposition \fm{-t} ‘at/to/around a point’
		though the two are now functionally independent;
	plausibly related to the obscure suffix \X{-át} but \fm{-t} normally does not occur
		with an epenthetic (filler) vowel
	\begin{enumerate}
	\item	repetitive suffix with any verb that denotes an event involving
			striking (or aiming at) a target;
		appears in a repetitive imperfective form
			or in forms derived from the repetitive imperfective
			(secondary aspectual derivations, 
			\citeauthor{leer:1991}’s (\citeyear{leer:1991}) “epiaspect”);
		attested with at least 35 different verb roots
			including both activity and achievement verbs,
			compare to verbs that occur with \X{-x̱aa};
		may be limited to transitive verbs but this is unverified;
		predominantly with \fm{∅} conjugation class though there are also examples
			with the three non-\fm{∅} classes (\fm{n}, \fm{g̱}, \fm{g});
		as with other repetitives there may be multiple entities involved (plural),
			or a single entity multiple times (pluractional),
			or both;
		\begin{itemize}
		\item	\fm{–.únt} ‘repeatedly shoot and hit (with gun)’
			from \fm{\rt{.un}} ‘shoot (gun)’ in
			\newline
			\vbform{a.únt}{rep impfv}[tr, \fm{∅}, ach]{she/he/it repeatedly shoots and hits him/her/it}
			\parencites[63]{leer:1963}[58]{story:1966}[02/312]{leer:1973}[154]{leer:1976}
				\vbmorph{a-&\rt[²]{.un}&-μH&\gm{-t}}
					{\xx{3>3}&\rt[²]{shoot}&\·\xx{var}&\·\xx{ict}}
			\exand \vbform{aawa.únt}{rep pfv}{she/he/it repeatedly shot and hit him/her/it}
			\parencite[02/312]{leer:1973}
				\vbmorph{a-&μʷ-&wa-&\rt[²]{.un}&-μH&\gm{-t}}
					{\xx{3>3}&\xx{pfv}&\xx{stv}&\rt[²]{shoot}&\·\xx{var}&\·\xx{rep}}
			\versus \vbform{a.únx̱}{rep impfv}{she/he/it is repeatedly shooting him/her/it}
			\parencites[02/312]{leer:1973}[153]{leer:1976}
				\vbmorph{a-&\rt[²]{.un}&-μH&-x̱}
					{\xx{3>3}&\rt[²]{shoot}&\·\xx{var}&\·\xx{rep}}
			\exand \vbform{aawa.ún}{pfv}{she/he/it shot him/her/it}
				\vbmorph{a-&μʷ-&wa-&\rt[²]{.un}&-μH}
					{\xx{3>3}&\xx{pfv}&\xx{stv}&\rt[²]{shoot}&\·\xx{var}}
			\newline
			also derived noun \fm{at.úndi} ‘shooter’ \parencites[58]{story:1966}[02/312]{leer:1973}
				\vbmorph{at=&\rt{.un}&-μH&\gm{-t}&-i}
					{\xx{ind.n.o}&\rt{shoot}&\·\xx{var}&\·\xx{ict}&\·\xx{nmz}}
		\item	\fm{–sʼélʼt} ‘repeatedly tear, rip’
			from \fm{\rt{sʼelʼ}} ‘tear, rip’ in
			\newline
			\vbform{asʼélʼt}{rep impfv}[tr, \fm{n}, \fm{-μμH} act]{she/he/it repeatedly tears, rips him/her/it}
			\parencites[224.3161]{story-naish:1973}[09/218]{leer:1973}[518]{leer:1976}
				\vbmorph{a-&\rt{sʼelʼ}&-μH&\gm{-t}}
					{\xx{3>3}&\rt{tear}&\·\xx{var}&\·\xx{ict}}
			\versus \vbform{asʼéilʼ}{impfv}{she/he/it is tearing, ripping him/her/it}
			\parencites[09/218]{leer:1973}[518]{leer:1976}
				\vbmorph{a-&\rt{sʼelʼ}&-μμH}
					{\xx{3>3}&\rt{tear}&\·\xx{var}}
			\newline
			also nouns \fm{sʼéilʼ} ‘rip, tear, wound’
				and \fm{sʼélʼ} ‘rubber’
		\item	roots attested with ictive \fm{-t} in a repetitive imperfective forms
			include \parencite[532]{crippen:2019}:
			\begin{inlinelist}
			\item	\fm{\rt{chʼex̱ʼ}} ‘point’
			\item	\fm{\rt{dax̱}} ‘adze out’
			\item	\fm{\rt{dlakw}} ‘scratch’
			\item	\fm{\rt{dzuᴸ}} ‘throw missile’ (\fm{–dzeit})
			\item	\fm{\rt{guᴴk}} ‘peck’
			\item	\fm{\rt{han}} ‘cut into strips’
			\item	\fm{\rt{jux̱ʼ}} ‘sling missile’
			\item	\fm{\rt{kitʼ}} ‘pry’
			\item	\fm{\rt{kʼix̱ʼ}} \~\ \fm{\rt{kʼex̱ʼ}} ‘gaff, snag’
			\item	\fm{\rt{ḵasʼ}} ‘split’
			\item	\fm{\rt{ḵʼish}} ‘swat’
			\item	\fm{\rt{sʼaxw}} ‘stack’
			\item	\fm{\rt{sʼelʼ}} ‘tear’
			\item	\fm{\rt{sʼuᴴw}} ‘chop’
			\item	\fm{\rt{taḵ}} ‘poke, spear’
			\item	\fm{\rt{teḵʼ}} ‘twist’
			\item	\fm{\rt{tex̱ʼ}} ‘wring’
			\item	\fm{\rt{taxʼ}} ‘bite’
			\item	\fm{\rt{tʼach}} ‘slap, swim’
			\item	\fm{\rt{tʼaxʼ}} ‘flick’
			\item	\fm{\rt{tʼax̱ʼ}} ‘pop’
			\item	\fm{\rt{tʼex̱ʼ}} ‘pound’
			\item	\fm{\rt{tʼiᴴÿ}} ‘elbow’
			\item	\fm{\rt{tʼuᴴk}} ‘shoot (bow)’
			\item	\fm{\rt{tsaḵ}} ‘poke’
			\item	\fm{\rt{tsix̱}} \~\ \fm{\rt{tsex̱}} ’kick’
			\item	\fm{\rt{tsix̱ʼ}} \~\ \fm{\rt{tsex̱ʼ}} ‘strangle’
			\item	\fm{\rt{tsuᴴw}} ‘push’
			\item	\fm{\rt{tsux̱}} ‘block’
			\item	\fm{\rt{tsʼikʼw}} \~\ \fm{\rt{tsʼukʼ}} ‘pinch’
			\item	\fm{\rt{.uᴴn}} ‘shoot (gun)’
			\item	\fm{\rt{walʼ}} ‘break’
			\item	\fm{\rt{xitʼ}} ‘sweep’
			\item	\fm{\rt{xʼasʼ}} ‘slice’
			\item	\fm{\rt{x̱ich}} ‘club’
			\item	\fm{\rt{x̱utʼ}} ‘adze’
			\end{inlinelist}
		\end{itemize}
	\item	repetitive suffix in some tendency state imperfectives based on CVC roots
			instead of \X{-k} \~\ \X{-kw}
			or \X{-kt} \~\ \X{-kwt} with CV roots (see below)
		\begin{itemize}
		\item	\fm{–níkt} ‘tattle’
			from \fm{\rt{nik}} ‘tell, report’ in
			\newline
			\vbform{dliníkt}{rep impfv}[subj intr, conj?, state]{she/he/it tattles}
			\parencites[58]{story:1966}[101.1296]{story-naish:1973}
				\vbmorph{d-&l-&i-&\rt{nik}&-μH&\gm{-t}}
					{\xx{mid}&\xx{xtn}&\xx{stv}&\rt{tell}&\·\xx{var}&\·\xx{ict}}
		\item	\fm{–.úkt} ‘boil easily’
			from \fm{\rt[²]{.uᴴk}} ‘boil’ in
			\newline
			\vbform{dli.úkt}{rep impfv}[obj intr, conj?, state]{she/he/it boils easily}
			\parencites[02/181]{leer:1973}[167]{leer:1976}
				\vbmorph{d-&l-&i-&\rt{.uᴴk}&-μH&\gm{-t}}
					{\xx{mid}&\xx{xtn}&\xx{stv}&\rt{boil}&\·\xx{var}&\·\xx{ict}}
		\item	\fm{–.áḵwt} ‘quick to plan’
			from \fm{\rt[²]{.aḵw}} ‘plan, direct, command, try’ in
			\newline
			\vbform{at kaÿa.áḵwt}{rep impfv}[tr, \fm{n}?, state]{she/he/it is quick to plan/suggest things}
			\parencite[123]{leer:1976}
				\vbmorph{at=&ka-&ÿa-&\rt[²]{.aḵw}&-μH&\gm{-t}}
					{\xx{ind.n.o}&\xx{qual}&\xx{stv}&\rt[²]{plan}&\·\xx{var}&\·\xx{ict}}
		\item	\fm{–níkwt} ‘gets sick easily’
			from \fm{\rt[¹]{nikw}} ‘sick’ in
			\newline
			\vbform{yaníkwt}{impfv}[obj intr, \fm{n}?, state]{she/he/it gets sick easily}
			\parencite[04/177]{leer:1973}
				\vbmorph{ÿa-&\rt[¹]{nikw}&-μH&\gm{-t}}
					{\xx{stv}&\rt[¹]{sick}&\·\xx{var}&\·\xx{ict}}
		\end{itemize}
	\item	unclear meaning with two verb roots of unknown meaning
		\begin{itemize}
		\item	\fm{–tált} ‘dissuade’
			from unknown \fm{\rt{tal}} in
			\newline
			\vbform{atált}{rep impfv}[tr, \fm{n}, inv act]{she/he/it is trying to dissuade, discourage him/her/it}
			\parencites[06/72]{leer:1973}[69.831]{story-naish:1973}
				\vbmorph{a-&\rt{tal}&-μH&\gm{-t}}
					{\xx{3>3}&\rt{dissuade?}&\·\xx{var}&\·\xx{ict}}
			\versus \vbform{ash wootált}{pfv}{she/he/it dissuaded, discouraged him/her/it}
			\parencites[06/72]{leer:1973}
				\vbmorph{ash=&wu-&μ-&\rt{tal}&-μH&\gm{-t}}
					{\xx{3prx.o}&\xx{pfv}&\xx{stv}&\rt{dissuade?}&\·\xx{var}&\·\xx{ict}}
		\item	\fm{–shéesht} ‘squeeze out’
			from unknown \fm{\rt{shish}} in
			\newline
			\vbform{alshéesht}{impfv}[tr, conj?, inv act]{she/he/it is cleaning him/her/it out by squeezing contents}
			\parencites[552]{leer:1976}
				\vbmorph{a-&lˢ-&\rt{shish}&-μμH&\gm{-t}}
					{\xx{3>3}&\xx{csv}&\rt{squeeze.out}&\·\xx{var}&\·\xx{ict}}
			\versus \vbform{awlishéesht}{pfv}{she/he/it is cleaned him/her/it out by squeezing contents}
			\parencites[10/88]{leer:1973}[552]{leer:1976}
				\vbmorph{a-&w-&lˢ-&i-&\rt{shish}&-μμH&\gm{-t}}
					{\xx{3>3}&\xx{pfv}&\xx{csv}&\xx{stv}&\rt{squeeze.out}&\·\xx{var}&\·\xx{ict}}
			\newline
			compare 
			\begin{inlinelist}
			\item	\fm{\rt{shisʼ}} ‘strip, squeeze out’
				(thus likely \fm{–shéesht} < \fm[*]{\rt{shishʼ}} with */\ipa{ʃʼ}/)
			\item	\fm{\rt{shiᴴsh}} ‘try to outdo in eating competition’
				(\fm{ᴴ} suggests \fm[*]{\rt{shiˀsh}} < \fm[*]{\rt{shishʼ}})
			\item	\fm{shéesht} ‘lucky gambling stick’
			\item	\fm{keishísh} ‘mountain alder’
			\item	\fm{sʼelasheesh} ‘flathead duck’
			\item	\fm{sheesh} ‘inconnu’
			\item	\fm{\rt{shish}} ‘skim off, sip, slurp’
			\end{inlinelist}
		\end{itemize}
	\item	in combination with \X{-kw} as \X{-kwt} \~\ \X{-kt},
		occurring with multipositional repetitive state imperfectives
		and with tendency state imperfectives;
		see \X{-kwt} \~\ \X{-kt} for details
	\item	in some complex nouns derived from verbs
		\begin{itemize}
		\item	\fm{atkʼíx̱ʼdi} ‘gaffer, gaff fisherman’
			from \fm{\rt[²]{kʼix̱ʼ}} \~\ \fm{\rt[²]{kʼex̱ʼ}} ‘gaff, snag’
			\parencites[f04/130]{leer:1973}
			\vbmorph{at=&\rt[²]{kʼix̱ʼ}&-μH&\gm{-t}&-i}
				{\xx{ind.n.o}&\rt[²]{gaff}&\·\xx{var}&\·\xx{ict}&\·\xx{nmz}}
			\newline
			compare \vbform{at kʼéx̱ʼt}{rep impfv}[tr, \fm{∅}/\fm{g}, ach]{she/he/it is repeatedly gaffing something}
			\parencites[91.1149]{story-naish:1973}[f04/130]{leer:1973}[778]{leer:1976}
				\vbmorph{at=&\rt[²]{kʼex̱ʼ}&-μH&\gm{-t}}
					{\xx{ind.n.o}&\rt[²]{gaff}&\·\xx{var}&\·\xx{ict}}
			\exand
			\begin{inlinelist}
			\item	\fm{kʼéx̱ʼaa} ‘gaff hook’
			\item	\fm{kakʼéx̱ʼaa} ‘crochet hook’
			\item	\fm{wóoshnáx̱ x̱ʼakakʼéix̱'i} ‘chain’
			\end{inlinelist}
		\item	\fm{kalḵásʼdi} ‘splitting knife/stick’
			from \fm{\rt[¹]{ḵasʼ}} ‘split; stick’
			\vbmorph{ka-&lˢ-&\rt[¹]{ḵasʼ}&-μH&\gm{-t}&-i}
				{\xx{qual}&\xx{csv}&\rt[¹]{split}&\·\xx{var}&\·\xx{ict}&\·\xx{nmz}}
			\newline
			compare \vbform{aklaḵásʼt}{rep impfv}[tr, \fm{∅}/\fm{n}, ach]{she/he/it splits him/her/it}
			\parencites[205.2866]{story-naish:1973}[f01/47]{leer:1973}
				\vbmorph{a-&k-&lˢa-&\rt[¹]{ḵasʼ}&-μH&\gm{-t}}
					{\xx{3>3}&\xx{qual}&\xx{csv}&\rt[¹]{split}&\·\xx{var}&\·\xx{ict}}
		\item	\fm{taay kasʼúkdi} ‘cooked fat’
			from \fm{\rt[¹]{sʼikw}} \~\ \fm{\rt[¹]{sʼuk}} ‘crisp, toast, fry to crisp’
			\vbmorph{taaÿ&ka-&\rt{sʼuk}&-μH&\gm{-t}&-i}
				{fat&\xx{qual}&\rt{crisp}&\·\xx{var}&\·\xx{ict}&\·\xx{poss}}
			\newline
			compare \vbform{akawlisʼúk}{pfv}[tr, \fm{∅}, ach]{she/he/it crisped, toasted, fried him/her/it}
			\parencites[97.1243, 231.3282]{story-naish:1973}[09/250–251]{leer:1973}[523]{leer:1976}
			\vbmorph{a-&ka-&w-&lˢ-&i-&\rt[¹]{sʼuk}&-μH}
				{\xx{3>3}&\xx{qual}&\xx{pfv}&\xx{csv}&\xx{stv}&\rt[¹]{crisp}&\·\xx{var}}
			\newline
			no verb based on this root is attested with \fm{-t} but this predicts
				repetitive imperfective forms like \fm{akasʼúkt} and \fm{aklasʼúkt}
		\item	\fm{xákwdi} ‘empty shell of house, empty container’
			probably from \fm{\rt{xak}} \~\ \fm{\rt{xakw}} ‘skeleton, empty shell; dessicated’
			\vbmorph{\rt{xakw}&-μH&\gm{-t}&-i}
				{\rt{dessicated}&\·\xx{var}&·\xx{ict}&\·\xx{nmz}}
			\newline
			compare
			\begin{inlinelist}
			\item	\fm{ḵaa shakaxaagú} ‘skull’
				\parencite[f03/37]{leer:1973}
			\item	\fm{xákw} ‘sandbar’
				\parencite[f03/36]{leer:1973};
			\end{inlinelist}
			further compare
			\begin{inlinelist}
			\item	\fm{xáak} ‘empty shell’
			\item	\fm{du xaagí} ‘her/his skeleton’
			\item	\vbform{wusixaak}{pfv}[obj intr, \fm{g̱}, ach]{she/he/it became a dried out shell, skeleton}
				\parencite[193.2687]{story-naish:1973}
			\item	\vbform{ḵukaawaxaak}{pfv}[impers, conj?, ach?]{weather became dry and crisp}
				\parencite[76.923]{story-naish:1973}
			\item	\fm{ḵuxaak} ‘dry weather’
				\parencite[f03/30]{leer:1973}
			\item	\vbform{ashxáak}{impfv}[tr, \fm{∅}, \fm{-μμH} act]{she/he/it is steaming him/her/it (shell) open}
				\parencite[f03/28]{leer:1973}
			\item	\vbform{x̱ʼawsixaak}{pfv}[obj intr?, conj?, ach?]{his/her/its mouth fell open (got dry?)}
				\parencite[f03/29]{leer:1973}
			\end{inlinelist}
		\item	\fm{atxáshdi} ‘cut leather’
			from \fm{\rt{xash}} ‘cut’
			\vbmorph{at=&\rt{xash}&-μH&\gm{-t}&-i}
				{\xx{ind.n.o}&\rt{cut}&\·\xx{var}&\·\xx{ict}&\·\xx{nmz}}
			\newline
			compare
			\vbform{axáash}{impfv}[tr, \fm{n}/\fm{∅}, \fm{-μμH} act]{she/he/it cuts him/her/it}
			\vbmorph{a-&\rt{xash}&-μμH}
				{\xx{3>3}&\rt{cut}&\·\xx{var}}
			\parencite[613]{leer:1976}
			\newline
			no verb based on this root is attested with \fm{-t} but this predicts
				repetitive imperfective forms like \fm{axásht} and \fm{akaxásht}
		\item	\fm{kaxʼásʼdi} ‘lumber’
			from \fm{\rt[²]{xʼasʼ}} ‘slice’
			\vbmorph{ka-&\rt[²]{xʼasʼ}&-μH&\gm{-t}&-i}
				{\xx{qual}&\rt{slice}&\·\xx{var}&\·\xx{ict}&\·\xx{nmz}}
			\exand \fm{sh kadaxʼásʼdi hít} ‘sawmill’ = ‘house that slices itself’
			\parencite[f04/20]{leer:1973}
			\vbmorph{sh=&ka-&da-&\rt[²]{xʼasʼ}&-μH&\gm{-t}&-i&hít}
				{\xx{rflx.o}&\xx{qual}&\xx{mid}&\rt[²]{slice}&\·\xx{var}&\·\xx{ict}&\·\xx{rel}&house}
			\newline
			compare \vbform{akaxʼásʼt}{rep impfv}[tr, \fm{∅}/\fm{g̱}, \fm{-μμH} act]{she/he/it repeatedly slices him/her/it}
			\parencites[173.2384]{story-naish:1973}[739]{leer:1976}
			\vbmorph{a-&ka-&\rt[²]{xʼasʼ}&-μH&\gm{-t}}
				{\xx{3>3}&\xx{qual}&\rt[²]{slice}&\·\xx{var}&\·\xx{rep}}
		\item	\fm{kax̱ʼeiltí} ‘crumbs’
			from \fm{\rt[²]{x̱ʼeᴴl}} ‘crumble, break into pieces’
			(but \X{-μμL})
			\parencite[f01/163]{leer:1973}
			\vbmorph{ka-&\rt[²]{x̱ʼeᴴl}&-μμL&\gm{-t}&-í}
				{\xx{sro}&\rt[²]{crumble}&\·\xx{var}&\·\xx{ict}&\·\xx{nmz}}
			\newline
			compare
			\vbform{akaawax̱ʼéil}{pfv}[tr, \fm{g̱}, ach]{she/he/it crumbled, broke him/her/it into pieces}
			\parencite[f01/163]{leer:1973}
			\vbmorph{a-&ka-&μʷ-&wa-&\rt[²]{x̱ʼeᴴl}&-μμH}
				{\xx{3>3}&\xx{sro}&\xx{pfv}&\xx{stv}&\rt[²]{crumble}&\·\xx{var}}
			\exand \vbform{yei kx̱ax̱ʼéilt}{rep impfv}{I’m (trying to) break it apart}
			\parencite[f01/163]{leer:1973}
			\vbmorph{yei=&k-&x̱a-&\rt[²]{x̱ʼeᴴl}&-μμH&\gm{-t}}
				{down&\xx{sro}&\xx{1sg.s}&\rt[²]{crumble}&\·\xx{var}&\·\xx{ict}}
		\end{itemize}
	\item	possibly analyzable in some CVCC nouns with unexplained coda /\ipa{t}/
		\begin{itemize}
		\item	\fm{káast} ‘barrel’
			from unknown \fm{\rt{kas}}
		\item	\fm{láḵt} ‘bentwood box’
			from unknown \fm{\rt{laḵ}};
			compare \fm{laax̱} ‘redcedar’
		\item	\fm{núkt} ‘dusky grouse’
			perhaps from \fm{\rt{nuk}} ‘sg.\ sit’;
			compare \fm{–núkts} ‘sweet’ (with \X{-ts})
			and \fm{–núkch} ‘helpless’ (with \X{-ch})
		\item	\fm{síxwdi} \~\ \fm{sáxwdi} \~\ \fm{súxdi} ‘handle, shaft’
			from unknown \fm{\rt{sixw}} \~\ \fm{\rt{saxw}};
			compare \fm{saaxw} ‘cockles’,
			\fm{séek} ‘belt’
		\item	\fm{shéesht} ‘lucky gambling stick’
			from unknown \fm{\rt{shish}};
			see \fm{–shéesht} ‘squeeze out’ above for more detail
		\end{itemize}
	\end{enumerate}

\item[-tʼ]\label{m:-tʼ}
	repetitive suffix only occurring with specific verbs;
	unlike the \X{-ch}, \X{-k}, and \X{-x̱} repetitive suffixes,
		this suffix is never specified by conjugation class
		or as part of a motion derivation;
	for some verbs \fm{-tʼ} occurs instead of
		the conjugation class-specific repetitive suffix (\fm{-ch}, \fm{-k}, \fm{-x̱})
		but other verbs can use either the conjugation class-specific suffix
		or \fm{-tʼ};
	this suffix seems to be productive
		and apparently involves the destruction of an entity
		but there are some puzzling exceptions (especially \fm{–kweitʼ} ‘know’ below);
	there may be a historical relationship 
		between \X{-lʼ}, \X{-sʼ}, \X{-tʼ}, and \X{-xʼ}
		since all of these are ejective
	\begin{enumerate}
	\item	repetitive suffix attested with eight roots,
		appearing in a repetitive imperfective form
		or in forms derived from the repetitive imperfective
			(secondary aspectual derivations, 
			\citeauthor{leer:1991}’s (\citeyear{leer:1991}) “epiaspect”);
		the meaning involves destruction or dissolution of an entity
			with the sole exception of \fm{–kweitʼ} ‘know’
			where \fm{-tʼ} is apparently just iterative;
		as with other repetitives there may be multiple entities involved (plural),
			or a single entity multiple times (pluractional),
			or both
		\begin{itemize}
		\item	\fm{–gántʼ} ‘burn’
			from \fm{\rt[¹]{gan}} ‘burn’ in
			\newline
			\vbform{aksagántʼ}{rep impfv}[tr, \fm{n}, ach]{she/he/it repeatedly burns him/her/it}
			\parencite[38.361]{story-naish:1973}
				\vbmorph{a-&k-&sa-&\rt[¹]{gan}&-μH&\gm{-tʼ}}
					{\xx{3>3}&\xx{qual}&\xx{csv}&\rt[¹]{burn}&\·\xx{var}&\·\xx{rep}}
			\versus \vbform{akawsigaan}{pfv}{she/he/it burned him/her/it}
				\vbmorph{a-&ka-&w-&s-&i-&\rt[¹]{gan}&-μμL}
					{\xx{3>3}&\xx{qual}&\xx{pfv}&\xx{csv}&\xx{stv}&\rt[¹]{burn}&\·\xx{var}}
			\newline
			also in nouns:
			\begin{itemize}
			\item	\fm{káx̱ gántʼi} ‘something roasted’
				\parencite[f05/20]{leer:1973}
				\vbmorph{ká&-x̱&\rt{gan}&-μH&\gm{-tʼ}&-i}
					{\xx{hsfc}&\·\xx{pert}&\rt{burn}&\·\xx{var}&\·\xx{rep}&\·\xx{nmz}}
			\item	\fm{koogántʼi} ‘burned area’
				\parencite[223.3147]{story-naish:1973}
				\vbmorph{ka-&u-&\rt{gan}&-μH&\gm{-tʼ}&-i}
					{\xx{hsfc}&\xx{irr}&\rt{burn}&\·\xx{var}&\·\xx{rep}&\·\xx{nmz}}
				\newline
				\citeauthor{story-naish:1973} mistranslate this as ‘windfall’;
				\textcite[T·39]{leer-hitch-ritter:2001} gives \fm{koogánti} 
					with \X{-t} instead of \X{-tʼ}
			\end{itemize}
		\item	\fm{–héitʼ} ‘erase’
			from \fm{\rt[¹]{ha}} ‘(dis)appear, move imperceptibly’ in
			\newline
			\vbform{aadáx̱ as.héitʼ}{rep impfv}[tr, \fm{n}, ach]{she/he/it is erasing it from it}
			\parencite[01/7]{leer:1973}
				\vbmorph{aa&-dáx̱&a-&s-&\rt[¹]{ha}&-μμᵉH&\gm{-tʼ}}
					{\xx{3n}&\·\xx{abl}&\xx{3>3}&\xx{csv}&\rt[¹]{disappear}&\·\xx{var}&\·\xx{rep}}
			\exand \vbform{aadáx̱ ashéix̱}{rep impfv}{she/he/it is erasing it from it}
			\parencite[7]{leer:1976}
				\vbmorph{aa&-dáx̱&a-&s-&\rt[¹]{ha}&-μμᵉH&-x̱}
					{\xx{3n}&\·\xx{abl}&\xx{3>3}&\xx{csv}&\rt[¹]{disappear}&\·\xx{var}&\·\xx{rep}}
			\versus \vbform{aadáx̱ awsihaa}{pfv}{she/he/it erased it from it}
			\parencite[01/7]{leer:1973}
				\vbmorph{aa&-dáx̱&a-&w-&s-&i-&\rt[¹]{ha}&-μμL}
					{\xx{3n}&\·\xx{abl}&\xx{3>3}&\xx{pfv}&\xx{csv}&\xx{stv}&\rt[¹]{disappear}&\·\xx{var}}
			\newline
			\textcite[7]{leer:1976} also translates these as
				“he is making off with them”
				and “made off with it (took it without owner’s knowledge)”;
			\textcite[80.990]{story-naish:1973} add “rub off”;
			with areal \X[ḵu-areal]{ḵu-} as the object the meaning is ‘remove dirt, polish’
				but this is only attested with \X{-x̱} and not \fm{-tʼ}
		\item	\fm{–kélʼtʼ} ‘ash’
			from \fm{\rt{kelʼ}} ‘ash’ in
			\newline
			\vbform{ashkélʼtʼ}{rep impfv}[tr, \fm{∅}/\fm{g̱}, ach]{she/he/it makes ash of him/her/it}
				\vbmorph{a-&sh-&\rt{kelʼ}&-μH&\gm{-tʼ}}
					{\xx{3>3}&\xx{pej}&\rt{ash}&\·\xx{var}&\·\xx{rep}}
			\versus \vbform{awshikélʼ}{pfv}[\fm{∅}]{she/he/it made ash of him/her/it}
				\vbmorph{a-&w-&sh-&i-&\rt{kelʼ}&-μH}
					{\xx{3>3}&\xx{pfv}&\xx{pej}&\xx{stv}&\rt{ash}&\·\xx{var}}
			\newline
			the root \fm{\rt{kelʼ}} can mean ‘undo, untie, unravel’
				and ‘flee, chase’
				as well as ‘ash’,
				but \fm{-tʼ} occurs only with the ‘ash’ meaning;
			the verb is attested both as \fm{∅} conjugation class (perfective stem \fm{-μH})
				and as \fm{g̱} conjugation class (\fm{yei=}…\fm{-ch}, perfective stem \fm{-μμH})
			\newline
			also in nouns:
			\begin{itemize}
			\item	\fm{kélʼtʼ} ‘ash’
				(also \fm{kéilʼ} ‘ash; dandruff’ (Southern, Tongass) without \fm{-tʼ})
				\vbmorph{\rt{kelʼ}&-μH&\gm{-tʼ}}
					{\rt{ash}&\·\xx{var}&\·\xx{rep}}
			\item	\fm{dleit kakélʼtʼ} ‘light snow’
				\vbmorph{dleit&ka-&\rt{kelʼ}&-μH&\gm{-tʼ}}
					{snow&\xx{hsfc}&\rt{ash}&\·\xx{var}&\·\xx{rep}}
			\item	\fm{sʼaḵkélʼtʼi} \~\ \fm{sʼax̱kélʼtʼi} ‘fine powder, bone ash’
				\vbmorph{sʼaḵ-&\rt{kelʼ}&-μH&\gm{-tʼ}&-i}
					{bone&\rt{ash}&\·\xx{var}&\·\xx{rep}&\·\xx{poss}}
			\item	\fm{sʼeeḵ x̱ʼakélʼtʼi} ‘cigarette ash’
				\vbmorph{sʼeeḵ&x̱ʼe-&\rt{kelʼ}&-μH&\gm{-tʼ}&-i}
					{smoke&mouth&\rt{ash}&\·\xx{var}&\·\xx{rep}&\·\xx{poss}}
			\end{itemize}
		\item	\fm{–kweitʼ} ‘know’
			from \fm{\rt{kuᴸ}} ‘know’ in
			\newline
			\vbform{ash ée askweitʼ}{rep impfv}[tr, \fm{∅⁺}, ach]{she/he/it is acquainting him/her/it with him/her/it}
				\parencite[f06/126]{leer:1973}
				\vbmorph{ash&ee&-H&a-&s-&s-&\rt{kuᴸ}&-μμᵉL&\gm{-tʼ}}
					{\xx{3prx}&\xx{base}&\·\xx{loc}&\xx{3>3}&\xx{appl}&\xx{xtn}&\rt{know}&\·\xx{var}&\·\xx{rep}}
			\versus \vbform{ash ée awsikóo}{pfv}{she/he/it acquainted him/her/it with him/her/it}
				\parencite[716]{leer:1976}
				\vbmorph{ash&ee&-H&a-&w-&s-&s-&i-&\rt{kuᴸ}&-μμH}
					{\xx{3prx}&\xx{base}&\·\xx{loc}&\xx{3>3}&\xx{pfv}&\xx{appl}&\xx{xtn}&\xx{stv}&\rt{know}&\·\xx{var}}
		\item	\fm{–láxwtʼ} ‘starve to death’
			from \fm{\rt[¹]{laxw}} ‘starve’ in
			\newline
			\vbform{a x̱oo aa láxwtʼ}{rep impfv}[obj intr, \fm{∅}, ach]{some among them are starving to death}
			\parencite[265.1]{swanton:1909}
				\vbmorph{a&x̱oo&aa&\rt{laxw}&-μH&\gm{-tʼ}}
					{\xx{3n}&among&\xx{part}&\rt{starve}&\·\xx{var}&\·\xx{rep}}\
			\versus \vbform{uwaláxw}{pfv}{she/he/it starved}
				\vbmorph{u-&wa-&\rt{laxw}&-μH}
					{\xx{zpfv}&\xx{stv}&\rt{starve}&\·\xx{var}}	
		\item	\fm{–náatʼ} ‘plural die’
			from \fm{\rt[¹]{na⁽ʷ⁾}} ‘die’ in
			\newline
			\vbform{has náatʼ}{rep impfv}[obj intr, \fm{n}, ach]{they (always) die}
			\parencite[56]{leer:1963}
				\vbmorph{has=&\rt[¹]{na⁽ʷ⁾}&-μμH&\gm{-tʼ}}
					{\xx{plh}&\rt[¹]{die}&\·\xx{var}&\·\xx{rep}}
			\exand \vbform{has woonáatʼ}{rep pfv}{they died}
			\parencite[58]{leer:1963}
				\vbmorph{has=&wu-&μ-&\rt[¹]{na⁽ʷ⁾}&-μμH&\gm{-tʼ}}
					{\xx{plh}&\xx{pfv}&\xx{stv}&\rt[¹]{die}&\·\xx{var}&\·\xx{rep}}
			\exand \vbform{dax̱ woonáatʼ}{rep pfv}{they died}
			\parencite[58]{leer:1963}
				\vbmorph{dax̱=&wu-&μ-&\rt[¹]{na⁽ʷ⁾}&-μμH&\gm{-tʼ}}
					{\xx{dpl}&\xx{pfv}&\xx{stv}&\rt[¹]{die}&\·\xx{var}&\·\xx{rep}}
			\versus \vbform{has woonaa}{pfv}{they died}
				\vbmorph{has=&wu-&μ-&\rt[¹]{na⁽ʷ⁾}&-μμL}
					{\xx{plh}&\xx{pfv}&\xx{stv}&\rt[¹]{die}&\·\xx{var}}
		\item	\fm{–.úwtʼ} ‘buy’
			from \fm{\rt[²]{.uw}} ‘buy’ in
			\newline
			\vbform{da.úwtʼ}{rep impfv}[subj intr, \fm{∅}?, ach]{she/he/it buys a bunch at a time}
			\parencite[149]{leer:1976}
				\vbmorph{da-&\rt[²]{.uw}&-μH&\gm{-tʼ}}
					{\xx{apsv}&\rt[²]{buy}&\·\xx{var}&\·\xx{rep}}
			\versus \vbform{aawa.úw}{pfv}[\fm{∅}]{she/he/it bought him/her/it}
				\vbmorph{a-&μʷ-&wa-&\rt[²]{.uw}&-μH}
					{\xx{3>3}&\xx{pfv}&\xx{stv}&\rt[²]{buy}&\·\xx{var}}
			\exand \vbform{a.óow}{impfv}[tr, \fm{n}, inv act]{she/he/it buys him/her/it}
				\vbmorph{a-&\rt[²]{.uw}&-μμH}
					{\xx{3>3}&\rt[²]{buy}&\·\xx{var}}
		\item	\fm{–xóoshtʼ} ‘scorch, singe’
			from \fm{\rt{xuᴴsh}} ‘scorch, singe’ in
			\newline
			\vbform{alxóoshtʼ}{rep impfv}[obj intr, \fm{n}/\fm{g̱}, ach]{she/he/it repeatedly scorches, singes it}
			\parencite[635]{leer:1976}
				\vbmorph{a-&lˢ-&\rt{xuᴴsh}&-μμH&\gm{-tʼ}}
					{\xx{3>3}&\xx{csv}&\rt{scorch}&\·\xx{var}&\·\xx{rep}}
			\exand \vbform{awlixóoshtʼ}{rep pfv}{she/he/it repeatedly scorched, singed him/her/it}
			\parencite[f03/125]{leer:1973}
				\vbmorph{a-&w-&lˢ-&i-&\rt{xuᴴsh}&-μμH&\gm{-tʼ}}
					{\xx{3>3}&\xx{pfv}&\xx{xtn}&\xx{stv}&\rt{scorch}&\·\xx{var}&\·\xx{rep}}
			\versus \vbform{awlixóosh}{pfv}{she/he/it scorched, singed it}
				\vbmorph{a-&w-&lˢ-&i-&\rt{xuᴴsh}&-μμH}
					{\xx{3>3}&\xx{pfv}&\xx{xtn}&\xx{stv}&\rt{scorch}&\·\xx{var}}
			\newline
			also noun noun \fm{xóoshtʼ} ‘scorched or singed matter’;
			the unusual stem variation \fm{-μμH} instead of \fm{-μH} expected with \fm{-tʼ}
				in the verb forms suggests a derivation process,
				probably from the noun \fm{xóoshtʼ} to the verbs,
				but this does not explain the stem variation of the noun;
			compare related \fm{\rt{xuᴴts}} ‘char’ and \fm{\rt{xwaᴴts}} ‘paint black’
		\end{itemize}
	\item	in some complex nouns apparently derived from unknown verbs
		\begin{itemize}
		\item	\fm{daakaleiltʼí} ‘husk, chaff, dried remains of berry’
			from \fm{\rt{lel}} ‘limp, lax, baggy, flabby’
			\vbmorph{daa-&ka-&\rt{lel}&-μμL&\gm{-tʼ}&-í}
				{around&\xx{hsfc}&\rt{limp}&\·\xx{var}&\·\xx{rep}&\·\xx{poss}}
			\newline
			compare
			\begin{inlinelist}
			\item	\vbform{kawlilél}{pfv}[obj intr, \fm{∅}, ach]{she/he/it became limp, baggy, flabby}
			\item	\fm{leilí} ‘flab; scrotum’
			\item	\fm{a daaleilí} ‘its saggy skin’;
			\end{inlinelist}
			related to \fm{létlʼk} \~\ \fm{lélʼk} ‘soft’
				and \fm{\rt{dletl}} ‘hang slack’
		\item	\fm{kayéx̱tʼi} ‘wood shavings’
			from \fm{\rt[²]{yex̱}} ‘shave; build’
			\vbmorph{ka-&\rt{yex̱}&-μH&\gm{-tʼ}&-i}
				{\xx{hsfc}&\rt{shave}&\·\xx{var}&\·\xx{rep}&\·\xx{nmz}}
			\newline
			compare \vbform{akayeix̱}{impfv}[tr, \fm{∅}, \fm{-μμL} act]{she/he/it is shaving, whittling, planing him/her/it}
				\vbmorph{a-&ka-&\rt[²]{yex̱}&-μμL}
					{\xx{3>3}&\xx{hsfc}&\rt[²]{shave}&\·\xx{var}}
			\newline
			the noun \fm{kayéx̱tʼi} implies a verb with \fm{-tʼ}
				but no other forms of \fm{\rt[²]{yex̱}} with \fm{-tʼ} are attested
		\item	\fm{kakúshtʼi} ‘pitch blister on tree’
			from unknown \fm{\rt{kush}}
			\vbmorph{ka&\rt{kush}&-μH&\gm{-tʼ}&-i}
				{\xx{hsfc}&\rt{\xx{unkn}}&\·\xx{var}&\·\xx{rep}&\·\xx{nmz}}
			\newline
			also in \fm{xʼaan kakúshtʼi} ‘cottonwood fluff with seeds’
			\vbmorph{xʼaan&ka&\rt{kush}&-μH&\gm{-tʼ}&-i}
				{branch.tip&\xx{hsfc}&\rt{\xx{unkn}}&\·\xx{var}&\·\xx{rep}&\·\xx{nmz}}
			\newline
			compare \fm{kóoshdaa} ‘land otter’
				(perhaps including \fm{dáa} ‘weasel’
					or alternatively \X{-t} + \X{-aa})
				and \fm{kweiḵ} ‘finger abscess, felon, whitlow’;
			perhaps related to \fm{\rt{ḵush}} ‘purulent, unclean’ in
				\begin{inlinelist}
				\item	\fm{ḵóosh} ‘open sore; unclean thing’
				\item	\fm{ḵóosh kadáan!} (interj.)\ ‘damn! woe!’
				\item	\vbform{wuliḵoosh}{pfv}[obj intr, \fm{n}, ach]{she/he/it became open sore}
				\item	\vbform{liḵooshí}{impfv}[obj intr, conj?, inv state]{she/he/it is unclean};
				\end{inlinelist}
			perhaps also related to \fm{\rt[²]{ḵush}} ‘handle blob’ in
				\vbform{x̱waaḵoosh}{pfv}[tr, \fm{n}?, mot]{I carried it (blob, blubber, mud, snake)}
				\parencite[874]{leer:1976}
				and thus \fm{\rt{ḵutlʼ}} ‘mud’ in
				\fm{ḵútlʼkw} ‘mud, mire’,
				\vbform{kawshiḵútlʼ}{pfv}[obj intr, \fm{∅}, ach]{she/he/it became muddy},
				and \vbform{kashiḵútlʼkw}{rep impfv}[obj intr, conj?, rep state]{she/he/it is muddy}
		\item	\fm{ḵʼalukakúltʼi} ‘slit, eyehole in thong end’
			from unknown \fm{\rt{kul}}
			\vbmorph{ḵʼa-&lu-&ka-&\rt{kul}&-μH&\gm{-tʼ}&-i}
				{mouth&nose&\xx{hsfc}&\rt{\xx{unkn}}&\·\xx{var}&\·\xx{rep}&\·\xx{nmz}}
			\newline
			compare
			\begin{inlinelist}
			\item	\fm{kool} ‘navel, whorl’
			\item	\fm{jikóol} ‘back of hand’
			\end{inlinelist}
		\item	\fm{at yasaḵéitʼi} ‘lawyer, judge’
			from \fm{\rt[¹]{ḵa}} ‘say, speak’
			\vbmorph{at=&ÿa-&sa-&\rt{ḵa}&-μμᵉH&\gm{-tʼ}&-i}
				{\xx{ind.n.o}&\xx{qual}&\xx{tr}&\rt[¹]{say}&\·\xx{var}&\·\xx{rep}&\·\xx{nmz}}
			\newline
			also in
			\fm{ḵuyasaḵéitʼi} ‘judge, commissioner’
			\vbmorph{ḵu-&ÿa-&sa-&\rt{ḵa}&-μμᵉH&\gm{-tʼ}&-i}
				{\xx{ind.h.o}&\xx{qual}&\xx{tr}&\rt[¹]{say}&\·\xx{var}&\·\xx{rep}&\·\xx{nmz}}
		\end{itemize}
	\item	possibly analyzable in some CVCC nouns with unexplained coda /\ipa{tʼ}/
		\begin{itemize}
		\item	\fm{íx̱tʼ} ‘shaman, medicine man, Indian doctor’
			from unknown \fm{\rt{.ix̱}}
			\vbmorph{\rt{.ix̱}&-μH&\gm{-tʼ}}
				{\rt{\xx{unkn}}&\·\xx{var}&\·\xx{rep}}
			\newline
			also in a few obscure verbs apparently derived from the noun
			\parencite[all from][02/281–282]{leer:1973}
			\begin{itemize}
			\item	\vbform{woosh da.íx̱tʼ}{rep impfv}[tr?, conj?, ach?]{they (shamans) are trying to outdo each other}
				\vbmorph{woosh=&da-&\rt{.ix̱}&-μH&\gm{-tʼ}}
					{\xx{recip.o}&\xx{mid}&\rt{\xx{unkn}}&\·\xx{var}&\·\xx{rep}}
			\item	\vbform{áa ḵuwdi.íx̱tʼ}{pfv}[subj intr?, conj?, ach?]{there got to be a lot of shamans there}
				\vbmorph{á&-μ&ḵu-&w-&d-&i-&\rt{.ix̱}&-μH&\gm{-tʼ}}
					{\xx{3n}&\·\xx{loc}&\xx{areal}&\xx{pfv}&\xx{mid}&\xx{stv}&\rt{\xx{unkn}}&\·\xx{var}&\·\xx{rep}}
			\item	\vbform{wuli.íx̱tʼ}{pfv}[obj intr?, conj?, ach?]{she/he/it became a shaman}
				\vbmorph{wu-&l-&i-&\rt{.ix̱}&-μH&\gm{-tʼ}}
					{\xx{pfv}&\xx{intr}&\xx{stv}&\rt{\xx{unkn}}&\·\xx{var}&\·\xx{rep}}
			\end{itemize}
		\item	\fm{sʼáxtʼ} ‘devils club’
			from unknown \fm{\rt{sʼax}}
			\vbmorph{\rt{sʼax}&-μH&\gm{-tʼ}}
				{\rt{\xx{unkn}}&\·\xx{var}&\·\xx{rep}}
			\newline
			compare
			\begin{inlinelist}
			\item	\fm{sʼáx} ‘starfish’
			\item	\fm{\rt{sʼaxw}} ‘stack’
			\item	\fm{\rt{sʼix}} ‘aged, fermented, rotten’
			\item	\fm{\rt{sʼixw}} \~\ \fm{\rt{sʼux}} ‘sour’
			\item	\fm{\rt{sʼikw}} \~\ \fm{\rt{sʼuk}} ‘crisp’
			\item	\fm{\rt{sʼixʼw}} ‘sticky’
			\item	\fm{\rt{tsʼikʼw}} \~\ \fm{\rt{tsʼikʼ}} ‘pinch’
			\item	\fm{\rt{tsʼikw}} ‘delicate’ in \fm{–tsʼígwaa} (see \X{-aa})
			\end{inlinelist}
		\item	\fm{shakalḵʼíshtʼ} ‘deer with two-point horns’
			from unknown \fm{\rt{ḵʼish}}
			\vbmorph{sha-&ka-&d-&l-&\rt{ḵʼish}&-μH&\gm{-tʼ}}
				{head&\xx{qual}&\xx{mid}&\xx{xtn}&\rt{\xx{unkn}}&\·\xx{var}&\·\xx{rep}}
			\newline
			compare
			\begin{inlinelist}
			\item	\fm{ḵʼeishtʼóo} ‘larva cyst in skin; hard ball’ (see below)
			\item	\fm{\rt{ḵʼichʼ}} ‘scabbed’ and noun \fm{ḵʼéechʼ} ‘scab, dried blood’
			\item	\fm{\rt{ḵʼish}} \~\ \fm{\rt{ḵʼesh}} ‘swat puck, ball’
				and nouns \fm{ḵʼísht} ‘puck’ (with \X{-t})
				and \fm{ḵʼeish} ‘batting, striking a ball’
			\item	\fm{\rt{ḵʼetlʼ}} ‘cut open, split down center’
			\end{inlinelist}
		\end{itemize}
	\item	possibly part of three nouns ending with \fm{…tʼu}
		\begin{itemize}
		\item	\fm{kéitʼu} ‘pick, pickaxe’ 
			also \fm{sheey kéitʼu} ‘prying tool (made from limb)’
			and \fm{ḵaashakéitʼu} ‘freshwater insect’
			from unknown \fm{\rt{ke}} or \fm{\rt{ketʼ}};
			compare
			\begin{inlinelist}
			\item	\fm{káatʼ} ‘digging stick’
			\item	\fm{\rt{kitʼ}} ‘jam; pry’ and noun \fm{kítʼaa} ‘prybar’
			\item	Haida \fm{kitʼuu} ‘harpoon, seafood spear’ \parencite[1062]{enrico:2005}
			\end{inlinelist}
		\item	\fm{ḵʼeishtʼóo} ‘larva cyst in skin; hard ball’
			from unknown \fm{\rt{ḵʼesh}};
			compare
			\begin{inlinelist}
			\item	\fm{\rt{ḵʼichʼ}} ‘scabbed’ and noun \fm{ḵʼéechʼ} ‘scab, dried blood’
			\item	\fm{\rt{ḵʼish}} \~\ \fm{\rt{ḵʼesh}} ‘swat puck, ball’
				and nouns \fm{ḵʼísht} ‘puck’ (with \X{-t})
				and \fm{ḵʼeish} ‘batting, striking a ball’
			\item	\fm{ḵʼeishkaháagu} ‘very small cranberry’
			\item	\fm{\rt{ḵʼetlʼ}} ‘cut open, split down center’
			\end{inlinelist}
		\item	\fm{x̱éitʼu} ‘white film on tongue’
			from unknown \fm{\rt{x̱e}} or \fm{\rt{x̱etʼ}};
			possibly includes \fm{\rt{wu}} ‘pale, fair’ like in
			\begin{inlinelist}
			\item	\fm{chichwú} ‘white porpoise’
			\item	\fm{chʼáatwu} ‘epidermis’
			\item	\fm{chʼeetwú} ‘white auklet or murrelet’
			\item	\fm{jánwu} \~\ \fm{jánu} ‘mountain goat’
			\item	\fm{kitwú} ‘white killerwhale’
			\item	\fm{lakʼeechʼwú} ‘scoter’
			\item	\fm{taanwú} \~\ \fm{taanú} ‘umbilical cord’
			\item	\fm{wú} ‘white side of it (flounder, halibut)’;
			\end{inlinelist}
			compare
			\begin{inlinelist}
			\item	\fm{x̱aatlʼáḵw} ‘mouth ulcer’ (with \X{-áḵw}?)
			\item	\fm{x̱aatlʼ} ‘freshwater grass’
			\item	\fm{x̱éetʼ} ‘giant clam’
			\end{inlinelist}
		\end{itemize}
	\end{enumerate}

\item[tu-]
	first person plural subject;
	note that \cite{story-naish:1973} write all cases of \fm{tu-} as \fm{too-}
		so they do not distinguish the two allomorphs
	\begin{itemize}
	\item	\fm{wutuwax̱áa} (pfv; tr, \fm{∅}, \fm{-μH} act) ‘we ate it’\newline
		versus \fm{toox̱á} (impfv) ‘we eat it; we are eating it’
	\end{itemize}

\item[tu-]
	incorporated noun ‘inside; mind, emotion, bodily spirit’;
	derived from relational noun \fm{tú} ‘inside of (hollow object)’
	used metaphorically as ‘mind, emotion, bodily spirit’ as in \fm{ax̱ toowú yanéekw} ‘my mind hurts’

\item[too-]
	allomorph of first person plural subject \fm{tu-}
	\begin{itemize}
	\item	\fm{toox̱á} (pfv; tr, \fm{∅}, \fm{-μH} act) ‘we eat it; we are eating it’\newline
		versus \fm{wutuwax̱áa} (pfv) ‘we ate it’
	\end{itemize}

\item[-ts]\label{m:-ts}
	unknown suffix which occurs only in the stem \fm{–núkts} ‘sweet, delicious’
		which is analyzed as \fm{\rt{nuk}-μH-ts};
	the underlying root may be \fm{\rt{nikw}} \~\ \fm{\rt{nuk}} ‘feel’
		but the composition of meaning is unclear;
	this suffix disappears when the stem \fm{–núkts} is combined with \X{-chʼán},
		which see
	\begin{itemize}
	\item	\vbform{linúkts}{impfv}[obj intr, \fm{g}, inv state]{she/he/it is sweet (tasting)}
			\vbmorph{lˢ-&i-&\rt[⁰]{nuk}&-μH&\gm{-ts}}
				{\xx{intr}&\xx{stv}&\rt[⁰]{sweet}&\·\xx{var}&\·\xx{unkn}}
	\end{itemize}
\end{morphdesc}

\subsection{U}\label{sec:alphalist-u}
\begin{morphdesc}[resume*=alphalist]
\item[u-]\label{m:u-irr}
	irrealis prefix

\item[u-]\label{m:u-pfv}
	\fm{∅} conjugation class perfective,
	occurring in some perfective aspect forms and some habitual aspect forms of
	\fm{∅} conjugation class verbs

\item[-uḵ]\label{m:-uḵ}
	allomorph of \X[-ḵ-dprv]{-ḵ} with labialization and epenthetic (filler) \fm{u}

\item[-úḵ]\label{m:-úḵ}
	allomorph of \X[-ḵ-dprv]{-ḵ} with labialization and epenthetic (filler) \fm{ú}

\end{morphdesc}

\subsection{W}\label{sec:alphalist-w}
\begin{morphdesc}[resume*=alphalist]
\item[ʷ-]\label{m:ʷ-pfv}
	allomorph of perfective \fm{wu-} when followed by
	first person singular subject \fm{x̱-} \~\ \fm{x̱a-},
	labializing its fricative to form \fm{x̱w} or \fm{x̱wa}
	(phonetically [\ipa{χʷ}] or [\ipa{χʷa}])

\item[ʷ-]\label{m:ʷ-irr}
	allomorph of irrealis \fm{u-}

\item[w-]\label{m:w-pfv}
	allomorph of perfective \fm{wu-} in coda of a syllable;
	in Inland Tlingit \fm{m-} is used instead, may also occur elsewhere in older Tlingit
		(e.g.\ song lyrics);
	19th century Tlingit occasionally has full \fm{wu-} rather than \fm{w-},
		e.g.\ \fm{awusikóo du éesh hídi} ‘she knew her father’s house’
		\parencite[255.7]{swanton:1909}
	\begin{itemize}
	\item	\fm{awsiteen} (pfv; tr, \fm{g̱}, ach) ‘s/he/it caught sight of (saw) him/her/it’\newline
		versus \fm{x̱at wusiteen} (pfv) ‘s/he/it caught sight of (saw) me’
	\end{itemize}

\item[w-]\label{m:w-irr}
	allomorph of irrealis \fm{u-}

\item[w̃-]\label{m:w̃-}
	variant form of perfective \fm{w-} allomorph in coda of a syllable;
	occurs in some varieties where \fm{m-} formerly occurred, compare \fm{m-} still used
	in Teslin and Carcross/Tagish Tlingit varieties

\item[wa-]\label{m:wa-}
	allomorph of stative \fm{ÿa-} when preceded by labialized (round) sound

\item[wu-]\label{m:wu-}
	perfective prefix used in most perfective aspect forms;
	see also \fm{∅} conjugation class perfective \fm{u-};
	the perfective prefix has a very complex phonology with many different patterns that depend
	on both	preceding and following prefixes;
	see individual entries for specific details and verb prefix charts for comprehensive patterns;
	reconstructed as \fm[*]{ŋu-} \~\ \fm[*]{ŋʷ-} cognate with Proto-Dene \fm[*]{ŋi-} perfective
	\newline
	allomorphs:
	\begin{allolist}
	\item[\X{m-}]	coda consonant following a vowel, variant form of \fm{w-} used in Teslin
			and Carcross/Tagish varieties
	\item[{\X[u-pfv]{u-}}]
			special allomorph only used with \fm{∅} conjugation class verbs
	\item[{\X[ʷ-pfv]{ʷ-}}]
			labialization of a consonant, with first person singular subject \fm{x̱-} \~\ \fm{x̱a-}
	\item[{\X[w-pfv]{w-}}]
			coda consonant following a vowel (i.e.\ \fm{u} of \fm{wu-} deleted)
	\item[\X{w̃-}]	coda consonant following a vowel, retaining the nasalization of former \fm{m-}
	\item[{\X[ÿ-pfv]{ÿ-}}]
			delabialized \fm{y} or \fm{ÿ} preceding or merged with a front vowel
	\item[\X{ÿu-}]	abstract representation reflecting labialized and delabialized forms
	\item[\X{μʷ-}]	lengthening of preceding vowel with labialization spread to other prefixes
	\item[\X{μw-}]	lengthening of preceding vowel with labial consonant coda
	\item[\X{μm-}]	lengthening of preceding vowel with \fm{m} consonant coda
	\end{allolist}
	combinations:
	\begin{allolist}
	\item[x̱w]	≡ \fm{ʷ-x̱-} with first person singular subject \fm{x̱-}
	\item[x̱wa]	≡ \fm{ʷ-x̱a-} with first person singular subject \fm{x̱a-}
	\item[ÿ]	≡ \fm{ÿ-i-} with second singular subject \X[i-2sg]{i-}
	\item[ÿee]	≡ \fm{ÿ-i-μ-} with second singular subject \X[i-2sg]{i-}
			and stative \X{μ-}
	\item[ÿeeÿ]	≡ \fm{ÿu-ÿi-} with second plural subject \X{ÿi-}
	\item[ÿeeÿ]	≡ \fm{ÿu-ÿi-μ-} with second plural subject \X{ÿi-}
			and stative \X{μ-}
	\item[ÿeeÿCi]	≡ \fm{ÿu-ÿi-C-i-} with second plural subject \X{ÿi-}
			and valency \X{s-} or \X{l-}/\X{lˢ-} or \X{sh-}
			and stative \X[i-stv]{i-}
	\item[ÿi]	≡ \fm{ÿ-i-} with second singular subject \X[i-2sg]{i-}
	\end{allolist}
	\begin{itemize}
	\item	\vbform{at wuduwax̱áa}{pfv}[tr, \fm{∅}, \fm{-μH} act]{someone/people ate something}
			\vbmorph{at=&wu-&du-&wa-&\rt[²]{x̱a}&-μμH}
				{\xx{ind.n.o}&\xx{pfv}&\xx{ind.h.s}&\xx{stv}&\rt[²]{eat}&\·\xx{var}}
		\versus \vbform{at dux̱á}{impfv}{someone/people is/are eating something}
			\vbmorph{at=&du-&\rt[²]{x̱a}&-μH}
				{\xx{ind.n.o}&\xx{ind.h.s}&\rt[²]{eat}&\·\xx{var}}
	\end{itemize}

\item[wush=]\label{m:wush=}
	allomorph of reciprocal object \fm{woosh=}

\item[wooch=]\label{m:wooch=}
	allomorph of reciprocal object \fm{woosh=}

\item[woosh=]\label{m:woosh=}
	reciprocal object
\end{morphdesc}

\subsection{X}\label{sec:alphalist-x}
\begin{morphdesc}[resume*=alphalist]
\item[-xʼ]\label{m:-xʼ}
	plural repetitive suffix describing either
		a plural number of eventualities (pluractional)
		or a plural number of entities (plural object),
		or perhaps both at the same time in some contexts;
	unlike the \X{-ch}, \X{-k}, and \X{-x̱} repetitive suffixes,
		this suffix is never specified by conjugation class
		or as part of a motion derivation;
	this suffix is identical to the plural \fm{-xʼ} \~\ \fm{-xʼw} of nouns
		but it occurs on verbs and so its meaning differs;
	it may be glossed as plural \xx{pl} or repetitive \xx{rep};
	although \fm{-xʼ} \~\ \fm{-xʼw} sometimes quantifies objects, 
		it should not be glossed as \xx{pl.o} (with \xx{o} for object)
		or similar because this can misleadingly imply that it
		is an object marker which is never the case
	\newline
	allomorphs:
	\begin{allolist}
	\item[-xʼ]	basic form
	\item[\X{-xʼw}]	form with labialization
	\end{allolist}
	\begin{enumerate}
	\item	repetitive suffix used with a wide variety of activity
			and achievement verb roots
			\parencite[534]{crippen:2019};
		verbs using this suffix may be of any conjugation class
			but there is a tendency for \fm{∅}
			and \fm{n} conjugation (probably due to regular frequency);
		most verbs with \fm{-xʼ} \~\ \fm{-xʼw} are transitive,
			but there are both object intransitive
			and subject intransitive verbs as well;
		in some cases the repetitive imperfective predicted from conjugation class
			is also documented, but in other cases it is absent
			and the only documented repetitive imperfective has \fm{-xʼ} \~\ \fm{-xʼw}
		\begin{itemize}
		\item	\vbform{achʼéen}{impfv}[tr, \fm{∅}, \fm{-μμH} act]{she/he is tying it (hair with ribbon)}
			\parencite[603]{leer:1976}
				\vbmorph{a-&\rt[²]{chʼin}&-μμH}
					{\xx{3>3}&\rt[²]{tie.ribbon}&\·\xx{var}}
			\versus \vbform{achʼínxʼ}{rep impfv}{she/he is tying it here and there}
			\parencite[603]{leer:1976}
				\vbmorph{a-&\rt[²]{chʼin}&-μH&\gm{-xʼ}}
					{\xx{3>3}&\rt[²]{tie.ribbon}&\·\xx{var}&\·\xx{pl}}
			\versus \vbform{achʼínxʼ}{rep impfv}{she/he is tying it repeatedly}
			\parencite[603]{leer:1976}
				\vbmorph{a-&\rt[²]{chʼin}&-μH&-x̱}
					{\xx{3>3}&\rt[²]{tie.ribbon}&\·\xx{var}&\·\xx{rep}}
		\item	\vbform{aÿadláḵxʼw}{rep impfv}[tr, \fm{n}, ach]{she/he/it is winning them}
			\parencite[459]{leer:1976}
				\vbmorph{a-&ÿa-&\rt[²]{dlaḵ}&-μH&\gm{-xʼw}}
					{\xx{3>3}&face&\rt[²]{win}&\·\xx{var}&\·\xx{pl}}
			\versus \vbform{aÿaawadlaaḵ}{pfv}{she/he won it/them}
				\vbmorph{a-&ÿa-&μʷ-&wa-&\rt[²]{dlaḵ}&-μμL}
					{\xx{3>3}&face&\xx{pfv}&\xx{stv}&\rt[²]{win}&\·\xx{var}}
		\item	roots attested with \fm{-xʼ} \~\ \fm{-xʼw}
			in a repetitive imperfective form include
			\parencite[534]{crippen:2019}:
			\begin{inlinelist}
			\item	\fm{\rt[²]{chʼin}} ‘tie bow’
			\item	\fm{\rt[²]{dlaḵ}} ‘win, obtain’
			\item	\fm{\rt[²]{gish}} ‘soak; kelp’
			\item	\fm{\rt[²]{han}} ‘cut into strips’
			\item	\fm{\rt[²]{hat}} ‘cover’
			\item	\fm{\rt[²]{hiᴸ}} ‘pay shaman’
			\item	\fm{\rt[²]{hits}} ‘singe’
			\item	\fm{\rt[¹]{kan}} ‘wave, flutter’
			\item	\fm{\rt[²]{kwach}} ‘handle handful’
			\item	\fm{\rt[²]{kʼwach}} ‘break’
			\item	\fm{\rt[²]{na}} ‘die; inherit’
			\item	\fm{\rt[²]{nal}} ‘steam’
			\item	\fm{\rt[²]{naᴴsh}} ‘shake off’
			\item	\fm{\rt[¹]{suᴴs}} ‘fall, scatter’
			\item	\fm{\rt[²]{sin}} ‘hide, conceal’
			\item	\fm{\rt[¹]{sʼis}} ‘wind blown’
			\item	\fm{\rt[²]{tiÿ}} ‘soak’
			\item	\fm{\rt[²]{tuᴴk}} ‘pop’
			\item	\fm{\rt[²]{tul}} ‘spin, drill’
			\item	\fm{\rt[¹]{tuᴴl}} ‘murmur’
			\item	\fm{\rt[²]{tuḵ}} ‘spit’
			\item	\fm{\rt[²]{tʼaᴸ}} ‘hot’
			\item	\fm{\rt[²]{tleḵw}} ‘snatch; finger’
			\item	\fm{\rt[²]{tsis}} ‘float’
			\item	\fm{\rt[²]{tsuᴴw}} ‘push, jab’
			\item	\fm{\rt[²]{xatʼ}} ‘drag’
			\item	\fm{\rt[²]{xwach}} ‘tan’
			\item	\fm{\rt[²]{xwen}} ‘ladle’
			\item	\fm{\rt[²]{x̱ach}} ‘tow’
			\item	\fm{\rt[²]{x̱ʼeᴴÿ}} ‘encourage’
			\item	\fm{\rt[²]{ya}} ‘backpack’
			\item	\fm{\rt[²]{yiḵ}} ‘mouth, bite’
			\item	\fm{\rt[²]{yiḵ}} ‘pull’
			\end{inlinelist}
		\end{itemize}
	\item	plural suffix used with dimensional state verbs when
			the object is plural;
		occurs together with \X{d-} for unknown reasons
			\parencites[93]{story:1966}[99]{leer:1991}[458]{crippen:2019};
		note that \fm{-xʼ} \~\ \fm{-xʼw} does not occur
			when using the comparative derivation
			with comparative \X[ka-cmpv]{ka-}
			(which see for discussion of comparatives)
			for a plural object even though \X{d-} still occurs
			suggesting that \fm{-xʼ} is somehow blocked
		\begin{itemize}
		\item	\vbform{yagéi}{impfv}[obj intr, \fm{g}, \fm{-μμH} state]{she/he/it is big, much}
				\vbmorph{ÿa-&\rt[¹]{ge}&-μμH}
					{\xx{stv}&\rt[¹]{big}&\·\xx{var}}
			\versus \vbform{digéixʼ}{impfv}{they are big, numerous}
				\vbmorph{d-&i-&\rt[¹]{ge}&-μμH&\gm{-xʼ}}
					{\xx{mid}&\xx{stv}&\rt[¹]{big}&\·\xx{var}&\·\xx{pl}}
		\item	\vbform{a yáanax̱ koogéi}{impfv}[obj intr, \fm{g}, \fm{-μμH} state]{she/he/it is bigger than it}
				\vbmorph{k-&u-&μ-&\rt[¹]{ge}&-μμH}
					{\xx{cmpv}&\xx{irr}&\xx{stv}&\rt[¹]{big}&\·\xx{var}}
			\versus \vbform{a yáanáx̱ kudigéi}{impfv}{they are bigger than it}
				\vbmorph{k-&u-&d-&i-&\rt[¹]{ge}&-μμH}
					{\xx{cmpv}&\xx{irr}&\xx{mid}&\xx{stv}&\rt[¹]{big}&\·\xx{var}}
				\andnot{\fm[*]{a yáanáx̱ kudigéixʼ}}
		\item	roots attested with \fm{-xʼ} \~\ \fm{-xʼw}
			in a dimensional state form include
			\parencite[459]{crippen:2019}:
			\begin{inlinelist}
			\item	\fm{\rt[¹]{dal}} ‘heavy’
			\item	\fm{\rt[¹]{ge}} ‘big, much’
			\item	\fm{\rt[¹]{kak}} ‘thick’
			\item	\fm{\rt[¹]{sa}} ‘narrow’
			\item	\fm{\rt[¹]{tla}} ‘stout’
			\item	\fm{\rt[¹]{wux̱ʼ}} ‘wide’
			\item	\fm{\rt[¹]{ÿatʼ}} ‘long’
			\end{inlinelist}
		\end{itemize}
	\end{enumerate}

\item[-xʼw]\label{m:-xʼw}
	allomorph of plural/repetitive suffix \X{-xʼ} with labialization
\end{morphdesc}

\subsection{X̱}\label{sec:alphalist-xh}
\begin{morphdesc}[resume*=alphalist]
\item[x̱-]\label{m:x̱-1sg}
	first person singular subject
	\newline
	allomorphs:
	\begin{allolist}
	\item[x̱-]	basic form
	\item[\X{x̱a-}]	form with epenthetic (filler) vowel \fm{á}
	\end{allolist}
	\begin{itemize}
	\item	\fm{laaḵʼásk kax̱satʼaak} (impfv; tr, \fm{∅}, \fm{-μμL} act) ‘I am pressing black seaweed’\newline
		versus \fm{laaḵʼásk x̱ax̱á} (impfv; tr, \fm{∅}, \fm{-μH} act) ‘I am eating black seaweed’
	\end{itemize}

\item[x̱-]\label{m:x̱-g̱cnj}
	allomorph of conjugation \fm{g̱-} when in a syllable coda
	\begin{itemize}
	\item	\fm{kax̱lax̱óotʼ} (imp; tr, \fm{g̱}, \fm{-μμH} act) ‘(you sg.)\ chop/adze it!’ in syllable \fm{kax̱}\newline
		(not \fm[*]{kag̱alax̱óotʼ} or \fm[*]{kaḵlax̱óotʼ})\newline
		versus \fm{kag̱aylax̱óotʼ} ‘you pl.\ chop/adze it!’ in syllable \fm{g̱ay}
	\end{itemize}

\item[x̱-]\label{m:x̱-mod}
	allomorph of modality \fm{g̱-} when in a syllable coda
	\begin{itemize}
	\item	\fm{at gax̱toox̱áa} (prosp; tr, \fm{∅}, \fm{-μH} act) ‘we will eat something’ with \fm{x̱-}\newline
		(not \fm[*]{at gag̱atoox̱áa} or \fm[*]{at gaḵtoox̱áa})\newline
		versus \fm{at gug̱ax̱áa} (prosp) ‘s/he/it will eat something’ with \fm{g̱a-}
	\end{itemize}

\item[-x̱]\label{m:-x̱}
	repetitive suffix;
	\newline
	allomorphs:
	\begin{allolist}
	\item[-x̱w]	with labialization
	\end{allolist}
	\begin{enumerate}
	\item	repetitive suffix predicted for \fm{∅} conjugation class verbs
	\item	repetitive suffix in derived aspect paradigms
	\item	repetitive suffix lexically required for some roots
	\item	possibly part of amissive \fm{-x̱aa} ‘miss target’
	\end{enumerate}

\item[x̱a-]\label{m:x̱a-}
	allomorph of first person singular subject \X[x̱-1sg]{x̱-} with epenthetic (filler) vowel

\item[-x̱aa]\label{m:-x̱aa}
	amissive suffix, indicates failure of an attempt to hit a target;
	part of the ‘miss target’ derivation made up of:
		qualifier \X[ÿa-qual]{ÿa-}
		+ extensional \X{s-}/\X{lˢ-}
		+ amissive \fm{-x̱aa} \~\ \fm{-x̱áa}
		with \fm{∅} conjugation class;
	probably derived from repetitive \X{-x̱} and unknown \X{-áa} \~\ \X{-aa};
	if treated as a single suffix \fm{-x̱aa} \~\ \fm{-x̱áa}
		then it is glossed as \xx{miss}
		otherwise \xx{rep} + \xx{unkn};
	applicable to any verb that denotes striking a target in some manner,
		meaning ‘attempt to strike target but miss’;
	attested with the roots
		\begin{inlinelist}
		\item	\fm{\rt{dzu}} ‘throw at’
		\item	\fm{\rt{gwal}} ‘punch, strike’
		\item	\fm{\rt{ḵʼish}} ‘slap with stick’
		\item	\fm{\rt{shaᴴt}} ‘grab’
		\item	\fm{\rt{tʼach}} ‘slap’
		\item	\fm{\rt{tʼuᴴk}} ‘shoot (arrow)’
		\item	\fm{\rt{.uᴴn}} ‘shoot (gun)’
		\item	\fm{\rt{x̱ich}} ‘club, spank’
		\end{inlinelist}
	\newline
	allomorphs:
	\begin{allolist}
	\item[-x̱aa]	L tone form occurs after an H tone stem
	\item[\X{-x̱áa}]	H tone form occurs after an L tone stem (unattested)
	\end{allolist}
	\begin{itemize}
	\item	\vbform{ayawsi.únx̱aa}{pfv}[tr, \fm{∅}, ach]{she/he/it shot at him/her/it and missed}
			\vbmorph{a-&ÿa-&w-&s-&i-&\rt[²]{.uᴴn}&-μH&\gm{-x̱aa}}
				{\xx{3>3}&\xx{qual}&\xx{pfv}&\xx{xtn}&\xx{stv}&\rt[²]{shoot}&\·\xx{var}&\·\xx{miss}}
		\versus \vbform{aawa.ún}{pfv}{she/he/it shot him/her/it}
			\vbmorph{a-&μʷ-&wa-&\rt[²]{.uᴴn}&-μH}
				{\xx{3>3}&\xx{pfv}&\xx{stv}&\rt[²]{shoot}&\·\xx{var}}
	\item	\vbform{ayawlishátx̱aa}{pfv}[tr, \fm{∅}, ach]{she/he/it grabbed at it and missed}
			\vbmorph{a-&ÿa-&w-&lˢ-&i-&\rt[²]{shaᴴt}&-μH&\gm{-x̱aa}}
				{\xx{3>3}&\xx{qual}&\xx{pfv}&\xx{xtn}&\xx{stv}&\rt[²]{grab}&\·\xx{var}&\·\xx{miss}}
		\versus \vbform{aawasháat}{pfv}[tr, \fm{g}, ach]{she/he/it grabbed him/her/it}
			\vbmorph{a-&μʷ-&wa-&\rt[²]{shaᴴt}&-μμH}
				{\xx{3>3}&\xx{pfv}&\xx{stv}&\rt[²]{grab}&\·\xx{var}}
	\item	\vbform{ayawlidzéix̱aa}{pfv}[tr, \fm{∅}, ach]{she/he/it threw at it and missed}
			\vbmorph{a-&ÿa-&w-&\gm{lˢ-}&i-&\rt[²]{dzu}&-μᵉμH&\gm{-x̱aa}}
				{\xx{3>3}&\xx{qual}&\xx{pfv}&\xx{xtn}&\xx{stv}&\rt[²]{throw}&\·\xx{var}&\·\xx{miss}}
		\versus \vbform{aawadzóo}{pfv}[tr, \fm{∅}, ach]{she/he/it threw at him/her/it}
			\vbmorph{a-&μʷ-&wa-&\rt[²]{dzu}&-μμH}
				{\xx{3>3}&\xx{pfv}&\xx{stv}&\rt[²]{throw}&\·\xx{var}}
	\end{itemize}

\item[-x̱áa]\label{m:-x̱áa}
	allomorph of \X{-x̱aa} with H tone, used after L tone syllable (polar tone);
	unattested but predicted from patterns of \X{-aa} \~\ \X{-áa}

\item[x̱at=]
	first person singular object;
	similar to independent pronoun \fm{x̱át} ‘me’ but with L tone instead of H tone
	\newline
	allomorphs:
	\begin{allolist}
	\item[x̱at=]	typical form
	\item[ax̱=]	possessor of incorporated noun
	\end{allolist}
	\begin{itemize}
	\item	\vbform{x̱at yisiteen}{pfv}[tr, \fm{g̱}, ach]{you saw me}
			\vbmorph{\gm{x̱at=}&ÿ-&i-&s-&i-&\rt[²]{tin}&-μμL}
				{\xx{1sg.o}&\xx{pfv}&\xx{2sg.s}&\xx{xtn}&\xx{stv}&\rt[²]{see}&\·\xx{var}}
		\versus \vbform{ix̱wsiteen}{pfv}{I saw you}
			\vbmorph{i-&ʷ-&x̱-&s-&i-&\rt[²]{tin}&-μμL}
				{\xx{1sg.o}&\xx{pfv}&\xx{1sg.s}&\xx{xtn}&\xx{stv}&\rt[²]{see}&\·\xx{var}}
	\end{itemize}

\item[-x̱w]
	allomorph with labialization of repetitive \fm{-x̱}

\item[x̱w]
	≡ \fm{ʷ-x̱-} combination of
		perfective \fm{ʷ-}
		and first person singular subject \fm{x̱-}

\item[x̱wa]
	≡ \fm{ʷ-x̱a-} combination of
		perfective \fm{ʷ-}
		and first person singular subject \fm{x̱a-}

\end{morphdesc}

\subsection{Y}\label{sec:alphalist-y}
\begin{morphdesc}[resume*=alphalist]
\item[ÿ-]\label{m:ÿ-2sg}
	allomorph of second person singular subject \X[i-2sg]{i-}

\item[ÿ-]\label{m:ÿ-2pl}
	allomorph of second person plural subject \X{ÿi-}
	
\item[ÿ-]\label{m:ÿ-pfv}
	allomorph of perfective \X{wu-}

\item[ÿ-]\label{m:ÿ-face}
	allomorph of incorporated noun \X[ÿa-face]{ÿa-} ‘face’

\item[ÿ-]\label{m:ÿ-qual}
	allomorph of qualifier \X[ÿa-qual]{ÿa-} of unknown meaning

\item[-ÿ]\label{m:-ÿ}
	sonorant suffix of unknown meaning

\item[ÿa-]\label{m:ÿa-stv}
	allomorph of stative \X[i-stv]{i-}
	\begin{itemize}
	\item	\vbform{yadál}{impfv}[obj intr, \fm{g}, \fm{-μH} state]{she/he/it is heavy}
			\vbmorph{\gm{ÿa-}&\rt[¹]{dal}&-μH}
				{\xx{stv}&\rt[¹]{heavy}&\·\xx{var}}
		\versus \vbform{si.áatʼ}{impfv}[obj intr, \fm{g}, \fm{-μμH} state]{she/he/it is cold}
			\vbmorph{s-&i-&\rt[⁰]{.atʼ}&-μμH}
				{\xx{intr}&\xx{stv}&\rt[⁰]{cold}&\·\xx{var}}
	\end{itemize}

\item[ÿa-]\label{m:ÿa-face}
	incorporated noun indicating vertical surface or face,
	derived from the relational noun \fm{ÿá} ‘face’;
	can occur together with qualifier \X[ÿa-qual]{ÿa-} of unknown meaning

\item[ÿa-]\label{m:ÿa-qual}
	qualifier of unknown meaning;
	can occur together with incorporated noun \X[ÿa-face]{ÿa-} ‘face’

\item[ÿaa=]
	directional preverb indicating progression or movement along a space
	(compare \fm{\rt[²]{ÿa}} ‘move’,
		directional noun \fm{diÿáa} ‘across, other side’,
		\fm{niÿaa} ‘direction’);
	\begin{enumerate}
	\item	progression, used in progressive aspect for \fm{∅} and \fm{n} conjugation class verbs
		\begin{itemize}
		\item	\fm{yaa x̱at nalnítl} (prog; obj intr, \fm{∅}, ach) ‘I am getting fat’\newline
			versus \fm{x̱at wudlinítl} (pfv) ‘I got fat’
		\end{itemize}
	\item	movement along a space
		\begin{enumerate}
		\item	motion derivation
				\fm{ÿaa} (\fm{g̱}, \fm{yei=…-ch} rep) ‘down along’
				(\fm{yei} in repetitive blocks \fm{ÿaa})
		\item	motion derivation
				\fm{ÿaa} \~\ \fm{ÿa-u-} (\fm{∅}, \fm{-ch} rep) ‘obliquely, circuitously’
		\end{enumerate}
	\end{enumerate}

\item[ÿaa=]
	preverb indicating mental phenomenon, limited to a couple of verbs;
	uncertain if it can occur together with \fm{ÿaa} ‘along’;
	possibly related to Proto-Dene \fm[*]{yən-} \~\ \fm[*]{yiːn-} ‘mind’ and Eyak \fm{ʔiːlih} ‘mind’
	\begin{itemize}
	\item	\fm{yaa ḵux̱dzigéi} (impfv; subj intr, \fm{g}, \fm{-μμH} state) ‘I am smart, wise’
	\item	\fm{yaa aḵoowlig̱át} (pfv; tr, \fm{∅}, ach) ‘s/he/it forgot him/her/it’
	\end{itemize}

\item[ÿaan=]
	incorporated noun indicating hunger,
	saturates object argument;
	derived from noun \fm{ÿaan} ‘hunger’ (now rare)
	\begin{itemize}
	\item	\fm{ax̱ éet yaan uwaháa} (pfv; obj intr, \fm{∅}, mot) ‘hunger appeared to me’ (i.e.\ ‘I got hungry’)\newline
		(not \fm[*]{yaan ax̱ éet uwaháa})
	\end{itemize}

\item[ÿan=]
	directional preverb indicating motion to shore, motion to ground, or termination;
	allomorphs are \fm{ÿax̱} and \fm{ÿánde}:
		\fm{ÿax̱} is used with repetitive,
		\fm{ÿánde} with progressive and prospective,
		and \fm{ÿan} elsewhere (e.g.\ pfv, imp);
	morphologically a specialization of the
		\fm{NP-\{t,x̱,dé\}} (\fm{∅}, \fm{-μμL} rep) ‘arriving at NP’
		motion derivation,
	so the \fm{ÿan} probably used to end with \fm{-t} ‘to a point’ punctual postposition
		as \fm[*]{ÿant};
	derived from noun \fm{ÿán} ‘shore’
		(< Pre-Tlingit \fm[*]{ŋanʰ} < Proto-Na-Dene \fm[*]{ŋənˀ} ‘ground, earth’)
	\begin{enumerate}
	\item	motion on water to shore,
		can be translated ‘ashore’;
		motion derivation
			\fm{ÿan} / \fm{yax̱} / \fm{ÿánde} (\fm{∅}, \fm{-μμL} rep) ‘ashore’
	\item	motion to ground or other horizontal surface,
		can be translated ‘down’ or ‘on ground’;
		motion derivation
			\fm{ÿan} / \fm{yax̱} / \fm{ÿánde} (\fm{∅}, \fm{-μμL} rep) ‘on ground’
			optionally with incorporates (\fm{kʼi-} ‘base’ for ‘setting up, erecting’,
			\fm{sha-} ‘head’ for ‘leaning against’)
	\item	termination of eventuality,
		can be translated ‘ending, terminating, finishing’;
		eventuality/motion derivation
			\fm{ÿan} \~\ \fm{yax̱} \~\ \fm{ÿánde} (\fm{∅}, \fm{-μμL} rep) ‘ending, finishing’
			optionally with \fm{NP-xʼ} ‘coming to rest at NP’;
		derives from metaphor of ‘shore’ as ‘end of journey’ and thus ‘end of event’
	\end{enumerate}

\item[ÿánde=]
	allomorph of directional preverb \fm{ÿan} ‘ashore’ or ‘ending’
	with allative postposition \fm{-dé} \~\ \fm{-de} ‘toward’
	\begin{itemize}
	\item	\fm{yánde gax̱tooḵóox̱} (prosp; subj intr, \fm{∅}, mot) ‘we are going to boat ashore’\newline
		versus \fm{yan wutuwaḵúx̱} (pfv) ‘we boated ashore’
	\end{itemize}

\item[ÿata=]
	incorporated noun ‘sleep’,
	saturates object argument;
	apparently derived from \fm{ÿá} ‘face’ and \fm{\rt[¹]{taᴸ}} ‘sg.\ sleep’
	\begin{itemize}
	\item	\fm{ax̱ éet yataawaháa} (pfv; obj intr, \fm{∅}, mot) ‘sleep appeared to me’, i.e. ‘I got sleepy’
	\item	\fm{ax̱ yaadáx̱ yataawahaa} (pfv; obj intr, \fm{g̱}, mot) ‘sleep disappeared from my face’,
		i.e.\ ‘I became wakeful’
	\end{itemize}

\item[ÿax̱=]
	allomorph of directional preverb \fm{ÿan} ‘ashore’ or ‘ending’
	with perlative postposition \fm{-x̱} ‘contacting’;
	used only with repetitive versus \fm{ÿánde} (prog, prosp) or \fm{ÿan} (pfv, imp, etc.)
	\begin{itemize}
	\item	\fm{yax̱ tooḵoox̱} (rep impfv; subj intr, \fm{∅}, mot) ‘we repeatedly boat ashore’\newline
		versus \fm{yan wutuwaḵúx̱} (pfv) ‘we boated ashore’
	\end{itemize}

\item[ÿee-]
	allomorph of second person singular subject \fm{ÿi-}

\item[ÿee=]
	allomorph of second person singular object \fm{ÿi-}

\item[ÿee=]\label{m:ÿee=time}
	incorporated noun indicating time;
	may or may not saturate the object argument of the verb;
	derived from a no longer independent noun \fm{ÿee} ‘time’
	that can also be identified in some nouns, adjectives, and adverbs
	including
		\begin{inlinelist}
		\item	\fm{hóochʼeenís} ‘for the last time’
		\item	\fm{ḵinxʼiyís} ‘just in case’
		\item	\fm{niyís} ‘in preparation for (time)’
		\item	\fm{yagiÿee} \~\ \fm{yakÿee} \~\ \fm{yagee} ‘day’
		\item	\fm{ÿeedát} ‘moment; now’
		\item	\fm{ÿeen} ‘during, in the middle of (duration)’
		\item	\fm{ÿées} ‘new, young’
		\item	\fm{ÿéeÿi} ‘former; subordinate clause past tense’
		\item	\fm{yeis} ‘autumn, fall’
		\item	\fm{ÿeisú} ‘still, yet, recently’
		\end{inlinelist}
	\begin{itemize}
	\item	\vbform{yeeyayátʼ}{impfv}[obj intr, \fm{n}, \fm{-μH} state]{it is a long time}
			\vbmorph{ÿee=&ÿa-&\rt[¹]{ÿatʼ}&-μH}
				{time&\xx{stv}&\rt[¹]{long}&\·\xx{var}}
		\versus \vbform{yayátʼ}{impfv}[obj intr, \fm{n}, \fm{-μH} state]{she/he/it is long}
			\vbmorph{ÿa-&\rt[¹]{ÿatʼ}&-μH}
				{\xx{stv}&\rt[¹]{long}&\·\xx{var}}
	\end{itemize}

\item[ÿee]
	≡ \fm{wu-i-μ}
	combination of perfective \fm{wu-}
		and second person singular subject \fm{i-}
		and stative \fm{μ-}

\item[ÿeeÿ]
	second person plural subject \fm{ÿi-} combined with either one or both of
		perfective \fm{wu-}
		and stative \fm{ÿa-} \~\ \fm{i-}
	\begin{enumerate}
	\item	\fm{ÿeeÿ} ≡ \fm{ÿi-ÿa-}
		with stative \fm{ÿa-}
	\item	\fm{ÿeeÿ} ≡ \fm{wu-ÿi-}
		with perfective \fm{wu-}
	\item	\fm{ÿeeÿ} ≡ \fm{wu-ÿi-ÿa-}
		with perfective \fm{wu-}
		and stative \fm{ÿa-}
	\item	\fm{ÿeeÿsi} ≡ \fm{ÿi-s-i-}
		with valency \fm{s-}
			(or \fm{ÿeeÿli} \fm{l-} or \fm{ÿeeÿshi} \fm{sh-})
		and stative \fm{i-}
	\item	\fm{ÿeeÿsi} ≡ \fm{wu-ÿi-s-i-}
		with perfective \fm{wu-}
		and valency \fm{s-}
			(or \fm{ÿeeÿli} \fm{l-} or \fm{ÿeeÿshi} \fm{sh-})
		and stative \fm{i-}
	\end{enumerate}

\item[ÿeeḵ=]
	directional preverb ‘beach’, variant forms \fm{ÿeiḵ=} and \fm{eèḵ=};
	derived from noun \fm{éeḵ} \~\ \fm{éiḵ} ‘beach’;
	compare \fm{éeg̱i=} \~\ \fm{éig̱i=}

\item[yei=]
	direction preverb ‘down’;
	may reflect \fm{g̱} conjugation class or a \fm{∅} conjugation class motion derivation;
	part of directional element paradigm of \fm{\rt{ÿiⁿ}} ‘down’:
		\fm{(di)ÿée} ‘below’, \fm{(di)ÿín-de} ‘to below’, \fm{(di)ÿee-naa} ‘downward’;
	related to \fm{ÿee} ‘beneath, below’
	\begin{enumerate}
	\item	reflects \fm{g̱} conjugation class in prospective, progressive, and repetitive imperfective
	\item	\fm{g̱} conjugation class motion derivation
	\item	\fm{∅} conjugation class motion derivation
	\end{enumerate}

\item[yéi=]
	manner preverb ‘thus, so’;
	derived from noun \fm{yéi} \~\ \fm{yé} ‘place, way, manner’

\item[ÿeiḵ=]
	variant form of directional preverb \fm{ÿeeḵ=} ‘beach’ used in some Northern varieties
	arises from uvular lowering of \fm{ée} to \fm{éi};
	derived from noun \fm{éeḵ} \~\ \fm{éiḵ} ‘beach’

\item[ÿi-]\label{m:ÿi-}
	second person plural subject or object; long vowel allomorphs are \fm{ÿee-} and \fm{ÿee=}
	\begin{enumerate}
	\item	second person plural subject
	\item	second person plural object
	\end{enumerate}

\item[ÿi]
	≡ \fm{wu-i-}
	combination of perfective \fm{wu-} and
		second person singular subject \fm{i-}

\item[-ÿi]\label{m:-ÿi-rel}
	allomorph of relative clause \fm{-i}

\item[-ÿi]\label{m:-ÿi-sub}
	allomorph of subordinate \fm{-i}

\item[ÿu-]\label{m:ÿu-}
	abstract representation of perfective \fm{wu-};
	this form does not actually occur in speech, instead see
		\fm{wu-}, \fm{w-}, \fm{m-}, \fm{μʷ-} \fm{ÿi}, \fm{ÿee}, \fm{ÿeeÿ}

\item[yoo=]
	alternating eventuality preverb

\item[yóo=]
	quotative preverb

\end{morphdesc}

\subsection{Symbols}\label{sec:alphalist-sym}
\begin{morphdesc}[resume*=alphalist]
\item[μ-]\label{m:μ-}
	allomorph of stative \fm{ÿa-} \~\ \fm{i-}
	\begin{itemize}
	\item	\fm{x̱at yatéen} (impfv; tr, \fm{∅}, \fm{-μμH} state) ‘s/he/it can see me’
			with \fm{ÿa-}\newline
		versus \fm{x̱aatéen} (impfv) ‘I can see him/her/it’
			with \fm{μ-}\newline
		(not normally \fm[*]{x̱ayatéen}, although some speakers also permit this form)
	\end{itemize}

\item[μʷ-]\label{m:μʷ-}
	allomorph of perfective \fm{wu-} when preceded by CV and followed by stative \fm{ÿa-}
	\begin{itemize}
	\item	\fm{aawajáḵ} (pfv; \fm{∅}, ach) ‘s/he/it killed him/her/it’
			with \fm{a-μʷ-wa-\rt[¹]{jaḵ}-μH}\newline
		versus \fm{wutuwajáḵ} (pfv) ‘we killed him/her/it’
			with \fm{wu-tu-wa-\rt[¹]{jaḵ}-μH}
	\end{itemize}

\item[μw-]\label{m:μw-}
	allomorph of perfective \fm{wu-}

\item[μm-]\label{m:μm-}
	allomorph of perfective \fm{wu-}

\item[-μL]\label{m:-μL}
	stem variation: short vowel (μ) with low tone (L) so [\ipa{V̀}]
	\begin{itemize}
	\item	\fm{neil uwagudi ḵáa} (pfv rel; subj intr, \fm{∅}, mot) ‘man who went home’
		with \fm{\rt[¹]{gut}} ‘sg.\ go’ and \fm{-μL}
		in \fm{∅} conjugation class perfective aspect relative clause
	\end{itemize}

\item[-μH]\label{m:-μH}
	stem variation: short vowel (μ) with high tone (H) so [\ipa{V́}]
	\begin{itemize}
	\item	\fm{neil uwagút} (pfv; subj intr, \fm{∅}, mot) ‘s/he/it went home’
		with \fm{\rt[¹]{gut}} ‘sg.\ go’ and \fm{-μH}
		in \fm{∅} conjugation class perfective aspect main clause
	\end{itemize}

\item[-μμL]\label{m:-μμL}
	stem variation: long vowel (μμ) with low tone (L) so [\ipa{V̀ː}]
	\begin{itemize}
	\item	\fm{neildé woogoot} (pfv; subj intr, \fm{n}, mot) ‘s/he/it went homeward’
		with \fm{\rt[¹]{gut}} ‘sg.\ go’ and \fm{-μμL}
		in \fm{n} conjugation class perfective aspect main clause
	\end{itemize}

\item[-μμH]\label{m:-μμH}
	stem variation: long vowel (μμ) with high tone (H) so [\ipa{V́ː}]
	\begin{itemize}
	\item	\fm{neildé gug̱agóot} (prosp; subj intr, \fm{∅}/\fm{n}, mot) ‘s/he/it will go home’
		with \fm{\rt[¹]{gut}} ‘sg.\ go’ and \fm{-μμH}
		in prospective aspect main clause
	\end{itemize}

\item[-μᵉμL]\label{m:-μᵉμL}
	stem variation: ablaut (/\ipa{a, u}/ > [\ipa{e}]) long vowel (μμ) with low tone (L) so [\ipa{èː}];
	normally occurs only with \fm{\rt{CVᴸ}} (Tongass \fm{\rt{CVʰ}}) roots
	\begin{itemize}
	\item	\fm{x̱ateix̱} (rep impfv; subj intr, \fm{n}, \fm{-μH} act) ‘I repeatedly sleep’
			with \fm{\rt[¹]{taᴸ}} ‘sg.\ sleep’ and \fm{-μᵉμL-x̱}\newline
		versus
		\fm{x̱atá} (impfv) ‘I am sleeping’
			with \fm{-μH}\newline
		but \fm{yaa nx̱atéin} (prog) ‘I am falling asleep’
			with \fm{-μᵉμH-n}
	\end{itemize}

\item[-μᵉμH]\label{m:-μᵉμH}
	stem variation: ablaut (/\ipa{a, u}/ > [\ipa{e}]) long vowel (μμ) with high tone (H) so [\ipa{éː}];
	normally occurs only with \fm{\rt{CV}} roots
	\begin{itemize}
	\item	\fm{x̱ax̱éix̱} (rep impfv; tr, \fm{∅}, \fm{-μH} act) ‘I repeatedly eat it’
			with \fm{\rt[²]{x̱a}} ‘eat’ and \fm*{-μᵉμH-x̱}\newline
		versus
		\fm{x̱ax̱á} (impfv) ‘I am eating it’
			with \fm{-μH}
	\end{itemize}

\item[-⊗]
	stem variation: irregular deletion (⊗) of final consonant and short vowel with high tone;
	only occurs in imperatives with \fm{\rt[¹]{gut}} ‘sg go’,
			\fm{\rt[¹]{.at}} ‘pl go’,
			\fm{\rt[¹]{nuk}} ‘sg sit’
	\begin{itemize}
	\item	\vbform{neildé nagú!}{imp}[subj intr, \fm{n}, mot]{(you sg.)\ go home!}
			\vbmorph{neil&-dé&na-&\rt[ˢ]{gu\gm{t}}&\gm{-⊗}}
				{home&\·\xx{all}&\xx{ncnj}&\rt[ˢ]{go·\xx{sg}}&\·\xx{var}}
		\versus \vbform{neildé yeegoot}{pfv}{you sg.\ went home}
			\vbmorph{neil&-dé&ÿ-&i-&μ-&\rt[ˢ]{gut}&-μμL}
				{home&\·\xx{all}&\xx{pfv}&\xx{2sg.s}&\xx{stv}&\rt[ˢ]{go·\xx{sg}}&\·\xx{var}}
	\item	\vbform{neildé nay.á!}{imp}[subj intr, \fm{n}, mot]{you guys go home!}
			\vbmorph{neil&-dé&na-&ÿ-&\rt[ˢ]{.a\gm{t}}&\gm{-⊗}}
				{home&\·\xx{all}&\xx{ncnj}&\xx{2pl.s}&\rt[ˢ]{go·\xx{pl}}&\·\xx{var}}
		\versus \vbform{neildé yeey.aat}{pfv}{you guys went home}
			\vbmorph{neil&-dé&ÿ-&ÿ-&μ-&\rt[ˢ]{.at}&-μμL}
				{home&\·\xx{all}&\xx{pfv}&\xx{2pl.s}&\xx{stv}&\rt[ˢ]{go·\xx{pl}}&\·\xx{var}}
	\item	\vbform{g̱anú!}{imp}[subj intr, \fm{g̱}, mot]{(you sg.)\ sit down!}
			\vbmorph{g̱a-&\rt[ˢ]{nu\gm{k}}&\gm{-⊗}}
				{\xx{g̱cnj}&\rt[ˢ]{sit·\xx{sg}}&\·\xx{var}}
		\versus \vbform{yeenook}{pfv}{you sg.\ sat down}
			\vbmorph{ÿ-&i-&μ-&\rt[ˢ]{nuk}&-μμL}
				{\xx{pfv}&\xx{2sg.s}&\xx{stv}&\rt[ˢ]{sit·\xx{sg}}&\·\xx{var}}
	\end{itemize}
\end{morphdesc}
