%!TEX root = ../lingnote-verbmorphs.tex

\subsection{G}\label{sec:alphalist-g}
\begin{morphdesc}[resume*=alphalist]
\item[g-]\label{m:g-conj}
	\fm{g} conjugation prefix, upward spatial orientation;
	prospective aspect prefix with irrealis \fm{w-} and modal \fm{g̱-};
	can occur together with comparative \fm{g-}

\item[g-]\label{m:g-cmpv}
	irregular allomorph of comparative \X[ka-cmpv]{ka-}
	\begin{itemize}
	\item	\vbform{chʼa yéi googéikʼ}{impfv}[obj intr, \fm{n}, \fm{-μμH} cmpv state]{just a little}
			\vbmorph{chʼa&yéi=&g-&u-&μ-&\rt[¹]{ge}&-μμH&-kʼ}
				{just&thus&\xx{cmpv}&\xx{irr}&\xx{stv}&\rt[¹]{big}&\·\xx{var}&\·\xx{dim}}
		\versus \vbform{yagéi}{impfv}[obj intr, \fm{g}, \fm{-μμH} state]{it is big}
			\vbmorph{ÿa-&\rt[¹]{ge}&-μμH}
				{\xx{stv}&\rt[¹]{big}&\·\xx{var}}
	\end{itemize}

\item[ga-]\label{m:ga-conj}
	allomorph of \fm{g} conjugation prefix \X[g-conj]{g-} with epenthetic (filler) vowel \fm{a}

\item[ga-]\label{m:ga-cmpv}
	irregular allomorph of comparative \X[ka-cmpv]{ka-}

\item[ga-]\label{m:ga-sben}
	self-benefactive prefix, occurs with transitive verbs and requires \fm{d-};
	unclear if a \fm{g-} allomorph is possible;
	predicted to cooccur with \fm{g-} conjugation prefix but not attested;
	unclear if cooccurrence with \fm{ga-} comparative is possible
	\begin{itemize}
	\item	\vbform{at gawtudzi.ée}{pfv}[tr, \fm{∅}, ach]{we cooked something for ourselves}
			\vbmorph{at=&\gm{ga-}&w-&tu-&\gm{d-}&s-&i-&\rt[¹]{.i}&-μμH}
				{\xx{ind.n.o}&\xx{sben}&\xx{pfv}&\xx{1pl.s}&\xx{mid}&\xx{csv}&\xx{stv}&\rt[¹]{cooked}&\·\xx{var}}
		\versus \vbform{at wutusi.ée}{pfv}{we cooked something}
			\vbmorph{at=&wu-&tu-&s-&i-&\rt[¹]{.i}&-μμH}
				{\xx{ind.n.o}&\xx{pfv}&\xx{1pl.s}&\xx{csv}&\xx{stv}&\rt[¹]{cooked}&\·\xx{var}}
	\end{itemize}

\item[gáakt=]\label{m:gáakt=}

\item[gáakx̱=]\label{m:gáakx̱=}

\item[gáande=]\label{m:gáande=}

\item[gáani=]\label{m:gáani=}
	Locational preverb ‘outside’ indicating location outside of a building or cave.
	Derived from the noun \fm{gáan} ‘outside’
		with the special locative postposition allomorph
		\X[-i-loc]{-i} \~\ \X[-í-loc]{-í} ‘at’
		(instead of the regular allomorphs \fm{-xʼ} and \X[-μ-loc]{-μ}).
	This locative allomorph is unique in that it only occurs with a handful of preverbs,
		for which see the detailed entry of \X[-í-loc]{-í}.
	Compare the directional preverbs \X{gáande=},
		\X{gáant=},
		and \X{gánde=}
		which are based on the same noun.
	As with other preverbs including \X[-i-loc]{-i}, \fm{gáani=} can be represented either
		fully segmented as \fm{gáan-i=}
		or unsegmented as \fm{gáani=}.
	\begin{enumerate}
	\item	In forms with \X{yux̱=} ‘out’ based on the motion derivation
			\motderiv{yux̱=}{n, \fm{yoo=i-…-k} rep}{out, outside}.
		\begin{itemize}
		\item	\vbform{gáani yux̱ kawdudli.oo}{pfv}[tr, \fm{n}, mot]{people made him live outside}
			\parencite[257.4]{swanton:1909}
				\vbmorph{\gm{gáan}&\gm{-i=}&\gm{yux̱=}&ka-&w-&du-&d-&l-&i-&\rt[¹]{.u}&-μμL}
					{outside&\·\xx{loc}&out&\xx{qual}&\xx{pfv}&\xx{ind.h.s}&\xx{mid}&\xx{csv}&\xx{stv}&\rt[¹]{live}&\·\xx{var}}
		\item	\vbform{gáani yux̱ woogoot}{pfv}[subj intr, \fm{n}, mot]{it went outside}
			\parencite[220.54]{dauenhauer-dauenhauer:1987}
				\vbmorph{\gm{gáan}&\gm{-i=}&\gm{yux̱=}&wu-&μ-&\rt[¹]{gut}&-μμL}
					{outside&\·\xx{loc}&out&\xx{pfv}&\xx{stv}&\rt[¹]{go.\xx{sg}}&\·\xx{var}}
		\end{itemize}
	\item	In forms with \X{ḵux̱=} ‘back’ based on the motion derivation
			\motderiv{ḵux̱=/ḵúx̱de= d-}{∅, \fm{-ch} rep}{back, returning}.
		\begin{itemize}
		\item	\vbform{gáani ḵux̱ wujiḵáḵ}{pfv}[subj intr, \fm{∅}, mot]{he fell back outside}
			\parencite[260.3]{swanton:1909}
				\vbmorph{\gm{gáan}&\gm{-i=}&ḵux̱=&wu-&d-&sh-&i-&\rt[¹]{ḵaḵ}&-μH}
					{outside&\·\xx{loc}&\xx{rev}&\xx{pfv}&\xx{mid}&\xx{pej}&\xx{stv}&\rt[¹]{squat}&\·\xx{var}}
		\item	\vbform{gáani ḵux̱ has wudikélʼ}{pfv}[subj intr, \fm{∅}, mot]{they fled back outside}
			\parencite[260.11]{swanton:1909}
				\vbmorph{\gm{gáan}&\gm{-i=}&ḵux̱=&has=&wu-&d-&i-&\rt[¹]{kelʼ}&-μH}
					{outside&\·\xx{loc}&\xx{rev}&\xx{plh}&\xx{pfv}&\xx{mid}&\xx{stv}&\rt[¹]{flee}&\·\xx{var}}
		\end{itemize}
	\end{enumerate}

\item[gáant=]\label{m:gáant=}

\item[gági=]\label{m:gági=}
	Locational preverb ‘emerging, out in the open’ indicating location in an open or visible area,
		usually as a result of movement from an enclosed or obscured location.
	Derived from the noun \fm{gáak} ‘protrusion’
		with the special locative postposition allomorph
		\X[-i-loc]{-i} \~\ \X[-í-loc]{-í} ‘at’
		(instead of the regular allomorphs \fm{-xʼ} and \X[-μ-loc]{-μ}).
	This locative allomorph is unique in that it only occurs with a handful of preverbs,
		for which see the detailed entry of \X[-í-loc]{-í}.
	Compare the directional preverbs
		\X{gáakt=},
		\X{gáakx̱=},
		and \X{gákx̱=}
		which are based on the same noun \fm{gáak} ‘protrusion’.
	As with other preverbs including \X[-i-loc]{-i}, \fm{gági=} can be represented either
		fully segmented as \fm{gág-i=}
		or unsegmented as \fm{gági=}.
	Glosses can be literal such as ‘protrusion-\xx{loc}’
		or more contextualized such as ‘emerging’ or ‘into open’.
	\begin{enumerate}
	\item	As part of the motion derivation
		\motderiv{gági}{\fm{∅}, \fm{-x̱} rep}{emerging, out into the open}.
		\begin{itemize}
		\item	\vbform{gági uwagút}{pfv}[subj intr, \fm{∅}, mot]{he appeared}
				\parencite[20.78]{story-naish:1973}
				\vbmorph{\gm{gág}&\gm{-i=}&u-&wa-&\rt[¹]{gut}&-μH}
					{emerge&\·\xx{loc}&\xx{zpfv}&\xx{stv}&\rt[¹]{go.\xx{sg}}&\·\xx{var}}
		\item	\fm{gági uwaháa du waḵshayeexʼ chʼáakʼ ḵuyéik}
				‘it emerged before his eyes, the eagle spirit’
				\parencite[01/6]{leer:1973}
			containing
			\vbform{gági uwaháa}{pfv}[obj intr, \fm{∅}, mot]{she/he/it emerged, appeared out in open}
				\vbmorph{\gm{gág}&\gm{-i=}&u-&wa-&\rt[¹]{haᴸ}&-μμH}
					{emerge&\·\xx{loc}&\xx{zpfv}&\xx{stv}&\rt[¹]{appear}&\·\xx{var}}
		\end{itemize}
	\item	In forms with unclear conjugation class and motion derivation.
		\begin{itemize}
		\item	\vbform{gági wdudli.aat}{pfv}[tr, \fm{n}?, mot]{people brought them out}
				\parencite[xxi]{nyman-leer:1993}
				\vbmorph{\gm{gág}&\gm{-i=}&w-&du-&d-&l-&i-&\rt[¹]{.at}&-μμL}
					{emerge&\·\xx{loc}&\xx{pfv}&\xx{ind.h.s}&\xx{mid}&\xx{csv}&\xx{stv}&\rt[¹]{go.\xx{pl}}&\·\xx{var}}
			\newline
			This form is puzzling because the \X{-μμL} stem variation entails that
				this is not a \fm{∅} conjugation class verb form;
				if it were the \X{-μH} stem would occur instead.
			The verb could be any one of \fm{n}, \fm{g}, or \fm{g̱} conjugation class,
				but this form is ambiguous so without examples in other aspects
				it is impossible to say which conjugation class this belongs to,
				and consequently what motion derivation has been applied to produce
				this form.
			The name \fm{Gágiwdul.aat} is a nominalization of this verb form.
		\end{itemize}
	\end{enumerate}

\item[gákx̱=]\label{m:gákx̱=}

\item[gánde=]\label{m:gánde=}

\item[gug̱a]
	≡ \fm{g-u-g̱a-}
	combination of conjugation \fm{g-},
		irrealis \fm{u-},
		and  modal \fm{g̱a-},
		together indicating prospective (‘future’) aspect;
	this form occurs when there is no subject prefix and no
		immediately preceding vowel from an incorporated noun, object prefix, preverb, etc.;
	\fm{kg̱wa} occurs instead if there is a preceding vowel;
	compare \fm{kuḵa} with first person singular subject \fm{x̱a-}
	\begin{itemize}
	\item	\vbform{at gug̱ax̱áa}{prosp}[tr, \fm{∅}, \fm{-μμH} act]{she/he/it will eat something}
			\vbmorph{at=&g-&u-&g̱a-&\rt[²]{x̱a}&-μμH}
				{\xx{ind.n.o}&\xx{gcnj}&\xx{irr}&\xx{mod}&\rt[²]{eat}&\·\xx{var}}
		\versus \vbform{akg̱wax̱áa}{prosp}{she/he/it will eat him/her/it}
			\vbmorph{a-&g-&u-&g̱a-&\rt[²]{x̱a}&-μμH}
				{\xx{3>3}&\xx{gcnj}&\xx{irr}&\xx{mod}&\rt[²]{eat}&\·\xx{var}}
	\end{itemize}
\end{morphdesc}
