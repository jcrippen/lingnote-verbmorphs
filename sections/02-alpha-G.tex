%!TEX root = ../lingnote-verbmorphs.tex

\subsection{G}\label{sec:alphalist-g}
\begin{morphdesc}[resume*=alphalist]
\item[g-]\label{m:g-conj}
	\fm{g} conjugation prefix, upward spatial orientation;
	prospective aspect prefix with irrealis \fm{w-} and modal \fm{g̱-};
	can occur together with comparative \fm{g-}

\item[g-]\label{m:g-cmpv}
	irregular allomorph of comparative \X[ka-cmpv]{ka-}
	\begin{itemize}
	\item	\vbform{chʼa yéi googéikʼ}{impfv}[obj intr, \fm{n}, \fm{-μμH} cmpv state]{just a little}
			\vbmorph{chʼa&yéi=&g-&u-&μ-&\rt[¹]{ge}&-μμH&-kʼ}
				{just&thus&\xx{cmpv}&\xx{irr}&\xx{stv}&\rt[¹]{big}&\·\xx{var}&\·\xx{dim}}
		\versus \vbform{yagéi}{impfv}[obj intr, \fm{g}, \fm{-μμH} state]{it is big}
			\vbmorph{ÿa-&\rt[¹]{ge}&-μμH}
				{\xx{stv}&\rt[¹]{big}&\·\xx{var}}
	\end{itemize}

\item[ga-]\label{m:ga-conj}
	allomorph of \fm{g} conjugation prefix \X[g-conj]{g-} with epenthetic (filler) vowel \fm{a}

\item[ga-]\label{m:ga-cmpv}
	irregular allomorph of comparative \X[ka-cmpv]{ka-}

\item[ga-]\label{m:ga-sben}
	self-benefactive prefix, occurs with transitive verbs and requires \fm{d-};
	unclear if a \fm{g-} allomorph is possible;
	predicted to cooccur with \fm{g-} conjugation prefix but not attested;
	unclear if cooccurrence with \fm{ga-} comparative is possible
	\begin{itemize}
	\item	\vbform{at gawtudzi.ée}{pfv}[tr, \fm{∅}, ach]{we cooked something for ourselves}
			\vbmorph{at=&\gm{ga-}&w-&tu-&\gm{d-}&s-&i-&\rt[¹]{.i}&-μμH}
				{\xx{ind.n.o}&\xx{sben}&\xx{pfv}&\xx{1pl.s}&\xx{mid}&\xx{csv}&\xx{stv}&\rt[¹]{cooked}&\·\xx{var}}
		\versus \vbform{at wutusi.ée}{pfv}{we cooked something}
			\vbmorph{at=&wu-&tu-&s-&i-&\rt[¹]{.i}&-μμH}
				{\xx{ind.n.o}&\xx{pfv}&\xx{1pl.s}&\xx{csv}&\xx{stv}&\rt[¹]{cooked}&\·\xx{var}}
	\end{itemize}

\item[gáande=]\label{m:gáande=}

\item[gáani=]\label{m:gáani=}

\item[gáant=]\label{m:gáant=}

\item[gági=]\label{m:gági=}
	directional preverb ‘emerging, out into the open’;
	derived from noun \fm{gáak} ‘protrusion’
		with special locative postposition \fm{-í} \~\ \fm{-i};
	occurs in motion derivation
		\motderiv{gági}{\fm{∅}, \fm{-x̱} rep}{emerging, out into the open}
	\begin{itemize}
	\item	\fm{gági uwaháa du waḵshayeexʼ chʼáakʼ ḵuyéik}
		‘it emerged before his eyes, the eagle spirit’
		\parencite[01/6]{leer:1973}
		containing \vbform{gági uwaháa}{pfv}[obj intr, \fm{∅}, mot]{she/he/it emerged, appeared out in open}
			\vbmorph{gági=&u-&wa-&\rt[¹]{haᴸ}&-μμH}
				{emerge=&\xx{zpfv}&\xx{stv}&\rt[¹]{appear}&\·\xx{var}}
	\end{itemize}

\item[gánde=]\label{m:gánde=}

\item[gug̱a]
	≡ \fm{g-u-g̱a-}
	combination of conjugation \fm{g-},
		irrealis \fm{u-},
		and  modal \fm{g̱a-},
		together indicating prospective (‘future’) aspect;
	this form occurs when there is no subject prefix and no
		immediately preceding vowel from an incorporated noun, object prefix, preverb, etc.;
	\fm{kg̱wa} occurs instead if there is a preceding vowel;
	compare \fm{kuḵa} with first person singular subject \fm{x̱a-}
	\begin{itemize}
	\item	\vbform{at gug̱ax̱áa}{prosp}[tr, \fm{∅}, \fm{-μμH} act]{she/he/it will eat something}
			\vbmorph{at=&g-&u-&g̱a-&\rt[²]{x̱a}&-μμH}
				{\xx{ind.n.o}&\xx{gcnj}&\xx{irr}&\xx{mod}&\rt[²]{eat}&\·\xx{var}}
		\versus \vbform{akg̱wax̱áa}{prosp}{she/he/it will eat him/her/it}
			\vbmorph{a-&g-&u-&g̱a-&\rt[²]{x̱a}&-μμH}
				{\xx{3>3}&\xx{gcnj}&\xx{irr}&\xx{mod}&\rt[²]{eat}&\·\xx{var}}
	\end{itemize}
\end{morphdesc}
