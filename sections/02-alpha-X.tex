%!TEX root = ../lingnote-verbmorphs.tex

\subsection{X}\label{sec:alphalist-x}
\begin{morphdesc}[resume*=alphalist]
\item[-xʼ]\label{m:-xʼ}
	plural/repetitive suffix describing either
		a plural number of eventualities (pluractional)
		or a plural number of entities (plural object),
		or perhaps both at the same time in some contexts;
	unlike the \X{-ch}, \X{-k}, and \X{-x̱} repetitive suffixes,
		this suffix is never specified by conjugation class
		or as part of a motion derivation;
	this suffix is identical to the plural \fm{-xʼ} \~\ \fm{-xʼw} of nouns
		but it occurs on verbs and so its meaning differs;
	it may be glossed as plural \xx{pl} or repetitive \xx{rep};
	although \fm{-xʼ} \~\ \fm{-xʼw} sometimes quantifies objects, 
		it should not be glossed as \xx{pl.o} (with \xx{o} for object)
		or similar because this can misleadingly imply that it
		is an object marker which is never the case;
	homophonous with the locative postposition \fm{-xʼ}
		but has a distinct distribution and the two can cooccur
	\newline
	allomorphs:
	\begin{allolist}
	\item[-xʼ]	basic form
	\item[\X{-xʼw}]	form with labialization
	\end{allolist}
	\begin{enumerate}
	\item	repetitive suffix used with a wide variety of activity
			and achievement verb roots
			\parencite[534]{crippen:2019};
		appears in a repetitive imperfective form
			or in forms derived from the repetitive imperfective
			(secondary aspectual derivations, 
			\citeauthor{leer:1991}’s (\citeyear{leer:1991}) “epiaspect”);
		verbs using this suffix may be of any conjugation class
			but there is a tendency for \fm{∅}
			and \fm{n} conjugation (probably due to regular frequency);
		most verbs with \fm{-xʼ} \~\ \fm{-xʼw} are transitive,
			but there are both object intransitive
			and subject intransitive verbs as well;
		in some cases the repetitive imperfective predicted from conjugation class
			is also documented, but in other cases it is absent
			and the only documented repetitive imperfective has \fm{-xʼ} \~\ \fm{-xʼw}
		\begin{itemize}
		\item	\vbform{achʼínxʼ}{rep impfv}{she/he is tying it (hair, with ribbon) repeatedly}
			\parencite[603]{leer:1976}
				\vbmorph{a-&\rt[²]{chʼin}&-μH&-x̱}
					{\xx{3>3}&\rt[²]{tie.ribbon}&\·\xx{var}&\·\xx{rep}}
			\exand \vbform{achʼínxʼ}{rep impfv}{she/he is tying it here and there}
			\parencite[603]{leer:1976}
				\vbmorph{a-&\rt[²]{chʼin}&-μH&\gm{-xʼ}}
					{\xx{3>3}&\rt[²]{tie.ribbon}&\·\xx{var}&\·\xx{pl}}
			\versus \vbform{achʼéen}{impfv}[tr, \fm{∅}, \fm{-μμH} act]{she/he is tying it}
			\parencite[603]{leer:1976}
				\vbmorph{a-&\rt[²]{chʼin}&-μμH}
					{\xx{3>3}&\rt[²]{tie.ribbon}&\·\xx{var}}
		\item	roots attested with \fm{-xʼ} \~\ \fm{-xʼw}
			in a repetitive imperfective form include
			\parencites[58]{story:1966}[534]{crippen:2019}:
			\begin{inlinelist}
			\item	\fm{\rt[²]{chʼin}} ‘tie bow’
			\item	\fm{\rt[²]{dlaḵ}} ‘win, obtain’
			\item	\fm{\rt[²]{gish}} ‘soak; kelp’
			\item	\fm{\rt[¹]{gut}} ‘sg.\ go’
			\item	\fm{\rt[²]{han}} ‘cut into strips’
			\item	\fm{\rt[²]{hat}} ‘cover’
			\item	\fm{\rt[²]{hiᴸ}} ‘pay shaman’
			\item	\fm{\rt[¹]{hik}} ‘filled’
			\item	\fm{\rt[²]{hisʼ}} ‘borrow’
			\item	\fm{\rt[²]{hits}} ‘singe’
			\item	\fm{\rt[²]{hut}} ‘plank boat’
			\item	\fm{\rt[²]{.in}} ‘handle container’
			\item	\fm{\rt[¹]{kan}} ‘wave, flutter’
			\item	\fm{\rt[¹]{keᴴn}} ‘jump’
			\item	\fm{\rt[²]{kwach}} ‘handle handful’
			\item	\fm{\rt[²]{kʼwach}} ‘break’
			\item	\fm{\rt[¹]{ḵux̱}} ‘go by boat’
			\item	\fm{\rt[²]{na}} ‘die; inherit’
			\item	\fm{\rt[²]{nal}} ‘steam’
			\item	\fm{\rt[²]{naᴴsh}} ‘shake off’
			\item	\fm{\rt[²]{sin}} ‘hide, conceal’
			\item	\fm{\rt[²]{shaʷ}} ‘marry; woman’
			\item	\fm{\rt[¹]{suᴴs}} ‘fall, scatter’
			\item	\fm{\rt[¹]{sʼis}} ‘wind blown’
			\item	\fm{\rt{tin}} ‘thing’ (from Eng.\ \fm{thing})
			\item	\fm{\rt{tʼit}} ‘	beachcomb, scavenge’
				(from \fm{\rt[²]{tʼiᴸ}} ‘find’ + \X{-t} but \fm{-μμH} not \fm{-μμL})
			\item	\fm{\rt[²]{tiÿ}} ‘soak’
			\item	\fm{\rt[²]{tuᴴk}} ‘pop’
			\item	\fm{\rt[²]{tul}} ‘spin, drill’
			\item	\fm{\rt[¹]{tuᴴl}} ‘murmur’
			\item	\fm{\rt[²]{tux̱}} ‘spit’
			\item	\fm{\rt[²]{tʼaᴸ}} ‘hot’
			\item	\fm{\rt[²]{tleḵw}} ‘snatch; finger’
			\item	\fm{\rt[²]{tsis}} ‘float’
			\item	\fm{\rt[²]{tsuᴴw}} ‘push, jab’
			\item	\fm{\rt[²]{.u}} ‘put, place, leave’ (see below)
			\item	\fm{\rt[²]{.uᴴn}} ‘shoot, fire (gun)’
			\item	\fm{\rt[²]{xatʼ}} ‘drag’
			\item	\fm{\rt[²]{xwach}} ‘tan’
			\item	\fm{\rt[²]{xweᴴn}} ‘ladle’
			\item	\fm{\rt[²]{x̱a}} ‘paddle’
			\item	\fm{\rt[²]{x̱ach}} ‘tow’
			\item	\fm{\rt[²]{x̱ʼeᴴÿ}} ‘encourage’
			\item	\fm{\rt[²]{ya}} ‘backpack’
			\item	\fm{\rt[²]{yeᴴn}} ‘wave’
			\item	\fm{\rt[²]{yesʼ}} ‘dye, stain, dark, discoloured’
			\item	\fm{\rt[²]{yiḵ}} ‘mouth, bite’
			\item	\fm{\rt[²]{yiḵ}} ‘pull’
			\end{inlinelist}
		\end{itemize}
	\item	plural suffix used with dimensional state verbs when
			the object is plural;
		occurs together with \X{d-} for unknown reasons
			\parencites[93]{story:1966}[99]{leer:1991}[458]{crippen:2019};
		note that \fm{-xʼ} \~\ \fm{-xʼw} does not occur
			when using the comparative derivation
			with comparative \X[ka-cmpv]{ka-}
			(which see for discussion of comparatives)
			for a plural object even though \X{d-} still occurs
			suggesting that \fm{-xʼ} is somehow blocked
		\begin{itemize}
		\item	\vbform{yagéi}{impfv}[obj intr, \fm{g}, \fm{-μμH} state]{she/he/it is big, much}
				\vbmorph{ÿa-&\rt[¹]{ge}&-μμH}
					{\xx{stv}&\rt[¹]{big}&\·\xx{var}}
			\versus \vbform{digéixʼ}{impfv}{they are big, numerous}
				\vbmorph{d-&i-&\rt[¹]{ge}&-μμH&\gm{-xʼ}}
					{\xx{mid}&\xx{stv}&\rt[¹]{big}&\·\xx{var}&\·\xx{pl}}
		\item	\vbform{a yáanax̱ koogéi}{impfv}[obj intr, \fm{g}, \fm{-μμH} state]{she/he/it is bigger than it}
				\vbmorph{k-&u-&μ-&\rt[¹]{ge}&-μμH}
					{\xx{cmpv}&\xx{irr}&\xx{stv}&\rt[¹]{big}&\·\xx{var}}
			\versus \vbform{a yáanáx̱ kudigéi}{impfv}{they are bigger than it}
				\vbmorph{k-&u-&d-&i-&\rt[¹]{ge}&-μμH}
					{\xx{cmpv}&\xx{irr}&\xx{mid}&\xx{stv}&\rt[¹]{big}&\·\xx{var}}
				\andnot{\fm[*]{a yáanáx̱ kudigéixʼ}}
		\item	\vbform{dlidálxʼíḵ}{impfv}[obj intr, \fm{g}, \fm{-μH} state]{(fish flesh) is heavy}
				\parencite[05/51]{leer:1973}
				\vbmorph{d-&l-&i-&\rt[¹]{dal}&-μH&-xʼ&-ḵ}
					{\xx{mid}&\xx{xtn}&\xx{stv}&\rt[¹]{heavy}&\·\xx{var}&\·\xx{pl}&\·\xx{opt}?}
			\begin{itemize}
			\item	this is a unique case with \fm{-xʼ} and \X[-ḵ-optphib]{-ḵ};
				see \X{-ḵxʼ} for a possibly related combination
			\end{itemize}
		\item	roots attested with \fm{-xʼ} \~\ \fm{-xʼw}
			in a dimensional state form include
			\parencite[459]{crippen:2019}:
			\begin{inlinelist}
			\item	\fm{\rt[¹]{dal}} ‘heavy’
			\item	\fm{\rt[¹]{ge}} ‘big, much’
			\item	\fm{\rt[¹]{kak}} ‘thick’
			\item	\fm{\rt[¹]{sa}} ‘narrow’
			\item	\fm{\rt[¹]{tla}} ‘stout’
			\item	\fm{\rt[¹]{wux̱ʼ}} ‘wide’
			\item	\fm{\rt[¹]{ÿatʼ}} ‘long’
			\end{inlinelist}
		\end{itemize}
	\item	plural suffix in some state verbs (excluding dimensional state verbs)
		\begin{itemize}
		\item	\fm{–téixʼ} ‘rocky’
			from \fm{\rt{te}} ‘rock, stone’ (noun \fm{té} ‘rock, stone’) in
			\newline
			\vbform{kadzitéixʼ}{impfv}[obj intr, conj?, state]{it is rocky}
			\parencite[06/123]{leer:1973}
				\vbmorph{ka-&d-&s-&i-&\rt{te}&-μμH&\gm{-xʼ}}
					{\xx{hsfc}&\xx{mid}&\xx{intr}&\xx{stv}&\rt{rock}&\·\xx{var}&\·\xx{pl}}
		\item	\fm{–sʼaaḵxʼ} ‘bony’
			from \fm{\rt{sʼaḵ}} ‘bone’ (noun \fm{sʼaaḵ} ‘bone’) in
			\newline
			\vbform{dlisʼaaḵxʼ}{impfv}[obj intr, conj?, state]{they are bony}
			\parencite[33.273]{story-naish:1973}
				\vbmorph{d-&lˢ-&i-&\rt{sʼaḵ}&-μμL&\gm{-xʼ}}
					{\xx{mid}&\xx{intr}&\xx{stv}&\rt{bone}&\·\xx{var}&\·\xx{pl}}
		\item	\fm{–tsínxʼ} ‘expensive’
			from \fm{\rt[¹]{tsin}} ‘alive; strong’ in
			\newline
			\vbform{dax̱ x̱ʼadlitsínxʼ}{impfv}{they are each expensive}
			\parencite[154.1173]{nyman-leer:1993}
				\vbmorph{dax̱=&x̱ʼe-&d-&lˢ-&i-&\rt[¹]{tsin}&-μH&\gm{-xʼ}}
					{\xx{distb}&mouth&\xx{mid}&\xx{xtn}&\xx{stv}&\rt[¹]{strong}&\·\xx{var}&\·\xx{pl}}
			\versus \vbform{x̱ʼalitseen}{impfv}{she/he/it is expensive}
				\vbmorph{x̱ʼe-&lˢ-&i-&\rt[¹]{tsin}&-μμL}
					{mouth&\xx{xtn}&\xx{stv}&\rt[¹]{strong}&\·\xx{var}}
		\end{itemize}
	\item	plural suffix in some nouns derived from verbs with \fm{-xʼ};
		crucially these are not plural nouns and instead have the \fm{-xʼ}
		as part of the verbs from which they are derived
		\begin{itemize}
		\item	\fm{dax̱áchxʼi} ‘tugboat’
			from \fm{\rt[²]{x̱ach}} ‘tow’
			\parencite[58]{story:1966}
				\vbmorph{da-&\rt[²]{x̱ach}&-μH&\gm{-xʼ}&-i}
					{\xx{apsv}&\rt[²]{tow}&\·\xx{var}&\·\xx{pl}&\·\xx{nmz}}
			\versus \vbform{dax̱áchxʼ}{rep impfv}[subj intr, \fm{n}?, act?]{she/he/it is towing}
			\parencite[790]{leer:1976}
				\vbmorph{da-&\rt[²]{x̱ach}&-μH&\gm{-xʼ}}
					{\xx{apsv}&\rt[²]{tow}&\·\xx{var}&\·\xx{pl}}
			\exalso \vbform{ax̱áchxʼ}{rep impfv}[tr, \fm{n}?, act?]{she/he/it is towing them}
			\parencite[790]{leer:1976}
				\vbmorph{a-&\rt[²]{x̱ach}&-μH&\gm{-xʼ}}
					{\xx{3>3}&\rt[²]{tow}&\·\xx{var}&\·\xx{pl}}
		\end{itemize}
	\item	repetitive suffix in the suffix combination \X{-x̱xʼ} ≡ \X{-x̱} + \fm{-xʼ};
		occurs with the two roots
			\fm{\rt[¹]{tiᴸ}} ‘be, exist’
			and \fm{\rt[²]{.u}} ‘put, place, leave’;
		see \X{-x̱xʼ} for more details
	\item	unclear meaning in the suffix combination \X{-ḵxʼ} ≡ \fm{-ḵ} + \X{-xʼ}
			with deprivative \X[-ḵ-dprv]{-ḵ};
		associated with an increase in degree
			but the composition of meaning is unclear;
		see \X{-ḵxʼ} for more details
	\item	plural suffix frozen in a few roots
		\begin{itemize}
		\item	\fm{\rt[¹]{gaxʼ}} ‘annoy by noise’
			probably from \fm{\rt{gaw}} ‘noise’ in
			\newline
			\vbform{wudigáxʼ}{pfv}[obj intr, \fm{∅}, ach]{she/he/it became annoyed by noise}
				\vbmorph{wu-&d-&i-&\rt[¹]{gaxʼ}&-μH}
					{\xx{pfv}&\xx{mid}&\xx{stv}&\rt[¹]{annoy.noise}&\·\xx{var}}
			\versus \vbform{ligaaw}{impfv}[obj intr, \fm{g}, \fm{-μμL} state]{she/he/it is noisy}
				\vbmorph{l-&i-&\rt[⁰]{gaw}&-μμL}
					{\xx{intr}&\xx{stv}&\rt[¹]{noise}&\·\xx{var}}
			\newline
			the root \fm{\rt{gaxʼw}} ‘riddle’ may also be related,
			and \fm{\rt{gaw}} ‘noise’ may further be related to
			\fm{\rt{ga}} ‘(neg.\ only) still, unmoving’ and \fm{\rt{ga}} ‘delay’
		\item	\fm{\rt[¹]{x̱exʼw}} ‘plural sleep‘
			from \fm{\rt[¹]{x̱i}} \~\ \fm{\rt[¹]{x̱e}} ‘overnight’ in
			\newline
			\vbform{has woox̱éixʼw}{pfv}[subj intr, \fm{n}, \fm{-μH} act]{they slept}
				\vbmorph{has=&wu-&μ-&\rt[¹]{x̱exʼw}&-μμH}
					{\xx{plh}&\xx{pfv}&\xx{stv}&\rt[¹]{sleep.\xx{pl}}&\xx{var}}
			\versus \vbform{uwax̱ée}{pfv}[subj intr, \fm{∅}, \fm{-μμL} act]{she/he/it overnights}
				\vbmorph{u-&wa-&\rt[¹]{x̱i}&-μμH}
					{\xx{zpfv}&\xx{stv}&\rt[¹]{overnight}&\·\xx{var}}
			\newline
			the labialization of \fm{-xʼw} in \fm{\rt{x̱exʼw}} is unexpected
		\end{itemize}
	\item	unexplained suffix in obscure “\fm{ka-dz-g̱eˋḵxʼ} (brush) is impenetrable”
		given without futher detail at \cite[72]{leer:1978b}, origin unknown
		and root unidentified
	\end{enumerate}

\item[-xʼw]\label{m:-xʼw}
	allomorph of plural/repetitive suffix \X{-xʼ} with labialization;
	as with other posterior (velar, uvular, glottal) consonants the orthography
		does not indicate predictable spread of labialization within a syllable;
	labialized [\ipa{xʼʷ}] is actually written as \fm{-xʼ} in most cases
		and so \fm{-xʼw} is only used when it is not predictable from a
		preceding labialized vowel (\fm{u}, \fm{oo}) or labialized consonant;
	for example \vbform{atsúwxʼ}{rep impfv}[tr, \fm{∅}, \fm{-μμH} act]{she/he/it is pushing, poking them}
		is pronounced [\ipa{ʔà.ˈtsʰúwxʼʷ}] with [\ipa{xʼʷ}] which is written \fm{xʼ}
		since labialization predictably spreads to it from [\ipa{ú}] and [\ipa{w}];
	thus when \fm{-xʼw} is written instead of \fm{-xʼ} it usually indicates
		unexpected labialization where there is no labialized vowel
		or consonant from which it would spread;
	compare \X{-kw}, \X{-kʼw} that show similar unexpected labialization patterns
	\begin{itemize}
	\item	\vbform{aÿadláḵxʼw}{rep impfv}[tr, \fm{n}, ach]{she/he/it is winning them}
		\parencite[459]{leer:1976}
			\vbmorph{a-&ÿa-&\rt[²]{dlaḵ}&-μH&\gm{-xʼw}}
				{\xx{3>3}&face&\rt[²]{win}&\·\xx{var}&\·\xx{pl}}
		\versus \vbform{aÿaawadlaaḵ}{pfv}{she/he won it/them}
			(unspecified number)
			\vbmorph{a-&ÿa-&μʷ-&wa-&\rt[²]{dlaḵ}&-μμL}
				{\xx{3>3}&face&\xx{pfv}&\xx{stv}&\rt[²]{win}&\·\xx{var}}
	\item	roots attested with unexpected \fm{-xʼw} include
		\parencite{leer:1976}:
		\begin{inlinelist}
		\item	\fm{\rt[¹]{gut}} ‘sg.\ go’
		\item	\fm{\rt[²]{dlaḵ}} ‘win, obtain’
		\item	\fm{\rt[²]{hits}} ‘singe’
		\item	\fm{\rt[²]{shaʷ}} ‘marry; woman’
		\item	\fm{\rt[²]{tiÿ}} ‘soak’
		\item	\fm{\rt[²]{tul}} ‘drill’
		\item	\fm{\rt[¹]{tuᴴl}} ‘murmur’
		\item	\fm{\rt[²]{ya}} ‘backpack’
		\end{inlinelist}
	\end{itemize}
\end{morphdesc}
