%!TEX root = ../lingnote-verbmorphs.tex

\subsection{U}\label{sec:alphalist-u}
\begin{morphdesc}[resume*=alphalist]
\item[u-]\label{m:u-irr}
	irrealis prefix

\item[u-]\label{m:u-pfv}
	\fm{∅} conjugation class perfective,
	occurring in some perfective aspect forms and some habitual aspect forms of
	\fm{∅} conjugation class verbs

\item[-uḵ]\label{m:-uḵ}
	allomorph of \X[-ḵ-dprv]{-ḵ} with labialization and epenthetic (filler) \fm{u}

\item[-úḵ]\label{m:-úḵ}
	allomorph of \X[-ḵ-dprv]{-ḵ} with labialization and epenthetic (filler) \fm{ú}

\item[ux̱=]\label{m:ux̱=}
	Manner preverb ‘out of control, blindly, amiss’
		indicating that the eventuality happens in a manner that is somehow unintentional,
		unconsidered, without clear vision, in error, without care, or unacceptable
		\parencite[134, 297]{leer:1991}.
	Most instances of this preverb occur together with the directional preverb
		\X{kei=} \~\ \X{kéi=} ‘up’ as part of a motion derivation detailed below.
	This preverb is poorly documented, with only a couple of attested instances,
		so it is difficult to provide a detailed description.
	It seems likely that this preverb was originally more productive but has become
		fossilized with a couple of verbs and is no longer usable in other contexts.
	The origin of this preverb is obscure, with no obvious correlates in verb roots or
		other areas of the lexicon.
	\newline
	Variants:
	\begin{allolist}
	\item[ux̱=]	basic form
	\item[\X{úx̱=}]	variant form with H tone attested from Southern varieties
	\end{allolist}
	\begin{enumerate}
	\item	In motion verbs as part of the motion derivation
			\motderiv{ux̱=kei=}{∅, \fm{-ch} rep}{out of control, blindly, amiss}.
		\begin{itemize}
		\item	\vbform{ux̱ kéi nasgít}{prog}[subj intr, \fm{g}, ach]{he is getting into trouble/mischief}
			\parencite[658]{leer:1976}
				\vbmorph{\gm{ux̱=}&kéi=&na-&d-&s-&\rt[¹]{git}&-μH}
					{blind&up&\xx{ncnj}&\xx{mid}&\xx{xtn}&\rt[¹]{fall.anim}&\·\xx{var}}
			\versus Southern \vbform{úx̱ kei gásgítch}{hab}{he always gets into mischief}
			\parencite[f05/93]{leer:1973}
				\vbmorph{\gm{úx̱=}&kei=&ga-&d-&s-&\rt[¹]{git}&-μH&-ch}
					{blind&up&\xx{gcnj}&\xx{mid}&\xx{xtn}&\rt[¹]{fall.anim}&\·\xx{var}&\·\xx{rep}}
		\item	\vbform{ux̱ kei uwanée}{pfv}[obj intr, \fm{∅}, mot]{it went wrong}
			\parencite[276]{leer:1976}
				\vbmorph{\gm{ux̱=}&kei=&u-&wa-&\rt[¹]{ni}&-μμH}
					{blind&up&\xx{zpfv}&\xx{stv}&\rt[¹]{happen}&\·\xx{var}}
			\exand Tongass \vbform{ux̱ keì u̥wanee}{pfv}{it went wrong}
			\parencite[04/121]{leer:1973}
				\vbmorph{\gm{ux̱=}&kei=&u-&wa-&\rt[¹]{ni}&-μμ}
					{blind&up&\xx{zpfv}&\xx{stv}&\rt[¹]{happen}&\·\xx{var}}
		\item	\vbform{ux̱ kéi uwatée wé ḵu.éexʼ}{pfv}[obj intr, \fm{∅}, mot]{the potlatch went berserk}
			\parencite[382]{leer:1976}
				\vbmorph{\gm{ux̱=}&kei=&u-&wa-&\rt[¹]{tiᴸ}&-μμH&wé&ḵu.éexʼ}
					{blind&up&\xx{zpfv}&\xx{stv}&\rt[¹]{be}&\·\xx{var}&\xx{mdst}&potlatch}
			\exalso \vbform{ux̱ kéi at uwatée}{pfv}{things went (the) wrong (way)}
			\parencite[382]{leer:1976}
				\vbmorph{\gm{ux̱}&kéi&at=&u-&wa-&\rt[¹]{tiᴸ}&-μμH}
					{blind&up&\xx{ind.n.o}&\xx{zpfv}&\xx{stv}&\rt[¹]{be}&\·\xx{var}}
		\item	\vbform{gu.aal kwshégé tlél ux̱ kéi utéeg̱ei}{rel phib impfv}[obj intr, \fm{∅}, mot]{hopefully it won’t be mistaken}
			(N.M.\ Dauenhauer 2013 l.\ 23)
				\vbmorph{gu.aal&=kwshé&=gé&tlél&\gm{ux̱=}&kéi=&u-&\rt[¹]{tiᴸ}&-μμH&-ḵ&-i&yéi}
					{\xx{opt}&\•\xx{dub}&\•\xx{yn}&\xx{neg}&blind&up&\xx{irr}&\rt[¹]{be}&\·\xx{var}&\·\xx{phib}&\·\xx{rel}&way}
			\exalso \vbform{wé a náx̱ ux̱ kéi uwatiyi x̱á}{sub pfv}[obj intr, \fm{∅}, mot]{the one from before he got lost, you see}
			\parencite[236.393]{dauenhauer-dauenhauer:1987}
				\vbmorph{wé&a&náḵ&\gm{ux̱=}&kéi=&u-&wa-&\rt[¹]{tiᴸ}&-μL&-yi&x̱á}
					{\xx{mdst}&\xx{3n}&away&blind&up&\xx{pfv}&\xx{stv}&\rt[¹]{be}&\·\xx{var}&\xx{sub}&\xx{pcl}}
		\end{itemize}
	\item	In a couple of obscure examples with two verbs based on the root
			\fm{\rt[²]{sʼiḵ}} \~\ \fm{\rt[²]{sʼeḵ}} ‘smoke’,
			describing the result of smoke tanning leather.
		These are the only examples of \fm{ux̱=} without a following \fm{kei=} \~\ \fm{kei=}.
		Because there is no further documentation of these examples aside from their
			appearance in \cite{leer:1973}, it is difficult to say much about these forms.
		It is possible that these are misreadings of a postposition phrase \fm{á-x̱} ‘it-of’
			with the third person nonhuman pronoun \fm{á} ‘it’
			and the pertingent postposition \fm{-x̱} ‘of, contacting’
			since the first Naish-Story orthography represented the sound \fm{a} [\ipa{a}]
			as orthographic \fm{u}.
		This postposition phrase \fm{á-x̱} ‘it-of’ would then plausibly be an applied argument
			with the applicative \fm{lˢ-} \~\ \fm{lˢa-}.
		The \fm{ux̱=} in these forms might also not be a misreading and instead could reflect
			a historical reanalysis of \fm{á-x̱} as identical to \fm{ux̱=}, but there is no
			obvious explanation for why this reanalysis would have occurred.
		\begin{itemize}
		\item	\vbform{ux̱ akasʼíḵx̱i yáx̱ yatee}{rep impfv}[tr?, \fm{∅}?, act?]{it is light tan like moccasins}
			\parencites[09/268]{leer:1973}[524]{leer:1976}
				\vbmorph{\gm{ux̱=}&a-&ka-&\rt[²]{sʼiḵ}&-μH&-x̱&-i}
					{blind&\xx{3>3}?&\xx{qual}&\rt[²]{smoke}&\·\xx{var}&\·\xx{rep}&\·\xx{sub}}
			\newline
			The lack of \fm{lˢ-} in this form is peculiar considering the other
				example below that looks like an applicative
				with \fm{lˢ-} \~\ \fm{lˢa-}.
		\item	Tongass \vbform{ax̱/ux̱? akàwli̥sʼeèḵ}{pfv}[tr?, \fm{n}?, act?]{he smoked it up}
			\parencites[09/268]{leer:1973}
				\vbmorph{\gm{ux̱=}&a-&ka-&w-&lˢ-&i-&\rt[²]{sʼiḵ}&-μμʰ}
					{blind&\xx{3>3}&\xx{qual}&\xx{pfv}&\xx{appl}?&\xx{stv}&\rt[²]{smoke}&\·\xx{var}}
			\versus \vbform{ax̱ akalasʼiḵx̱}{rep impfv}{he’s smoking it}
			\parencites[09/268]{leer:1973}
				\vbmorph{a&-x̱&a-&ka-&lˢa-&\rt[²]{sʼiḵ}&-μ&-x̱}
					{\xx{3n}&\·\xx{pert}&\xx{3>3}&\xx{qual}&\xx{appl}&\rt[²]{smoke}&\·\xx{var}&\·\xx{rep}}
			\exalso \vbform{ax̱ akawli̥sʼeèḵ}{rep impfv}{he smoked it}
			\parencites[09/268]{leer:1973}
				\vbmorph{a&-x̱&a-&ka-&w-&lˢ-&i-&\rt[²]{sʼiḵ}&-μμ}
					{\xx{3n}&\·\xx{pert}&\xx{3>3}&\xx{qual}&\xx{pfv}&\xx{appl}&\xx{stv}&\rt[²]{smoke}&\·\xx{var}}
		\end{itemize}
	\end{enumerate}
	
\item[úx̱=]\label{m:úx̱=}
	Variant form of manner preverb \X{ux̱=} ‘out of control, blindly’
		attested from Southern varieties.
	All instances of this variant form occur together with the directional preverb \X{kei=} ‘up’
		and never its variant form \X{kéi=},
		probably due to the alternating tone patterns (Obligatory Contour Principle)
		of Southern Tlingit prefix phonology.
	See \X{ux̱=} for more details.
	\begin{itemize}
	\item	\vbform{úx̱ kei násgít}{prog}[subj intr, \fm{g}, ach]{he’s getting into mischief}
		\parencite[f05/93]{leer:1973}
			\vbmorph{\gm{úx̱=}&kei=&na-&d-&s-&\rt[¹]{git}&-μH}
				{blind&up&\xx{ncnj}&\xx{mid}&\xx{xtn}&\rt[¹]{fall.anim}&\·\xx{var}}
		\versus \vbform{úx̱ kei gásgítch}{hab}{he always gets into mischief}
		\parencite[f05/93]{leer:1973}
			\vbmorph{\gm{úx̱=}&kei=&ga-&d-&s-&\rt[¹]{git}&-μH&-ch}
				{blind&up&\xx{gcnj}&\xx{mid}&\xx{xtn}&\rt[¹]{fall.anim}&\·\xx{var}&\·\xx{rep}}
	\end{itemize} 

\end{morphdesc}
