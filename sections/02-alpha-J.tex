%!TEX root = ../lingnote-verbmorphs.tex

\subsection{J}\label{sec:alphalist-j}
\begin{morphdesc}[resume*=alphalist]
\item[-j]\label{m:-j}
	orthographic variant of \X{-ch} when followed by a vowel;
	occurs as part of \X{-jaa} \~\ \X{-jáa} and \X{-ja} \~\ \X{-já}

\item[-ja]\label{m:-ja}
	allomorph of unknown suffix \X{-jaa} \~\ \X{-jáa};
	occurs between H tone syllable and \fm{-ÿi} so that \fm{-jaa} has a short vowel

\item[-já]\label{m:-já}
	allomorph of unknown suffix \X{-jaa} \~\ \X{-jáa};
	occurs between L tone syllable and \fm{-ÿi} so that \fm{-jáa} has a short vowel
	\begin{itemize}
	\item	\fm{shaa kʼeeljáyi} ‘windy storm of a mountain’
			\vbmorph*{\rt{kʼil}&-μμL&\gm{-ch}&\gm{-á}&-ÿi}
				{\rt{storm?}&\·\xx{var}&\·\xx{rep}&\·\xx{instr}&\·\xx{poss}}
		\versus \fm{kʼeeljáa} ‘windy storm; strong south wind’
			\vbmorph*{\rt{kʼil}&-μμL&\gm{-ch}&\gm{-áa}}
				{\rt{storm?}&\·\xx{var}&\·\xx{rep}&\·\xx{instr}}
	\end{itemize}

\item[-jaa]\label{m:-jaa}
	suffix with unknown meaning, 
		apparently a combination of repetitive \X{-ch}
		and instrument \X{-aa} \~\ \X{-áa},
		though the composition of meaning is unclear;
	as with \X{-aa} \~\ \X{-áa} this suffix has
		polar tone opposite the preceding syllable
		so \fm{-jaa} after an H tone syllable
		and \X{-jáa} after an L tone syllable,
	and becomes a short vowel when followed by \fm{-ÿi}
		so \X{-ja} + \fm{-ÿi}
			(not \fm[*]{-ja-ÿí})
		and \X{-já} + \fm{-ÿi};
	\newline
	allomorphs:
	\begin{allolist}
	\item[-jaa]	L tone form used after H tone syllable
	\item[\X{-jáa}]	H tone form used after L tone syllable
	\item[\X{-ja}]	short vowel L tone form when followed by \fm{-ÿi}
	\item[\X{-já}]	short vowel H tone form when followed by \fm{-ÿi}
	\end{allolist}
	\begin{enumerate}
	\item	in some verbs perhaps derived from nouns
		\begin{itemize}
		\item	\fm{–.ítʼjaa}
			from noun \fm{ítʼch} ‘glass’
			probably from \fm{\rt[¹]{.itʼ}} ‘soaked’ in
			\vbform{kawli.ítʼjaa}{pfv}[obj intr, \fm{g}, inv state]{it shone with reflected light}
				\vbmorph{ka-&w-&lˢ-&i-&\rt[¹]{.itʼ}&-μH&-ch&-aa}
					{\xx{hsfc}&\xx{pfv}&\xx{intr}&\xx{stv}&\rt[¹]{shine}&\·\xx{var}&\·\xx{rep}&\·\xx{inst}}
			\versus	\vbform{kadli.ítʼch}{impfv}[obj intr, \fm{g}, inv state]{it sparkles, reflects light}
					\vbmorph{ka-&d-&lˢ-&i-&\rt[¹]{.itʼ}&-μH&-ch}
						{\xx{hsfc}&\xx{mid}&\xx{intr}&\rt[¹]{sparkle}&\·\xx{var}&\·\xx{rep}}
		\end{itemize}
	\end{enumerate}

\item[-jáa]\label{m:-jáa}
	allomorph of unknown suffix \X{-jaa} with H tone,
		used after L tone syllable (polar tone);
	only one noun is attested with this suffix as shown below,
		with no corresponding verb root known
	\begin{itemize}
	\item	\fm{kʼeeljáa} ‘windy storm; strong south wind’
		\vbmorph*{\rt{kʼil}&-μμL&\gm{-ch}&\gm{-áa}}
			{\rt{storm?}&\·\xx{var}&\·\xx{rep}&\·\xx{instr}}
	\end{itemize}

\item[ji-]\label{m:ji-}
	incorporated noun indicating hand or possession;
	qualifier indicating object with extended projections (fingers);
	derived from relational nouns \fm{jín} ‘hand’ and \fm{jee} ‘possession’

\item[ji]\label{m:ji}
	≡ \fm{d-sh-i-}
	combination of voice \X{d-},
		valency \X{sh-},
		and stative \X[i-stv]{i-}
\end{morphdesc}
