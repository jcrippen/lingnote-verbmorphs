%!TEX root = ../lingnote-verbmorphs.tex
%%
%% Miscellaneous LaTeX packages.
%%

%% Microtypographic features.
%% Only turn this on for final drafts because it slows down compilation.
%\usepackage[protrusion=true,expansion=false]{microtype}

%% Colour.
\usepackage{xcolor}

%% Tweak the date and time formats.
\usepackage[UKenglish,cleanlook]{isodate}

%% Various symbols in text mode.
\usepackage{textcomp}

%% Several packages use up all 16 of LaTeX's \write registers.
%% This adds more \write registers and a dissertation-draft.mw file.
%\usepackage{morewrites}

%% Multiple column environments.
%%
%% NOTE: This has nothing to do with multiple column spans in tables. That feature
%%   is obtained with \multicolumn{<align>}{<text>} and is built into the tabular
%%   environment from memoir.
%%
%% Usage: \begin{multicols}{<num>} ... \columnbreak ... \end{multicols}
%% The \columnbreak is optional; when absent multicol breaks automatically.
\usepackage{multicol}
%% Length of the separation space between columns.
\setlength{\columnsep}{0ex}
%% This controls the space that appears before and after an instance of multicols.
%% The skips from ExPex are ignored in this environment, so this value is identical
%% with the values of aboveexskip and belowexskip for ExPex.
%\setlength{\multicolsep}{0.5em plus 0.375em minus 0.25em}
%% By default multicol tries to balance the tops and bottoms of columns, thus
%% justifying them. We never want this. The default is \flushcolumns.
\raggedcolumns

%% Control widows and orphans easily.
\usepackage[all,defaultlines=2]{nowidow}

%% Multiple row spanning cells in tables.
%%
%% Usage: \multirow{2}{*}{Two rows}
%\usepackage{multirow}

%% Big delimiters for multirow in tables.
%%
%% Usage: \ldelim<delimiter>{rows}{width/*}[centred text]
%\usepackage{bigdelim}

%% Rotating boxes.
\usepackage{rotating}

%% Tabbing.
%\usepackage{tabto}

%% Make list-like tables that pretend to be made with itemize or enumerate.
%\usepackage{listliketab}

%% Tool for adjusting boxes.
\usepackage{adjustbox}

%% Debugging tool to print LaTeX lengths.
%%
%% \printlength{<length>} does what you'd expect.
%% \uselengthunit{<unit>} sets the output unit, in \{pt, pc, in, mm, cm, bp, dd, cc\}.
%%   The default unit is pt.
%\usepackage{printlen}

%% TeX logos.
%%
%% The values here were determined empirically. They are specific to the Brill font.
\usepackage{metalogo}
\setLaTeXa{\scshape a}
\setlogodrop{0.4375ex}
\setlogokern{Xe}{-0.125ex}
\setlogokern{eL}{-0.1875ex}
\setlogokern{La}{-0.59375ex}
\setlogokern{aT}{-0.1875ex}
\setlogokern{Te}{-0.1875ex}
\setlogokern{eX}{-0.125ex}
\setlogokern{eT}{-0.1875ex}
