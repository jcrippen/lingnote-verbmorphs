%!TEX root = ../lingnote-verbmorphs.tex
%!TEX encoding = UTF-8 Unicode
%%%
%%% Hyperlinks.
%%%

%% URL formatting.
\usepackage{url}
\urlstyle{sf}

%% Hyperlinks.
%%
%% Hyperref controls the colours of links (citations, URLs, etc.) which are produced by BibLaTeX.
%% To turn off colouring completely, set both colorlinks=false and pdfborder={0 0 0}.
\usepackage[unicode=true,pdfencoding=auto,psdextra]{hyperref}

%% HyperXMP.
%%
%% This package supports the encoding of Adobe’s XMP metadata in documents via hyperref.
%% As of TeX Live 2023 this must be loaded after hyperref but before hypersetup below.
%%
%% Note: “the xdvipdfmx back end compresses all pdf objects, including the ones containing
%%   XMP metadata. While Adobe Acrobat can still detect and utilize the XMP metadata,
%%   non-PDF-aware applications are unlikely to see the metadata. Three options to consider
%%   are to (1) use a different program (e.g., LuaLaTeX), (2) pass the --output-driver=
%%   "xdvipdfmx -z0" option to XeLaTeX to instruct xdvipdfmx to turn off all compression
%%   (which will of course make the PDF file substantially larger), or (3) postprocess the
%%   generated pdf file by loading it into the commercial version of Adobe Acrobat and
%%   re-saving it with the Save As… menu option.”
\usepackage{hyperxmp}

%% Hyperref configuration. There are good reasons for this to be separate from the
%% \usepackage invocation above.
\hypersetup{%
	% Link everything in TOC, LOF, and LOT.
	linktoc=all,
	% Break links across lines into two separate identical links.
	breaklinks=true,
	% Whether to colour link text.
	colorlinks=false,
	%colorlinks=true,
	% Colours of link text {R G B}.
	linkcolor={rgb:blue,1;black,1},
	%linkcolor=black,
	anchorcolor=black,
	%citecolor=green,
	citecolor=black,
	filecolor=cyan,
	menucolor=red,
	runcolor=cyan,
	urlcolor=blue,
	% How links behave when clicked: /O outline, /I inverse, /N nothing, /P "pressed" inset
	% Note that many PDF viewers seem to ignore this.
	pdfhighlight=/P,
	% Thickness of border around links. (An alternative to coloured text.)
	% This is automatically disabled if colorlinks=true.
	%pdfborder={0 0 1},
	pdfborder={0 0 0},
	% Colours of borders, {R G B}
	citebordercolor={0 1 0},
	filebordercolor={1 .5 .5},
	linkbordercolor={1 0 0},
	menubordercolor={1 0 0},
	urlbordercolor={0 0 1},
	runbordercolor={0 .7 .7},
	%% Bookmarks configuration.
	%bookmarks=true,
	% List section numbers in bookmarks.
	bookmarksnumbered=true,
	% How deep bookmarks should go w.r.t. sectioning.
	bookmarksdepth=6,
	% Whether bookmarks are open by default.
	bookmarksopen=true,
	% How deep bookmarks should be open by default.
	bookmarksopenlevel=4,
	%% HyperXMP stuff.
	pdftitle={Tlingit verb morpheme catalogue},
	pdfauthor={James A. Crippen},
	pdfcopyright={Copyright © 2024 James A. Crippen},
	pdfsubject={Tlingit language},
	pdfkeywords={grammar, morphology, semantics, Na-Dene, Tlingit},
	pdfcaptionwriter={James A. Crippen},
	pdfcontactaddress={McGill Linguistics, 1085 Dr. Penfield},
	pdfcontactcity={Montreal},
	pdfcontactregion={QC},
	pdfcontactpostcode={H3A 1A7},
	pdfcontactcountry={Canada},
	pdfcontactphone={1-514-398-4222},
	pdfcontactemail={james.crippen@mcgill.ca},
	pdfcontacturl={https://www.mcgill.ca/linguistics/},
	pdflang={en-CA}
}

%% Bookmarks.
%%
%% This gives more control over PDF bookmarks, which are used to link to all the sectional divisions.
\usepackage[final]{bookmark}

%% A hack to get hyperref to shut up.
%% FIXME: There are probably ways to do this by redefining \fm using the LaTeX3 xparse package
%%   and \DeclareExpandableDocumentCommand. The problem is with moving arguments, I think.
%%   See hypbmsec.sty (loaded below) for another workaround.
%\pdfstringdefDisableCommands{\def\fm#1{#1}}

%% Sectioning (bookmarks) syntax extension.
%%
%% This extends the syntax of the sectioning commands (\part, \section, \paragraph, etc.)
%% so that they have an additional argument. This new argument, delimited by parentheses,
%% contains the text to be emitted in a PDF bookmark.
%%
%% Usage:
%%  \section(PDF bookmark string)[ToC entry string]{Section title}
\usepackage{hypbmsec}
